\documentclass[10pt,a4paper]{article}

\usepackage[pdftex,
pdfauthor={Gianfranco Zamboni},
pdftitle={Resumen: Paradigmas de Lenguajes de Programación},
pdfsubject={},
pdfkeywords={Resumen , Computacion, FCEyN, UBA, Paradigmas de Lenguajes de Programación, Imperativo, Funcional, Cálculo Lambda, Programación Orientada a Objetos, Objetos, Programación Lógica},
pdfproducer={Latex with hyperref},
pdfcreator={pdflatex}]{hyperref}

\usepackage{amsmath}
\usepackage{ amssymb }
\usepackage{bussproofs}

\usepackage[spanish]{babel}


\usepackage[utf8]{inputenc} % para poder usar tildes en archivos UTF-8
\usepackage{graphicx}
\usepackage{xcolor}
\usepackage{pifont}

\usepackage{lscape}
\usepackage{minted}
\usepackage{a4wide} % márgenes un poco más anchos que lo usual
\usepackage[titletoc,toc,page]{appendix}
\usepackage{tikz}
\usepackage{forest}
\usepackage{multicol}

\setlength{\columnsep}{1cm}

\ifthenelse{\paperwidth < \paperheight}{\usepackage{fancyhdr}
\pagestyle{fancy}

%\renewcommand{\chaptermark}[1]{\markboth{#1}{}}
\renewcommand{\sectionmark}[1]{\markright{\thesection\ - #1}}

\fancyhf{}

\fancyhead[LO]{Sección \rightmark} % \thesection\ 
\fancyfoot[LO]{\small{Paradigmas de lenguajes de programación}}
\fancyfoot[RO]{\thepage}
\renewcommand{\headrulewidth}{0.5pt}
\renewcommand{\footrulewidth}{0.5pt}
\setlength{\hoffset}{-0.25in}
\setlength{\textwidth}{16cm}
%\setlength{\hoffset}{-1.1cm}
%\setlength{\textwidth}{16cm}
\setlength{\headsep}{0.5cm}
\setlength{\textheight}{25cm}
\setlength{\voffset}{-0.4in}
\setlength{\headwidth}{\textwidth}
\setlength{\headheight}{13.1pt}

\renewcommand{\baselinestretch}{1.1}  % line spacing}{\usepackage{fancyhdr}
\pagestyle{fancy}

%\renewcommand{\chaptermark}[1]{\markboth{#1}{}}
\renewcommand{\sectionmark}[1]{\markright{\thesection\ - #1}}

\fancyhf{}

\fancyhead[LO]{\rightmark} % \thesection\ 
\fancyfoot[LO]{\small{PLP - Prácticas}}
\fancyfoot[RO]{\thepage}
\renewcommand{\headrulewidth}{0.5pt}
\renewcommand{\footrulewidth}{0.5pt}
\setlength{\hoffset}{-0.25in}
\setlength{\textwidth}{25cm}
%\setlength{\hoffset}{-1.1cm}
%\setlength{\textwidth}{16cm}
\setlength{\headsep}{0.5cm}
\setlength{\textheight}{16cm}
\setlength{\voffset}{-0.4in}
\setlength{\headwidth}{\textwidth}
\setlength{\headheight}{13.1pt}

\renewcommand{\baselinestretch}{1.1}  % line spacing
}





\newenvironment{centrado}
    {
     \begin{center}
     \begin{minipage}{0.8\textwidth}
 }    
    {
     \end{minipage}
     \end{center}
    }

\newcommand{\rel}{\ensuremath{\mathcal{R}}}

\newcommand{\equalDef}{\overset{def}{=}}
\newcommand{\equalDot}{\overset{\cdot}{=}}

\newcommand{\lambdaAbs}[3]{\lambda #1: #2 . #3}
\newcommand{\lambdaAssign}[2]{#1~:=~#2}
\newcommand{\lambdaApp}[2]{#1~#2}
\newcommand{\lambdaIf}[3]{if~ #1~ then~ #2~ else~ #3}
\newcommand{\lambdaTrue}{true}
\newcommand{\lambdaFalse}{false}
\newcommand{\lambdaLet}[4]{let~#1:#2 = #3~in~#4}
\newcommand{\lambdaRef}[1]{ref~#1}
\newcommand{\lambdaVar}[1]{#1}
\newcommand{\lambdaValue}[1]{\color{red}#1\color{black}}
\newcommand{\lambdaFix}[1]{fix~#1}

\newcommand{\lambdaAbsI}[2]{\lambda #1. #2}



\newcommand{\blue}[1]{\color{blue}#1\color{black}}
\newcommand{\replaceBy}[3]{#1\{#2\leftarrow#3\}}

\newcommand{\judgeType}[3]{#1\triangleright #2 : #3}


\newenvironment{scprooftree}[1]%
{\gdef\scalefactor{#1}\begin{center}\proofSkipAmount \leavevmode}%
    {\scalebox{\scalefactor}{\DisplayProof}\proofSkipAmount \end{center} }


\tikzset{
    every leaf node/.style={text=red, align=center},
    every tree node/.style={text=blue, align=center},
}

\forestset{tikzQtree/.style={for tree={if n children=0{
                node options=every leaf node/.try}{node options=every tree node/.try}, text centered}}}
                
                
\DeclareMathOperator{\Erase}{Erase}
\DeclareMathOperator{\Nat}{Nat}
\DeclareMathOperator{\Bool}{Bool}
\DeclareMathOperator{\Union}{Union}

\newcommand{\WFunc}{\mathbb{W}}

\newcommand{\red}[1]{{\color{red}#1}}%\renewcommand{\appendixtocname}{Apéndices}
\newcommand{\green}[1]{{\color{green!40!black}#1}}%\renewcommand{\appendixpagename}{Apéndices}





\setcounter{section}{6}


\begin{document}
  \title{PLP - Práctica 6: Resolución en Lógica}

  \date{\today}

  \author{Zamboni, Gianfranco}

  \maketitle
  \setcounter{page}{1}

  \section*{\ Resolución en Lógica Proposicional}
  \subsection{Ejercicio 1}
    \subsubsection{FNC:}
    \begin{enumerate}
    \item $-p \vee p$
    \item $-p \vee -q \vee p$
    \item $(-p \vee p) \wedge (-q \vee p)$
    \item $(-p \vee p) \wedge (p \vee -p)$
    \item $(q \vee -p \vee -q) \wedge (p \vee -p \vee -q)$
    \item $(p \vee p) \wedge (p \vee r) \wedge (q \vee p) \wedge (q \vee r)$
    \item $(r \vee -p) \wedge (-q \vee r)$
    \item $-p \vee -q \vee r$
    \end{enumerate}
    \subsubsection{\blue{FNClausal:}}
  \subsection{Ejercicio 2}
  \subsubsection{}
    \begin{enumerate}
        \item $(-p \vee p)$ es una tautología $  \Leftrightarrow \{\{p\},\{-p\}\} $ es insatisfacible \\
        $S=\{\{p\},\{-p\}\}$ \\
        $\{p\}-(p) \cup \{-p\}-(-p)=\{\}$ \\
        $S=\{\{p\},\{-p\},\{\}\} \Rightarrow $ es insatisfacible
        \item $((-p \vee -q) \vee p) $ es una tautología$ \Leftrightarrow \{\{p\},\{q\},\{-p\}\} $ es insatisfacible \\
        $S=\{\{p\},\{q\},\{-p\}\}$ \\
        $\{p\}-(p) \cup \{-p\}-(-p)=\{\}$ \\
        $S=\{\{p\},\{q\},\{-p\},\{\}\} \Rightarrow $ es insatisfacible
        \item $((-p \vee p)\wedge(-q \vee p))$ es una tautología $ \Leftrightarrow \{\{p, q\},\{p, -p\},\{-p, q\},\{-p, -p\}\} $ es insatisfacible \\
        $S=\{\{p, q\},\{p, -p\},\{-p, q\},\{-p, -p\}\}$ \\
        $\{p, q\}-(p) \cup \{-p, q\}-(-p)=\{q\}$ \\
        $S=\{\{p, q\},\{p, -p\},\{-p, q\},\{-p, -p\},\{q\}\}$ \\
        No podemos aplicar ningún paso de resolución a S es decir generar una cláusula nueva, por lo tanto, no puede llegarse a una refutación a partir S. Entonces, S debe ser satisfacible. Sea v(p)=F,v(q)=V.
        \item $((-p \vee p)\wedge(p \vee -p))$ es una tautología $ \Leftrightarrow \{\{p, -p\},\{p, p\},\{-p, -p\},\{-p, p\}\} $ es insatisfacible \\
        $S=\{\{p, -p\},\{p, p\},\{-p, -p\},\{-p, p\}\}$ \\
        $\{p, -p\}-(p,-p) \cup \{-p, p\}-(p,-p)=\{\}$ \\
        $S=\{\{p, -p\},\{p, p\},\{-p, -p\},\{-p, p\},\{\}\} \Rightarrow $ es insatisfacible
        \item $((q \vee -p \vee -q)\wedge(p \vee -p \vee -q))$ es una tautología $ \Leftrightarrow \{\{-q, -p\},\{-q, p\},\{-q, q\},\{p, -p\},\{p, p\},\{p, q\},\{q, -p\},\{q, p\},\{q, q\}\} es$ insatisfacible
        $S=\{\{-q, -p\},\{-q, p\},\{-q, q\},\{p, -p\},\{p, p\},\{p, q\},\{q, -p\},\{q, p\},\{q, q\}\}$ \\
        $\{-q, -p\}-(-q, -p) \cup \{p, q\}-(p, q)=\{\}$ \\
        $S=\{\{-q, -p\},\{-q, p\},\{-q, q\},\{p, -p\},\{p, p\},\{p, q\},\{q, -p\},\{q, p\},\{q, q\},\{\}\} \Rightarrow $ es insatisfacible
        \item $((p \vee p)\wedge(p \vee r)\wedge(q \vee p)\wedge(q \vee r))$ es una tautología $ \Leftrightarrow \{\{-p, -p, -q, -q\},\{-p, -p, -q, -r\},\{-p, -p, -p, -q\},\{-p, -p, -p, -r\},\{-p, -p, -q, -q\},\{-p, -p, -q, -r\},\{-p, -p, -p, -q\},\{-p, -p, -p, -r\},\{-p, -r, -q, -q\},\{-p, -r, -q, -r\},\{-p, -r, -p, -q\},\{-p, -r, -p, -r\},\{-p, -r, -q, -q\},\{-p, -r, -q, -r\},\{-p, -r, -p, -q\},\{-p, -r, -p, -r\}\} $ es insatisfacible \\
        $S=\{\{-p, -p, -q, -q\},\{-p, -p, -q, -r\},\{-p, -p, -p, -q\},\{-p, -p, -p, -r\},\{-p, -p, -q, -q\},\{-p, -p, -q, -r\},\{-p, -p, -p, -q\},\{-p, -p, -p, -r\},\{-p, -r, -q, -q\},\{-p, -r, -q, -r\},\{-p, -r, -p, -q\},\{-p, -r, -p, -r\},\{-p, -r, -q, -q\},\{-p, -r, -q, -r\},\{-p, -r, -p, -q\},\{-p, -r, -p, -r\}\}$ \\
        No podemos aplicar ningún paso de resolución a S, por lo tanto, no puede llegarse a una refutación a partir S. Entonces, S debe ser satisfacible. Sea v(p)=v(q)=v(r)=F.
        \item $((r \vee -p)\wedge(-q \vee r))$ es una tautología $ \Leftrightarrow \{\{-r, q\},\{p, q\},\{-r, -r\},\{p, -r\}\} $ es insatisfacible \\
        $S=\{\{-r, q\},\{p, q\},\{-r, -r\},\{p, -r\}\}$ \\
        No se puede generar el conjunto vacío
        \item $(-p \vee -q \vee r)$ es una tautología $ \Leftrightarrow \{\{p\},\{q\},\{-r\}\} $ es insatisfacible \\
        $S=\{\{p\},\{q\},\{-r\}\}$ \\
        No podemos aplicar ningún paso de resolución a S, por lo tanto, no puede llegarse a una refutación a partir S. Entonces, S debe ser satisfacible. Sea v(p)=v(q)=T,v(r)=F.
    \end{enumerate}
\subsubsection{}
$((-p \Rightarrow q) \wedge (p \Rightarrow q) \wedge (-p \Rightarrow -q)) \Rightarrow (p \wedge q)$ es una tautología $\Leftrightarrow$ \\
$ \{\{-p, p, -p, p\},\{-p, p, q, p\},\{-p, p, -p, q\},\{-p, p, q, q\},\{-q, p, -p, p\},\{-q, p, q, p\},\{-q, p, -p, q\},$ \\
$\{-q, p, q, q\} \{-p, -q, -p, p\},\{-p, -q, q, p\},\{-p, -q, -p, q\},\{-p, -q, q, q\} \{-q, -q, -p, p\},\{-q, -q, q, p\},$ \\
$\{-q, -q, -p, q\},\{-q, -q, q, q\}\}$ es insatisfacible \\

$S=\{\{-p, p, -p, p\},\{-p, p, q, p\},\{-p, p, -p, q\},\{-p, p, q, q\},\{-q, p, -p, p\}, \\ \{-q, p, q, p\},\{-q, p, -p, q\},\{-q, p, q, q\} \{-p, -q, -p, p\},\{-p, -q, q, p\}, \\ \{-p, -q, -p, q\},\{-p, -q, q, q\} \{-q, -q, -p, p\},\{-q, -q, q, p\},\{-q, -q, -p, q\},\{-q, -q, q, q\}\}$ \\ \\
$S1=\{-q, -q, q, p\}\{-q, p, q, p\}=\{-q, p, p, p\}$ \\
$S2=\{-p, -q, -p, q\}\{-q, p, p, p\}=\{-q, p\}$ \\
$S3=\{-q, p, -p, q\}\{-p, p, -p, q\}=\{q, -p\}$ \\
$S4=\{-q, p\}\{q, -p\}=\{\} $ \\ \\
$\Rightarrow$ \\
$S\cup S1 \cup S2 \cup S3 \cup S4 = \{\{-p, p, -p, p\},\{-p, p, q, p\},\{-p, p, -p, q\},\{-p, p, q, q\},\{-q, p, -p, p\}, \\ \{-q, p, q, p\},\{-q, p, -p, q\},\{-q, p, q, q\} \{-p, -q, -p, p\},\{-p, -q, q, p\}, \\ \{-p, -q, -p, q\},\{-p, -q, q, q\},\{-q, -q, -p, p\},\{-q, -q, q, p\},\{-q, -q, -p, q\}, \\ \{-q, -q, q, q\}\},\{-q, p, p, p\},\{-q, p\},\{q, -p\},\{\}$ es insatisfacible

  \subsection{Ejercicio 3}
  \section*{\ Unificación en Lógica de Primer Orden}
  \subsection{Ejercicio 4}
    \begin{enumerate}
    \item $P(f(x)) $ unifica con: \\
    $P(f(a)) $ si $ x \doteq a$
    \item $P(a) $ unifica con: \\
    $P(x) $ si $ x \doteq a$ \\
    $P(g(z)) $ si $ g(z) \doteq a$
    \item $P(y) $ unifica con: \\
    $P(x) $ si $ y \doteq x$ \\
    $P(f(a)) $ si $ y \doteq f(a)$ \\
    $P(g(z)) $ si $ y \doteq g(z)$
    \item $Q(x,f(y)) $ unifica con: \\
    $Q(f(y),x) $ si $ x \doteq f(y)$
    \item $Q(x,f(z)) $ unifica con: \\
    $Q(f(y),x) $ si $ x \doteq f(y) $ y $ z \doteq y$ \\
    $Q(f(y),f(x)) $ si $ x \doteq f(y) $ y $ z \doteq f(y)$ \\
    $Q(f(y),y) $ si $ y \doteq f(z) y x \doteq f(f(z))$
    \item $Q(x,f(a)) $ unifica con: \\
    $Q(f(y),x) $ si $ x \doteq f(a) $ y $ y \doteq a$ \\
    $Q(f(y),f(x)) $ si $ x \doteq a $ y $ a \doteq f(y)$ \\
    $Q(f(y),y) $ si $ y \doteq f(a) $ y $ x \doteq f(f(a))$
    \end{enumerate}
  \subsection{Ejercicio 5}
    \begin{enumerate}
    \item $f(x,x,y) \doteq f(a,b,z)$ $\sigma:{x/a}$ \\
    $f(a,a,y) \doteq f(a,b,z)$ \\
    como $a ! \doteq b \Rightarrow$ Falla 
    \item $f(x) \doteq y$ si $\sigma:{y/f(x)}$
    \item $f(g(c,y),x) \doteq f(z,g(z,a))$ $\sigma:{z/g(c,y)}$
    $f(g(c,y),x) \doteq f(g(c,y),g(g(c,y),a))$ $\sigma_{2}:{x/g(g(c,y),a)}$
    \item $f(a) \doteq g(y)$ si $g(y)=f(a)$
    \item $f(x) \doteq x$ Falla porque $x \in FV(f(x))$
    \item $g(x,y) \doteq g(f(y),f(x))$ $\sigma:{x/f(y)}$
     $g(f(y),y) \doteq g(f(y),f(f(y)))$ Falla porque $y \in FV(f(f(y)))$
    \end{enumerate}
\section*{\ Resolución en Lógica de Primer Orden}
  \subsection{Ejercicio 6}
    \begin{enumerate}
    \item $\forall x.  \forall y. ( \neg Q(x,y) \Rightarrow \neg P(x,y)$ \\
    $\forall x.  \forall y. ( \neg \neg Q(x,y) \vee \neg P(x,y)$ \\
    $\forall x.  \forall y. ( Q(x,y) \vee \neg P(x,y)$
    \item $\forall x.  \forall y. ((P(x,y) \wedge Q(x,y)) \Rightarrow R(x,y))$ \\
    $\forall x.  \forall y. (\neg (P(x,y) \wedge Q(x,y)) \vee R(x,y))$ \\
    $\forall x.  \forall y. ((\neg P(x,y) \vee \neg Q(x,y)) \vee R(x,y))$
    \item $\forall x.  \exists y. (P(x,y) \Rightarrow Q(x,y))$ \\
    $\forall x.  \exists y. (\neg P(x,y) \vee Q(x,y))$
    \end{enumerate}
  \subsection{Ejercicio 7}
\begin{enumerate}
\item 
$\exists x. \exists y. (x<y)$ \\
$\exists x. (x<f(x))$ \\
$(f_{2}(c)<f(x))$ \\
Forma clausal: $\{\{f_{2}(c)<f(x)\}\}$
\item 
$\forall x. \exists y. (x<y)$ \\
$\forall x. (x<f(x))$ \\
Forma clausal: $\{\{x<f(x)\}\}$
\item 
$\forall x. \neg(P(x) \wedge \forall y.(\neg P(y) \vee Q(y)))$ \\
$\forall x. \forall y. \neg(P(x) \wedge (\neg P(y) \vee Q(y)))$ \\
$\forall x. \forall y. (\neg P(x) \vee \neg(\neg P(y) \vee Q(y)))$ \\
$\forall x. \forall y. (\neg P(x) \vee (\neg \neg P(y) \wedge \neg Q(y)))$ \\
$\forall x. \forall y. (\neg P(x) \vee (P(y) \wedge \neg Q(y)))$ \\
$\forall x. \forall y. ( (\neg P(x) \vee P(y)) \wedge (\neg P(x) \vee \neg Q(y) ) )$ \\
Forma clausal: $\{\{\neg P(x). P(y)\},\{\neg P(x), \neg Q(y)\}\}$
\item 
$\exists x. \forall y. (P(x,y) \wedge Q(x) \wedge \neg R(y))$ \\
$\forall y. (P(f(c),y) \wedge Q(f(c)) \wedge \neg R(y))$ \\
Forma clausal: $\{\{P(f(c),y), Q(f(c)), \neg R(y)\}\}$
\item 
$\forall x. (P(x) \wedge \exists y. (Q(y) \vee \forall z. \exists w. (P(z) \wedge \neg Q(w)))$ \\
$\forall x. \forall z. (P(x) \wedge (Q(f(c)) \vee (P(z) \wedge \neg Q(f(z))))$ \\
$\forall x. \forall z. (P(x) \wedge (Q(f(c)) \vee P(z) ) \wedge (Q(f(c)) \vee \neg Q(f(z)) ) )$ \\
Forma clausal: $\{\{ P(x)\} , \{Q(f(c)), P(z)\} , \{Q(f(c)), \neg Q(f(z))\} \}$
\end{enumerate}
  \subsection{Ejercicio 8}
  \subsection{Ejercicio 9}
\begin{enumerate}
\item
$\exists x. \forall y. R(x,y) \Rightarrow \forall y. \exists x. R(x,y)$ \\
$\neg \exists x. \forall y. R(x,y) \vee \forall y. \exists x. R(x,y)$ \\
Negado: \\
$\neg (\neg \exists x. \forall y. R(x,y) \vee \forall y. \exists x. R(x,y))$ \\
$ \neg \neg \exists x. \forall y. R(x,y) \wedge \neg \forall y. \exists x. R(x,y) $ \\
$ \exists x. \forall y. R(x,y) \wedge \neg \forall y. \exists x. R(x,y) $ \\
$ \forall y. R(f(y),y) \wedge \exists y. \neg \exists x. R(x,y) $ \\
$ \forall y. R(f(y),y) \wedge \exists y. \forall x. \neg R(x,y) $ \\
$ \forall y. R(f(y),y) \wedge \forall x. \neg R(x,f(x)) $ \\
$ \forall y. \forall x.  R(f(y),y) \wedge \neg R(x,f(x)) $ \\
Forma clausal: $S=\{\{ R(f(y),y)\},\{ \neg R(x,f(x))\} \}$ \\
Resolución: No podemos aplicar ningún paso de resolución a S, por lo tanto, no puede llegarse a una refutación a partir S. Entonces, S debe ser satisfacible. Sea $v(R(f(y),y))=T$, $v(\neg R(x,f(x)))=F$.
\item
$\forall x. \exists y. R(x,y) \Rightarrow \exists y. \forall x. R(x,y)$ \\
$\neg \forall x. \exists y. R(x,y) \vee \exists y. \forall x. R(x,y)$ \\
Negado: \\
$\neg (\neg \forall x. \exists y. R(x,y) \vee \exists y. \forall x. R(x,y))$ \\
$\neg \neg \forall x. \exists y. R(x,y) \wedge \neg \exists y. \forall x. R(x,y)$ \\
$\forall x. \exists y. R(x,y) \wedge \forall y. \neg \forall x. R(x,y)$ \\
$\forall x. \exists y. R(x,y) \wedge \forall y. \exists x. \neg R(x,y)$ \\
$\forall x. R(x,f(x)) \wedge \forall y. \neg R(f(y),y)$ \\
Forma clausal: $S=\{ \{ R(x,f(x))\},\{ \neg R(f(y),y) \}\}$ \\
Resolución: No podemos aplicar ningún paso de resolución a S, por lo tanto, no puede llegarse a una refutación a partir S. Entonces, S debe ser satisfacible. Sea $v(R(x,f(x)))=T$, $v(\neg R(f(y),y))=F$.
\item
$\exists x. [P(x) \Rightarrow \forall x. P(x)]$ \\
$\exists x'. [\neg P(x') \vee \forall x. P(x)]$ \\
Negado: \\
$\neg (\exists x'. [\neg P(x') \vee \forall x. P(x)])$ \\
$ \forall x'. \neg [\neg P(x') \vee \forall x. P(x)]$ \\
$ \forall x'. [ \neg \neg P(x') \wedge \neg \forall x. P(x)]$ \\
$ \forall x'. [ P(x') \wedge \exists x. \neg P(x)]$ \\
$ \forall x'. [ P(x') \wedge \neg P(f(c))]$ \\
Forma clausal: $S=\{ \{ P(x') \} , \{ \neg P(f(c)) \} \}$ \\
Resolución: \\
$S=\{ \{P(x')\}, \{\neg P(f(c))\}\}$\\
Sea $\sigma(x' / f(c)) \Rightarrow \sigma(S)=\{ P(f(c)), \neg P(f(c))\} \Rightarrow \{\}$ \\
$S=\{ \{P(x')\}, \{\neg P(f(c))\}, \{\} \} \Rightarrow $ S es insatisfacible
\item
$ \exists x. [P(x) \vee Q(x)] \Rightarrow [\exists P(x) \vee \exists Q(x)]$ \\
$ \exists x. \neg [P(x) \vee Q(x)] \vee [\exists x. P(x) \vee \exists x. Q(x)]$ \\
Negado: \\
$ \neg (\exists x. \neg [P(x) \vee Q(x)] \vee [\exists x. P(x) \vee \exists x. Q(x)])$ \\
$ \forall x. (\neg \neg [P(x) \vee Q(x)]) \vee \neg [\exists x. P(x) \vee \exists x. Q(x)]$ \\
$ \forall x. [P(x) \vee Q(x)] \vee [\neg \exists x. P(x) \wedge \neg \exists x. Q(x)]$ \\
$ \forall x. [P(x) \vee Q(x)] \vee [\forall x. \neg P(x) \wedge \forall x. \neg Q(x)]$ \\
$ \forall x. \forall y. \forall z. [P(x) \vee Q(x)] \vee [\neg P(z) \wedge \neg Q(y)]$ \\
$ \forall x. \forall y. \forall z. [(P(x) \vee Q(x) \vee \neg P(z)) \wedge (P(x) \vee Q(x) \vee \neg Q(y))]$ \\
Forma clausal: $S=\{\{P(x), Q(x), \neg P(z)\}, \{P(x), Q(x), \neg Q(y)\}\}$ \\
Resolución: $\blue{???}$
\item
$ \forall x. [P(x) \vee Q(x)] \Rightarrow [\forall P(x) \vee \forall Q(x)]$ \\
$ \forall x. \neg [P(x) \vee Q(x)] \vee [\forall P(x) \vee \forall Q(x)]$ \\
Negado: \\
$ \neg \forall x. \neg [P(x) \vee Q(x)] \vee [\forall x. P(x) \vee \forall x. Q(x)]$ \\
$ \exists x. \neg \neg [P(x) \vee Q(x)] \wedge \neg [\forall x. P(x) \vee \forall x. Q(x)]$ \\
$ \exists x. [P(x) \vee Q(x)] \wedge [\neg \forall x. P(x) \wedge \neg \forall x. Q(x)]$ \\
$ \exists x. [P(x) \vee Q(x)] \wedge [\exists x. \neg P(x) \wedge \exists x. \neg Q(x)]$ \\
$ [P(f(c)) \vee Q(f(c))] \wedge [\neg P(f(b)) \wedge \neg Q(f(a))]$ \\
$ [P(f(c)) \vee Q(f(c))] \wedge [\neg P(f(b)) \wedge \neg Q(f(a))]$ \\
Forma clausal: $\{ \{P(f(c)), Q(f(c))\}, \{\neg P(f(b)), \neg Q(f(a))\} \}$ \\
Resolución:\\
$S=\{P(f(c)) \vee Q(f(c))\},\{\neg P(f(b))\}$ $\sigma= c \doteq b$ resolvente = $\{Q(f(b))\}$ \\
$S=\{ P(f(c)) \vee Q(f(c)), \neg P(f(b)), \neg Q(f(a)), Q(f(b)) \}$ \\
$S=\{ \neg Q(f(a))\},\{Q(f(b)) \}$ $ \sigma= a \doteq b$ resolvente = $ \{\} $
$S=\{ P(f(c)) \vee Q(f(c)), \neg P(f(b)), \neg Q(f(a)), Q(f(b)), \{\} \}$ $\Rightarrow$ S es insatisfacible.
\item
$ [\exists x. P(x) \wedge \forall x. Q(x)] \Rightarrow \exists x. [P(x) \wedge Q(x)] $ \\
$ \neg [\exists x. P(x) \wedge \forall x. Q(x)] \vee \exists x. [P(x) \wedge Q(x)] $ \\
$ [\neg \exists x. P(x) \wedge \neg \forall x. Q(x)] \vee \exists x. [P(x) \wedge Q(x)] $ \\
$ [\forall x. \neg P(x) \wedge \exists x. \neg Q(x)] \vee [P(f(a)) \wedge Q(f(a))] $ \\
$ \forall x. [\neg P(x) \wedge \neg Q(f(b))] \vee [P(f(a)) \wedge Q(f(a))] $ \\
$ \forall x. (\neg P(x) \vee P(f(a))) \wedge (\neg Q(f(b)) \vee P(f(a))) \wedge (\neg P(x)\vee Q(f(a))) \wedge (\neg Q(f(b)) \vee Q(f(a))) $ \\
Forma clausal: $\{ \{\neg P(x), P(f(a))\}, \{\neg Q(f(b)), P(f(a))\}, \{\neg P(x), Q(f(a))\}, \{\neg Q(f(b)), Q(f(a))\}\}$
Resolución: \\
$S=\{(\neg Q(f(b)) \vee P(f(a))), (\neg P(x)\vee Q(f(a)))\}$ $\sigma=(f(a)) \doteq x \doteq f(b)))$ resolvente = $ \{\} $ \\
$S=\{ (\neg P(x) \vee P(f(a))), (\neg Q(f(b)) \vee P(f(a))), (\neg P(x)\vee Q(f(a))), (\neg Q(f(b)) \vee Q(f(a))), \{\}\}$ $\Rightarrow$ S es insatisfacible.
\item
$\forall x. \exists y. \forall z. \exists w. [P(x,y) \vee \neg P(w,z)]$ \\
Negado: \\
$\neg \forall x. \exists y. \forall z. \exists w. [P(x,y) \vee \neg P(w,z)]$ \\
$\exists x. \neg \exists y. \forall z. \exists w. [P(x,y) \vee \neg P(w,z)]$ \\
$\exists x. \forall y. \neg \forall z. \exists w. [P(x,y) \vee \neg P(w,z)]$ \\
$\exists x. \forall y. \exists z. \neg \exists w. [P(x,y) \vee \neg P(w,z)]$ \\
$\exists x. \forall y. \exists z. \forall w. \neg [P(x,y) \vee \neg P(w,z)]$ \\
$\exists x. \forall y. \exists z. \forall w. [\neg P(x,y) \wedge \neg \neg P(w,z)]$ \\
$\forall y. \forall w. [\neg P(f(b),y) \wedge P(w,f(a))]$ \\
Forma clausal: $\{\{ \neg P(f(b),y) \}, \{P(w,f(a)) \} \}$
Resolución: \\
$S=\{ \neg P(f(b),y), P(w,f(a)) \}$ $\sigma=(w \doteq f(b), y \doteq f(a))$ resolvente = $ \{\} $ \\
$S=\{ \neg P(f(b),y), P(w,f(a)), \{\} \}$ $\Rightarrow$ S es insatisfacible.
\item
$\forall x. \forall y. \forall z. [\neg P(f(a)) \vee \neg P(y) \vee Q(y)] \wedge P(f(z)) \wedge [\neg P(f(f(x))) \vee \neg Q(f(x))]$ \\
Negado: \\
$\neg (\forall x. \forall y. \forall z. [\neg P(f(a)) \vee \neg P(y) \vee Q(y)] \wedge P(f(z)) \wedge [\neg P(f(f(x))) \vee \neg Q(f(x))])$ \\
$\exists x. \neg \forall y. \forall z. [\neg P(f(a)) \vee \neg P(y) \vee Q(y)] \wedge P(f(z)) \wedge [\neg P(f(f(x))) \vee \neg Q(f(x))]$ \\
$\exists x. \exists y. \neg \forall z. [\neg P(f(a)) \vee \neg P(y) \vee Q(y)] \wedge P(f(z)) \wedge [\neg P(f(f(x))) \vee \neg Q(f(x))]$ \\
$\exists x. \exists y. \exists z. \neg [\neg P(f(a)) \vee \neg P(y) \vee Q(y)] \wedge P(f(z)) \wedge [\neg P(f(f(x))) \vee \neg Q(f(x))]$ \\
$\exists x. \exists y. \exists z. [ \neg (\neg P(f(a)) \vee \neg P(y) \vee Q(y))] \vee \neg P(f(z)) \vee [\neg (\neg P(f(f(x))) \vee \neg Q(f(x)))]$ \\
$\exists x. \exists y. \exists z. [ (\neg \neg P(f(a)) \wedge \neg \neg P(y) \wedge \neg Q(y))] \vee \neg P(f(z)) \vee [\neg \neg P(f(f(x))) \wedge \neg \neg Q(f(x))]$ \\
$\exists x. \exists y. \exists z. [ (P(f(a)) \wedge P(y) \wedge \neg Q(y))] \vee \neg P(f(z)) \vee [P(f(f(x))) \wedge Q(f(x))]$ \\
$\exists x. \exists y. \exists z. [ ( (P(f(a)) \vee \neg P(f(z)) ) \wedge ( P(y) \vee \neg P(f(z)) ) \wedge ( \neg Q(y)) \vee \neg P(f(z)) )]  \vee [P(f(f(x))) \wedge Q(f(x))]$ \\
$\exists x. \exists y. \exists z. \\
P(f(a)) \vee \neg P(f(z)) \vee P(f(f(x)) \wedge \\
P(f(a)) \vee \neg P(f(z)) \vee Q(f(x))  \wedge \\
P(y) \vee \neg P(f(z)) \vee P(f(f(x)) \wedge \\
P(y) \vee \neg P(f(z)) \vee Q(f(x)) \wedge \\
\neg Q(y)) \vee \neg P(f(z)) \vee P(f(f(x)) \wedge \\
\neg Q(y)) \vee \neg P(f(z)) \vee Q(f(x)) $ \\
Forma clausal: $\{ \\
\{P(f(a)), \neg P(f(z)), P(f(f(x))\}, \\
\{P(f(a)), \neg P(f(z)), Q(f(x)) \}, \\
\{P(y), \neg P(f(z)), P(f(f(x))\}, \\
\{P(y), \neg P(f(z)), Q(f(x))\}, \\
\{\neg Q(y)), \neg P(f(z)), P(f(f(x))\}, \\
\{\neg Q(y)), \neg P(f(z)), Q(f(x))\} \\
\}$ \\
Resolución: \\
$\{P(y) \vee \neg P(f(z)) \vee Q(f(x)) \},\{\neg Q(y')) \vee \neg P(f(z')) \vee P(f(f(x'))\}$ \\
$\sigma=(y \doteq f(z'), f(z) \doteq f(f(x)), f(x) \doteq y')$ \\
Resolvente: $\{\}$ \\
$S=\{ P(f(a)) \vee \neg P(f(z)) \vee P(f(f(x)), \\
P(f(a)) \vee \neg P(f(z)) \vee Q(f(x)) , \\
P(y) \vee \neg P(f(z)) \vee P(f(f(x)), \\
P(y) \vee \neg P(f(z)) \vee Q(f(x)), \\
\neg Q(y)) \vee \neg P(f(z)) \vee P(f(f(x)), \\
\neg Q(y)) \vee \neg P(f(z)) \vee Q(f(x)), \\
\{\} $ $\}$ $\Rightarrow$ S es insatisfacible.

\end{enumerate}



  \subsection{Ejercicio 10}
\subsubsection{Modus Ponens: $((P \rightarrow Q) \wedge P) \rightarrow Q$}
$((P \rightarrow Q) \wedge P) \rightarrow Q$ \\
$\neg ((P \rightarrow Q) \wedge P) \vee Q$ \\
$\neg ((\neg P \vee Q) \wedge P) \vee Q$ \\
$(\neg (\neg P \vee Q) \vee \neg P) \vee Q$ \\
$((P \vee Q) \wedge (\neg Q \vee Q )) \vee (\neg P \vee Q)$ \\
$((P \vee Q \vee \neg P \vee Q) \wedge (\neg Q \vee Q \vee \neg P \vee Q))$ \\
Forma clausal: \\
$\{\{P, Q, \neg P, Q\}\{\neg Q, Q, \neg P, Q\}\}$ \\

\subsubsection{Modus Tollens: $((P \rightarrow Q) \wedge \neg Q) \rightarrow \neg P$}
$((P \rightarrow Q) \wedge \neg Q) \rightarrow \neg P$ \\
$(\neg (\neg P \vee Q) \wedge \neg Q) \vee \neg P$ \\
$((\neg \neg P \wedge \neg Q) \vee \neg \neg Q) \vee \neg P$ \\
Forma clausal: \\
$\{\{P, Q, \neg P\}\{\neg Q , Q, \neg P\}\} $ \\

  \subsection{Ejercicio 11}
  \subsection{Ejercicio 12}
Condiciones son necesarias para que una demostración por resolución sea SLD:
\begin{itemieze}
\item Realizarse de manera lineal (utilizando en cada paso el resolvente obtenido en el paso anterior).
\item Utilizar únicamente cláusulas de Horn.
\item Empezar por una cláusula objetivo (sin literales positivos).
\item Empezar por una cláusula que provenga de la negación de lo que se quiere demostrar.
\item Recorrer el espacio de búsqueda de arriba hacia abajo y de izquierda a derecha.
\item Utilizar la regla de resolución binaria en lugar de la general.
\end{itemieze}

Innecesaria:
\begin{itemieze}
\item Utilizar cada cláusula a lo sumo una vez.
\end{itemieze}

  \subsection{Ejercicio 13}
  \subsection{Ejercicio 14}
	\begin{itemize}
	    \item Cláusulas: \\
		$1.{\neg suma(x1,y1,z1), suma(x1,suc(y1),suc(z1))}$ \\
		$2.{suma(x2,cero,x2)}$ \\
		$3.{\neg suma(x3,y3,z3), par(y3)}$
	    \item Quiero probar: \\
		${par(suc(suc(cero)))}$
	    \item Entonces, la niego: \\
		$4.{\neg par(suc(suc(cero)))}$
	    \item Demostración: \\
		$5.4y3. \sigma={y3 \leftarrow suc(suc(cero))}$ y la resolvente es: $ {\neg suma(x3,suc(suc(cero)),z3)}$ \\
		$6.5y1. \sigma={x1 \leftarrow x3,suc(y1) \leftarrow suc(suc(cero)),z3 \leftarrow suc(z1))}$ y la resolvente es: ${\neg suma(x3,suc(cero),z1)}$ \\
		$7.6y1. \sigma={x1' \leftarrow x3, suc(y1') \leftarrow suc(cero), suc(z1') \leftarrow z1}$ y la resolvente es: $ {\neg suma(x3,cero, z1)}$ \\
		$8.7y2. \sigma={x2 \leftarrow z1, x3 \leftarrow x2} $ y la resolvente es: $ \{\}$
	    \item Aplique resolución SLD, porque eran Cláusulas de horn, utilicé la resolucion binaria y en cada paso utilicé la última resolvente creada.
	\end{itemize}
  \subsection{Ejercicio 15}
  \subsection{Ejercicio 16}
	\begin{itemize}
	    \item Para Identificar errores lo pruebo: $(\exists x. enBAr(x)) \Rightarrow \exists y. ( enBAr(y) \ wedge ( bebe(y) \Rightarrow \forall z. ( enBAr(z) \Rightarrow bebe(z) ) ) )$
	    \item Eliminar implicaciones: $\neg (\exists x. enBAr(x)) \vee \exists y. ( enBAr(y) \ wedge (\neg bebe(y) \vee \forall z. ( \neg enBAr(z) \vee bebe(z) ) ) )$
	    \item Forma normal negada: $(\forall x. \neg enBAr(x)) \vee \exists y. ( enBAr(y) \ wedge (\neg bebe(y) \vee \forall z. ( \neg enBAr(z) \vee bebe(z) ) ) )$
	    \item Forma normal de skolem: $\forall z. \forall x. (\neg enBAr(x)) \vee ( enBAr(f(c)) \ wedge (\neg bebe(f(c)) \vee ( \neg enBAr(z) \vee bebe(z) ) ) )$
	    \item Forma normal clausal: $\forall z. \forall x. ((\neg enBAr(x)) \vee enBAr(f(c))) \ wedge (\neg enBAr(x)) \vee \neg bebe(f(c)) \vee \neg enBAr(z) \vee bebe(z) ) )$ \\
	    $\{\{\neg enBAr(x)), enBAr(f(c))\},\{\neg enBAr(x), \neg bebe(f(c)), \neg enBAr(z), bebe(z)\}\}$
	\end{itemize}
	\blue{1} la negacion del existencial \\
	\blue{2} la separación en terminos de la forma clausal \\
	\blue{3} cuando genera una resolvente a partir de dos cláusulas que son negaciones ($\neg enBar()$) \\
	\blue{4} cuando define a una sustitución de una constante por una variable($\{k \rightarrow Z\}$) \\
	\blue{5} se olvido de negar lo que queria probar.
  \subsection{Ejercicio 17}
\begin{itemize}
    \item $\forall x. \exists y. esContacto(x,y)$ = $\{esContacto(x,y)\}$
    \item $\forall x. \forall y. esContacto(x,y) \Rightarrow esContacto(y,x)$ = $\neg esContacto(x,y) \vee esContacto(y,x)$  = $\{\neg esContacto(x,y), esContacto(y,x)\}$ 
    \item Queremos ver: $\forall x. esContacto(x,x)$ 
    \item La negamos: $\neg \forall x. esContacto(x,x)$ = $\exists x. \neg esContacto(x,x)$ = $\neg esContacto(f(c),f(c))$
    \item Rta: Es correcta. \\ 
    \item Queremos ver: $\forall y. \exists x. esContacto(x,y)$ 
    \item La negamos: $\neg \forall y. \exists x. esContacto(x,y)$ = $\exists y. \neg \exists x. esContacto(x,y)$ = $\exists y. \forall x. \neg esContacto(x,y)$ = $\forall x. \neg esContacto(x,f(c))$  
    \item El punto 4) está mal. 
    \item Queda: del 2 y 3, $esContacto(d,d)$ y ahi se traba porque va a usar simepre la $resolvente_{i}$ con 2.
\end{itemize}
  \subsection{Ejercicio 18}
\begin{itemize}
\item 1$.\{\neg Progenitor(x1,y1),Descenciente(y1,x1)\}$ 
\item 2$.\{\neg Descenciente(x2,y2), \neg Descenciente(y2,z2), Descenciente(x2,z2)\}$ 
\item 3$.\{\neg Abuelo(x3,y3), Progenitor(x3, medio(x3,y3))\}$ 
\item 4$.\{\neg Abuelo(x4,y4), Progenitor(medio(x4,y4),y4)\}$ 
\item Queremos ver que: $\forall x. \forall y. (Abuelo(x,y) \Rightarrow Descenciente(y,x))$ 
\item $\forall x. \forall y. (Abuelo(x,y) \Rightarrow Descenciente(y,x))$ 
\item $\forall x. \forall y. (\neg Abuelo(x,y) \vee Descenciente(y,x))$ 
\item Lo niego: $\neg \forall x. \forall y. (\neg Abuelo(x,y) \vee Descenciente(y,x))$ 
\item $\exists x. \neg \forall y. (\neg Abuelo(x,y) \vee Descenciente(y,x))$ 
\item $\exists x. \exists y. \neg (\neg Abuelo(x,y) \vee Descenciente(y,x))$ 
\item $\exists x. \exists y. (\neg \neg Abuelo(x,y) \wedge \neg Descenciente(y,x))$ 
\item $\exists x. \exists y. (Abuelo(x,y) \wedge \neg Descenciente(y,x))$ 
\item 5$.\{Abuelo(x5,y5)\} $  
\item 6$.\{\neg Descenciente(y6,x6)\} $ \\
\item Resolución: 
\item 7$. 5y3. \{x3 \leftarrow x6, y3 \leftarrow y6\}$ y la resolvente es: $\{Progenitor(x3, medio(x3,y3))\}$ 
\item 8$. 7y1. \{x1 \leftarrow x3, y1 \leftarrow medio(x3,y3)\}$ y la resolvente es: $\{Descenciente( medio(x3,y3),x3)\}\}$ 
\item 9.$ 8y6. \{y6 \leftarrow medio(x3,y3), x6 \leftarrow x3\}$ y la resolvente es: $\{\}$, por lo tanto probamos que $\forall x. \forall y. (\neg Abuelo(x,y) \vee Descenciente(y,x))$ es unna tautología.
\end{itemize}
  \subsection{\blue{Ejercicio 19 COMPLETAR}}
$\forall x. \neg R(x,x)$ = $\{\neg R(x,x)\}$ \\
$\forall x. \forall y. (R(x,y) \Rightarrow R(y,x))$ = $\forall x. \forall y. (\neg R(x,y) \vee R(y,x))$ = $\{\neg R(x,y), R(y,x)\}$ \\
$\forall x. \forall y. \forall z. ((R(x,y) \wedge R(y,z)) \Rightarrow R(x,z))$ = $\forall x. \forall y. \forall z. ( \neg (R(x,y) \wedge R(y,z)) \vee R(x,z))$ = $\forall x. \forall y. \forall z. ( (\neg R(x,y) \vee \neg R(y,z)) \vee R(x,z))$ = $\{\neg R(x,y), \neg R(y,z), R(x,z)\}$ \\
QVQ\\
$\forall x. \neg \exists y. R(x,y)$ = $\forall x. \forall y. \neg R(x,y)$ \\
Lo niego: $\exists x. \neg \forall y. \neg R(x,y)$ = $\exists x. \exists y. \neg \neg R(x,y)$ = $R(f(c),f(d))$ = $\{R(f(c),f(d))\}$ \\
Cláusulas: \\
$\{\neg R(x,x)\}$ \\
$\{\neg R(x,y), R(y,x)\}$ \\
$\{\neg R(x,y), \neg R(y,z), R(x,z)\}$ \\
$\{R(f(c),f(d))\}$
  \subsection{Ejercicio 20}
	A) \\
	Definiciones:
	\begin{itemieze}
	\item natural(cero).
	\item natural(suc(X)) $:-$ natural(X).
	\item mayorOIgual(suc(X),Y) $:-$ mayorOIgual(X,Y).
	\item mayorOIgual(X,X) $:-$ natural(X).
	\end{itemieze}

	Que sucede con la consulta: mayorOIgual(suc(suc(N)), suc(cero))?
	mayorOIgual(suc(suc(N)), suc(cero)) trata de unificar con, mayorOIgual(suc(X),Y), y eso sucede si y solo si, mayorOIgual(X,Y)=mayorOIgual(suc(N),suc(cero)).
	mayorOIgual(suc(N),suc(cero)): trata de unificar con, mayorOIgual(suc(X),Y), y eso sucede si y solo si, mayorOIgual(X,Y)=mayorOIgual(N,suc(cero)).
	mayorOIgual(N,suc(cero)): trata de unificar con, mayorOIgual(suc(X),Y), pero no puede porque N no tiene estructura de suc(), por lo tanto intenta unificar con mayorOIgual(N,suc(cero)) y no puede porque no sabe si N es suc(cero), entonces se cuelga. 
	B) \\

	\begin{enumerate}
	\item $\{natural(cero) \}$
	\item $\{natural(suc(X2)), \neg natural(X2) \}$
	\item $\{mayorOIgual(suc(X3),Y3), \neg mayorOIgual(X3,Y3) \}$
	\item $\{mayorOIgual(X4,X4), \neg natural(X4) \}$
	\item $\{\neg mayorOIgual(suc(suc(N5)),suc(cero))\}$
	\item 5y3. $\{X3 \leftarrow suc(N5), Y3 \leftarrow suc(cero)\}$ y la resolvente es $\{\neg mayorOIgual(suc(N5),suc(cero))\}$
	\item 6y3. $\{X3 \leftarrow N5, Y3 \leftarrow suc(cero)\}$ y la resolvente es $\{\neg mayorOIgual(N5,suc(cero))\}$
	\item 7y4. $\{X4 \leftarrow suc(cero),N5 \leftarrow X5\}$ y la resolvente es $\{natural(suc(cero))\}$
	\item 8y2. $\{X2 \leftarrow cero\}$ y la resolvente es $\{\neg natural(cero)\}$
	\item 9y1. la resolvente es $\{\}$
	\end{enumerate}
	Es resolución SLD porque usé clausulas de horn, resolución binaria y en cada paso de resolución usé la resolvente alterior. También usé el orden de resolución de prolog.
  \subsection{Ejercicio 21}
Quiero probar: $\exists x.$ inteligente(X) $\wedge$ analfabeto(X) entonces lo niego: $\neg \exists x.$ inteligente(X) $\wedge$ analfabeto(X) = $\forall x. \neg ($inteligente(X) $\wedge$ analfabeto(X)) = $\forall x. \neg$ inteligente(X) $\vee \neg$ analfabeto(X) = $\{\neg$ inteligente(X), $\neg$ analfabeto(X)$\}$
\begin{enumerate}
\item $\{analfabeto(X1), \neg vivo(X1), \neg noSabeLeer(X1) \}$
\item $\{noSabeLeer(X2), \neg mesa(X2)\}$
\item $\{vivo(X3), \neg delfín(X3) \}$
\item $\{noSabeLeer(X4), \neg delfín(X4)\}$
\item $\{inteligente(flipper)\}$
\item $\{delfín(flipper)\}$
\item $\{inteligente(alan)\}$
\item $\{\neg inteligente(X8), \neg analfabeto(X8)\}$
\item 8y1. $\{X1 \leftarrow X8 \}$ resolvente: $\{\neg inteligente(X8), \neg vivo(X8), \neg noSabeLeer(X8) \}$
\item 9y3. $\{X3 \leftarrow X8\}$ resolvente: $\{\neg delfín(X3),\neg inteligente(X8), \neg noSabeLeer(X8)\}$
\item 10y4.$\{X4 \leftarrow X8\}$ resolvente: $\{\neg delfín(X3),\neg inteligente(X8), \neg delfín(X8)\}$
\item 11y6.$\{X8 \leftarrow flipper\}$ resolvente: $\{\neg delfín(X3),\neg inteligente(X8)\}$
\item 12y6.$\{X8 \leftarrow flipper\}$ resolvente: $\{\neg inteligente(X8)\}$
\item 13y5.$\{X8 \leftarrow flipper\}$ resolvente: $\{\}$
\end{enumerate}
  \subsection{Ejercicio 22}

\end{document}
