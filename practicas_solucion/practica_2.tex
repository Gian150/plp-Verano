\documentclass[10pt,a4paper, landscape]{article}

\usepackage[pdftex,
pdfauthor={Gianfranco Zamboni},
pdftitle={Resumen: Paradigmas de Lenguajes de Programación},
pdfsubject={},
pdfkeywords={Resumen , Computacion, FCEyN, UBA, Paradigmas de Lenguajes de Programación, Imperativo, Funcional, Cálculo Lambda, Programación Orientada a Objetos, Objetos, Programación Lógica},
pdfproducer={Latex with hyperref},
pdfcreator={pdflatex}]{hyperref}

\usepackage{amsmath}
\usepackage{ amssymb }
\usepackage{bussproofs}
\usepackage{pifont}

\usepackage[spanish]{babel}


\usepackage[utf8]{inputenc} % para poder usar tildes en archivos UTF-8
\usepackage{graphicx}
\usepackage{xcolor}

\usepackage{lscape}
\usepackage{minted}
\usepackage{a4wide} % márgenes un poco más anchos que lo usual
\usepackage[titletoc,toc,page]{appendix}
\usepackage{tikz}
\usepackage{forest}
\usepackage{multicol}

\usepackage{fancyhdr}
\pagestyle{fancy}

%\renewcommand{\chaptermark}[1]{\markboth{#1}{}}
\renewcommand{\sectionmark}[1]{\markright{\thesection\ - #1}}

\fancyhf{}

\fancyhead[LO]{\rightmark} % \thesection\ 
\fancyfoot[LO]{\small{PLP - Prácticas}}
\fancyfoot[RO]{\thepage}
\renewcommand{\headrulewidth}{0.5pt}
\renewcommand{\footrulewidth}{0.5pt}
\setlength{\hoffset}{-0.25in}
\setlength{\textwidth}{25cm}
%\setlength{\hoffset}{-1.1cm}
%\setlength{\textwidth}{16cm}
\setlength{\headsep}{0.5cm}
\setlength{\textheight}{16cm}
\setlength{\voffset}{-0.4in}
\setlength{\headwidth}{\textwidth}
\setlength{\headheight}{13.1pt}

\renewcommand{\baselinestretch}{1.1}  % line spacing

\newenvironment{centrado}
    {
     \begin{center}
     \begin{minipage}{0.8\textwidth}
 }    
    {
     \end{minipage}
     \end{center}
    }

\newcommand{\rel}{\ensuremath{\mathcal{R}}}

\newcommand{\equalDef}{\overset{def}{=}}
\newcommand{\equalDot}{\overset{\cdot}{=}}

\newcommand{\lambdaAbs}[3]{\lambda #1: #2 . #3}
\newcommand{\lambdaAssign}[2]{#1~:=~#2}
\newcommand{\lambdaApp}[2]{#1~#2}
\newcommand{\lambdaIf}[3]{if~ #1~ then~ #2~ else~ #3}
\newcommand{\lambdaTrue}{true}
\newcommand{\lambdaFalse}{false}
\newcommand{\lambdaLet}[4]{let~#1:#2 = #3~in~#4}
\newcommand{\lambdaRef}[1]{ref~#1}
\newcommand{\lambdaVar}[1]{#1}
\newcommand{\lambdaValue}[1]{\color{red}#1\color{black}}
\newcommand{\lambdaFix}[1]{fix~#1}

\newcommand{\lambdaAbsI}[2]{\lambda #1. #2}



\newcommand{\blue}[1]{\color{blue}#1\color{black}}
\newcommand{\replaceBy}[3]{#1\{#2\leftarrow#3\}}

\newcommand{\judgeType}[3]{#1\triangleright #2 : #3}


\newenvironment{scprooftree}[1]%
{\gdef\scalefactor{#1}\begin{center}\proofSkipAmount \leavevmode}%
    {\scalebox{\scalefactor}{\DisplayProof}\proofSkipAmount \end{center} }


\tikzset{
    every leaf node/.style={text=red, align=center},
    every tree node/.style={text=blue, align=center},
}

\forestset{tikzQtree/.style={for tree={if n children=0{
                node options=every leaf node/.try}{node options=every tree node/.try}, text centered}}}
                
                
\DeclareMathOperator{\Erase}{Erase}
\DeclareMathOperator{\Nat}{Nat}
\DeclareMathOperator{\Bool}{Bool}
\DeclareMathOperator{\Union}{Union}

\newcommand{\WFunc}{\mathbb{W}}

\newcommand{\red}[1]{{\color{red}#1}}%\renewcommand{\appendixtocname}{Apéndices}
\newcommand{\green}[1]{{\color{green!40!black}#1}}%\renewcommand{\appendixpagename}{Apéndices}






\setcounter{section}{2}


\begin{document}
\title{PLP - Práctica 2: Introducción al Cálculo Lambda Tipado}

\date{\today}

\author{Zamboni, Gianfranco}

\maketitle
\setcounter{page}{1}

\section*{Sintaxis}
\subsection{Ejercicio 1}
En este ejercicio, nos piden identificar las expresiones sitacticamente válidas. Tenemos que tener cuidado de no confundir estas expresiones con las expresiones correctamente tipadas. Todas las expresiones que nos permite escribir el conjunto de términos son expresiones válidas sintacticamente, aún si estas no pueden ser tipadas.

\vspace*{5mm}
\begin{center}
    \begin{tabular}{c|c|c}
        \textbf{Expresiones de términos} & \textbf{Expresiones de tipo} & \textbf{No válidas}\\
        
        $x$ & $Bool$ & $M$ \\ 
        $x~x$  &  $Bool\to Bool$  &  $\sigma$ \\
        $M~M$   & $Bool\to Bool\to Nat$ & $\lambda x.isZero(x)$ \\ 
        $true~false$ &  $(Bool\to Bool)\to Nat$ & $\lambda x:\lambdaIf{true}{Bool}{Nat.x}$ \\ 
        $true~succ(true~false)$     &   & $succ~true$ \\ 
        ${\lambdaAbs{x}{\sigma}{succ(x)}}$ &   & \\ 
        $\lambdaAbs{x}{Bool}{succ(x)}$     &  & \\ 
        $\lambdaAbs{x}{Bool}{\lambdaIf{0}{true}{\lambdaApp{0}{succ(true)}}}$ &  & \\ 
    \end{tabular}
\end{center}

\vspace*{\fill}
\begin{multicols}{2}
\subsection{Ejercicio 2}

    \begin{forest} tikzQtree,
        [$isZero(pre(succ(\lambdaApp{0}{\lambdaIf{true}{false}{\lambdaAbs{x}{Nat}{x}}})))$,
        [$pre(succ(\lambdaApp{0}{\lambdaIf{true}{false}{\lambdaAbs{x}{Nat}{x}}}))$
        [$succ(\lambdaApp{0}{\lambdaIf{true}{false}{\lambdaAbs{x}{Nat}{x}}})$
        [$\lambdaApp{0}{\lambdaIf{true}{false}{\lambdaAbs{x}{Nat}{x}}}$
        [$0$]
        [$\lambdaIf{true}{false}{\lambdaAbs{x}{Nat}{x}}$
        [$true$]
        [$false$]
        [$\lambdaAbs{x}{Nat}{x}$
        [x]
        ]
        ]                
        ]
        ]
        ]
        ]
    \end{forest}

\subsection{Ejercicio 3}
\paragraph{1.}
$\lambdaAbs{x}{Nat}{succ((\lambdaApp{\lambdaAbs{x}{Nat}{\blue{x}})}{\lambdaValue{x}})}$

\paragraph{2. }En el término $\lambdaAbs{x_1}{Nat}{succ(x_2)}$, $x_1$ no aparece como subtérmino.

\paragraph{3. } La expresión $x~(y~z)$ no sucede en la expresión $u~x~(y~z)$

\begin{center}
    \begin{forest} tikzQtree,
        [$u~x~(y~z)$
        [$u~x$
        [$u$]
        [$x$]
        ]
        [$y~z$
        [$y$]
        [$z$]
        ]
        ]
    \end{forest}
\end{center}

\end{multicols}
\vspace*{\fill}

\newpage
\subsection{Ejercicio 4}
Marcamos con \blue{azul} las variables ligadas y con \red{rojo} las variables libres
\paragraph{a)}

\begin{forest} tikzQtree,
    [$\red{(}\blue{(}\red{u}~\red{x}\blue{)}(\red{y}~\red{z})\red{)}~(\lambdaAbs{v}{Bool}{\blue{v}})$
    [$\blue{(}\red{u}~\red{x}\blue{)}(\red{y}~\red{z})$
    [$\red{u}~\red{x}$
    [$\red{u}$]
    [$\red{x}$]
    ]
    [$\red{y}~\red{z}$
    [$\red{y}$]
    [$\red{z}$]
    ]
    ]
    [$\lambdaAbs{v}{Bool}{\blue{v}}$
    [$\red{v}$]
    ]  
    ]
\end{forest}

\paragraph{b)} En esta expresión aparece $(\lambdaAbs{x}{Bool\to Nat\to Bool}{\lambdaAbs{y}{Bool\to Nat}{\lambdaAbs{z}{Bool}{x~z~(y~z)}}})~u$, la marco con una \xmark~ cuando aparece.

\vspace*{5mm}
\begin{forest}tikzQtree,
[$\red{(}\blue{(}(\lambdaAbs{x}{Bool\to Nat \to Bool}{\lambdaAbs{y}{Bool\to Nat}{\lambdaAbs{z}{Bool}{\red{(}\blue{(}\blue{x}~\blue{z}\blue{)}(\blue{y}~\blue{z})\red{)}}}})~\red{u}\blue{)}~\red{v}\red{)}~ \red{w}$
    [$\blue{(}(\lambdaAbs{x}{Bool\to Nat\to Bool}{\lambdaAbs{y}{Bool\to Nat}{\lambdaAbs{z}{Bool}{\red{(}\blue{(}\blue{x}~\blue{z}\blue{)}(\blue{y}~\blue{z})\red{)}}}})~\red{u}\blue{)}~\red{v}$
        [ \xmark $(\lambdaAbs{x}{Bool\to Nat\to Bool}{\lambdaAbs{y}{Bool\to Nat}{\lambdaAbs{z}{Bool}{\red{(}\blue{(}\blue{x}~\blue{z}\blue{)}(\blue{y}~\blue{z})\red{)}}}})~\red{u}$ ,
            [$\lambdaAbs{x}{Bool\to Nat\to Bool}{\lambdaAbs{y}{Bool\to Nat}{\lambdaAbs{z}{Bool}{\red{(}\blue{(}\blue{x}~\blue{z}\blue{)}(\blue{y}~\blue{z})\red{)}}}}$
                [$\lambdaAbs{y}{Bool\to Nat}{\lambdaAbs{z}{Bool}{\red{(}\blue{(}\red{x}~\blue{z}\blue{)}(\blue{y}~\blue{z})\red{)}}}$
                    [$\lambdaAbs{z}{Bool}{\red{(}\blue{(}\red{x}~\blue{z}\blue{)}(\red{y}~\blue{z})\red{)}}$
                        [$\blue{(}\red{x}~\red{z}\blue{)}(\red{y}~\red{z})$
                            [$\red{x}~\red{z}$
                                [$\red{x}$]
                                [$\red{z}$]
                            ]
                            [$\red{y}~\red{z}$
                                [$\red{y}$]
                                [$\red{z}$]
                            ]
                        ]
                    ]
                ]
            ]
    [$\red{u}$]
    ]
    [$\red{v}$]
    ]
    [$\red{w}$]
]
\end{forest}

\paragraph{c)}
\begin{forest}tikzQtree,
[$\red{(}\blue{(}\red{w}~(\lambdaAbs{x}{Bool\to Nat\to Bool}{\lambdaAbs{y}{Bool\to Nat}{\lambdaAbs{z}{Bool}{\green{(}\red{(}\blue{x}~\blue{z}\red{)}~(\blue{y}~ \blue{z})\green{)}}}})\blue{)}~\red{u}\red{)}~\red{v}$
    [$\blue{(}\red{w}~(\lambdaAbs{x}{Bool\to Nat\to Bool}{\lambdaAbs{y}{Bool\to Nat}{\lambdaAbs{z}{Bool}{\green{(}\red{(}\blue{x}~\blue{z}\red{)}~(\blue{y}~ z)\green{)}}}})\blue{)}~\red{u}$
        [$\red{w}~(\lambdaAbs{x}{Bool\to Nat\to Bool}{\lambdaAbs{y}{Bool\to Nat}{\lambdaAbs{z}{Bool}{\green{(}\red{(}\blue{x}~\blue{z}\red{)}~(\blue{y}~ \blue{z})\green{)}}}})$
            [$\red{w}$]
            [$\lambdaAbs{x}{Bool\to Nat\to Bool}{\lambdaAbs{y}{Bool\to Nat}{\lambdaAbs{z}{Bool}{\green{(}\red{(}\blue{x}~\blue{z}\red{)}~(\blue{y}~ \blue{z})\green{)}}}}$
                [$\lambdaAbs{y}{Bool\to Nat}{\lambdaAbs{z}{Bool}{\green{(}\red{(}\red{x}~\blue{z}\red{)}~(\blue{y}~ \blue{z})\green{)}}}$
                    [$\lambdaAbs{z}{Bool}{\green{(}\red{(}\red{x}~\blue{z}\red{)}~(\red{y}~ \blue{z})\green{)}}$
                        [$\red{(}\red{x}~\red{z}\red{)}~(\red{y}~ \red{z})$
                            [$\red{x}~\red{z}$
                                [$\red{x}$]
                                [$\red{z}$]
                            ]
                            [$\red{y}~ \red{z}$
                                [$\red{y}$]
                                [$\red{z}$]
                            ]
                        ]
                    ]
                ]
            ]
        ]
        [$\red{u}$]
    ]
    [$\red{v}$]
]
\end{forest}
%\subsection{Ejercicio 5}
%Al final de la práctica estan las soluciones de este ejercicio, hay que leerla con las hojas apaisadas. 
%
%\subsection{Ejercicio 6}
%
%$$\judgeType{\phi}{succ(0)}{Nat}$$
%
%$$\judgeType{\phi}{isZero(succ(0))}{Bool}$$
%
%$$\judgeType{\phi}{\lambdaIf{\lambdaIf{true}{false}{false}}{0}{succ(0)}}{Nat}$$
%

%\begin{landscape}
%\subsection*{Tipado}
%\subsection{Ejercicio 5} \label{practica2:ejercicio5}
%\begin{center}
%  \begin{scprooftree}
%      \def\extraVskip{5pt}
%      \AxiomC{}
%      \RightLabel{T-True}
%      \UnaryInfC{$\judgeType{\phi}{true}{Bool}$}
%      
%      \AxiomC{}
%      \RightLabel{T-Zero}
%      \UnaryInfC{$\judgeType{\phi}{0}{Nat}$}
%      
%      \AxiomC{}
%      \RightLabel{T-Zero}
%      \UnaryInfC{$\judgeType{\phi}{0}{Nat}$}
%      \RightLabel{T-Succ}
%      \UnaryInfC{$\judgeType{\phi}{succ(0)}{Nat}$}
%      
%      \TrinaryInfC{$\judgeType{\phi}{\lambdaIf{true}{0}{succ(0)}}{Nat}$}
%    \end{scprooftree}
%\end{center}
%
%\vspace*{2cm}
%En la siguiente demostración $\Gamma =\{x:Nat,~y:Bool\}$
%\begin{center}
%   \begin{scprooftree}
%       \def\extraVskip{5pt}
%       
%       \AxiomC{}
%       \RightLabel{T-True}
%       \UnaryInfC{$\judgeType{\Gamma}{true}{Bool}$}
%       
%       \AxiomC{}
%       \RightLabel{T-False}
%       \UnaryInfC{$\judgeType{\Gamma}{false}{Bool}$}
%       
%       \AxiomC{$z:Bool\in \Gamma,z:Bool$}
%       \RightLabel{T-Var}
%       \UnaryInfC{$\judgeType{\Gamma,z:Bool}{z}{Bool}$}            
%       \RightLabel{T-Abs}
%       \UnaryInfC{$\judgeType{\Gamma}{\lambdaAbs{z}{Bool}{z}}{Bool\to Bool}$}
%       
%       \AxiomC{}
%       \RightLabel{T-True}
%       \UnaryInfC{$\judgeType{\Gamma}{true}{Bool}$}
%       \RightLabel{T-App}
%       \BinaryInfC{$\judgeType{\Gamma}{\lambdaApp{(\lambdaAbs{z}{Bool}{z})}{true}}{Bool}$}
%       
%       \RightLabel{T-If}
%       \TrinaryInfC{$\judgeType{\Gamma}{\lambdaIf{true}{false}{\lambdaApp{(\lambdaAbs{z}{Bool}{z})}{true}}}{Bool}$}
%    \end{scprooftree}
%\end{center}
%
%\vspace*{2cm}
%\begin{center}
%\begin{scprooftree}
%    \def\extraVskip{5pt}
%    
%    \AxiomC{\red{$\judgeType{\phi,x:Bool}{x}{Bool}\Rightarrow \judgeType{\phi}{\lambdaAbs{x}{Bool}{x}}{Bool\to\tau}$}}
%    \RightLabel{T-Abs}
%    \UnaryInfC{$\judgeType{\phi}{\lambdaAbs{x}{Bool}{x}}{Bool}$}
%    
%    \AxiomC{$\judgeType{\phi}{0}{Nat}$}
%    
%    \AxiomC{$\judgeType{\phi}{succ(0)}{Nat}$}
%    
%    \RightLabel{T-If}
%    \TrinaryInfC{$\judgeType{\phi}{\lambdaIf{\lambdaAbs{x}{Bool}{x}}{0}{succ(0)}}{Nat}$}
%\end{scprooftree}
%\end{center}
%
%\newpage
%En la próxima demostración $\Gamma = \{x : Bool \to Nat, y : Bool\}$ 
%\begin{center}
%\begin{scprooftree}
%    \def\extraVskip{5pt}
%    
%    \AxiomC{$x:Bool\to Nat\in\Gamma$}
%    \RightLabel{T-Var}
%    \UnaryInfC{$\judgeType{\Gamma}{x}{Bool \to Nat}$}
%    
%    \AxiomC{$y:Bool\in\Gamma$}
%    \RightLabel{T-Var}
%    \UnaryInfC{$\judgeType{\Gamma}{y}{Bool}$}
%    \RightLabel{T-App}
%    \BinaryInfC{$\judgeType{\Gamma}{\lambdaIf{\lambdaAbs{x}{Bool}{x}}{0}{succ(0)}}{Nat}$}
%\end{scprooftree}
%\end{center}
%
%
%\subsection{Ejercicio 6}
%\begin{center}
%\begin{scprooftree}
%    \def\extraVskip{5pt}
%    \AxiomC{}
%    \RightLabel{T-Zero}
%    \UnaryInfC{$\judgeType{\phi}{0}{Nat}$}
%    
%    \RightLabel{T-Succ}
%    \UnaryInfC{$\judgeType{\phi}{succ(0)}{Nat}$}
%\end{scprooftree}
%\hspace{2cm}
%\begin{scprooftree}
%    \def\extraVskip{5pt}
%        \AxiomC{}
%        \RightLabel{Ya demostrado}
%        \UnaryInfC{$\judgeType{\phi}{succ(0)}{Nat}$}
%    
%    \RightLabel{T-Zero}
%    \UnaryInfC{$\judgeType{\phi}{isZero(succ(0))}{Bool}$}
%\end{scprooftree}
%
%\vspace*{2cm}
%\begin{scprooftree}
%   \def\extraVskip{5pt}
%            \AxiomC{}
%        \RightLabel{T-True}    
%        \UnaryInfC{$\judgeType{\phi}{true}{Bool}$}
%
%            \AxiomC{}
%        \RightLabel{T-True}    
%        \UnaryInfC{$\judgeType{\phi}{false}{Bool}$}
%
%            \AxiomC{}
%       \RightLabel{T-True}    
%       \UnaryInfC{$\judgeType{\phi}{false}{Bool}$}
%
%    \RightLabel{T-If}
%     \TrinaryInfC{$\judgeType{\Gamma}{\lambdaIf{true}{false}{false}}{Bool}$}
%
%        \AxiomC{}
%    \RightLabel{T-Zero}
%    \UnaryInfC{$\judgeType{\phi}{0}{Nat}$}
%    
%         \AxiomC{}
%    \RightLabel{Ya demostrado}
%    \UnaryInfC{$\judgeType{\phi}{succ(0)}{Nat}$}
%            
%\RightLabel{T-If}
%\TrinaryInfC{$\judgeType{\phi}{\lambdaIf{\lambdaIf{true}{false}{false}}{0}{succ(0)}}{Nat}$}
%    \end{scprooftree}
%\end{center}
%
%\newpage
%\subsection{Ejercicio 7}
%En la próxima demostración $\Gamma = \{x:\sigma\}$
%\begin{center}
%\begin{scprooftree}
%       \def\extraVskip{5pt}
%\AxiomC{$x:Nat\in \Gamma$}
%\RightLabel{T-Var}
%\UnaryInfC{$\judgeType{\Gamma}{x}{Nat}$}
%\RightLabel{T-Succ}
%\UnaryInfC{$\judgeType{\Gamma}{succ(x)}{Nat}$}
%\RightLabel{T-Succ}
%\UnaryInfC{$\judgeType{\Gamma}{isZero(succ(x))}{\tau}$}
%\end{scprooftree}
%\end{center}
%\vspace*{5mm}
%Entonces, para que la demostración tenga sentido debe pasar que $\sigma = Nat$ y $\tau = Bool$
%
%\vspace*{2cm}
%\begin{center}
%    \begin{scprooftree}
%   \def\extraVskip{5pt}
%        
%            \AxiomC{$\judgeType{\{x:\sigma\}}{x}{\sigma}$}
%       \RightLabel{T-Abs}
%        \UnaryInfC{$\judgeType{\phi}{\lambdaAbs{x}{\sigma}{x}}{\sigma\to\sigma}$}
%        
%            \AxiomC{$\judgeType{\{y:Bool\}}{0}{Nat}$}
%        \RightLabel{T-Abs}
%        \UnaryInfC{$\judgeType{\phi}{\lambdaAbs{y}{Bool}{0}}{\sigma}$}
%
%\RightLabel{T-App}
%\BinaryInfC{$\judgeType{\phi}{\lambdaApp{(\lambdaAbs{x}{\sigma}{x})}{(\lambdaAbs{y}{Bool}{0})}}{\sigma}$}
%    \end{scprooftree}
%\end{center}
%
%\vspace*{5mm}
%El árbol de la segunda abstracción nos dice que $\sigma = Bool\to Nat$ y como el desgloce de la primera abstracción, no impuso ninguna otra condición sobre $\sigma$, entonces con este tipo funciona.
%
%\end{landscape}

%\end{landscape}

\end{document}