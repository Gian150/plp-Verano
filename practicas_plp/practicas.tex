\documentclass[10pt,a4paper]{article}

\usepackage[pdftex,
pdfauthor={Gianfranco Zamboni},
pdftitle={Resumen: Paradigmas de Lenguajes de Programación},
pdfsubject={},
pdfkeywords={Resumen , Computacion, FCEyN, UBA, Paradigmas de Lenguajes de Programación, Imperativo, Funcional, Cálculo Lambda, Programación Orientada a Objetos, Objetos, Programación Lógica},
pdfproducer={Latex with hyperref},
pdfcreator={pdflatex}]{hyperref}

\usepackage{amsmath}
\usepackage[spanish]{babel}
\usepackage[utf8]{inputenc} % para poder usar tildes en archivos UTF-8
\usepackage{minted}
\usepackage{a4wide} % márgenes un poco más anchos que lo usual
\usepackage[titletoc,toc,page]{appendix}

\newenvironment{centrado}
    {
     \begin{center}
     \begin{minipage}{0.8\textwidth}
 }    
    {
     \end{minipage}
     \end{center}
    }

\newcommand{\rel}{\ensuremath{\mathcal{R}}}

\newcommand{\equalDef}{\overset{def}{=}}
\newcommand{\equalDot}{\overset{\cdot}{=}}

\newcommand{\lambdaAbs}[3]{\lambda #1: #2 . #3}
\newcommand{\lambdaAssign}[2]{#1~:=~#2}
\newcommand{\lambdaApp}[2]{#1~#2}
\newcommand{\lambdaIf}[3]{if~ #1~ then~ #2~ else~ #3}
\newcommand{\lambdaTrue}{true}
\newcommand{\lambdaFalse}{false}
\newcommand{\lambdaLet}[4]{let~#1:#2 = #3~in~#4}
\newcommand{\lambdaRef}[1]{ref~#1}
\newcommand{\lambdaVar}[1]{#1}
\newcommand{\lambdaValue}[1]{\color{red}#1\color{black}}
\newcommand{\lambdaFix}[1]{fix~#1}

\newcommand{\lambdaAbsI}[2]{\lambda #1. #2}



\newcommand{\blue}[1]{\color{blue}#1\color{black}}
\newcommand{\replaceBy}[3]{#1\{#2\leftarrow#3\}}

\newcommand{\judgeType}[3]{#1\triangleright #2 : #3}


\newenvironment{scprooftree}[1]%
{\gdef\scalefactor{#1}\begin{center}\proofSkipAmount \leavevmode}%
    {\scalebox{\scalefactor}{\DisplayProof}\proofSkipAmount \end{center} }


\tikzset{
    every leaf node/.style={text=red, align=center},
    every tree node/.style={text=blue, align=center},
}

\forestset{tikzQtree/.style={for tree={if n children=0{
                node options=every leaf node/.try}{node options=every tree node/.try}, text centered}}}
                
                
\DeclareMathOperator{\Erase}{Erase}
\DeclareMathOperator{\Nat}{Nat}
\DeclareMathOperator{\Bool}{Bool}
\DeclareMathOperator{\Union}{Union}

\newcommand{\WFunc}{\mathbb{W}}

\newcommand{\red}[1]{{\color{red}#1}}%\renewcommand{\appendixtocname}{Apéndices}
\newcommand{\green}[1]{{\color{green!40!black}#1}}%\renewcommand{\appendixpagename}{Apéndices}



\usepackage{fancyhdr}
\pagestyle{fancy}

%\renewcommand{\chaptermark}[1]{\markboth{#1}{}}
\renewcommand{\sectionmark}[1]{\markright{\thesection\ - #1}}

\fancyhf{}

\fancyhead[LO]{Sección \rightmark} % \thesection\ 
\fancyfoot[LO]{\small{Paradigmas de lenguajes de programación}}
\fancyfoot[RO]{\thepage}
\renewcommand{\headrulewidth}{0.5pt}
\renewcommand{\footrulewidth}{0.5pt}
\setlength{\hoffset}{-0.25in}
\setlength{\textwidth}{16cm}
%\setlength{\hoffset}{-1.1cm}
%\setlength{\textwidth}{16cm}
\setlength{\headsep}{0.5cm}
\setlength{\textheight}{25cm}
\setlength{\voffset}{-0.4in}
\setlength{\headwidth}{\textwidth}
\setlength{\headheight}{13.1pt}

\renewcommand{\baselinestretch}{1.1}  % line spacing

\setcounter{section}{-1}

\begin{document}

\title{Prácticas: Paradigmas de Lenguajes de Programación}

\date{\today}

\author{Zamboni, Gianfranco}

\maketitle
\tableofcontents

\newpage
\setcounter{page}{1}

\section{Práctica 0}

\subsection{Ejercicio 1}
\begin{centrado}
\begin{minted}[breaklines]{haskell}
null :: Foldable t => t a -> Bool 
-- Indica si una estructura está vacía. El tipo a debe ser de la clase Foldable, esto es, son tipos a los que se les puede aplicar la función foldr. La notación "t a" indica que es un tipo parámetrico, es decir, un tipo t que usa a otro tipo a, por ejemplo, si le pasamos a la función una lista de enteros, entonces a = Int y t = [Int]

head :: [a] -> a
-- Devuelve el primer elemento de una lista.

tail :: [a] -> [a]
-- Devuelve los últimos elementos de una lista (todos los elementos, salvo el primero).

init :: [a] -> [a]
-- Devuelve los primeros elementos de una lista (todos los elementos salvo el último).

last :: [a] -> a
-- Devuelve el último elemento de una lista.

take :: Int -> [a] -> [a]
-- Devuelve los primeros n elementos de una lista

drop :: Int -> [a] -> [a]
-- Devuelve los últimos n elementos de una lista

(++) :: [a] -> [a] -> [a]
-- Concatena dos listas

concat :: Foldable t => t [a] -> [a]
-- Concatena todas las listas de un contenedor de listas que soporte la operación foldr.

(!!) :: [a] -> Int -> a
-- Dado una lista L y un entero N, devuelve el elemento de L que se encuentra en la N-ésima posición. La numeración comienza desde 0.

elem :: (Eq a, Foldable t) => a -> t a -> Bool
-- Dada una estructura T que soporta la operación foldr y que almacene elementos del tipo a que puedan ser comparados por medio de la igualdad y dado un elemento A de ese tipo, indica si A aparecen en T.

\end{minted}
\end{centrado}

\newpage
\subsection{Ejercicio 2}
\begin{centrado}
\begin{minted}[breaklines]{haskell}
valorAbsoluto :: Float -> Float
valorAbsoluto x | x < 0     = -x
                | otherwise =  x

bisiesto :: Int -> Bool
bisiesto x = (x `mod` 4) == 0

factorial :: Int -> Int
factorial 1 = 1
factorial x = x * factorial (x-1)

cantDivisoresPrimos :: Int -> Int
cantDivisoresPrimos x = 
    length (filter (\x -> length (divisores x) == 2) (divisores x))
        where divisores x = [ y | y <- [1..x], x `mod` y == 0 ];
\end{minted}
\end{centrado}

\subsection{Ejercicio 3}
\begin{centrado}
\begin{minted}[breaklines]{haskell}
inverso :: Float -> Maybe Float
inverso 0 = Nothing
inverso x = Just (1/x)

aEntero :: Either Int Bool -> Int
aEntero (Left x) = x
aEntero (Right x) | x == True = 1
                  | otherwise = 0
\end{minted}
\end{centrado}

\subsection{Ejercicio 4}
\begin{centrado}
\begin{minted}[breaklines]{haskell}
limpiar :: String -> String -> String
limpiar xs ys = [ y | y <- ys, not(elem y xs) ]

difPromedio :: [Float] -> [Float]
difPromedio xs = map (\y -> y - promedio xs) xs 
    where promedio xs = (sum xs) / (genericLength xs)

todosIguales :: [Int] -> Bool
todosIguales xs = foldr (\y  rec -> ((length xs == 1) || (y == (head xs))) && rec) True xs
\end{minted}
\end{centrado}

\subsection{Ejercicio 5}
\begin{centrado}
\begin{minted}[breaklines]{haskell}
data AB a = Nil | Bin (AB a) a (AB a)

vacioAB:: AB a -> Bool
vacioAB Nil = True
vacioAB (Bin _ _ _) = False

negacionAB :: AB Bool -> AB Bool
negacionAB Nil = Nil
negacionAB (Bin l x r) = Bin (negacionAB l) (not x) (negacionAB r)

productoAB :: AB Int -> Int
productoAB Nil = 1
productoAB (Bin l x r) = x * (productoAB l) * (productoAB r)
\end{minted}
\end{centrado}

\newpage
\section{Práctica 1}
\subsection{Ejercicio 1}
\begin{centrado}
\begin{minted}[breaklines]{haskell}
max2 ::(Float, Float) -> Float
max2 (x, y) | x >= y = x        
            | otherwise = y

-- max2 currificada --
max2 :: Float -> Float -> Float
max2 x y | x >= y = x
         | otherwise = y

normaVectorial :: (Float, Float) -> Float
normaVectorial (x, y) = sqrt (x^2 + y^2)

-- normaVectorial currificada --
normaVectorial :: Float -> Float -> Float
normaVectorial x y = sqrt (x^2 + y^2)

subtract :: Float -> Float -> Float
subtract = flip (-)

predecesor :: Float -> Float
predecesor = subtract 1

evaluarEnCero :: (Float -> b) -> b
evaluarEnCero = \f -> f 0

dosVeces :: (a -> a) -> (a -> a)
dosVeces = \f -> f.f

flipAll :: [a -> b -> c] -> [ b -> a -> c]
flipAll = map flip

flipRaro :: b -> ( a -> b -> c ) -> a -> c
flipRaro = flip flip
\end{minted}
\end{centrado}

\subsection{Ejercicio 2}
\begin{centrado}
\begin{minted}[breaklines]{haskell}
[ x | x <- [1..3], y <- [x..3], ( x + y ) `mod` 3 == 0 ] = [ 1, 3 ]
\end{minted}
\end{centrado}

\newpage
\subsection{Ejercicio 3}
\begin{centrado}
\begin{minted}[breaklines]{haskell}
pitagóricas :: [(Integer, Integer, Integer)]
pitagóricas = 
    [(a, b, c) | a <- [1..], b <-[1..], c <- [1..], a^2 + b^2 == c^2]
\end{minted}
\end{centrado}

Esta definición agrega la tupla \texttt{(1,1,1)} a la lista y luego aumenta \texttt{c} infinitamente, sin encontrar ningun nueva coincidencia. Si cambiamos el orden en el que se recorren las listas y agregando algunas cotas de la siguiente forma:
\begin{centrado}
\begin{minted}[breaklines]{haskell}
pitagoricas :: [(Integer, Integer, Integer)]
pitagoricas = [ (a, b, c)
            | c <- [1..], b <-[1..c], a <- [1..c], a^2 + b^2 == c^2]
\end{minted}
\end{centrado}
En este caso, para cada número probamos todas las combinaciones de pares \texttt{(a,b)} tales que la suma de sus cuadrados podría llegar a dar \texttt{c}. Como \texttt{a} y \texttt{b} están acotados por \texttt{c}, ya que claramante $\texttt{c}^2 + \texttt{c}^2 > \texttt{c}^2$, la cantidad de pruebas de pares para cada número es finita ($2^c$ pares) y es posible pasar al siguiente número una vez realizados estos chequeos.

\end{document}