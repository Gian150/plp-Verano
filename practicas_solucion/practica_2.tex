\section{Introducción al Cálculo Lambda Tipado}
\subsection*{Sintaxis}
\subsection{Ejercicio 1}
En este ejercicio, nos piden identificar las expresiones sitacticamente válidas. Tenemos que tener cuidado de no confundir estas expresiones con las expresiones correctamente tipadas. Todas las expresiones que nos permite escribir el conjunto de términos son expresiones válidas sintacticamente, aún si estas no pueden ser tipadas.

\vspace*{5mm}
\begin{center}
\begin{tabular}{cc}
    \textbf{Expresiones de términos} & \textbf{Expresiones de tipo}\\

    $x$ & $\lambdaValue{\sigma}$ \\ 
    \lambdaValue{$M$} & $Bool$ \\ 
    $x~x$ & $Bool\to Bool$ \\ 
    $M~M$ & $Bool\to Bool\to Bool$  \\ 
    $true~false$ & $Bool\to Bool\to Nat$ \\ 
    $true~succ(true~false)$ & $(Bool\to Bool)\to Nat$  \\ 
    $\lambdaValue{\lambdaAbs{x}{\sigma}{succ(x)}}$ &  \\ 
    $\lambdaAbs{x}{Bool}{succ(x)}$ &  \\ 
$\lambdaAbs{x}{Bool}{\lambdaIf{0}{true}{\lambdaApp{0}{succ(x)}}}$ &  \\

\end{tabular}

\vspace*{5mm}
\begin{tabular}{c}
    \textbf{No validas} \\
    $\lambda x.isZero(x)$ \\
    $\lambda x:\lambdaIf{true}{Bool}{Nat.x}$ \\
    $succ~true$ \\
\end{tabular}
\end{center}

\subsection{Ejercicio 2}
\begin{centrado}
\begin{forest} tikzQtree,
    [$isZero(pre(succ(\lambdaApp{0}{\lambdaIf{true}{false}{\lambdaAbs{x}{Nat}{x}}})))$,
        [$pre(succ(\lambdaApp{0}{\lambdaIf{true}{false}{\lambdaAbs{x}{Nat}{x}}}))$
            [$succ(\lambdaApp{0}{\lambdaIf{true}{false}{\lambdaAbs{x}{Nat}{x}}})$
                [$\lambdaApp{0}{\lambdaIf{true}{false}{\lambdaAbs{x}{Nat}{x}}}$
                    [$0$]
                    [$\lambdaIf{true}{false}{\lambdaAbs{x}{Nat}{x}}$
                        [$true$]
                        [$false$]
                        [$\lambdaAbs{x}{Nat}{x}$
                            [x]
                        ]
                    ]                
                ]
            ]
        ]
    ]
\end{forest}
\end{centrado}

\subsection{Ejercicio 3}

$$\lambdaAbs{x}{Nat}{succ((\lambdaApp{\lambdaAbs{x}{Nat}{\blue{x}})}{\lambdaValue{x}})}$$

En el término $\lambdaAbs{x_1}{Nat}{succ(x_2)}$, $x_1$ no aparece como subtérmino.]

\subsection{Ejercicio 4}
A la espera...


%\subsection{Ejercicio 5}
%Al final de la práctica estan las soluciones de este ejercicio, hay que leerla con las hojas apaisadas. 
%
%\subsection{Ejercicio 6}
%
%$$\judgeType{\phi}{succ(0)}{Nat}$$
%
%$$\judgeType{\phi}{isZero(succ(0))}{Bool}$$
%
%$$\judgeType{\phi}{\lambdaIf{\lambdaIf{true}{false}{false}}{0}{succ(0)}}{Nat}$$
%

\begin{landscape}
\subsection*{Tipado}
\subsection{Ejercicio 5} \label{practica2:ejercicio5}
\begin{center}
  \begin{scprooftree}
      \def\extraVskip{5pt}
      \AxiomC{}
      \RightLabel{T-True}
      \UnaryInfC{$\judgeType{\phi}{true}{Bool}$}
      
      \AxiomC{}
      \RightLabel{T-Zero}
      \UnaryInfC{$\judgeType{\phi}{0}{Nat}$}
      
      \AxiomC{}
      \RightLabel{T-Zero}
      \UnaryInfC{$\judgeType{\phi}{0}{Nat}$}
      \RightLabel{T-Succ}
      \UnaryInfC{$\judgeType{\phi}{succ(0)}{Nat}$}
      
      \TrinaryInfC{$\judgeType{\phi}{\lambdaIf{true}{0}{succ(0)}}{Nat}$}
    \end{scprooftree}
\end{center}

\vspace*{2cm}
En la siguiente demostración $\Gamma =\{x:Nat,~y:Bool\}$
\begin{center}
   \begin{scprooftree}
       \def\extraVskip{5pt}
       
       \AxiomC{}
       \RightLabel{T-True}
       \UnaryInfC{$\judgeType{\Gamma}{true}{Bool}$}
       
       \AxiomC{}
       \RightLabel{T-False}
       \UnaryInfC{$\judgeType{\Gamma}{false}{Bool}$}
       
       \AxiomC{$z:Bool\in \Gamma,z:Bool$}
       \RightLabel{T-Var}
       \UnaryInfC{$\judgeType{\Gamma,z:Bool}{z}{Bool}$}            
       \RightLabel{T-Abs}
       \UnaryInfC{$\judgeType{\Gamma}{\lambdaAbs{z}{Bool}{z}}{Bool\to Bool}$}
       
       \AxiomC{}
       \RightLabel{T-True}
       \UnaryInfC{$\judgeType{\Gamma}{true}{Bool}$}
       \RightLabel{T-App}
       \BinaryInfC{$\judgeType{\Gamma}{\lambdaApp{(\lambdaAbs{z}{Bool}{z})}{true}}{Bool}$}
       
       \RightLabel{T-If}
       \TrinaryInfC{$\judgeType{\Gamma}{\lambdaIf{true}{false}{\lambdaApp{(\lambdaAbs{z}{Bool}{z})}{true}}}{Bool}$}
    \end{scprooftree}
\end{center}

\vspace*{2cm}
\begin{center}
\begin{scprooftree}
    \def\extraVskip{5pt}
    
    \AxiomC{\red{$\judgeType{\phi,x:Bool}{x}{Bool}\Rightarrow \judgeType{\phi}{\lambdaAbs{x}{Bool}{x}}{Bool\to\tau}$}}
    \RightLabel{T-Abs}
    \UnaryInfC{$\judgeType{\phi}{\lambdaAbs{x}{Bool}{x}}{Bool}$}
    
    \AxiomC{$\judgeType{\phi}{0}{Nat}$}
    
    \AxiomC{$\judgeType{\phi}{succ(0)}{Nat}$}
    
    \RightLabel{T-If}
    \TrinaryInfC{$\judgeType{\phi}{\lambdaIf{\lambdaAbs{x}{Bool}{x}}{0}{succ(0)}}{Nat}$}
\end{scprooftree}
\end{center}

\newpage
En la próxima demostración $\Gamma = \{x : Bool \to Nat, y : Bool\}$ 
\begin{center}
\begin{scprooftree}
    \def\extraVskip{5pt}
    
    \AxiomC{$x:Bool\to Nat\in\Gamma$}
    \RightLabel{T-Var}
    \UnaryInfC{$\judgeType{\Gamma}{x}{Bool \to Nat}$}
    
    \AxiomC{$y:Bool\in\Gamma$}
    \RightLabel{T-Var}
    \UnaryInfC{$\judgeType{\Gamma}{y}{Bool}$}
    \RightLabel{T-App}
    \BinaryInfC{$\judgeType{\Gamma}{\lambdaIf{\lambdaAbs{x}{Bool}{x}}{0}{succ(0)}}{Nat}$}
\end{scprooftree}
\end{center}


\subsection{Ejercicio 6}
\begin{center}
\begin{scprooftree}
    \def\extraVskip{5pt}
    \AxiomC{}
    \RightLabel{T-Zero}
    \UnaryInfC{$\judgeType{\phi}{0}{Nat}$}
    
    \RightLabel{T-Succ}
    \UnaryInfC{$\judgeType{\phi}{succ(0)}{Nat}$}
\end{scprooftree}
\hspace{2cm}
\begin{scprooftree}
    \def\extraVskip{5pt}
        \AxiomC{}
        \RightLabel{Ya demostrado}
        \UnaryInfC{$\judgeType{\phi}{succ(0)}{Nat}$}
    
    \RightLabel{T-Zero}
    \UnaryInfC{$\judgeType{\phi}{isZero(succ(0))}{Bool}$}
\end{scprooftree}

\vspace*{2cm}
\begin{scprooftree}
   \def\extraVskip{5pt}
            \AxiomC{}
        \RightLabel{T-True}    
        \UnaryInfC{$\judgeType{\phi}{true}{Bool}$}

            \AxiomC{}
        \RightLabel{T-True}    
        \UnaryInfC{$\judgeType{\phi}{false}{Bool}$}

            \AxiomC{}
       \RightLabel{T-True}    
       \UnaryInfC{$\judgeType{\phi}{false}{Bool}$}

    \RightLabel{T-If}
     \TrinaryInfC{$\judgeType{\Gamma}{\lambdaIf{true}{false}{false}}{Bool}$}

        \AxiomC{}
    \RightLabel{T-Zero}
    \UnaryInfC{$\judgeType{\phi}{0}{Nat}$}
    
         \AxiomC{}
    \RightLabel{Ya demostrado}
    \UnaryInfC{$\judgeType{\phi}{succ(0)}{Nat}$}
            
\RightLabel{T-If}
\TrinaryInfC{$\judgeType{\phi}{\lambdaIf{\lambdaIf{true}{false}{false}}{0}{succ(0)}}{Nat}$}
    \end{scprooftree}
\end{center}

\newpage
\subsection{Ejercicio 7}
En la próxima demostración $\Gamma = \{x:\sigma\}$
\begin{center}
\begin{scprooftree}
       \def\extraVskip{5pt}
\AxiomC{$x:Nat\in \Gamma$}
\RightLabel{T-Var}
\UnaryInfC{$\judgeType{\Gamma}{x}{Nat}$}
\RightLabel{T-Succ}
\UnaryInfC{$\judgeType{\Gamma}{succ(x)}{Nat}$}
\RightLabel{T-Succ}
\UnaryInfC{$\judgeType{\Gamma}{isZero(succ(x))}{\tau}$}
\end{scprooftree}
\end{center}
\vspace*{5mm}
Entonces, para que la demostración tenga sentido debe pasar que $\sigma = Nat$ y $\tau = Bool$

\vspace*{2cm}
\begin{center}
    \begin{scprooftree}
   \def\extraVskip{5pt}
        
            \AxiomC{$\judgeType{\{x:\sigma\}}{x}{\sigma}$}
       \RightLabel{T-Abs}
        \UnaryInfC{$\judgeType{\phi}{\lambdaAbs{x}{\sigma}{x}}{\sigma\to\sigma}$}
        
            \AxiomC{$\judgeType{\{y:Bool\}}{0}{Nat}$}
        \RightLabel{T-Abs}
        \UnaryInfC{$\judgeType{\phi}{\lambdaAbs{y}{Bool}{0}}{\sigma}$}

\RightLabel{T-App}
\BinaryInfC{$\judgeType{\phi}{\lambdaApp{(\lambdaAbs{x}{\sigma}{x})}{(\lambdaAbs{y}{Bool}{0})}}{\sigma}$}
    \end{scprooftree}
\end{center}

\vspace*{5mm}
El árbol de la segunda abstracción nos dice que $\sigma = Bool\to Nat$ y como el desgloce de la primera abstracción, no impuso ninguna otra condición sobre $\sigma$, entonces con este tipo funciona.

\end{landscape}
