\documentclass[10pt,a4paper]{article}


\usepackage[pdftex,
pdfauthor={Gianfranco Zamboni},
pdftitle={Resumen: Paradigmas de Lenguajes de Programación},
pdfsubject={},
pdfkeywords={Resumen , Computacion, FCEyN, UBA, Paradigmas de Lenguajes de Programación, Imperativo, Funcional, Cálculo Lambda, Programación Orientada a Objetos, Objetos, Programación Lógica},
pdfproducer={Latex with hyperref},
pdfcreator={pdflatex}]{hyperref}

\usepackage{amsmath}
\usepackage{ amssymb }
\usepackage{bussproofs}

\usepackage[spanish]{babel}


\usepackage[utf8]{inputenc} % para poder usar tildes en archivos UTF-8
\usepackage{graphicx}
\usepackage{xcolor}
\usepackage{pifont}

\usepackage{lscape}
\usepackage{minted}
\usepackage{a4wide} % márgenes un poco más anchos que lo usual
\usepackage[titletoc,toc,page]{appendix}
\usepackage{tikz}
\usepackage{forest}
\usepackage{multicol}

\setlength{\columnsep}{1cm}

\ifthenelse{\paperwidth < \paperheight}{\usepackage{fancyhdr}
\pagestyle{fancy}

%\renewcommand{\chaptermark}[1]{\markboth{#1}{}}
\renewcommand{\sectionmark}[1]{\markright{\thesection\ - #1}}

\fancyhf{}

\fancyhead[LO]{Sección \rightmark} % \thesection\ 
\fancyfoot[LO]{\small{Paradigmas de lenguajes de programación}}
\fancyfoot[RO]{\thepage}
\renewcommand{\headrulewidth}{0.5pt}
\renewcommand{\footrulewidth}{0.5pt}
\setlength{\hoffset}{-0.25in}
\setlength{\textwidth}{16cm}
%\setlength{\hoffset}{-1.1cm}
%\setlength{\textwidth}{16cm}
\setlength{\headsep}{0.5cm}
\setlength{\textheight}{25cm}
\setlength{\voffset}{-0.4in}
\setlength{\headwidth}{\textwidth}
\setlength{\headheight}{13.1pt}

\renewcommand{\baselinestretch}{1.1}  % line spacing}{\usepackage{fancyhdr}
\pagestyle{fancy}

%\renewcommand{\chaptermark}[1]{\markboth{#1}{}}
\renewcommand{\sectionmark}[1]{\markright{\thesection\ - #1}}

\fancyhf{}

\fancyhead[LO]{\rightmark} % \thesection\ 
\fancyfoot[LO]{\small{PLP - Prácticas}}
\fancyfoot[RO]{\thepage}
\renewcommand{\headrulewidth}{0.5pt}
\renewcommand{\footrulewidth}{0.5pt}
\setlength{\hoffset}{-0.25in}
\setlength{\textwidth}{25cm}
%\setlength{\hoffset}{-1.1cm}
%\setlength{\textwidth}{16cm}
\setlength{\headsep}{0.5cm}
\setlength{\textheight}{16cm}
\setlength{\voffset}{-0.4in}
\setlength{\headwidth}{\textwidth}
\setlength{\headheight}{13.1pt}

\renewcommand{\baselinestretch}{1.1}  % line spacing
}





\newenvironment{centrado}
    {
     \begin{center}
     \begin{minipage}{0.8\textwidth}
 }    
    {
     \end{minipage}
     \end{center}
    }

\newcommand{\rel}{\ensuremath{\mathcal{R}}}

\newcommand{\equalDef}{\overset{def}{=}}
\newcommand{\equalDot}{\overset{\cdot}{=}}

\newcommand{\lambdaAbs}[3]{\lambda #1: #2 . #3}
\newcommand{\lambdaAssign}[2]{#1~:=~#2}
\newcommand{\lambdaApp}[2]{#1~#2}
\newcommand{\lambdaIf}[3]{if~ #1~ then~ #2~ else~ #3}
\newcommand{\lambdaTrue}{true}
\newcommand{\lambdaFalse}{false}
\newcommand{\lambdaLet}[4]{let~#1:#2 = #3~in~#4}
\newcommand{\lambdaRef}[1]{ref~#1}
\newcommand{\lambdaVar}[1]{#1}
\newcommand{\lambdaValue}[1]{\color{red}#1\color{black}}
\newcommand{\lambdaFix}[1]{fix~#1}

\newcommand{\lambdaAbsI}[2]{\lambda #1. #2}



\newcommand{\blue}[1]{\color{blue}#1\color{black}}
\newcommand{\replaceBy}[3]{#1\{#2\leftarrow#3\}}

\newcommand{\judgeType}[3]{#1\triangleright #2 : #3}


\newenvironment{scprooftree}[1]%
{\gdef\scalefactor{#1}\begin{center}\proofSkipAmount \leavevmode}%
    {\scalebox{\scalefactor}{\DisplayProof}\proofSkipAmount \end{center} }


\tikzset{
    every leaf node/.style={text=red, align=center},
    every tree node/.style={text=blue, align=center},
}

\forestset{tikzQtree/.style={for tree={if n children=0{
                node options=every leaf node/.try}{node options=every tree node/.try}, text centered}}}
                
                
\DeclareMathOperator{\Erase}{Erase}
\DeclareMathOperator{\Nat}{Nat}
\DeclareMathOperator{\Bool}{Bool}
\DeclareMathOperator{\Union}{Union}

\newcommand{\WFunc}{\mathbb{W}}

\newcommand{\red}[1]{{\color{red}#1}}%\renewcommand{\appendixtocname}{Apéndices}
\newcommand{\green}[1]{{\color{green!40!black}#1}}%\renewcommand{\appendixpagename}{Apéndices}





\begin{document}
\title{Resumen: Paradigmas de Lenguajes de Programación}

\date{\today}

\author{Zamboni, Gianfranco}

\section{Funciones de haskell}
\begin{centrado}
	\begin{minted}{haskell}
foldr, foldl :: Foldable t => (a -> b -> b) -> b -> t a -> b
foldr1, foldl1 :: Foldable t => (a -> a -> a) -> t a -> a
	\end{minted}
\end{centrado} 
\begin{centrado}
	\begin{minted}{haskell}
map :: (a -> b) -> [a] -> [b]
zipWith :: (a -> b -> c) -> [a] -> [b] -> [c]
	\end{minted}
\end{centrado} 
\begin{centrado}
	\begin{minted}{haskell}
(++) :: [a] -> [a] -> [a]
(!!) :: [a] -> Int -> a
head, last :: [a] -> a
init, tail :: [a] -> [a]
length :: Foldable t => t a -> Int
	\end{minted}
\end{centrado} 
\begin{centrado}
	\begin{minted}{haskell}
reverse :: [a] -> [a]
concat :: Foldable t => t [a] -> [a]
union :: Eq a => [a] -> [a] -> [a]
	\end{minted}
\end{centrado} 
\begin{centrado}
	\begin{minted}{haskell}
all, any :: Foldable t => (a -> Bool) -> t a -> Bool
	\end{minted}
\end{centrado} 
\begin{centrado}
	\begin{minted}{haskell}
null :: Foldable t => t a -> Bool -- Es vacio
elem :: (Eq a, Foldable t) => a -> t a -> Bool
    
nub :: Eq a => [a] -> [a] -- Elimina duplicados
sort :: Ord a => [a] -> [a] -- Ordena la lista
	\end{minted}
\end{centrado} 
\begin{centrado}
	\begin{minted}{haskell}
concatMap :: Foldable t => (a -> [b]) -> t a -> [b]
find :: (a -> Bool) -> [a] -> Maybe a
filter :: (a -> Bool) -> [a] -> [a]
iterate :: (a -> a) -> a -> [a]
span :: (a -> Bool) -> [a] -> ([a], [a])
replicate :: Int -> a -> [a]
take, drop :: Int -> [a] -> [a]
takeWhile, dropWhile :: (a -> Bool) -> [a] -> [a]
	\end{minted}
\end{centrado} 
\begin{centrado}
	\begin{minted}{haskell}
and, or :: Foldable t => t Bool -> Bool
maximum, minimum :: (Ord a, Foldable t) => t a -> a
sum :: (Num a, Foldable t) => t a -> a
	\end{minted}
\end{centrado} 
\begin{centrado}
	\begin{minted}{haskell}
max, min :: Ord a => a -> a -> a
rem :: Integral a => a -> a -> a
ord :: Char -> Int
chr :: Int -> Char
	\end{minted}
\end{centrado} 
\begin{centrado}
	\begin{minted}{haskell}
fromJust :: Maybe a -> a
isNothing :: Maybe a -> Bool
lookup :: Eq a => a -> [(a, b)] -> Maybe b
maybe :: b -> (a -> b) -> Maybe a -> b


	\end{minted}
\end{centrado} 

\newpage
\section{Esquemas de recursión}

\begin{centrado}
	\begin{minted}{haskell}
map :: (a -> b) -> [a] -> [b]
map _ [] = []
map f (x:xs) = (f x) : (map f xs)
	\end{minted}
\end{centrado} 

\begin{centrado}
	\begin{minted}{haskell}
filter :: (a -> Bool) -> [a] -> [a]
filter _ [] = []
filter p (x:xs) | (p x)     = x : (filter p xs)
                | otherwise = filter p xs  
	\end{minted}
\end{centrado}

\begin{centrado}
	\begin{minted}{haskell}
foldr :: (a -> b -> b) -> b -> [a] -> b
foldr _ z [] = z
foldr f z (x:xs) = f x (foldr f z xs)
		
foldl :: (b -> a -> b) -> b -> [a] -> b
foldl f z [] = z
foldl f z (x : xs) = foldl f (f z x) xs
\end{minted}
\end{centrado}

\begin{centrado}
	\begin{minted}{haskell}
recr :: b -> (a -> [a] -> b -> b) -> [a] -> b
recr z _ []= z
recr z f (x:xs) = f x xs (recr z f xs)
	\end{minted}
\end{centrado}

\begin{centrado}
	\begin{minted}[breaklines]{haskell}
type DivideConquer a b  = (a -> Bool) -> (a -> b) -> (a -> [a]) -> 
                          ([b] -> b) -> a -> b                         
\end{minted}
\end{centrado}
\begin{centrado}
	\begin{minted}[breaklines]{haskell}
divideConquerListas :: DivideConquer [a] b
-- Esto significa que DivideConquerLista es de tipo 
-- ([a] -> Bool) -> ([a] -> b) -> ([a] -> [[a]]) -> ([b] -> b)
-- -> [a] -> b

divideConquerListas esTrivial resolver repartir combinar l =
	if (esTrivial l) then resolver l
	else combinar (map dc (repartir l))
where dc = divideConquerListas esTrivial resolver repartir combinar
                        
	\end{minted}
\end{centrado}

\newpage
\section{Cálculo lambda $\lambda^{bn}$}
\subsubsection*{Función Free Variables}
\begin{equation*}
\begin{split}
FV(x) &\equalDef {x} \\
FV(true) = FV(false) &\equalDef \emptyset \\
FV(\lambdaIf{M}{P}{Q}) &\equalDef FV(M)\cup FV(P)\cup FV(Q) \\
FV(\lambdaApp{M}{N}) &\equalDef FV(M)\cup FV(N) \\
FV(\lambdaAbs{x}{\sigma}{M}) &\equalDef FV(M) \backslash \{x\}
\end{split}
\end{equation*}
\subsubsection*{Sustitución}
\begin{equation*}
\begin{split}
\replaceBy{x}{x}{N} &\equalDef N \\
\replaceBy{a}{x}{N} &\equalDef a \text{ si } a \in \{true,false\}\cup\mathcal{X}\backslash\{x\} \\
\replaceBy{(\lambdaIf{M}{P}{Q})}{x}{N} &\equalDef \lambdaIf{\replaceBy{M}{x}{N}}{\replaceBy{P}{x}{N}}{\replaceBy{Q}{x}{N}}\\
\replaceBy{(\lambdaApp{M_1}{M_2})}{x}{N} &\equalDef \lambdaApp{\replaceBy{M_1}{x}{N}}{\replaceBy{M_2}{x}{N}}\\
\replaceBy{(\lambdaAbs{y}{\sigma}{M})}{x}{N} &\equalDef \lambdaAbs{y}{\sigma}{\replaceBy{M}{x}{N}}~x\neq y,~y\notin~FV(N)
\end{split}
\end{equation*}

\subsubsection{Tipos}
$\sigma, \tau ~::=~ Bool~|~Nat~|~\sigma\to\tau$
\subsubsection*{Expresiones}
\begin{equation}
\begin{split}
M, P, Q ~ ::&= ~ true~|~ false~|~ \lambdaIf{M}{P}{Q} \\
&|~ \lambdaApp{M}{N}~|~ \lambdaAbs{x}{\sigma}{M} \\
&|~x~|~0~|~succ(M)~|~pred(M)~|~isZero(M)
\end{split}
\end{equation}




\subsubsection{Reglas de tipado}

\begin{equation*}
\begin{gathered}
    \frac{}{\judgeType{\Gamma}{\lambdaValue{true}}{Bool}}(\text{T-True}) \hspace*{2cm} \frac{}{\judgeType{\Gamma}{\lambdaValue{false}}{Bool}}(\text{T-False})\hspace*{2cm} \\
    \vspace*{5mm} \\
    \frac{x:\sigma\in\Gamma}{\judgeType{\Gamma}{\lambdaValue{x}}{\sigma}}(\text{T-Var})\hspace*{2cm}
    \frac{\judgeType{\Gamma}{M}{Bool}\hspace*{5mm}\judgeType{\Gamma}{P}{\sigma}\hspace*{5mm}\judgeType{\Gamma}{Q}{\sigma}}{\judgeType{\Gamma}{\lambdaIf{M}{P}{Q}}{\sigma}}(\text{T-If}) \\
    \vspace*{5mm} \\
    \frac {\judgeType{\Gamma,x:\sigma}{M}{\tau}}
          {\judgeType{\Gamma}{\lambdaValue{\lambdaAbs{x}{\sigma}{M}}}{\sigma\to\tau}}(\text{T-Abs})\hspace*{2cm}
    \frac{\judgeType{\Gamma}{M}{\sigma\to\tau}\hspace*{5mm}\judgeType{\Gamma}{N}{\sigma}}{\judgeType{\Gamma}{\lambdaApp{M}{N}}{\tau}}(\text{T-App})
\end{gathered}
\end{equation*}

\vspace*{5mm}
\begin{equation*}
\frac{}{\judgeType{\Gamma}{0}{Nat}}(\text{T-Zero})
\end{equation*}
\vspace*{5mm}
\begin{equation*}
\frac{\judgeType{\Gamma}{M}{Nat}}
{\judgeType{\Gamma}{succ(M)}{Nat}}(\text{T-Succ})\hspace*{1cm}
\frac{\judgeType{\Gamma}{M}{Nat}}{\judgeType{\Gamma}{pred(M)}{Nat}}(\text{T-Pred})
\end{equation*}
\vspace*{5mm}
\begin{equation*}
\frac{\judgeType{\Gamma}{M}{Nat}}{\judgeType{\Gamma}{isZero(M)}{Bool}}(\text{T-IsZero})
\end{equation*}


\subsubsection*{Valores}
$ V~::=~true~|~false~|~\lambdaAbs{x}{\sigma}{M}|~\underline{n}\text{ donde } \underline{n} \text{ abrevia } succ^n(0)$

\subsection*{Reglas de semánticas}
\begin{equation*}
\frac{}{\lambdaIf{\lambdaValue{true}}{M_1}{M_2} \to M_1}(\text{E-IfTrue})
\end{equation*}
\vspace*{5mm}
\begin{equation*}
\frac{}{\lambdaIf{\lambdaValue{false}}{M_1}{M_2} \to M_2}(\text{E-IfFalse})
\end{equation*}
\vspace*{5mm}
\begin{equation*}
\frac{M_1\to M'_1}{\lambdaIf{M_1}{M_2}{M_3}\to\lambdaIf{M'_1}{M_2}{M_3}}(\text{E-If})
\end{equation*}
\vspace*{5mm}
\begin{equation*}
\frac{M_1\to M'_1}{\lambdaApp{M_1}{M_2} \to 
\lambdaApp{M'_1}{M_2}}(\text{E-App1}~ /~ \mu)
\hspace*{5mm}
\frac{M_2 \to M'_2}{\lambdaApp{\lambdaValue{V_1}}{M_2} \to 
	\lambdaApp{\lambdaValue{V_1}}{M'_2}}(\text{E-App2}~/~v)
\end{equation*}	
\vspace*{5mm}
\begin{equation*}
\frac{}{\lambdaApp{(\lambdaAbs{x}{\sigma}{M})}{\lambdaValue{V}} \to 
	\replaceBy{M}{x}{\lambdaValue{V}}}(\text{E-App3}~/~\beta)
\end{equation*}
\vspace*{5mm}
\begin{equation*}
\frac{M_1\to M_1'}{succ(M_1)\to succ(M'_1)}(\text{E-Succ})
\end{equation*}
\vspace*{5mm}
\begin{equation*}
\frac{}{pred(0)\to 0}(\text{E-PredZero})\hspace*{1cm}\frac{}{pred(succ(\underline{n}))\to \underline{n}}(\text{E-PredSucc})
\end{equation*}
\vspace*{5mm}
\begin{equation*}
\frac{M_1\to M_1'}{pred(M_1)\to pred(M'_1)}(\text{E-Pred})
\end{equation*}
\vspace*{5mm}
\begin{equation*}
\frac{}{isZero(0)\to true}(\text{E-IsZeroZero})\hspace*{1cm}\frac{}{isZero(succ(\underline{n}))\to false}(\text{E-isZeroSucc})
\end{equation*}
\vspace*{5mm}
\begin{equation*}
\hspace*{1cm}\frac{M_1\to M'_1}{isZero(M_1)\to isZero(M_1')}(\text{E-isZero})
\end{equation*}

\newpage
\section{Extensión con memoria $\lambda^{bnu}$}
\paragraph{Tipos}
$$\sigma, \tau ~::=~ Bool~|Nat~|~\blue{Unit}~|~\blue{Ref~\sigma}~|~\sigma\to\tau$$

\paragraph{Términos}

$$ M~::=~ \dots~|~unit~|~\lambdaRef{M}~|~!M~|~\lambdaAssign{M}{N}!|~    l$$

\paragraph{Axiomas y reglas de tipado}
\begin{equation*}
\frac{}{\judgeType{\Gamma|\Sigma}{unit}{Unit}}(\text{T-Unit})\hspace*{1cm}\frac{\judgeType{\Gamma|\Sigma}{M_1}{\sigma}}{\judgeType{\Gamma|\Sigma}{\lambdaRef{M_1}}{Ref~\sigma}}(\text{T-Ref})
\end{equation*}
\vspace*{5mm}
\begin{equation*}
\frac{\judgeType{\Gamma|\Sigma}{M_1}{Ref~\sigma}}{\judgeType{\Gamma}{!M_1}{\sigma}}(\text{T-DeRef})
\end{equation*}
\vspace*{5mm}
\begin{equation*}
\frac{\judgeType{\Gamma|\Sigma}{M_1}{Ref~\sigma}\hspace*{5mm}\judgeType{\Gamma|\Sigma}{M_2}{\sigma}}{\judgeType{\Gamma}{\lambdaAssign{M_1}{M_2}}{Unit}}(\text{T-Assing})
\end{equation*}
\vspace*{5mm}
\begin{equation*}
\frac{\Sigma(l) = \sigma}{\judgeType{\Gamma|Signa}{l}{Ref~\sigma}}(\text{T-Loc})
\end{equation*}

\paragraph{Valores}
$$V~::=~\dots~|~unit~|~l$$

\paragraph{Axiomas y reglas semánticas}
\begin{equation*}
\frac{M_1|\mu\to M'_1|\mu'}{\lambdaApp{M_1}{M_2}|\mu\to \lambdaApp{M'_1}{M_2}|\mu'}(\text{E-App1})\hspace*{1cm}\frac{M_2|\mu\to M'_2|\mu'}{\lambdaApp{\lambdaValue{V_1}}{M_2}|\mu\to \lambdaApp{\lambdaValue{V_1}}{M'_2}|\mu'}(\text{E-App2})
\end{equation*}
\vspace*{5mm}
\begin{equation*}
\frac{}{(\lambdaApp{\lambdaAbs{x}{\sigma}{M})}{\lambdaValue{V}}|\mu\to \replaceBy{M}{x}{\lambdaValue{V}}|\mu'}(\text{E-AppAbs})
\end{equation*}
\vspace*{5mm}
\begin{equation*}
\frac{M_1|\mu\to M'_1|\mu'}{!M_1|\mu\to !M_1'|\mu'}(\text{E-DeRef})\hspace*{1cm}
\frac{\mu(l) = \lambdaValue{V}}{!l|\mu\to V|\mu}(\text{E-DerefLoc})
\end{equation*}
\vspace*{5mm}
\begin{equation*}
\frac{M_1|\mu\to M'_1|\mu'}{\lambdaAssign{M_1}{M_2}~|\mu\to \lambdaAssign{M'_1}{M_2}|\mu'}(\text{E-Assign1})
\end{equation*}
\vspace*{5mm}
\begin{equation*}
\frac{M_2|\mu\to M'_2|\mu'}{\lambdaAssign{\lambdaValue{V}}{M_2}|\mu\to \lambdaAssign{\lambdaValue{V}}{M'_2}|\mu'}(\text{E-Assign2})
\end{equation*}
\vspace*{5mm}
\begin{equation*}
\frac{}{\lambdaAssign{l}{\lambdaValue{V}}|\mu\to unit|\mu[l\to \lambdaValue{V}]}(\text{E-Assign})
\end{equation*}
\vspace*{5mm}
\begin{equation*}
\frac{M_1|\mu\to M'_1|\mu'}{\lambdaRef{M_1}|\mu\to \lambdaRef{M'_1}|\mu'}(\text{E-Ref})\hspace*{1cm}
\frac{l\notin Dom(\mu)}{\lambdaRef{\lambdaValue{V}}|\mu\to l|\mu\oplus(l\to \lambdaValue{V})}(\text{E-RefV})
\end{equation*}

\newpage
\section{Extensión con recursión $\lambda^{..r}$}
\paragraph{Términos}
$$M~:=~\dots~|~\lambdaFix{M}$$

\paragraph{Regla de tipado}
\begin{equation*}
    \frac{\judgeType{\Gamma}{M}{\sigma\to\sigma}}{\judgeType{\Gamma}{\lambdaFix{M}}{\sigma}}(\text{T-Fix})
\end{equation*}

\paragraph{Reglas de evaluación}
\begin{equation*}
    \frac{M_1\to M'_1}{\lambdaFix{M_1}\to\lambdaFix{M'_1}}(\text{E-Fix})
\end{equation*}
\vspace*{5mm}
\begin{equation*}
\frac{}{\lambdaFix{(\lambdaAbs{x}{\sigma}{M})}\to\replaceBy{M}{x}{\lambdaFix{\lambdaAbs{x}{\sigma}{M}}}}(\text{E-FixBeta})
\end{equation*}

\section{Extensión con Declaraciones Locales (\texorpdfstring{$\lambda^{...let}$}{lambda ...let})}\label{extension_lambda:let}

Con esta extensión, agregamos al lenguaje el término $\lambdaLet{x}{\sigma}{M}{N}$, que evalúa $M$ a un valor, liga $x$ a $V$ y, luego, evalúa $N$. Este término solo mejora la legibilidad de los programas que ya podemos definir con el lenguaje hasta ahora definido.

\paragraph{Términos}
$$ M~::=~ \dots~|~\lambdaLet{x}{\sigma}{M}{N} $$


\paragraph{Axiomas y reglas de tipado}
\begin{equation*}
\frac{\judgeType{\Gamma}{M}{\sigma_1}\hspace*{5mm}\judgeType{\Gamma,x:\sigma_1}{N}{\sigma_2}}{\judgeType{\Gamma}{\lambdaLet{x}{\sigma_1}{M}{N}}{\sigma_2}}(\text{T-Let})
\end{equation*}

\paragraph{Axiomas y reglas de evaluación}

\begin{equation*}
\frac{M_1\to M_1'}{\lambdaLet{x}{\sigma}{M_1}{M_2}\to \lambdaLet{x}{\sigma}{M'_1}{M_2}}(\text{E-Let})
\end{equation*}
\vspace*{5mm}
\begin{equation*}
\frac{}{\lambdaLet{x}{\sigma}{\lambdaValue{V_1}}{M_2}\to \replaceBy{M_2}{x}{\lambdaValue{V_1}}}(\text{E-LetV})
\end{equation*}

\subsubsection{Construcción \textit{let} recursivo (Letrec)}
Una construcción alternativa para definir funciones recursivas es 
$$letrec~f:\sigma\to\sigma = \lambdaAbs{x}{\sigma}{M~in~N}$$

Y $letRec$ se puede definir  en base a $let$ y $fix$ (definido en \ref{lambda_calculo:recursion}) de la siguiente forma:

$$\lambdaLet{f}{\sigma\to\sigma}{(\lambdaFix{\lambdaAbs{f}{\sigma\to\sigma}{\lambdaAbs{x}{\sigma}{M}}})}{N}$$


\newpage
\section{Extensión con Registros \texorpdfstring{$\lambda^{...r}$}{lambda ...r+}}

\paragraph{Tipos}
$$\sigma, \tau ~::=~...~|~\{l_i : \sigma_i ~^{i\in 1..n}\}$$

El tipo $\{l_i : \sigma_i^{i\in 1..n}\}$ representan las estructuras con $n$ atributos tipados, por ejemplo: $\{nombre : String,edad:Nat\}$
\paragraph{Términos}
$$ M~::=~ \dots~|~\{l_i = M_i ~^{i\in 1..n}\}~|~M.l $$

Los términos significan:
\begin{itemize}
    \item El registro $\{l_i = M_i ~^{i\in 1..n}\}$ evalua $\{l_i = V_i ~^{i\in 1..n}\}$  donde $V_i$ es el valores al que evalúa $M_i$ para $i\in 1..n$.
    \item $M.l$: Proyecta el valor de la etiqueta $l$ del registro $M$
\end{itemize}

\paragraph{Axiomas y reglas de tipado}
\begin{equation*}
\frac{\judgeType{\Gamma}{M_i}{\sigma_i} \text{ para cada } i \in 1..n}{\judgeType{\Gamma}{\{l_i = M_i ~^{i\in 1..n}\}}{\{l_i : \sigma_i ~^{i\in 1..n}\}}}(\text{T-RCD})
\end{equation*}
\vspace*{5mm}
\begin{equation*}
\frac{\judgeType{\Gamma}{M}{\{l_i : \sigma_i ~^{i\in 1..n}\}}\hspace*{5mm} j \in 1..n}
{\judgeType{\Gamma}{M.l_j}{\sigma_j}}(\text{T-Proj})
\end{equation*}

\paragraph{Valores}
$$V~::=~\dots~|~\{l_i = V_i ~^{i\in 1..n}\}$$

\paragraph{Axiomas y reglas de evaluación}

\begin{equation*}
\frac{j\in 1..n}{\{l_i = \lambdaValue{V_i} ~^{i\in 1..n}\}.l_j \to \lambdaValue{V_j}}(\text{E-ProjRcd})
\end{equation*}
\vspace*{5mm}
\begin{equation*}
\frac{M \to M'}{M.l \to M'.l}(\text{E-Proj})
\end{equation*}

\vspace*{5mm}
\begin{equation*}
\frac{M_j\to M_j'}{\{l_i = \lambdaValue{V_i}~^{i\in 1..j-1}, l_j = M_j, l_i = M_i ~^{i\in j+1..n}\} \to \{l_i = \lambdaValue{V_i}~^{i\in 1..j-1}, l_j = M'_j, l_i = M_i ~^{i\in j+1..n}\}}(\text{E-RCD})
\end{equation*}
\vspace*{5mm}

\newpage
\section{Extensión con tuplas}

\paragraph{Tipos}
$$\sigma,\tau~::= \dots~|~\sigma\times\tau$$

\paragraph{Términos}
$$M,~N~::=~\dots~|~<M,N>~|~\pi_1(M)~|~\pi_2(M)$$
\paragraph{Axiomas y reglas de tipado}
\begin{equation*}
    \frac{\judgeType{\Gamma}{M}{\sigma}\hspace*{5mm}\judgeType{\Gamma}{N}{\tau}}{\judgeType{\Gamma}{<M,N>}{\sigma\times\tau}}(\text{T-Tupla})
\end{equation*}
\vspace*{5mm}
\begin{equation*}
\frac{\judgeType{\Gamma}{M}{\sigma\times\tau}}{\judgeType{\Gamma}{\pi_1(M)}{\sigma}}(\text{T-}\pi_1)\hspace*{1cm}\frac{\judgeType{\Gamma}{M}{\sigma\times\tau}}{\judgeType{\Gamma}{\pi_2(M)}{\tau}}(\text{T-}\pi_2)
\end{equation*}

\paragraph{Valores}
$$V~::=~\dots~|~<V,V>$$

\paragraph{Axiomas y reglas de evaluación}
\begin{equation*}
\frac{M\to M'}{<M,N>\to<M',N>}(\text{E-Tuplas})\hspace*{1cm}\frac{N\to N'}{<\lambdaValue{V},N>\to<\lambdaValue{V},N'>}(\text{E-Tuplas1})
\end{equation*}
\vspace*{5mm}
\begin{equation*}
\frac{M\to M'}{\pi_1(M)\to\pi_1(M')}(\text{E-}\pi_1)\hspace*{1cm}\frac{}{\pi_1(<\lambdaValue{V_1}, \lambdaValue{V_2}>)\to\lambdaValue{V_1}}(\text{E-}\pi'_1)
\end{equation*}
\vspace*{5mm}
\begin{equation*}
\frac{M\to M'}{\pi_2(M)\to\pi_2(M')}(\text{E-}\pi_2)\hspace*{1cm}\frac{}{\pi_2(<\lambdaValue{V_1}, \lambdaValue{V_2}>)\to\lambdaValue{V_2}}(\text{E-}\pi'_2)
\end{equation*}

\newpage
\section{Extensión con árboles binarios}

\paragraph{Tipos}
$$\sigma,\tau~::= \dots~|~AB_\sigma$$

\paragraph{Términos}
$$M,~N~::=~\dots~|~\text{Nil}_\sigma~|~\text{Bin}(M, N, O)~|~\text{raiz}(M)~|~\text{der}(M)~|~\text{izq}(M)~|~\text{esNil}(M)$$
\paragraph{Axiomas y reglas de tipado}
\begin{equation*}
\begin{gathered}
    \frac{}{\judgeType{\Gamma}{\text{Nil}_\sigma}{AB_\sigma}}(\text{T-Nil})\hspace*{1cm}
\frac{\judgeType{\Gamma}{M}{AB_\sigma}\hspace*{5mm}\judgeType{\Gamma}{N}{\sigma}\hspace*{5mm}\judgeType{\Gamma}{O}{AB_\sigma}}{\judgeType{\Gamma}{\text{Bin}(M, N, O)}{AB_\sigma}}(\text{T-Bin}) \\
\vspace*{5mm}\\
\frac{\judgeType{\Gamma}{M}{AB_\sigma}}{\judgeType{\Gamma}{\text{raiz}(M)}{\sigma}}(\text{T-raiz})\hspace*{1cm}
\frac{\judgeType{\Gamma}{M}{AB_\sigma}}{\judgeType{\Gamma}{\text{der}(M)}{AB_\sigma}}(\text{T-der})
\vspace*{5mm} \\
\frac{\judgeType{\Gamma}{M}{AB_\sigma}}{\judgeType{\Gamma}{\text{izq}(M)}{AB_\sigma}}(\text{T-izq})
\hspace*{1cm}
\frac{\judgeType{\Gamma}{M}{AB_\sigma}}{\judgeType{\Gamma}{\text{isNil}(M)}{Bool}}(\text{T-isNil})
\end{gathered}
\end{equation*}

\paragraph{Valores}
$$V~::=~\dots~|~\text{Nil}~|~\text{Bin}(V,V,V)$$

\paragraph{Axiomas y reglas de evaluación}
\begin{equation*}
\frac{M\to M'}{\text{Bin}(M,N,O)\to \text{Bin}(M',N,O)}(\text{E-Bin1})\hspace*{1cm}\frac{N\to N'}{\text{Bin}(V,N,O)\to \text{Bin}(V,N',O)}(\text{E-Bin2})
\end{equation*}
\vspace*{5mm}
\begin{equation*}
\frac{O\to O'}{\text{Bin}(V_1,V_2,O)\to \text{Bin}(V_1,V_2,O')}(\text{E-Bin3})
\end{equation*}
\vspace*{5mm}
\begin{equation*}
\frac{M\to M'}{\text{raiz}(M)\to\text{raiz}(M')}(\text{E-Raiz1})\hspace*{1cm}\frac{}{\text{raiz}(\text{Bin}(V_1,V_2,V_3))\to V_2}(\text{E-Bin3})
\end{equation*}
\vspace*{5mm}
\begin{equation*}
\frac{M\to M'}{\text{der}(M)\to\text{der}(M')}(\text{E-Der1})\hspace*{1cm}\frac{}{\text{der}(\text{Bin}(V_1,V_2,V_3))\to V_3}(\text{E-Der2})
\end{equation*}
\vspace*{5mm}
\begin{equation*}
\frac{M\to M'}{\text{izq}(M)\to\text{izq}(M')}(\text{E-Izq1})\hspace*{1cm}\frac{}{\text{izq}(\text{Bin}(V_1,V_2,V_3))\to V_1}(\text{E-Izq2})
\end{equation*}
\hspace*{5mm}
\begin{equation*}
\frac{}{\text{isNil}(M)\to\text{izq}(M')}(\text{E-isNil1})\hspace*{1cm}\frac{}{\text{isNil}(\text{Bin}(V_1,V_2,V_3))\to false}(\text{E-isNilBin})
\end{equation*}
\hspace*{5mm}
\begin{equation*}
\frac{}{\text{isNil}(\text{Bin}(V_1,V_2,V_3))\to true}(\text{E-isNilNil})
\end{equation*}

\newpage
\section{Algoritmo de Martelli-Montanari}
\begin{enumerate}
\item \textbf{Descomposición}

$\{\sigma_1\to\sigma_2 \equalDot\tau_1\to\tau_2\}\cup G\mapsto \{\sigma_1\equalDot\tau_1,~\sigma_2 \equalDot\tau_2\}\cup G$
\item \textbf{Eliminación de par trivial}

$\{ Nat \equalDot Nat\}\cup G\mapsto G$

$\{ Bool \equalDot Bool\}\cup G\mapsto G$

$\{\text{s} \equalDot\text{s}\}\cup G\mapsto G$
\item \textbf{Swap} Si $\sigma$ no es una variable,

$\{\sigma \equalDot\text{s}\}\cup G\mapsto \{\text{s}\equalDot\sigma\}\cup G$

\item \textbf{Eliminación de variable} Si $s\notin FV(\sigma)$

$\{\text{s}\equalDot\sigma\}\cup G\mapsto_{\sigma/s} G[\sigma/s]$

\item \textbf{Falla}

$\{\sigma\equalDot\tau\}\cup G\mapsto \red{\texttt{falla}}$, con $(\sigma,\tau)\in T\cup T^{-1}$ y $T =\{( Bool, Nat), (Nat, \sigma_1\to\sigma_2), ( Bool, \sigma_1\to\sigma_2\}$. Acá, la notación $T^{-1}$ se refiere al conjunto con cada tupla de $T$ invertida.

\item \textbf{Occur Check} Si $s\neq\sigma$ y $s\in FV(\sigma)$

$\{\text{s}\equalDot\sigma\}\cup G\mapsto \red{\texttt{falla}}$
\end{enumerate}

\section{Función $\WFunc$}
\subsubsection*{Constantes y variables}
\begin{equation*}
\begin{split}
\WFunc(\red{true}) &\equalDef \judgeType{\emptyset}{true}{Bool} \\
\WFunc(\red{false}) &\equalDef \judgeType{\emptyset}{false}{Bool} \\
\WFunc(\red{x}) &\equalDef \judgeType{\{x:s\}}{x}{s}, \text{ \textit{s} variable fresca } \\
\WFunc(\red{0}) &\equalDef \judgeType{\emptyset}{0}{Nat} \\
\end{split}
\end{equation*}
\begin{multicols}{2}

\subsubsection*{Caso \textit{succ}}
$\WFunc(\red{succ(U)}) \equalDef \judgeType{S\Gamma}{S~succ(M)}{Nat}$
\begin{centrado}
\begin{itemize}
\item $\WFunc(U) = \judgeType{\Gamma}{M}{\tau}$
\item $S = MGU\{\tau\equalDot Nat\}$
\end{itemize}
\end{centrado}

\subsubsection*{Caso \textit{pred}}
$\WFunc(\red{pred(U)}) \equalDef \judgeType{S\Gamma}{S~pred(M)}{Nat}$
\begin{centrado}
\begin{itemize}
\item $\WFunc(U) = \judgeType{\Gamma}{M}{\tau}$
\item $S = MGU\{\tau\equalDot Nat\}$
\end{itemize}
\end{centrado}

\end{multicols}

\subsubsection*{Caso \textit{isZero}}
$\WFunc(\red{isZero(U)}) \equalDef \judgeType{S\Gamma}{S~isZero(M)}{Bool}$
\begin{centrado}
\begin{itemize}
\item $\WFunc(U) = \judgeType{\Gamma}{M}{\tau}$
\item $S = MGU\{\tau\equalDot Nat\}$
\end{itemize}
\end{centrado}

\subsubsection*{Caso \textit{ifThenElse}}
$\WFunc(\red{\lambdaIf{U}{V}{W}}) \equalDef \judgeType{S\Gamma_1\cup S\Gamma_2\cup S\Gamma_3}{S~(\lambdaIf{M}{P}{Q})}{S\sigma}$
\begin{centrado}
\begin{itemize}
\item $\WFunc(U) = \judgeType{\Gamma_1}{M}{\rho}$
\item $\WFunc(V) = \judgeType{\Gamma_2}{P}{\sigma}$
\item $\WFunc(W) = \judgeType{\Gamma_3}{Q}{\tau}$
\item $S = MGU\{\sigma_1\equalDot \sigma_2~|~x:\sigma_1\in\Gamma_i~\land~x:\sigma_2\in\Gamma_j,~i\neq j\}\cup\{\sigma\equalDot\tau\,~\rho\equalDot Bool\}$
\end{itemize}
\end{centrado}

\subsubsection*{Caso aplicación}
$\WFunc(\red{U~V}) \equalDef \judgeType{S\Gamma_1\cup S\Gamma_2}{S~(M~N)}{St}$
\begin{centrado}
\begin{itemize}
\item $\WFunc(U) = \judgeType{\Gamma_1}{M}{\tau}$
\item $\WFunc(V) = \judgeType{\Gamma_2}{N}{\rho}$
\item $S = MGU\{\sigma_1\equalDot \sigma_2~|~x:\sigma_1\in\Gamma_i~\land~x:\sigma_2\in\Gamma_j,~i\neq j\}\cup\{\tau\equalDot\rho\to t\}$ con $t$ variable fresca
\end{itemize}
\end{centrado}

\subsubsection*{Caso abstracción}
Sea $\WFunc(U) = \judgeType{\Gamma}{M}{\rho}$, si $\Gamma$ tiene información de tipos para $x$, es decir $x:\tau\in\Gamma$ para algún $\tau$, entonces:

$$\WFunc(\red{\lambdaAbsI{x}{U}}) \equalDef \judgeType{\Gamma \backslash\{x:\tau\}}{\lambdaAbs{x}{\tau}{M}}{\tau\to\rho}$$

Si $\Gamma$ no tiene información de tipos para $x$ ($x\notin \text{Dom}(\Gamma)$), entonces elegimos una variable fresca $s$ y

$$\WFunc(\red{\lambdaAbsI{x}{U}}) \equalDef \judgeType{\Gamma}{\lambdaAbs{x}{s}{M}}{s\to\rho}$$

\subsubsection*{Caso \textit{fix}}
$\WFunc(\red{\lambdaFix{(U)}}) \equalDef \judgeType{S\Gamma}{S~\lambdaFix{(M)}}{St}$
\begin{centrado}
\begin{itemize}
\item $\WFunc(U) = \judgeType{\Gamma_1}{M}{\tau}$
\item $S = MGU\{\tau\equalDot t\to t\}$ con $t$ variable fresca
\end{itemize}
\end{centrado}

\newpage
\section{Subtipado}

\begin{align*}
\frac{}{Nat <: Float}(\text{S-NatFloat}) \hspace*{5mm}\frac{}{Int <: Float}(\text{S-IntFloat}) \hspace*{5mm}\frac{}{Bool <: Nat}(\text{S-BoolNat})
\end{align*}

$$\frac{\sigma' <: \sigma\hspace*{5mm}\tau <: \tau'}{\sigma\to\tau <: \sigma'\to\tau'}(\text{S-Func})$$

$$\frac{}{\sigma <: \sigma}(\text{S-Refl}) \hspace*{1cm}\frac{\sigma <:\tau\hspace*{5mm}\tau <: \rho}{\sigma <: \rho}(\text{S-Trans})$$

$$\frac{\sigma <: \tau\hspace*{5mm} \tau <: \sigma}{Ref~\tau <: Ref~\sigma}$$

\begin{align*}
\frac{\sigma <: \tau}{Source~\sigma <: Source~\tau}(\text{S-Source})\hspace*{1cm}\frac{\tau <: \sigma}{Sink~\sigma <: Sink~\tau}(\text{S-Sink})
\end{align*}

\begin{align*}
\frac{}{Ref~\tau <: Source~\tau}(\text{S-RefSource})\hspace*{1cm}\frac{}{Ref~\tau <: Sink~\tau}(\text{S-RefSink})
\end{align*}
\subsection{Reglas de reduccion con subtipado}

\begin{equation*}
\begin{gathered}
    \frac{x:\sigma\in\Gamma}{\judgeTypeS{\Gamma}{x}{\sigma}}(\text{T-Var})\hspace*{2cm}
\vspace*{5mm} \\
    \frac {\judgeTypeS{\Gamma,x:\sigma}{M}{\tau}}
          {\judgeTypeS{\Gamma}{\lambdaAbs{x}{\sigma}{M}}{\sigma\to\tau}}(\text{T-Abs})\hspace*{2cm}
    \frac{\judgeTypeS{\Gamma}{M}{\sigma\to\tau}\hspace*{5mm}\judgeTypeS{\Gamma}{N}{\blue{\rho}}\hspace*{5mm}\blue{\rho <: \sigma}}{\judgeTypeS{\Gamma}{M~N}{\tau}}(\text{T-App})
\end{gathered}
\end{equation*}

\newpage
\section{Objetos}
\subsubsection{Sintaxis}
\begin{tabular}{lllll}
	$a,b$ &$::=$& &$x$ & Variables \\
 	      &     & $|$ &$[\OOAtributo{l_i}{x_i}{b_i}^{i\in1..n}]$ &  Objetos\\
 	      &     & $|$ &$a.l$ &  Selección/ Envío de mensajes \\
 	      &     & $|$ &$\OORedefinicion{a.l}{x}{b}$ &  Redefinición de un método.	       	       	      
\end{tabular}

\subsection{Variables libres}
\begin{tabular}{ll}
	$\text{fv}(\varsigma(x)b)$ &$= \text{fv}(b)\backslash \{x\} $\\
	$\text{fv}(x)$ &$= \{x\} $\\
	$\text{fv}([\OOAtributo{l_i}{x_i}{b_i}^{i\in 1..n}])$ &$=  \bigcup^{1\in 1..n} \text{fv}(\varsigma(x)b)$\\
	$\text{fv}(a.l)$ &$= \text{fv}(a) $\\
	$\text{fv}(\OORedefinicion{a.l}{x}{b})$ &$= \text{fv}(a.l)\cup \text{fv}(\varsigma(x)b) $\\
\end{tabular}

\subsection{Sustitución}
\begin{center}
\begin{tabular}{lll}
	$x\{x \leftarrow c\}$ &$= c$ & \\
	$y\{x \leftarrow c\}$ &$= y$ & si $x\neq y$\\
	$([\OOAtributo{l_i}{x_i}{b_i}^{i\in 1..n}])\{x \leftarrow c\}$ &$=  [l_i = (\varsigma(x_i)b_i)\{x \leftarrow c\}^{i\in 1..n}]$ & \\
	$(a.l)\{x \leftarrow c\}$ &$= (a\{x \leftarrow c\}).l $ & \\
	$(\OORedefinicion{a.l}{x}{b})\{x \leftarrow c\}$ &$= (a\{x \leftarrow c\}).l \leftleftharpoons (\varsigma(x)b)\{x \leftarrow c\} $ & \\
	$(\varsigma(y)b)\{x \leftarrow c\}$ &$= (\varsigma(y')(b\{y \leftarrow y'\}\{x \leftarrow c\})) $ & si $y'\notin$fv$(\varsigma(y)b)\cup$fv$(c)\cup\{x\}$ \\
\end{tabular}
\end{center}

\subsection{Semantica operacional}
$$V~::=~[\OOAtributo{l_i}{x_i}{b_i}^{1\in 1..n}]$$

$$\frac{}{v\longrightarrow v}[\text{Obj}]$$
\vspace*{5mm}
$$\frac{a\longrightarrow v'\hspace*{5mm} v'\equiv [\OOAtributo{l_i}{x_i}{b_i}^{i\in 1..n}]\hspace*{5mm} b_j\{x_j\leftarrow v'\}\longrightarrow v\hspace*{5mm} j\in1..n}{a.l_j\longrightarrow v}[\text{Sel}]$$

\vspace*{5mm}
$$\frac{a\longrightarrow [\OOAtributo{l_i}{x_i}{b_i}^{i\in 1..n}]\hspace*{5mm} j\in1..n}{\OORedefinicion{a.l_j}{x}{b}\longrightarrow [\OOAtributo{l_j}{x}{b},~\OOAtributo{l_i}{x_i}{b_i}^{i\in 1..n-\{j\}}]}[\text{Upd}]$$

\paragraph{Indefinido: } $[\OOAtributo{a}{x}{x.a}].a$

\subsubsection{Codificacion de funciones}
\begin{align*}
\OORep{x} &\equalDef x\\
\OORep{M~N} &\equalDef \OORep{M}.arg :=~\OORep{N}\\
\OORep{\lambdaAbsI{x}{M}} &\equalDef 
[\OOAtributo{val}{y}{\OORep{M}\{x\leftarrow y.arg\}},~\OOAtributo{arg}{y}{y.arg}]\\
\end{align*}


\newpage
\section{Resolución}
\subsection{Lógica propocisional}
$$\frac{C_1 = \{A_1,\dots,A_m,L\}\hspace*{5mm} C_2 = \{B_1,\dots,B_m, \overline{L}\}}{C = \{A_1,\dots,A_m, B_1,\dots,B_n\}}$$

\subsection{Lógica de primer orden}
Transformar la formula:
\begin{enumerate}
\item Eliminar las implicaciones, es decir, si aparece una clausula de la forma $(A\supset B)$, reescribirla como $(\lnot A \lor B)$.
\item Pasar a \textbf{forma normal negada}.
\item Pasar a \textbf{forma normal prenexa}.
\item Pasar a \textbf{forma normal de Skolem}.
\item Pasar a \textbf{forma normal conjuntiva}.
\item \textbf{Distribuir} cuantificadores universales.
\end{enumerate}

\subsubsection{Skolemización}

Sea $A$ una sentencia rectificada en forma normal negada, la \textbf{forma normal de Skolem de A} (\textbf{SK(A)}) se define recursivamente como sigue:

Sea $A'$ cualquier subfórmula de $A$, 
\begin{itemize}
\item Si $A'$ es una fórmula atómica o su negación, \textbf{SK}$(A') = A'$.
\item Si $A'$ es de la forma $(B\star C)$ con $\star \in \{\land,\lor\}$, entonces $\textbf{SK}(A') = (\textbf{SK}(B)\star \textbf{SK}(C))$.
\item Si $A'$ es de la forma $\forall x.B$, entonces $\textbf{SK}(A') = \forall x.\textbf{SK}(B)$.
\item Si $A'$ es de la forma $\exists x.B$ y $\{x,y_1,\dots,y_m\}$ son las variables libres de $B$, entonces:
\begin{enumerate}
\item Si $m>0$, crear un \textbf{símbolo de función de Skolem}, $f_x$ de aridad $m$ y definir:

$$\textbf{SK}(A') = \textbf{SK}(B\{ x \leftarrow f(y_1,\dots,y_m)\})$$

\item Si $m=0$, crear una nueva \textbf{constante de Skolem} $c_x$ y

$$\textbf{SK}(A') = \textbf{SK}(B\{ x \leftarrow c_x\})$$

\end{enumerate}
\end{itemize}

\subsection{Reglas de resolucion de primer orden}
$$\frac{\{\blue{B_1,\dots,B_k},A_1,\dots,A_n\}\hspace*{5mm}\{\blue{\lnot D_1,\dots,\lnot D_k},A_1,\dots,A_n\}}{\red{\sigma(\{A_1,\dots A_m, C_1,\dots,C_n\})}}$$

donde $\sigma$ es el \textbf{unificador más general} (MGU) de $\{B_1,\dots,B_k,\lnot D_1,\dots,\lnot D_k\}$ y \\ $\sigma(\{A_1,\dots A_m, C_1,\dots,C_n\})$ es el \textbf{resolvente}.

\subsection{Regla de resolucion binaria y factorizacion}
$$\frac{\{\blue{B_1},A_1,\dots,A_n\}\hspace*{5mm}\{\blue{\lnot D_1},A_1,\dots,A_n\}}{\red{\sigma(\{A_1,\dots A_m, C_1,\dots,C_n\})}}$$

$$\frac{\{\blue{B_1,\dots,B_k},A_1,\dots,A_n\}}{\red{\sigma(\{B_1,A_1,\dots A_m\})}}$$

\section{Prolog predicados}
\paragraph{Predicados:} 
=, sort, msort, length, nth1, nth0, member, append, last, between, is\_list, list\_to\_set, is\_set, union, intersection, subset, subtract, select, delete, reverse,  atom,  number, numlist, sumlist, flatten, help

\paragraph{Operaciones extra-lógicas}: is, $\backslash =$,  $==$,  $=:=$, $=\backslash=$, $>$, $<$, $=<$, $>=$, abs, max, min, gcd,  var,  nonvar, ground, trace, notrace

\paragraph{Metapredicados:} bagof, setof, maplist, include, not, forall, assert, retract, listing
\end{document}
