
\documentclass[10pt,a4paper]{article}

\usepackage[pdftex,
pdfauthor={Gianfranco Zamboni},
pdftitle={Resumen: Paradigmas de Lenguajes de Programación},
pdfsubject={},
pdfkeywords={Resumen , Computacion, FCEyN, UBA, Paradigmas de Lenguajes de Programación, Imperativo, Funcional, Cálculo Lambda, Programación Orientada a Objetos, Objetos, Programación Lógica},
pdfproducer={Latex with hyperref},
pdfcreator={pdflatex}]{hyperref}

\usepackage{amsmath}
\usepackage{ amssymb }
\usepackage{bussproofs}

\usepackage[spanish]{babel}


\usepackage[utf8]{inputenc} % para poder usar tildes en archivos UTF-8
\usepackage{graphicx}
\usepackage{xcolor}
\usepackage{pifont}

\usepackage{lscape}
\usepackage{minted}
\usepackage{a4wide} % márgenes un poco más anchos que lo usual
\usepackage[titletoc,toc,page]{appendix}
\usepackage{tikz}
\usepackage{forest}
\usepackage{multicol}

\setlength{\columnsep}{1cm}

\ifthenelse{\paperwidth < \paperheight}{\usepackage{fancyhdr}
\pagestyle{fancy}

%\renewcommand{\chaptermark}[1]{\markboth{#1}{}}
\renewcommand{\sectionmark}[1]{\markright{\thesection\ - #1}}

\fancyhf{}

\fancyhead[LO]{Sección \rightmark} % \thesection\ 
\fancyfoot[LO]{\small{Paradigmas de lenguajes de programación}}
\fancyfoot[RO]{\thepage}
\renewcommand{\headrulewidth}{0.5pt}
\renewcommand{\footrulewidth}{0.5pt}
\setlength{\hoffset}{-0.25in}
\setlength{\textwidth}{16cm}
%\setlength{\hoffset}{-1.1cm}
%\setlength{\textwidth}{16cm}
\setlength{\headsep}{0.5cm}
\setlength{\textheight}{25cm}
\setlength{\voffset}{-0.4in}
\setlength{\headwidth}{\textwidth}
\setlength{\headheight}{13.1pt}

\renewcommand{\baselinestretch}{1.1}  % line spacing}{\usepackage{fancyhdr}
\pagestyle{fancy}

%\renewcommand{\chaptermark}[1]{\markboth{#1}{}}
\renewcommand{\sectionmark}[1]{\markright{\thesection\ - #1}}

\fancyhf{}

\fancyhead[LO]{\rightmark} % \thesection\ 
\fancyfoot[LO]{\small{PLP - Prácticas}}
\fancyfoot[RO]{\thepage}
\renewcommand{\headrulewidth}{0.5pt}
\renewcommand{\footrulewidth}{0.5pt}
\setlength{\hoffset}{-0.25in}
\setlength{\textwidth}{25cm}
%\setlength{\hoffset}{-1.1cm}
%\setlength{\textwidth}{16cm}
\setlength{\headsep}{0.5cm}
\setlength{\textheight}{16cm}
\setlength{\voffset}{-0.4in}
\setlength{\headwidth}{\textwidth}
\setlength{\headheight}{13.1pt}

\renewcommand{\baselinestretch}{1.1}  % line spacing
}





\newenvironment{centrado}
    {
     \begin{center}
     \begin{minipage}{0.8\textwidth}
 }    
    {
     \end{minipage}
     \end{center}
    }

\newcommand{\rel}{\ensuremath{\mathcal{R}}}

\newcommand{\equalDef}{\overset{def}{=}}
\newcommand{\equalDot}{\overset{\cdot}{=}}

\newcommand{\lambdaAbs}[3]{\lambda #1: #2 . #3}
\newcommand{\lambdaAssign}[2]{#1~:=~#2}
\newcommand{\lambdaApp}[2]{#1~#2}
\newcommand{\lambdaIf}[3]{if~ #1~ then~ #2~ else~ #3}
\newcommand{\lambdaTrue}{true}
\newcommand{\lambdaFalse}{false}
\newcommand{\lambdaLet}[4]{let~#1:#2 = #3~in~#4}
\newcommand{\lambdaRef}[1]{ref~#1}
\newcommand{\lambdaVar}[1]{#1}
\newcommand{\lambdaValue}[1]{\color{red}#1\color{black}}
\newcommand{\lambdaFix}[1]{fix~#1}

\newcommand{\lambdaAbsI}[2]{\lambda #1. #2}



\newcommand{\blue}[1]{\color{blue}#1\color{black}}
\newcommand{\replaceBy}[3]{#1\{#2\leftarrow#3\}}

\newcommand{\judgeType}[3]{#1\triangleright #2 : #3}


\newenvironment{scprooftree}[1]%
{\gdef\scalefactor{#1}\begin{center}\proofSkipAmount \leavevmode}%
    {\scalebox{\scalefactor}{\DisplayProof}\proofSkipAmount \end{center} }


\tikzset{
    every leaf node/.style={text=red, align=center},
    every tree node/.style={text=blue, align=center},
}

\forestset{tikzQtree/.style={for tree={if n children=0{
                node options=every leaf node/.try}{node options=every tree node/.try}, text centered}}}
                
                
\DeclareMathOperator{\Erase}{Erase}
\DeclareMathOperator{\Nat}{Nat}
\DeclareMathOperator{\Bool}{Bool}
\DeclareMathOperator{\Union}{Union}

\newcommand{\WFunc}{\mathbb{W}}

\newcommand{\red}[1]{{\color{red}#1}}%\renewcommand{\appendixtocname}{Apéndices}
\newcommand{\green}[1]{{\color{green!40!black}#1}}%\renewcommand{\appendixpagename}{Apéndices}





\setcounter{section}{6}


\begin{document}
  \title{PLP - Práctica 7: Programación Lógica}

  \date{\today}

  \author{Zamboni, Gianfranco}

  \maketitle
  \setcounter{page}{1}


\section*{\ El motor de búsqueda de prolog}

\subsection{Ejercicio 1}
Base de conocimiento:
\begin{multicols}{2}
    \begin{itemize}
        \item padre(juan, carlos).
        \item padre(juan, luis).
        \item padre(carlos, daniel).
        \item padre(carlos, diego).
	    	\columnbreak
        \item padre(luis, pablo).
        \item padre(luis, manuel).
        \item padre(luis, ramiro).
        \item abuelo(X,Y) :- padre(X,Z), padre(Z,Y).
    \end{itemize}
\end{multicols}
Respuestas:
\begin{enumerate}
\item abuelo(X, manuel): padre(X,Z), padre(Z,manuel) \\
Juan
\item Definición:
hijo(X,Y) :- padre(Y,X). \\
hermano(X,Y) :- padre(Z,X), padre(Z,Y). \\
descendiente(X,Y) :- padre(Y,X). \\
descendiente(X,Y) :- padre(X,Z), descendiente(Z,Y).
\item \blue{árbol}
\item abuelo(Juan,X).
\item hermano(Pablo,X).
\item Regla: \\
ancestro(X, X). \\
ancestro(X, Y) :- ancestro(Z, Y), padre(X, Z).
\item el problema es que se genera una consulta que no termina, porque todo el tiempo el árbol se abre por ancestro().
\item Solucion: \\
ancestro(X, X). \\
ancestro(X, Y) :- padre(X, Z), ancestro(Z, Y).
\end{enumerate}

\subsection{Ejercicio 2}
Base de conocimiento:
\begin{itemize}
\item vecino(X, Y, [X|[Y|Ls]]).
\item vecino(X, Y, [W|Ls]) :- vecino(X, Y, Ls).
\end{itemize}
Respuestas: vecino(5, Y, [5,6,5,3]).
\begin{enumerate}
\item Entra por la primera regla e instancia y = 6, luego entra por la segunda regla y saca el 5 y el 6, vuelve a usar la primera regla e instancia y = 3, y luego da false.
\item Instancia en orden opuesto, ya que primero se resuelven las consultas mas profundas. y da Y=3, Y=6.
\end{enumerate}

\subsection{Ejercicio 3}
    Base de conocimiento:
    \begin{itemize}
        \item natural(0).
        \item natural(suc(X)) :- natural(X).
        \item menorOIgual(X, suc(Y)) :- menorOIgual(X, Y).
        \item menorOIgual(X,X) :- natural(X).
    \end{itemize}
    
    Respuestas:
    \begin{enumerate}
        \item menorOIgual(0,X1): prolog trata de unificar con la regla menorOIgual(X, suc(Y)) :- menorOIgual(X, Y), instancia a X1 como suc(Y) y consulta menorOIgual(0, Y), y así infinitas veces.
        \item Prolog puede generar un ciclo infinito cuando no tiene bien definido el caso base, hay que chequearlo antes de el caso "recursivo".
    \end{enumerate}
    
    Redefinición: cambiar el orden de las reglas.
    \begin{itemize}
        \item menorOIgual(X,X) :- natural(X).
        \item menorOIgual(X, suc(Y)) :- menorOIgual(X, Y).
    \end{itemize}

\section*{\ Estructuras, instanciación y reversibilidad}

\subsection{Ejercicio 4}
\begin{itemize}
%concatenar(?Lista1,?Lista2,?Lista3),
\item concatenar([],L1,C):-C=L1.
\item concatenar([E|T],L1,C):- concatenar(T,L1,C2),append([[E],C2],C).
\end{itemize}

\subsection{Ejercicio 5}
\begin{itemize}
\item last(?L, ?U). \\
last([],[]). \\
last([E],E). \\
last([E|T],E2):-last(T,E2).
\item reverse(+L, -L1). \\
reverse([],[]). \\
reverse([E|T],L):- reverse(T,L2),append([L2,[E]],L).
\item maxlista(+L, -M). \\
maxlista([E],E). \\
maxlista([E|T],E):-maxlista(T,M),E>=M. \\
maxlista([E|T],M):-maxlista(T,M),E=<M.
\item maxlista(+L, -M).\\
minlista([E],E). \\
minlista([E|T],M):-minlista(T,M),E>=M. \\
minlista([E|T],E):-minlista(T,M),E=<M.
\item prefijo(?P, +L). \\
prefijo([], _). \\
prefijo([E|T2], [E|T]):-prefijo(T2,T).
\item sufijo(?S, +L). \\
sufijo(S, L):-reverse(S,S2),reverse(L,L2),prefijo(S2,L2).
\item sublista(?S, +L). \\
sublistas(Sub,L) :- \\
length(Sub,Largo), \\
append(Aux,Der,L), \\
append(Izq,Sub,Aux), \\
Largo >= 1.
\item pertenece(?X, +L). \\
pertenece(X, [X|_]). \\
pertenece(X, [_|T]):-pertenece(X,T).
\end{itemize}

\subsection{Ejercicio 6}
aplanar([],[]). \\
aplanar([[X1|T1]|T2], Y):- aplanar([X1|T1],Y1),aplanar(T2,Y2),!, append([Y1,Y2],Y). \\
aplanar([X|T], Y):- aplanar(T,Y2),append([[X],Y2],Y).

\subsection{Ejercicio 7}
\begin{itemize}
\item 
%palindromo(+L, -L1)
palindromo([], []). \\
palindromo([X|T], P):- palindromo(T, P1), append([[X],P1,[X]],P).
\item 
%doble(?L, ?L1)
doble([],[]). \\
doble([X|T], L):- doble(T, L1), append([[X,X],L1],L).
\item 
%iesimo(?I, +L, -X)
iesimo(1, [X|_], X). \\
iesimo(N, [E|T], X):- length([E|T],L), L>=N, iesimo(N1, T, X), N1 is N-1.
\end{itemize}

\subsection{Ejercicio 8}
\begin{itemize}
\item desde2(X,Y):- nonvar(Y), X=<Y. \\
desde2(X,Y) :- var(Y), desde(X,Y).
\item desde(X,X). \\
desde(X,Y) :- N is X+1, desde(N,Y).
\end{itemize}

\subsection{Ejercicio 9}
\begin{itemize}
\item difSimetrica(L1,L2,L):- append([L1,L2],L3), delete_rep(L3,L).
\item delete_rep([],[]). \\
delete_rep([X|XS],L):- member(X,XS), !, delete(XS,X,YS), delete_rep(YS,L). \\
delete_rep([X|XS],[X|L]):- delete_rep(XS,L).
\item sinRepetidos([],[]). \\
sinRepetidos([X|L3],L33):-member(X,L3),!,sinRepetidos(L3,L33). \\
sinRepetidos([X|L3],[X|L33]):-sinRepetidos(L3,L33).
\item interseccion(+L1, +L2, -L3). \\
interseccion(L1,L2,L33):-interseccionAux(L1,L2,L3), sinRepetidos(L3,L33).
\item interseccionAux([],_,[]). \\
interseccionAux([X|L1],L2,[X|L3]):-member(X,L2),!,interseccionAux(L1,L2,L3). \\
interseccionAux([_|L1],L2,L3):-interseccionAux(L1,L2,L3).
\item split(N, L, L1, L2). \\
split(0, L, [], L). \\
split(N, [X|L], [X|L1], L2) :- N2 is N-1, length(L,Largo), N2=<Largo, split(N2, L, L1, L2).
\item borrar(+ListaOriginal, +X, -ListaSinXs). \\
borrar([], _, []). \\
borrar([X|L], E, LsinX):- X==E,!,borrar(L, E, LsinX). \\
borrar([X|L], E, [X|LsinX]):- borrar(L, E, LsinX).
\item sacarDuplicados(+L1, -L2). \\
sacarDuplicados(L1, L2):-setof(X,member(X,L1),L2).
\item reparto(+L, +N, -LListas). \\
repartoSinVaciasAux?
\item repartoSinVacias(+L, -LListas). \\
desde(X,X). \\
desde(X,Y) :- N is X+1, desde(N,Y).
\item repartoSinVacias(L, LListas):-desde(1,S),repartoSinVaciasAux(L, S, LListas). \\
repartoSinVaciasAux(_, 0, []). \\
repartoSinVaciasAux(L, N, [X|LListas]):- sublista(L,X), length([X|LListas],N), N2 is N-1, repartoSinVaciasAux(L, N2, LListas).
\item sublista(L,X):- sublistaAux(L,Sub), sacarDuplicados(Sub, X). \\
sublistaAux(L,Sub):-append(_,Medio,L),append(Sub, _,Medio).
\end{itemize}

\subsection{Ejercicio 10}
\begin{itemize}
\item
\end{itemize}

\subsection{Ejercicio 11}
\begin{itemize}
\item
\end{itemize}

\subsection{Ejercicio 12}
\begin{itemize}
\item inorder(nil,[]). \\
inorder(bin(Izq, V, Der), XS):-inorder(Izq, L1),inorder(Der,L2),append([L1,[V],L2],XS).
\item arbolConInorder([],nil). \\
arbolConInorder(XS,bin(Izq,X, Der)):-append(L1,[X|L2],XS),arbolConInorder(L1,Izq),arbolConInorder(L2,Der).
\item 
\item 
\end{itemize}
\section*{\ Generate and test}

\subsection{Ejercicio 13}
\begin{itemize}
\item coprimos(X,Y):- desde(2,X), between(2,X,Y), 1 is gcd(X,Y).
\end{itemize}

\subsection{Ejercicio 14}
\begin{itemize}
\item
lista_que_suman([],0,0). \\
lista_que_suman([X|XS],S,N):- \\
    between(0,S,X),  \\
    N2 is N-1, \\
    length(XS,N2),  \\
    S2 is S-X,  \\
    lista_que_suman(XS,S2,N2). \\
cuadradoSemiLatino(N,XS):-length(XS,N),desde(0,S),cuadradoSemiLatinoAux(N,N,XS,S). \\
cuadradoSemiLatinoAux(_,0,[],_). \\
cuadradoSemiLatinoAux(M,N,[X|XS],S):- N2 is N-1, lista_que_suman(X,S,M), cuadradoSemiLatinoAux(M,N2,XS,S).
\end{itemize}

\subsection{Ejercicio 15}
\begin{itemize}
\item
\end{itemize}
\section*{\ Negación por falla}

\subsection{Ejercicio 16}
\begin{itemize}
\item
\end{itemize}

\subsection{Ejercicio 17}
\begin{itemize}
\item
\end{itemize}

\subsection{Ejercicio 18}
\begin{itemize}
\item
\end{itemize}

\subsection{Ejercicio 19}
\begin{itemize}
\item
\end{itemize}


\subsection{Ejercicio 20}
\begin{itemize}
\item
\end{itemize}

\subsection{Ejercicio 21}
\begin{itemize}
\item
\end{itemize}

\subsection{Ejercicio 22}
\begin{itemize}
\item
\end{itemize}

\end{document}
