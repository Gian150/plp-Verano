
\usepackage[pdftex,
pdfauthor={Gianfranco Zamboni},
pdftitle={Resumen: Paradigmas de Lenguajes de Programación},
pdfsubject={},
pdfkeywords={Resumen , Computacion, FCEyN, UBA, Paradigmas de Lenguajes de Programación, Imperativo, Funcional, Cálculo Lambda, Programación Orientada a Objetos, Objetos, Programación Lógica},
pdfproducer={Latex with hyperref},
pdfcreator={pdflatex}]{hyperref}

\usepackage{amsmath}
\usepackage{ amssymb }
\usepackage{bussproofs}

\usepackage[spanish]{babel}


\usepackage[utf8]{inputenc} % para poder usar tildes en archivos UTF-8
\usepackage{graphicx}
\usepackage{xcolor}
\usepackage{pifont}

\usepackage{lscape}
\usepackage{minted}
\usepackage{a4wide} % márgenes un poco más anchos que lo usual
\usepackage[titletoc,toc,page]{appendix}
\usepackage{tikz}
\usepackage{forest}
\usepackage{multicol}


\usemintedstyle{borland}    

\setlength{\columnsep}{5mm}

\ifthenelse{\paperwidth < \paperheight}{\usepackage{fancyhdr}
\pagestyle{fancy}

%\renewcommand{\chaptermark}[1]{\markboth{#1}{}}
\renewcommand{\sectionmark}[1]{\markright{\thesection\ - #1}}

\fancyhf{}

\fancyhead[LO]{Sección \rightmark} % \thesection\ 
\fancyfoot[LO]{\small{Paradigmas de lenguajes de programación}}
\fancyfoot[RO]{\thepage}
\renewcommand{\headrulewidth}{0.5pt}
\renewcommand{\footrulewidth}{0.5pt}
\setlength{\hoffset}{-0.25in}
\setlength{\textwidth}{16cm}
%\setlength{\hoffset}{-1.1cm}
%\setlength{\textwidth}{16cm}
\setlength{\headsep}{0.5cm}
\setlength{\textheight}{25cm}
\setlength{\voffset}{-0.4in}
\setlength{\headwidth}{\textwidth}
\setlength{\headheight}{13.1pt}

\renewcommand{\baselinestretch}{1.1}  % line spacing}{\usepackage{fancyhdr}
\pagestyle{fancy}

%\renewcommand{\chaptermark}[1]{\markboth{#1}{}}
\renewcommand{\sectionmark}[1]{\markright{\thesection\ - #1}}

\fancyhf{}

\fancyhead[LO]{\rightmark} % \thesection\ 
\fancyfoot[LO]{\small{PLP - Prácticas}}
\fancyfoot[RO]{\thepage}
\renewcommand{\headrulewidth}{0.5pt}
\renewcommand{\footrulewidth}{0.5pt}
\setlength{\hoffset}{-0.25in}
\setlength{\textwidth}{25cm}
%\setlength{\hoffset}{-1.1cm}
%\setlength{\textwidth}{16cm}
\setlength{\headsep}{0.5cm}
\setlength{\textheight}{16cm}
\setlength{\voffset}{-0.4in}
\setlength{\headwidth}{\textwidth}
\setlength{\headheight}{13.1pt}

\renewcommand{\baselinestretch}{1.1}  % line spacing
}

\newenvironment{centrado}
    {
     \begin{center}
     \begin{minipage}{0.8\textwidth}
 }    
    {
     \end{minipage}
     \end{center}
    }

\newcommand{\rel}{\ensuremath{\mathcal{R}}}

\newcommand{\equalDef}{\overset{def}{=}}
\newcommand{\equalDot}{\overset{\cdot}{=}}
\newcommand{\alphaEq}[1]{\underset{#1}{\overset{\alpha-eq}{\leadsto}}}

\newcommand{\goesTo}[2]{\underset{#2}{\overset{#1}{\leadsto}}}
\newcommand{\reduceTo}[1]{\underset{#1}{\to}}


\DeclareMathOperator{\Erase}{Erase}
\DeclareMathOperator{\Nat}{Nat}
\DeclareMathOperator{\Bool}{Bool}
\DeclareMathOperator{\Union}{Union}

\newcommand{\WFunc}{\mathbb{W}}

\newcommand{\List}[1]{[~]_{#1}}

\newcommand{\lambdaAbs}[3]{\lambda #1: #2 \text{.} #3}
\newcommand{\lambdaAssign}[2]{#1~:=~#2}
\newcommand{\lambdaApp}[2]{#1~#2}
\newcommand{\lambdaIf}[3]{\text{if } #1 \text{ then } #2 \text{ else } #3}
\newcommand{\lambdaTrue}{true}
\newcommand{\lambdaFalse}{false}
\newcommand{\lambdaLet}[4]{\text{let}~#1:#2 = #3~in~#4}
\newcommand{\lambdaRef}[1]{\text{ref}~#1}
\newcommand{\lambdaVar}[1]{#1}
\newcommand{\lambdaValue}[1]{{\color{red}#1}}
\newcommand{\lambdaFix}[1]{\text{fix } #1}


\newcommand{\lambdaAbsI}[2]{\lambda #1\text{.} #2}
\newcommand{\lambdaLetI}[3]{let~#1 = #2~in~#3}

\newcommand{\lambdaLetrec}[3]{\text{letrec } #1 = #2 \text{ in }~#3}

\newcommand{\lambdaListCase}[3]{\text{case } #1 \text{ of } \{[~]\leadsto#2~|~h::t\leadsto#3\}}
\newcommand{\lambdaListFold}[3]{\text{foldr } #1 \text{ base }\leadsto#2;~\text{rec}(h, r)\leadsto#3}

\newcommand{\blue}[1]{{\color{blue}#1}}
\newcommand{\red}[1]{{\color{red}#1}}
\newcommand{\green}[1]{{\color{green}#1}}

\newcommand{\falla}{\red{\texttt{falla}}}

\newcommand{\replaceBy}[3]{#1\{#2\leftarrow#3\}}
\newcommand{\multiReplaceBy}[2]{#1\{#2\}}

\newcommand{\judgeType}[3]{#1\triangleright #2 : #3}
\newcommand{\judgeTypeS}[3]{#1\mapsto #2 : #3}

\newcommand{\xmark}{\ding{55}}%

\newenvironment{scprooftree}
{\leavevmode\hbox\bgroup}
{\DisplayProof\egroup}

\tikzset{
    every tree node/.style={align=center},
}

\forestset{tikzQtree/.style={
        for tree={node options=every tree node/.try}, text centered},
        begin draw/.code={\begin{tikzpicture}[baseline=(current bounding box.center)]},
        }


%\newenvironment{scforestset}[1]%
%{\gdef\scalefactor{#1}\proofSkipAmount \leavevmode\hbox\bgroup}%
%    {\scalebox{\scalefactor}{\DisplayProof}\egroup\proofSkipAmount }

%\renewcommand{\appendixtocname}{Apéndices}
%\renewcommand{\appendixpagename}{Apéndices}

