\documentclass[10pt,a4paper]{article}

\usepackage[pdftex,
pdfauthor={Gianfranco Zamboni},
pdftitle={Resumen: Paradigmas de Lenguajes de Programación},
pdfsubject={},
pdfkeywords={Resumen , Computacion, FCEyN, UBA, Paradigmas de Lenguajes de Programación, Imperativo, Funcional, Cálculo Lambda, Programación Orientada a Objetos, Objetos, Programación Lógica},
pdfproducer={Latex with hyperref},
pdfcreator={pdflatex}]{hyperref}

\usepackage{amsmath}
\usepackage{ amssymb }
\usepackage{bussproofs}

\usepackage[spanish]{babel}


\usepackage[utf8]{inputenc} % para poder usar tildes en archivos UTF-8
\usepackage{graphicx}
\usepackage{xcolor}

\usepackage{lscape}
\usepackage{minted}
\usepackage{a4wide} % márgenes un poco más anchos que lo usual
\usepackage[titletoc,toc,page]{appendix}
\usepackage{tikz}
\usepackage{forest}

\usepackage{fancyhdr}
\pagestyle{fancy}

%\renewcommand{\chaptermark}[1]{\markboth{#1}{}}
\renewcommand{\sectionmark}[1]{\markright{\thesection\ - #1}}

\fancyhf{}

\fancyhead[LO]{Sección \rightmark} % \thesection\ 
\fancyfoot[LO]{\small{Paradigmas de lenguajes de programación}}
\fancyfoot[RO]{\thepage}
\renewcommand{\headrulewidth}{0.5pt}
\renewcommand{\footrulewidth}{0.5pt}
\setlength{\hoffset}{-0.25in}
\setlength{\textwidth}{16cm}
%\setlength{\hoffset}{-1.1cm}
%\setlength{\textwidth}{16cm}
\setlength{\headsep}{0.5cm}
\setlength{\textheight}{25cm}
\setlength{\voffset}{-0.4in}
\setlength{\headwidth}{\textwidth}
\setlength{\headheight}{13.1pt}

\renewcommand{\baselinestretch}{1.1}  % line spacing
\newenvironment{centrado}
    {
     \begin{center}
     \begin{minipage}{0.8\textwidth}
 }    
    {
     \end{minipage}
     \end{center}
    }

\newcommand{\rel}{\ensuremath{\mathcal{R}}}

\newcommand{\equalDef}{\overset{def}{=}}
\newcommand{\equalDot}{\overset{\cdot}{=}}

\newcommand{\lambdaAbs}[3]{\lambda #1: #2 . #3}
\newcommand{\lambdaAssign}[2]{#1~:=~#2}
\newcommand{\lambdaApp}[2]{#1~#2}
\newcommand{\lambdaIf}[3]{if~ #1~ then~ #2~ else~ #3}
\newcommand{\lambdaTrue}{true}
\newcommand{\lambdaFalse}{false}
\newcommand{\lambdaLet}[4]{let~#1:#2 = #3~in~#4}
\newcommand{\lambdaRef}[1]{ref~#1}
\newcommand{\lambdaVar}[1]{#1}
\newcommand{\lambdaValue}[1]{\color{red}#1\color{black}}
\newcommand{\lambdaFix}[1]{fix~#1}

\newcommand{\lambdaAbsI}[2]{\lambda #1. #2}



\newcommand{\blue}[1]{\color{blue}#1\color{black}}
\newcommand{\replaceBy}[3]{#1\{#2\leftarrow#3\}}

\newcommand{\judgeType}[3]{#1\triangleright #2 : #3}


\newenvironment{scprooftree}[1]%
{\gdef\scalefactor{#1}\begin{center}\proofSkipAmount \leavevmode}%
    {\scalebox{\scalefactor}{\DisplayProof}\proofSkipAmount \end{center} }


\tikzset{
    every leaf node/.style={text=red, align=center},
    every tree node/.style={text=blue, align=center},
}

\forestset{tikzQtree/.style={for tree={if n children=0{
                node options=every leaf node/.try}{node options=every tree node/.try}, text centered}}}
                
                
\DeclareMathOperator{\Erase}{Erase}
\DeclareMathOperator{\Nat}{Nat}
\DeclareMathOperator{\Bool}{Bool}
\DeclareMathOperator{\Union}{Union}

\newcommand{\WFunc}{\mathbb{W}}

\newcommand{\red}[1]{{\color{red}#1}}%\renewcommand{\appendixtocname}{Apéndices}
\newcommand{\green}[1]{{\color{green!40!black}#1}}%\renewcommand{\appendixpagename}{Apéndices}




\setcounter{section}{0}

\begin{document}

\title{PLP - Práctica 0: Pre-Práctica de Programación Funcional}

\date{\today}

\author{Zamboni, Gianfranco}

\maketitle
\setcounter{page}{1}

\subsection{Ejercicio 1}
\mintinline{haskell}{null :: Foldable t => t a -> Bool} indica si una estructura está vacía. El tipo a debe ser de la clase Foldable, esto es, son tipos a los que se les puede aplicar la función foldr. La notación "t a" indica que es un tipo parámetrico, es decir, un tipo t que usa a otro tipo a, por ejemplo, si le pasamos a la función una lista de enteros, entonces a = Int y t = [Int]

\vspace*{5mm}
\mintinline{haskell}{head :: [a] -> a} devuelve el primer elemento de una lista.

\vspace*{5mm}
\mintinline{haskell}{tail :: [a] -> [a]} devuelve los últimos elementos de una lista (todos los elementos, salvo el primero).

\vspace*{5mm}
\mintinline{haskell}{init :: [a] -> [a]} devuelve los primeros elementos de una lista (todos los elementos salvo el último).

\vspace*{5mm}
\mintinline{haskell}{last :: [a] -> a} devuelve el último elemento de una lista.


\vspace*{5mm}
\mintinline{haskell}{take :: Int -> [a] -> [a]} devuelve los primeros n elementos de una lista

\vspace*{5mm}
\mintinline{haskell}{drop :: Int -> [a] -> [a]} devuelve los últimos n elementos de una lista

\vspace*{5mm}
\mintinline{haskell}{(++) :: [a] -> [a] -> [a]} concatena dos listas

\vspace*{5mm}
\mintinline{haskell}{concat :: Foldable t => t [a] -> [a]} concatena todas las listas de un contenedor de listas que soporte la operación foldr.

\vspace*{5mm}
\mintinline{haskell}{(!!) :: [a] -> Int -> a} devuelve el elemento de una lista 
\mintinline{haskell}{l} que se encuentra en la \mintinline{haskell}{n}-ésima posición. La numeración comienza desde 0.

\vspace*{5mm}
\mintinline{haskell}{elem :: (Eq a, Foldable t) => a -> t a -> Bool}: Dada una estructura T que soporta la operación foldr y que almacene elementos del tipo a que puedan ser comparados por medio de la igualdad y dado un elemento A de ese tipo, indica si A aparecen en T.

\subsection{Ejercicio 2}
\begin{centrado}
\begin{minted}[breaklines]{haskell}
-- a) La función abs de Prelude ya hace esto
valorAbsoluto :: Float -> Float
valorAbsoluto x | x < 0     = -x
                | otherwise =  x
\end{minted}
\end{centrado}
\begin{centrado}
	\begin{minted}[breaklines]{haskell}
-- b) 
bisiesto :: Int -> Bool
bisiesto x = (x `mod` 4) == 0
	\end{minted}
\end{centrado}
\begin{centrado}
	\begin{minted}[breaklines]{haskell}
--c)
factorial :: Int -> Int
factorial 1 = 1
factorial x = x * factorial (x-1)
	\end{minted}
\end{centrado}
\begin{centrado}
	\begin{minted}[breaklines]{haskell}
cantDivisoresPrimos :: Int -> Int
cantDivisoresPrimos x = length (filter esPrimo (divisores x))
	\end{minted}
\end{centrado}

\begin{centrado}
	\begin{minted}[breaklines]{haskell}
-- Auxiliares 

esPrimo :: Int -> Bool
esPrimo x = length (divisores x) == 2

divisores :: Int -> [Int]
divisores x = [ y | y <- [1..x], x `mod` y == 0 ];
	\end{minted}
\end{centrado}

\subsection{Ejercicio 3}
\begin{centrado}
\begin{minted}[breaklines]{haskell}
--a)
inverso :: Float -> Maybe Float
inverso 0 = Nothing
inverso x = Just (1/x)
\end{minted}
\end{centrado}

\begin{centrado}
\begin{minted}[breaklines]{haskell}
-- b)
aEntero :: Either Int Bool -> Int
aEntero (Left x) = x
aEntero (Right x) | x == True = 1
                  | otherwise = 0
\end{minted}
\end{centrado}

\subsection{Ejercicio 4}
\begin{centrado}
\begin{minted}[breaklines]{haskell}
--a)
limpiar :: String -> String -> String
limpiar xs ys = [ y | y <- ys, not(elem y xs) ]
\end{minted}
\end{centrado}
\begin{centrado}
	\begin{minted}[breaklines]{haskell}
-- b)
difPromedio :: [Float] -> [Float]
difPromedio xs = map (\y -> y - promedio xs) xs 
    where promedio xs = (sum xs) / (genericLength xs)
\end{minted}
\end{centrado}
\begin{centrado}
	\begin{minted}[breaklines]{haskell}
-- c)
todosIguales :: [Int] -> Bool
todosIguales = 
	foldr (\y  rec -> ((length xs == 1) || (y == (head xs))) 
		&& rec) True
\end{minted}
\end{centrado}

\subsection{Ejercicio 5}
\begin{centrado}
\begin{minted}[breaklines, tabsize=4]{haskell}
data AB a = Nil | Bin (AB a) a (AB a)
\end{minted}
\end{centrado}
\begin{centrado}
	\begin{minted}[breaklines, tabsize=4]{haskell}
-- a)
vacioAB:: AB a -> Bool
vacioAB Nil = True
vacioAB (Bin _ _ _) = False
\end{minted}
\end{centrado}
\begin{centrado}
	\begin{minted}[breaklines, tabsize=4]{haskell}
-- b)
negacionAB :: AB Bool -> AB Bool
negacionAB Nil = Nil
negacionAB (Bin l x r) = 
	Bin (negacionAB l) (not x) (negacionAB r)
\end{minted}
\end{centrado}
\begin{centrado}
	\begin{minted}[breaklines, tabsize=4]{haskell}
-- c)
productoAB :: AB Int -> Int
productoAB Nil = 1
productoAB (Bin l x r) = x * (productoAB l) * (productoAB r)
\end{minted}
\end{centrado}
\end{document}