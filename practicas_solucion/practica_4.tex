\documentclass[10pt,a4paper,landscape]{article}

\usepackage[pdftex,
pdfauthor={Gianfranco Zamboni},
pdftitle={Resumen: Paradigmas de Lenguajes de Programación},
pdfsubject={},
pdfkeywords={Resumen , Computacion, FCEyN, UBA, Paradigmas de Lenguajes de Programación, Imperativo, Funcional, Cálculo Lambda, Programación Orientada a Objetos, Objetos, Programación Lógica},
pdfproducer={Latex with hyperref},
pdfcreator={pdflatex}]{hyperref}

\usepackage{amsmath}
\usepackage{ amssymb }
\usepackage{bussproofs}

\usepackage[spanish]{babel}


\usepackage[utf8]{inputenc} % para poder usar tildes en archivos UTF-8
\usepackage{graphicx}
\usepackage{xcolor}
\usepackage{pifont}

\usepackage{lscape}
\usepackage{minted}
\usepackage{a4wide} % márgenes un poco más anchos que lo usual
\usepackage[titletoc,toc,page]{appendix}
\usepackage{tikz}
\usepackage{forest}
\usepackage{multicol}

\setlength{\columnsep}{1cm}

\ifthenelse{\paperwidth < \paperheight}{\usepackage{fancyhdr}
\pagestyle{fancy}

%\renewcommand{\chaptermark}[1]{\markboth{#1}{}}
\renewcommand{\sectionmark}[1]{\markright{\thesection\ - #1}}

\fancyhf{}

\fancyhead[LO]{Sección \rightmark} % \thesection\ 
\fancyfoot[LO]{\small{Paradigmas de lenguajes de programación}}
\fancyfoot[RO]{\thepage}
\renewcommand{\headrulewidth}{0.5pt}
\renewcommand{\footrulewidth}{0.5pt}
\setlength{\hoffset}{-0.25in}
\setlength{\textwidth}{16cm}
%\setlength{\hoffset}{-1.1cm}
%\setlength{\textwidth}{16cm}
\setlength{\headsep}{0.5cm}
\setlength{\textheight}{25cm}
\setlength{\voffset}{-0.4in}
\setlength{\headwidth}{\textwidth}
\setlength{\headheight}{13.1pt}

\renewcommand{\baselinestretch}{1.1}  % line spacing}{\usepackage{fancyhdr}
\pagestyle{fancy}

%\renewcommand{\chaptermark}[1]{\markboth{#1}{}}
\renewcommand{\sectionmark}[1]{\markright{\thesection\ - #1}}

\fancyhf{}

\fancyhead[LO]{\rightmark} % \thesection\ 
\fancyfoot[LO]{\small{PLP - Prácticas}}
\fancyfoot[RO]{\thepage}
\renewcommand{\headrulewidth}{0.5pt}
\renewcommand{\footrulewidth}{0.5pt}
\setlength{\hoffset}{-0.25in}
\setlength{\textwidth}{25cm}
%\setlength{\hoffset}{-1.1cm}
%\setlength{\textwidth}{16cm}
\setlength{\headsep}{0.5cm}
\setlength{\textheight}{16cm}
\setlength{\voffset}{-0.4in}
\setlength{\headwidth}{\textwidth}
\setlength{\headheight}{13.1pt}

\renewcommand{\baselinestretch}{1.1}  % line spacing
}





\newenvironment{centrado}
    {
     \begin{center}
     \begin{minipage}{0.8\textwidth}
 }    
    {
     \end{minipage}
     \end{center}
    }

\newcommand{\rel}{\ensuremath{\mathcal{R}}}

\newcommand{\equalDef}{\overset{def}{=}}
\newcommand{\equalDot}{\overset{\cdot}{=}}

\newcommand{\lambdaAbs}[3]{\lambda #1: #2 . #3}
\newcommand{\lambdaAssign}[2]{#1~:=~#2}
\newcommand{\lambdaApp}[2]{#1~#2}
\newcommand{\lambdaIf}[3]{if~ #1~ then~ #2~ else~ #3}
\newcommand{\lambdaTrue}{true}
\newcommand{\lambdaFalse}{false}
\newcommand{\lambdaLet}[4]{let~#1:#2 = #3~in~#4}
\newcommand{\lambdaRef}[1]{ref~#1}
\newcommand{\lambdaVar}[1]{#1}
\newcommand{\lambdaValue}[1]{\color{red}#1\color{black}}
\newcommand{\lambdaFix}[1]{fix~#1}

\newcommand{\lambdaAbsI}[2]{\lambda #1. #2}



\newcommand{\blue}[1]{\color{blue}#1\color{black}}
\newcommand{\replaceBy}[3]{#1\{#2\leftarrow#3\}}

\newcommand{\judgeType}[3]{#1\triangleright #2 : #3}


\newenvironment{scprooftree}[1]%
{\gdef\scalefactor{#1}\begin{center}\proofSkipAmount \leavevmode}%
    {\scalebox{\scalefactor}{\DisplayProof}\proofSkipAmount \end{center} }


\tikzset{
    every leaf node/.style={text=red, align=center},
    every tree node/.style={text=blue, align=center},
}

\forestset{tikzQtree/.style={for tree={if n children=0{
                node options=every leaf node/.try}{node options=every tree node/.try}, text centered}}}
                
                
\DeclareMathOperator{\Erase}{Erase}
\DeclareMathOperator{\Nat}{Nat}
\DeclareMathOperator{\Bool}{Bool}
\DeclareMathOperator{\Union}{Union}

\newcommand{\WFunc}{\mathbb{W}}

\newcommand{\red}[1]{{\color{red}#1}}%\renewcommand{\appendixtocname}{Apéndices}
\newcommand{\green}[1]{{\color{green!40!black}#1}}%\renewcommand{\appendixpagename}{Apéndices}





\setcounter{section}{4}


\begin{document}
\title{PLP - Práctica 4: Subtipado}

\date{\today}

\author{Zamboni, Gianfranco}

\maketitle
\setcounter{page}{1}

\section*{Reglas de subtipado}
\subsection{Ejercicio 1}
\paragraph{1)}
\begin{center}
   \begin{scprooftree}
       \def\extraVskip{5pt}
       
       \AxiomC{$\{y\}\subseteq\{x,y,z\}$}

       \AxiomC{$y:Nat = y:Nat$}
       
       \RightLabel{S-Rcd}
       \BinaryInfC{$\{x: Nat,~y:Nat,~z:Nat\} <: \{y:Nat\}$}
    \end{scprooftree}
\end{center}

Esta demostración no es única porque existe la regla $S-Trans$ que nos permite probar la transitividad de los tipos, si hubiesemos decidido usarla para primero conseguir un supertipo $\{x:Nat,y:Nat\}$ del primer término y luego probar que ese supertipo es subtipo de $\{y:Nat\}$ entonces también hubiese sido una demostración válida.

\paragraph{2)}

\begin{center}
   \begin{scprooftree}
       \def\extraVskip{5pt}
       
       \AxiomC{$\emptyset\subseteq\{x,y,z\}$}
       \RightLabel{S-Rcd}
       \UnaryInfC{$\{x: Nat,~y:Nat\} <: \{\}$}
    \end{scprooftree}
\end{center}

\hspace*{5mm}
\begin{center}
   \begin{scprooftree}
       \def\extraVskip{5pt}

            \AxiomC{$\{x\}\subseteq\{x,y\}$}
            \AxiomC{$x:Nat = x:Nat$}
           \RightLabel{S-Rcd}
           \BinaryInfC{$\{x: Nat,~y:Nat\} <: \{x:Nat\}$}
           
               \AxiomC{$\emptyset\subseteq\{x,y,z\}$}
           \RightLabel{S-Rcd}            
           \UnaryInfC{$\{x:Nat\} <: \{\}$}

       \RightLabel{S-Trans}
       \BinaryInfC{$\{x: Nat,~y:Nat\} <: \{\}$}
    \end{scprooftree}
\end{center}

\subsection{Ejercicio 2}
\paragraph{1)} Los registros tienen subtipos infinitos porque dado el tipo de un registro $\omega = \{l_i:\sigma_i\}_{i\in 1..n}$, entonces cualquier tipo de la forma $\omega' = \{l_i:\tau_i\}_{i\in 1..k}$ con $k \geq n$ tal que  $\tau_i <: \sigma_i^{i\in 1..n}$ es subtipo del tipo de $\omega$.

$Top$ tiene como subtipo a los registros y los registros tienen infinitos subtipos, entonces $Top$ tiene infinitos subtipos.

Por S-Arrow, los subtipos de una función $\sigma\to\tau$ son los tipos de la forma $\sigma' \to \tau'$ tal que $\sigma <: \sigma'$ y $\tau' <: \tau$. En particular si $\tau$ es de tipo registro, entonces $\tau$ tiene infinitos subtipos, por lo que $\sigma\to\tau$ tambien los tiene (son las funciones que devuelven registros). 

\paragraph{2)} $Top$ no tiene supertipos.

Los registros tiene una cantidad finita de supertipos, siendo el máximo registro $\{\} <: Top$

Otra vez, hay casos en que las funciones tienen infinitos supertipos y es cuando toman como parámetro a un registro. Esto es porque la regla S-Arrow es contravariante respecto del tipo del párametro de la función, es decir, para que un tipo $\sigma\to\tau$ sea supertipo de $\sigma'\to\tau$ tiene que valer que $\sigma <: \sigma'$. Y si $sigma'$ es un registro, entonces tiene infinitos subtipos.

\subsection{Ejercicio 3}
\paragraph{1)} $S = Top$
\paragraph{2)} Si solo consideramos los tipos básicos $Bool,~Nat,~Int,~Float$, entonces $S=Bool$, pero cuando empezamos a considerar registros o listas u otros tipos, entonces, los ``mínimos'' de cada tipo no están relacionados de ninguna forma, incluso, en el caso de los registros, ese mínimo nisiquiera existe.
\paragraph{3)} Por S-Arrow, tenemos que $S_1\to S_2 <: T_1\to T_2$ si $T_1 <: S_1$ y $T_2 <: S_2$. 

El primer caso es el punto \textbf{1)}, el segundo es el punto \textbf{2)}.
\paragraph{4)} Es similar al anterior pero con los casos invertidos.

\subsection{Ejercicio 4}
\paragraph{1)} $T<:S~\underset{\text{S-Trans}}{\iff} ~ T<:T~\land~T<:S \underset{\text{S-Arrow}}{\iff} S\to T <: T\to T$
\paragraph{2)} Si $S = Bool$ y $T = Top$, entonces $\{x:Bool, y:Top\}$ tiene $26$ supertipos y el tipo $Bool\to Top$ solo tiene como supertipo a $Top$, porque $Bool$ no tiene subtipos y $Top$ no tiene supertipos.
\paragraph{3)} Si $S = Top$ y $T = Top$, entonces $\{x:Top, y:Top\}$ tiene como supertipos a $\{x:Top\}$, $\{y:Top\}$, $\{\}$ y a $Top$ y el tipo $Top\to Top$ tiene infinitos por el ejercicio 2.

\subsection{\blue{Ejercicio 5}}

subtipos de Bool: {Bool} -> finitos
subtipos de Nat: {Bool} {Nat} -> finitos
subtipos de a->b: {supertipos de a}x{subtipos de b}, con a y b en {nat,bool,funciones}

\section*{Subtipado en el contexto de tipado}

\subsection{Ejercicio 6}
\hspace*{5mm}
\begin{center}
   \begin{scprooftree}
       \def\extraVskip{5pt}
        
        \AxiomC{$y:Nat \in \{x:Bool,y:Nat\}$}
        \RightLabel{T-Var}
        \UnaryInfC{$\judgeTypeS{\{x:Bool,y:Nat\}}{y}{Nat}$}       
        \RightLabel{T-Succ}       \UnaryInfC{$\judgeTypeS{\{x:Bool,y:Nat\}}{succ(y)}{Nat}$}
        \RightLabel{T-Abs}
        \UnaryInfC{$\judgeTypeS{\{x:Bool\}}{\lambdaAbs{y}{Nat}{succ(y)}}{Nat \rightarrow Nat}$}


        \AxiomC{$x:Bool \in \{x:Bool\}$}
        \RightLabel{T-Var}
        \UnaryInfC{$\judgeTypeS{\{x:Bool\}}{x:Bool}{}$}

        \AxiomC{}
        \RightLabel{S-BoolNat}
        \UnaryInfC{$Bool <: Nat$}
        
        \RightLabel{T-Subs}
        \BinaryInfC{$\judgeTypeS{\{x:Bool\}}{x:nat}{Nat}$} 
        \RightLabel{T-App}
        \BinaryInfC{$\judgeTypeS{\{x:Bool\}}{\lambdaAbs{y}{Nat}{succ(y)} x}{Nat}$}
        
        \RightLabel{T-Abs}
        \UnaryInfC{$\judgeTypeS{\{\}}{\lambdaAbs{x}{Bool}{\lambdaAbs{y}{Nat}{succ(y)} x}}{Bool \rightarrow Nat}$}
       
\end{scprooftree}
\end{center}
\newpage

\vspace*{5mm}
\begin{center}
   \begin{scprooftree}
       \def\extraVskip{5pt}
        \AxiomC{Demo a)}
        \AxiomC{Demo b)}

        \RightLabel{T-App}
        \BinaryInfC{$\judgeTypeS{\{\}}{(\lambdaAbs{r}{\{l1:Bool,l2:Float\}}{if$ $r.l1$ $then$ $r.l2$ $else$ $5,5})$  $\{l1:True,l2:-8,l3:9,0\}}{Float}$}
\end{scprooftree}    
\end{center}


\par{Demo a):}

\vspace*{5mm}
\begin{center}
   \begin{scprooftree}
       \def\extraVskip{5pt}

        \AxiomC{$\judgeTypeS{\Gamma}{5,5}{Float}$}

        \AxiomC{$r.l2:Float \in \Gamma$}
        \RightLabel{T-Var}        \UnaryInfC{$\judgeTypeS{\Gamma}{r.l2}{Float}$}
        
        \AxiomC{$r.l1:Bool \in \Gamma$}
        \RightLabel{T-Var}        \UnaryInfC{$\judgeTypeS{\Gamma}{r.l1}{Bool}$}
        
        \RightLabel{T-If} \TrinaryInfC{$\judgeTypeS{\Gamma = r:\{l1:Bool,l2:Float\}}{if$ $r.l1$ $then$ $r.l2$ $else$ $5,5}{Float}$}
        \RightLabel{T-Abs}
        \UnaryInfC{$\judgeTypeS{\{\}}{\lambdaAbs{r}{\{l1:Bool,l2:Float\}}{if$ $r.l1$ $then$ $r.l2$ $else$ $5,5}}{\{l1:Bool,l2:Float\} \rightarrow Float}$}
\end{scprooftree}    
\end{center}

\par{Demo b):}

\vspace*{5mm}
\begin{center}
   \begin{scprooftree}
       \def\extraVskip{5pt}
        \AxiomC{}
        \RightLabel{S-RCDWidth}       \UnaryInfC{$\{l1:Bool,l2:Int,l3:Float\}$ $<:$ $\{l1:Bool,l2:Float\} $}


        \AxiomC{$\judgeTypeS{\{\}}{9,0:Float}{}$}
        \AxiomC{$\judgeTypeS{\{\}}{-8:Int}{}$}
        \AxiomC{$\judgeTypeS{\{\}}{True:Bool}{}$}
        \RightLabel{T-RCD}
        \TrinaryInfC{$\judgeTypeS{\{\}}{\{l1:True,l2:-8,l3:9,0\}}{\{l1:Bool,l2:Int,l3:Float\}} $}

        \RightLabel{T-Subs}
        \BinaryInfC{$\{l1:True,l2:-8,l3:9,0\}$ : $\{l1:Bool,l2:Int,l3:Float\}$}
\end{scprooftree}    
\end{center}

\newpage
\subsection{Ejercicio 7}

Inserte su explicacion formal aquí $:)$

Por lo tanto, el término xx no es tipable en el cálculo $\lambda$ clásico.

\vspace*{5mm}
\begin{center}
   \begin{scprooftree}
       \def\extraVskip{5pt}

    \AxiomC{$x:\sigma \rightarrow \tau \in \Gamma$}
    \UnaryInfC{$\judgeType{\Gamma}{x}{\sigma \rightarrow \tau}$}
    
    \AxiomC{$x:\theta \in \Gamma$}
    \UnaryInfC{$\judgeType{\Gamma}{x}{\theta}$}

    \AxiomC{$\theta <: \sigma$}

    \RightLabel{T-App}
    \TrinaryInfC{$\judgeTypeS{\Gamma}{x$ $x}{\tau}$}
\end{scprooftree}    
\end{center}

Si $\theta=\sigma \rightarrow \tau$ y $\sigma=Top$ entonces vale $\theta <: Top$, es decir, $Top \rightarrow \tau <: Top$.
       
\subsection{Ejercicio 8}
\par{a)}
\vspace*{5mm}
\begin{center}
   \begin{scprooftree}
       \def\extraVskip{5pt}
    \AxiomC{$U<:V$}
    \AxiomC{$S<:T$}
    \RightLabel{S-Arrow'}    
    \BinaryInfC{$T \rightarrow V <: S \rightarrow U$}
\end{scprooftree}    
\end{center}

Sea $M=(\lambdaAbs{x}{Int}{\lambdaIfThenElse{x>2}{2,5}{0,3}}: Int->Bool$, por la regla S-Arrow', podríamos tipar el término: M true, ya que podríamos ingresarle a M un elemento de un subtipo de Int.
Pero esto no tenrdía sentido, ya que no se podría evaluar: true>2.

Demostración:
\vspace*{5mm}
\begin{center}
   \begin{scprooftree}
       \def\extraVskip{5pt}

    \AxiomC{$\judgeType{\Gamma}{0,3}{Float}$}
    \AxiomC{$\judgeType{\Gamma}{2,5}{Float}$}
    \AxiomC{$\judgeType{\Gamma}{x>2}{Bool}$}
    \RightLabel{T-if}
    \TrinaryInfC{$\judgeType{\Gamma=\{x:Int\}}{if$ $(x>2)$ $then$ $2,5$ $else$ $0,3}{Float}$}
    \RightLabel{T-abs}    \UnaryInfC{$\judgeType{\{\}}{\lambdaAbs{x}{Int}{\lambdaIfThenElse{x>2}{2,5}{0,3}}}{Int \rightarrow Float}$}
    
    \AxiomC{$\judgeType{\{\}}{True}{Bool}$}
    \AxiomC{$Bool<:Int$}
    \RightLabel{T-Subs}
    \BinaryInfC{$\judgeType{\{\}}{True}{Int}$}
    
    \RightLabel{T-app}
    \BinaryInfC{$\judgeType{\{\}}{(\lambdaAbs{x}{Int}{\lambdaIfThenElse{x>2}{2,5}{0,3}$ $True}}{Float}$}
\end{scprooftree}    
\end{center}

\par{b)}
\vspace*{5mm}
\begin{center}
   \begin{scprooftree}
       \def\extraVskip{5pt}
    \AxiomC{$U<:V$}
    \AxiomC{$S<:T$}
    \RightLabel{S-Arrow''}
    \BinaryInfC{$S \rightarrow U <: T \rightarrow V$}
\end{scprooftree}    
\end{center}

\subsection{Ejercicio 9}
\subsection{Ejercicio 10}
\subsection{Ejercicio 11}
\subsection{Ejercicio 12}
\subsection{Ejercicio 13}
\subsection{\blue{Ejercicio 14}}
\end{document}
