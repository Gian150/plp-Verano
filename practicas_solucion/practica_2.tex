\documentclass[10pt,a4paper, landscape]{article}

\usepackage[pdftex,
pdfauthor={Gianfranco Zamboni},
pdftitle={Resumen: Paradigmas de Lenguajes de Programación},
pdfsubject={},
pdfkeywords={Resumen , Computacion, FCEyN, UBA, Paradigmas de Lenguajes de Programación, Imperativo, Funcional, Cálculo Lambda, Programación Orientada a Objetos, Objetos, Programación Lógica},
pdfproducer={Latex with hyperref},
pdfcreator={pdflatex}]{hyperref}

\usepackage{amsmath}
\usepackage{ amssymb }
\usepackage{bussproofs}

\usepackage[spanish]{babel}


\usepackage[utf8]{inputenc} % para poder usar tildes en archivos UTF-8
\usepackage{graphicx}
\usepackage{xcolor}
\usepackage{pifont}

\usepackage{lscape}
\usepackage{minted}
\usepackage{a4wide} % márgenes un poco más anchos que lo usual
\usepackage[titletoc,toc,page]{appendix}
\usepackage{tikz}
\usepackage{forest}
\usepackage{multicol}

\setlength{\columnsep}{1cm}

\ifthenelse{\paperwidth < \paperheight}{\usepackage{fancyhdr}
\pagestyle{fancy}

%\renewcommand{\chaptermark}[1]{\markboth{#1}{}}
\renewcommand{\sectionmark}[1]{\markright{\thesection\ - #1}}

\fancyhf{}

\fancyhead[LO]{Sección \rightmark} % \thesection\ 
\fancyfoot[LO]{\small{Paradigmas de lenguajes de programación}}
\fancyfoot[RO]{\thepage}
\renewcommand{\headrulewidth}{0.5pt}
\renewcommand{\footrulewidth}{0.5pt}
\setlength{\hoffset}{-0.25in}
\setlength{\textwidth}{16cm}
%\setlength{\hoffset}{-1.1cm}
%\setlength{\textwidth}{16cm}
\setlength{\headsep}{0.5cm}
\setlength{\textheight}{25cm}
\setlength{\voffset}{-0.4in}
\setlength{\headwidth}{\textwidth}
\setlength{\headheight}{13.1pt}

\renewcommand{\baselinestretch}{1.1}  % line spacing}{\usepackage{fancyhdr}
\pagestyle{fancy}

%\renewcommand{\chaptermark}[1]{\markboth{#1}{}}
\renewcommand{\sectionmark}[1]{\markright{\thesection\ - #1}}

\fancyhf{}

\fancyhead[LO]{\rightmark} % \thesection\ 
\fancyfoot[LO]{\small{PLP - Prácticas}}
\fancyfoot[RO]{\thepage}
\renewcommand{\headrulewidth}{0.5pt}
\renewcommand{\footrulewidth}{0.5pt}
\setlength{\hoffset}{-0.25in}
\setlength{\textwidth}{25cm}
%\setlength{\hoffset}{-1.1cm}
%\setlength{\textwidth}{16cm}
\setlength{\headsep}{0.5cm}
\setlength{\textheight}{16cm}
\setlength{\voffset}{-0.4in}
\setlength{\headwidth}{\textwidth}
\setlength{\headheight}{13.1pt}

\renewcommand{\baselinestretch}{1.1}  % line spacing
}





\newenvironment{centrado}
    {
     \begin{center}
     \begin{minipage}{0.8\textwidth}
 }    
    {
     \end{minipage}
     \end{center}
    }

\newcommand{\rel}{\ensuremath{\mathcal{R}}}

\newcommand{\equalDef}{\overset{def}{=}}
\newcommand{\equalDot}{\overset{\cdot}{=}}

\newcommand{\lambdaAbs}[3]{\lambda #1: #2 . #3}
\newcommand{\lambdaAssign}[2]{#1~:=~#2}
\newcommand{\lambdaApp}[2]{#1~#2}
\newcommand{\lambdaIf}[3]{if~ #1~ then~ #2~ else~ #3}
\newcommand{\lambdaTrue}{true}
\newcommand{\lambdaFalse}{false}
\newcommand{\lambdaLet}[4]{let~#1:#2 = #3~in~#4}
\newcommand{\lambdaRef}[1]{ref~#1}
\newcommand{\lambdaVar}[1]{#1}
\newcommand{\lambdaValue}[1]{\color{red}#1\color{black}}
\newcommand{\lambdaFix}[1]{fix~#1}

\newcommand{\lambdaAbsI}[2]{\lambda #1. #2}



\newcommand{\blue}[1]{\color{blue}#1\color{black}}
\newcommand{\replaceBy}[3]{#1\{#2\leftarrow#3\}}

\newcommand{\judgeType}[3]{#1\triangleright #2 : #3}


\newenvironment{scprooftree}[1]%
{\gdef\scalefactor{#1}\begin{center}\proofSkipAmount \leavevmode}%
    {\scalebox{\scalefactor}{\DisplayProof}\proofSkipAmount \end{center} }


\tikzset{
    every leaf node/.style={text=red, align=center},
    every tree node/.style={text=blue, align=center},
}

\forestset{tikzQtree/.style={for tree={if n children=0{
                node options=every leaf node/.try}{node options=every tree node/.try}, text centered}}}
                
                
\DeclareMathOperator{\Erase}{Erase}
\DeclareMathOperator{\Nat}{Nat}
\DeclareMathOperator{\Bool}{Bool}
\DeclareMathOperator{\Union}{Union}

\newcommand{\WFunc}{\mathbb{W}}

\newcommand{\red}[1]{{\color{red}#1}}%\renewcommand{\appendixtocname}{Apéndices}
\newcommand{\green}[1]{{\color{green!40!black}#1}}%\renewcommand{\appendixpagename}{Apéndices}





\setcounter{section}{3}


\begin{document}
\title{PLP - Práctica 3: Inferencia de tipos}

\date{\today}

\author{Zamboni, Gianfranco}

\maketitle
\setcounter{page}{1}


\section*{\centering Sintaxis}
\subsection{Ejercicio 1}
\begin{multicols}{2}

\end{multicols}

\vspace*{5mm}
\begin{center}
    \begin{tabular}{c|c|c}
        \textbf{Expresiones de términos} & \textbf{Expresiones de tipo} & \textbf{No válidas}\\
       $x$ & $ Bool$ & $M$ \\ 
       $x~x$ & $ Bool\to  Bool$ & $M~M$ \\      
       $true~false$ & $ Bool\to  Bool\to  Nat$ & $\lambda x.isZero(x)$ \\
       $true~succ(true~false)$ & $( Bool\to  Bool)\to  Nat$ & ${\lambdaAbs{x}{\sigma}{succ(x)}}$\\
       $\lambdaAbs{x}{ Bool}{succ(x)}$ & & $\lambda x:\lambdaIf{true}{ Bool}{ Nat.x}$ \\
       $\lambdaAbs{x}{ Bool}{\lambdaIf{0}{true}{0~succ(true)}}$ & & $\sigma$ \\
       & & $succ~true$
    \end{tabular}
\end{center}

\vspace*{\fill}
\begin{multicols}{2}
\subsection{Ejercicio 2}

    \begin{forest} tikzQtree,
        [$isZero(pre(succ(0~\lambdaIf{true}{false}{\lambdaAbs{x}{ Nat}{x}})))$,
        [$pre(succ(0~\lambdaIf{true}{false}{\lambdaAbs{x}{ Nat}{x}}))$
        [$succ(0~\lambdaIf{true}{false}{\lambdaAbs{x}{ Nat}{x}})$
        [$0~\lambdaIf{true}{false}{\lambdaAbs{x}{ Nat}{x}}$
        [$0$]
        [$\lambdaIf{true}{false}{\lambdaAbs{x}{ Nat}{x}}$
        [$true$]
        [$false$]
        [$\lambdaAbs{x}{ Nat}{x}$
        [x]
        ]
        ]                
        ]
        ]
        ]
        ]
    \end{forest}

\subsection{Ejercicio 3}
\paragraph{1.}
$\lambdaAbs{x}{ Nat}{succ((\lambdaAbs{x}{ Nat}{\blue{x}})~\lambdaValue{x})}$

\paragraph{2. }En el término $\lambdaAbs{x_1}{ Nat}{succ(x_2)}$, $x_1$ no aparece como subtérmino.

\paragraph{3. } La expresión $x~(y~z)$ no sucede en la expresión $u~x~(y~z)$

\begin{center}
    \begin{forest} tikzQtree,
        [$u~x~(y~z)$
        [$u~x$
        [$u$]
        [$x$]
        ]
        [$y~z$
        [$y$]
        [$z$]
        ]
        ]
    \end{forest}
\end{center}

\end{multicols}
\vspace*{\fill}

\newpage
\subsection{Ejercicio 4}
Marcamos con \blue{azul} las variables ligadas y con \red{rojo} las variables libres
\paragraph{a)}

\begin{forest} tikzQtree,
    [$\red{(}\blue{(}\red{u}~\red{x}\blue{)}(\red{y}~\red{z})\red{)}~(\lambdaAbs{v}{ Bool}{\blue{v}})$
    [$\blue{(}\red{u}~\red{x}\blue{)}(\red{y}~\red{z})$
    [$\red{u}~\red{x}$
    [$\red{u}$]
    [$\red{x}$]
    ]
    [$\red{y}~\red{z}$
    [$\red{y}$]
    [$\red{z}$]
    ]
    ]
    [$\lambdaAbs{v}{ Bool}{\blue{v}}$
    [$\red{v}$]
    ]  
    ]
\end{forest}

\paragraph{b)} En esta expresión aparece $(\lambdaAbs{x}{ Bool\to  Nat\to  Bool}{\lambdaAbs{y}{ Bool\to  Nat}{\lambdaAbs{z}{ Bool}{x~z~(y~z)}}})~u$, la marco con una \xmark~ cuando aparece.

\vspace*{5mm}
\begin{forest}tikzQtree,
[$\red{(}\blue{(}(\lambdaAbs{x}{ Bool\to  Nat \to  Bool}{\lambdaAbs{y}{ Bool\to  Nat}{\lambdaAbs{z}{ Bool}{\red{(}\blue{(}\blue{x}~\blue{z}\blue{)}(\blue{y}~\blue{z})\red{)}}}})~\red{u}\blue{)}~\red{v}\red{)}~ \red{w}$
    [$\blue{(}(\lambdaAbs{x}{ Bool\to  Nat\to  Bool}{\lambdaAbs{y}{ Bool\to  Nat}{\lambdaAbs{z}{ Bool}{\red{(}\blue{(}\blue{x}~\blue{z}\blue{)}(\blue{y}~\blue{z})\red{)}}}})~\red{u}\blue{)}~\red{v}$
        [ \xmark $(\lambdaAbs{x}{ Bool\to  Nat\to  Bool}{\lambdaAbs{y}{ Bool\to  Nat}{\lambdaAbs{z}{ Bool}{\red{(}\blue{(}\blue{x}~\blue{z}\blue{)}(\blue{y}~\blue{z})\red{)}}}})~\red{u}$ ,
            [$\lambdaAbs{x}{ Bool\to  Nat\to  Bool}{\lambdaAbs{y}{ Bool\to  Nat}{\lambdaAbs{z}{ Bool}{\red{(}\blue{(}\blue{x}~\blue{z}\blue{)}(\blue{y}~\blue{z})\red{)}}}}$
                [$\lambdaAbs{y}{ Bool\to  Nat}{\lambdaAbs{z}{ Bool}{\red{(}\blue{(}\red{x}~\blue{z}\blue{)}(\blue{y}~\blue{z})\red{)}}}$
                    [$\lambdaAbs{z}{ Bool}{\red{(}\blue{(}\red{x}~\blue{z}\blue{)}(\red{y}~\blue{z})\red{)}}$
                        [$\blue{(}\red{x}~\red{z}\blue{)}(\red{y}~\red{z})$
                            [$\red{x}~\red{z}$
                                [$\red{x}$]
                                [$\red{z}$]
                            ]
                            [$\red{y}~\red{z}$
                                [$\red{y}$]
                                [$\red{z}$]
                            ]
                        ]
                    ]
                ]
            ]
    [$\red{u}$]
    ]
    [$\red{v}$]
    ]
    [$\red{w}$]
]
\end{forest}

\paragraph{c)}
\begin{forest}tikzQtree,
[$\red{(}\blue{(}\red{w}~(\lambdaAbs{x}{ Bool\to  Nat\to  Bool}{\lambdaAbs{y}{ Bool\to  Nat}{\lambdaAbs{z}{ Bool}{\green{(}\red{(}\blue{x}~\blue{z}\red{)}~(\blue{y}~ \blue{z})\green{)}}}})\blue{)}~\red{u}\red{)}~\red{v}$
    [$\blue{(}\red{w}~(\lambdaAbs{x}{ Bool\to  Nat\to  Bool}{\lambdaAbs{y}{ Bool\to  Nat}{\lambdaAbs{z}{ Bool}{\green{(}\red{(}\blue{x}~\blue{z}\red{)}~(\blue{y}~ z)\green{)}}}})\blue{)}~\red{u}$
        [$\red{w}~(\lambdaAbs{x}{ Bool\to  Nat\to  Bool}{\lambdaAbs{y}{ Bool\to  Nat}{\lambdaAbs{z}{ Bool}{\green{(}\red{(}\blue{x}~\blue{z}\red{)}~(\blue{y}~ \blue{z})\green{)}}}})$
            [$\red{w}$]
            [$\lambdaAbs{x}{ Bool\to  Nat\to  Bool}{\lambdaAbs{y}{ Bool\to  Nat}{\lambdaAbs{z}{ Bool}{\green{(}\red{(}\blue{x}~\blue{z}\red{)}~(\blue{y}~ \blue{z})\green{)}}}}$
                [$\lambdaAbs{y}{ Bool\to  Nat}{\lambdaAbs{z}{ Bool}{\green{(}\red{(}\red{x}~\blue{z}\red{)}~(\blue{y}~ \blue{z})\green{)}}}$
                    [$\lambdaAbs{z}{ Bool}{\green{(}\red{(}\red{x}~\blue{z}\red{)}~(\red{y}~ \blue{z})\green{)}}$
                        [$\red{(}\red{x}~\red{z}\red{)}~(\red{y}~ \red{z})$
                            [$\red{x}~\red{z}$
                                [$\red{x}$]
                                [$\red{z}$]
                            ]
                            [$\red{y}~ \red{z}$
                                [$\red{y}$]
                                [$\red{z}$]
                            ]
                        ]
                    ]
                ]
            ]
        ]
        [$\red{u}$]
    ]
    [$\red{v}$]
]
\end{forest}

\newpage
\section*{\centering Tipado}
\subsection{Ejercicio 5}
\paragraph{1)}\begin{scprooftree}
      \def\extraVskip{5pt}
      \AxiomC{}
      \RightLabel{T-True}
      \UnaryInfC{$\judgeType{\emptyset}{true}{ Bool}$}
      
      \AxiomC{}
      \RightLabel{T-Zero}
      \UnaryInfC{$\judgeType{\emptyset}{0}{ Nat}$}
      
      \AxiomC{}
      \RightLabel{T-Zero}
      \UnaryInfC{$\judgeType{\emptyset}{0}{ Nat}$}
      \RightLabel{T-Succ}
      \UnaryInfC{$\judgeType{\emptyset}{succ(0)}{ Nat}$}
      
      \TrinaryInfC{$\judgeType{\emptyset}{\lambdaIf{true}{0}{succ(0)}}{ Nat}$}
    \end{scprooftree}

\vspace*{5mm}
\paragraph{2)}
En la siguiente demostración $\Gamma =\{x: Nat,~y: Bool\}$
\begin{center}
   \begin{scprooftree}
       \def\extraVskip{5pt}
       
       \AxiomC{}
       \RightLabel{T-True}
       \UnaryInfC{$\judgeType{\Gamma}{true}{ Bool}$}
       
       \AxiomC{}
       \RightLabel{T-False}
       \UnaryInfC{$\judgeType{\Gamma}{false}{ Bool}$}
       
       \AxiomC{$z: Bool\in \Gamma,z: Bool$}
       \RightLabel{T-Var}
       \UnaryInfC{$\judgeType{\Gamma,z: Bool}{z}{ Bool}$}            
       \RightLabel{T-Abs}
       \UnaryInfC{$\judgeType{\Gamma}{\lambdaAbs{z}{ Bool}{z}}{ Bool\to  Bool}$}
       
       \AxiomC{}
       \RightLabel{T-True}
       \UnaryInfC{$\judgeType{\Gamma}{true}{ Bool}$}
       \RightLabel{T-App}
       \BinaryInfC{$\judgeType{\Gamma}{(\lambdaAbs{z}{ Bool}{z})~true}{ Bool}$}
       
       \RightLabel{T-If}
       \TrinaryInfC{$\judgeType{\Gamma}{\lambdaIf{true}{false}{(\lambdaAbs{z}{ Bool}{z})~true}}{ Bool}$}
    \end{scprooftree}
\end{center}

\vspace*{5mm}
\paragraph{3)}%
\begin{scprooftree}
    \def\extraVskip{5pt}
    
    \AxiomC{\red{$\judgeType{\emptyset,x: Bool}{x}{ Bool}\Rightarrow \judgeType{\emptyset}{\lambdaAbs{x}{ Bool}{x}}{ Bool\to\tau}$}}
    \RightLabel{T-Abs}
    \UnaryInfC{$\judgeType{\emptyset}{\lambdaAbs{x}{ Bool}{x}}{ Bool}$}
    
    \AxiomC{$\judgeType{\emptyset}{0}{ Nat}$}
    
    \AxiomC{$\judgeType{\emptyset}{succ(0)}{ Nat}$}
    
    \RightLabel{T-If}
    \TrinaryInfC{$\judgeType{\emptyset}{\lambdaIf{\lambdaAbs{x}{ Bool}{x}}{0}{succ(0)}}{ Nat}$}
\end{scprooftree}

\vspace*{5mm}
\paragraph{4)} En la próxima demostración $\Gamma = \{x :  Bool \to  Nat, y :  Bool\}$ 
\begin{center}
\begin{scprooftree}
    \def\extraVskip{5pt}
    
    \AxiomC{$x: Bool\to  Nat\in\Gamma$}
    \RightLabel{T-Var}
    \UnaryInfC{$\judgeType{\Gamma}{x}{ Bool \to  Nat}$}
    
    \AxiomC{$y: Bool\in\Gamma$}
    \RightLabel{T-Var}
    \UnaryInfC{$\judgeType{\Gamma}{y}{ Bool}$}
    \RightLabel{T-App}
    \BinaryInfC{$\judgeType{\Gamma}{\lambdaIf{\lambdaAbs{x}{ Bool}{x}}{0}{succ(0)}}{ Nat}$}
\end{scprooftree}
\end{center}

\newpage
\subsection{Ejercicio 6}

\begin{multicols}{2}
\paragraph{1)}\label{p2:e6:s1}\begin{scprooftree}
    \def\extraVskip{5pt}
    \AxiomC{\red{$\sigma =  Nat$}}
    \RightLabel{T-Zero}
    \UnaryInfC{$\judgeType{\emptyset}{0}{\sigma}$}
    
    \RightLabel{T-Succ}
    \UnaryInfC{$\judgeType{\emptyset}{succ(0)}{\sigma}$}
\end{scprooftree}$\Rightarrow \sigma =  Nat$

\paragraph{2)} \begin{scprooftree}
    \def\extraVskip{5pt}
        \AxiomC{}
        \RightLabel{Ejercicio \ref{p2:e6:s1}}
        \UnaryInfC{$\judgeType{\emptyset}{succ(0)}{ Nat}\Rightarrow \red{\sigma =  Bool}$}
    
    \RightLabel{T-IsZero}
    \UnaryInfC{$\judgeType{\emptyset}{isZero(succ(0))}{\sigma}$}
\end{scprooftree}$\Rightarrow \sigma =  Bool$

\end{multicols}

\paragraph{3)}
\begin{center}
\begin{scprooftree}
   \def\extraVskip{5pt}
            \AxiomC{}
        \RightLabel{T-True}    
        \UnaryInfC{$\judgeType{\emptyset}{true}{ Bool}$}

            \AxiomC{}
        \RightLabel{T-True}    
        \UnaryInfC{$\judgeType{\emptyset}{false}{ Bool}$}

            \AxiomC{}
       \RightLabel{T-True}    
       \UnaryInfC{$\judgeType{\emptyset}{false}{ Bool}$}

    \RightLabel{T-If}
     \TrinaryInfC{$\judgeType{\Gamma}{\lambdaIf{true}{false}{false}}{ Bool}$}

        \AxiomC{\red{$\sigma =  Nat$}}
    \RightLabel{T-Zero}
    \UnaryInfC{$\judgeType{\emptyset}{0}{\sigma}$}
    
         \AxiomC{\red{$\sigma =  Nat$}}
    \RightLabel{Ejercicio \ref{p2:e6:s1}}
    \UnaryInfC{$\judgeType{\emptyset}{succ(0)}{\sigma}$}
            
\RightLabel{T-If}
\TrinaryInfC{$\judgeType{\emptyset}{\lambdaIf{\lambdaIf{true}{false}{false}}{0}{succ(0)}}{\sigma}$}
    \end{scprooftree}$\Rightarrow \sigma =  Nat$
\end{center}

\subsection{Ejercicio 7}
\begin{multicols}{2}
\paragraph{1)}En la próxima demostración $\Gamma = \{x:\sigma\}$

\vspace*{5mm}
\begin{scprooftree}
       \def\extraVskip{5pt}
\AxiomC{$x: Nat\in \Gamma$}
\RightLabel{T-Var}
\UnaryInfC{$\judgeType{\Gamma}{x}{ Nat}$}
\RightLabel{T-Succ}
\UnaryInfC{$\judgeType{\Gamma}{succ(x)}{ Nat}$}
\RightLabel{T-Succ}
\UnaryInfC{$\judgeType{\Gamma}{isZero(succ(x))}{\tau}$}
\end{scprooftree}

\vspace*{5mm}
Entonces, $\sigma =  Nat$ y $\tau =  Bool$

\vspace*{5mm}
\paragraph{2)}  \hfil

    \begin{scprooftree}
   \def\extraVskip{5pt}
        
            \AxiomC{$\judgeType{\{x:\sigma\}}{x}{\sigma}$}
       \RightLabel{T-Abs}
        \UnaryInfC{$\judgeType{\emptyset}{\lambdaAbs{x}{\sigma}{x}}{\sigma\to\sigma}$}
        
            \AxiomC{$\judgeType{\{y: Bool\}}{0}{ Nat}$}
        \RightLabel{T-Abs}
        \UnaryInfC{$\judgeType{\emptyset}{\lambdaAbs{y}{ Bool}{0}}{\sigma}$}

\RightLabel{T-App}
\BinaryInfC{$\judgeType{\emptyset}{(\lambdaAbs{x}{\sigma}{x})~(\lambdaAbs{y}{ Bool}{0})}{\sigma}$}
    \end{scprooftree}

\vspace*{5mm}
El árbol de la segunda abstracción nos dice que $\sigma =  Bool\to  Nat$. Por lo que la primera abstracción seria la función identidad de funciones del tipo $ Bool\to  Nat$.
\end{multicols}

\paragraph{3)} $\Gamma = \{y:\tau\}$

\vspace*{5mm}
\begin{center}
    \begin{scprooftree}
   \def\extraVskip{5pt}
        
            \AxiomC{\red{$\judgeType{\Gamma,x:\sigma}{x}{\sigma}\Rightarrow \judgeType{\Gamma,x:\sigma}{\lambdaAbs{x}{\sigma}{x}}{\sigma\to  Bool}$}}
       \RightLabel{T-Abs}
        \UnaryInfC{$\judgeType{\Gamma}{\lambdaAbs{x}{\sigma}{x}}{ Bool}$}
        
        \AxiomC{$\judgeType{\Gamma}{y}{\sigma}$}
        
        \AxiomC{$\judgeType{\Gamma}{succ(0)}{\sigma}$}

\RightLabel{T-If}
\TrinaryInfC{$\judgeType{\Gamma}{\lambdaIf{(\lambdaAbs{x}{\sigma}{x})}{y}{succ(0)}}{\sigma}$}
    \end{scprooftree}
\end{center}


\vspace*{5mm}
Entonces, la expresión no es tipable.

\begin{multicols}{2}
\paragraph{4)} $\Gamma = \{x:\sigma\}$

\vspace*{5mm}
    \begin{scprooftree}
   \def\extraVskip{5pt}
        
           \AxiomC{$x: \sigma_1\to\tau\in\Gamma \Rightarrow \sigma = \sigma_1\to\tau$}
       \RightLabel{T-Var}
        \UnaryInfC{$\judgeType{\Gamma}{x}{\sigma_1\to\tau}$}
        
            \AxiomC{\red{$y:\sigma_1\in\Gamma$}}
        \RightLabel{T-Var}
        \UnaryInfC{$\judgeType{\Gamma}{y}{\sigma_1}$}
        

\RightLabel{T-App}
\BinaryInfC{$\judgeType{\Gamma}{x~y}{\tau}$}
    \end{scprooftree}

En este caso, $\sigma = \sigma_1\to\tau$ para cualquier $\tau$ y cualquier $\sigma_1$. Sin embargo, el juicio de tipado no es derivable ya que el sistema de tipado no nos permite asegurar nada sobre el tipo de $y$.
\vspace*{5mm}
Acá no podemos hacer juicio de tipado porque no podemos asegurar que $y$ sea de tipo $\sigma_1$

\paragraph*{5)}$\Gamma = \{x:\sigma, y:\tau\}$

\vspace*{5mm}
    \begin{scprooftree}
   \def\extraVskip{5pt}
        
           \AxiomC{$x: \sigma_1\to\tau\in\Gamma \Rightarrow \sigma = \sigma_1\to\tau$}
       \RightLabel{T-Var}
        \UnaryInfC{$\judgeType{\Gamma}{x}{\sigma_1\to\tau}$}
        
            \AxiomC{$y:\sigma_1\in\Gamma\Rightarrow \sigma_1 = \tau$}
        \RightLabel{T-Var}
        \UnaryInfC{$\judgeType{\Gamma}{y}{\sigma_1}$}
        

\RightLabel{T-App}
\BinaryInfC{$\judgeType{\Gamma}{x~y}{\tau}$}
    \end{scprooftree}

\vspace*{5mm}
Entonces $\sigma = \tau\to\tau$ para cualquier tipo $\tau$

\paragraph{6)} $\Gamma = \{x:\sigma\}$

\vspace*{5mm}
    \begin{scprooftree}
   \def\extraVskip{5pt}
        
           \AxiomC{$x:  Bool\to\tau\in\Gamma \Rightarrow \sigma =  Bool\to\tau$}
       \RightLabel{T-Var}
        \UnaryInfC{$\judgeType{\Gamma}{x}{ Bool\to\tau}$}
        
        \RightLabel{T-True}
        \AxiomC{$\judgeType{\Gamma}{true}{ Bool}$}
        

\RightLabel{T-App}
\BinaryInfC{$\judgeType{\Gamma}{x~true}{\tau}$}
    \end{scprooftree}
    
\vspace*{5mm}
Entonces $\sigma =  Bool\to\tau$ para cualquier tipo $\tau$

\paragraph{7)} $\Gamma = \{x:\sigma\}$

\vspace*{5mm}
    \begin{scprooftree}
   \def\extraVskip{5pt}
        
           \AxiomC{\red{$x:  Bool\to\sigma\in\Gamma$}}
       \RightLabel{T-Var}
        \UnaryInfC{$\judgeType{\Gamma}{x}{ Bool\to\sigma}$}
        
        \RightLabel{T-True}
        \AxiomC{$\judgeType{\Gamma}{true}{ Bool}$}
        

\RightLabel{T-App}
\BinaryInfC{$\judgeType{\Gamma}{x~true}{\sigma}$}
    \end{scprooftree}
    
\vspace*{5mm}

Tenemos que $x$ debe ser de tipo $\sigma$ y $ Bool\to\sigma$ al mismo tiempo, por lo que no es posible dar un tipo a esta expresión

\paragraph{8)} $\Gamma = \{x:\sigma\}$

\vspace*{5mm}
    \begin{scprooftree}
   \def\extraVskip{5pt}
        
           \AxiomC{\red{$x: \sigma_1\to\tau\in\Gamma$}}
       \RightLabel{T-Var (1)}
        \UnaryInfC{$\judgeType{\Gamma}{x}{\sigma_1\to\tau}$}
        
            \AxiomC{\red{$x: \sigma_1\in\Gamma$}}
        \RightLabel{T-Var (2)}
        \UnaryInfC{$\judgeType{\Gamma}{x}{\sigma_1}$}
        

\RightLabel{T-App}
\BinaryInfC{$\judgeType{\Gamma}{x~x}{\tau}$}
    \end{scprooftree}
    
\vspace*{5mm}

Tenemos que $x$ debe ser de tipo $\sigma$ y $ Bool\to\sigma$ al mismo tiempo, por lo que no es posible dar un tipo a esta expresión

\end{multicols}

\subsection{Ejercicio 8}

La expresión más simple que se me ocurre es $true~false$.

\newpage
\subsection{Ejercicio 9}
El juicio de tipado puede ser $\judgeType{\Gamma}{\lambdaAbs{x}{\sigma}{x}}{\sigma\to\sigma}$

\begin{multicols}{2}
Con T-Abs tenemos:
    \begin{scprooftree}
   \def\extraVskip{5pt}
        
          \AxiomC{$x: \sigma\in\Gamma,x: \sigma$}
        \RightLabel{T-Var}
        \UnaryInfC{$\judgeType{\Gamma,x:\sigma}{x}{\sigma}$}
        

\RightLabel{T-Abs}
\UnaryInfC{$\judgeType{\Gamma}{\lambdaAbs{x}{\sigma}{x}}{\sigma\to\sigma}$}
    \end{scprooftree}

Con T-Abs2:
    \begin{scprooftree}
   \def\extraVskip{5pt}
        
          \AxiomC{\red{$x: \sigma\in\Gamma$}}
        \RightLabel{T-Var}
        \UnaryInfC{$\judgeType{\Gamma}{x}{\sigma}$}
        

\RightLabel{T-Abs2}
\UnaryInfC{$\judgeType{\Gamma}{\lambdaAbs{x}{\sigma}{x}}{\sigma\to\sigma}$}
    \end{scprooftree}

\vspace*{5mm}
No hay nada que nos asegure que $x:\sigma$ pertenece a $\Gamma$
\end{multicols}

\section*{\centering Semántica}
\subsection{Ejercicio 10}
\paragraph{1)} $\replaceBy{(\lambdaAbs{y}{\sigma}{x~(\lambdaAbs{x}{\tau}{x})})}{x}{(\lambdaAbs{y}{\rho}{x~y})}
\alphaEq{y=w;~x=z}\replaceBy{(\lambdaAbs{w}{\sigma}{x~(\lambdaAbs{z}{\tau}{z})})}{x}{(\lambdaAbs{y}{\rho}{x~y})} 
\equalDef \lambdaAbs{w}{\sigma}{(\lambdaAbs{y}{\rho}{x~y})~(\lambdaAbs{z}{\tau}{z})}$

\paragraph{2)}
\begin{equation*}
\begin{split}
\replaceBy{(y~(\lambdaAbs{v}{\sigma}{x~v}))}{x}{(\lambdaAbs{y}{\tau}{v~y})} 
&\equalDef \replaceBy{y}{x}{(\lambdaAbs{y}{\tau}{v~y})}~\replaceBy{(\lambdaAbs{v}{\sigma}{x~v})}{x}{(\lambdaAbs{y}{\tau}{v~y})}
\equalDef
y~\replaceBy{(\lambdaAbs{v}{\sigma}{x~v})}{x}{(\lambdaAbs{y}{\tau}{v~y})} \\
&\alphaEq{v=z}
y~\replaceBy{(\lambdaAbs{z}{\sigma}{x~z})}{x}{(\lambdaAbs{y}{\tau}{v~y}}) 
\equalDef y~(\lambdaAbs{z}{\sigma}{(\lambdaAbs{y}{\tau}{v~y})~z})
\end{split}
\end{equation*}

\subsection{Ejercicio 11}
\begin{center}
\begin{tabular}{c|c}
\textbf{Es un valor} & \textbf{No es un valor} \\
\hline
$\lambdaAbs{x}{ Bool}{\underline{2}}$ & $(\lambdaAbs{x}{ Bool}{x})~true$ \\
$\lambdaAbs{x}{ Bool}{pred(\underline{2})}$ &  $x$ \\
$\lambdaAbs{y}{ Nat}{(\lambdaAbs{x}{ Bool}{pred(\underline{2})})~true}$ &  \\
$succ(succ(0))$ &  \\
\end{tabular}
\end{center}

\newpage
\subsection{Ejercicio 12}
Los programas en \red{rojo} son los que dan error y los programas en \blue{azul} los que son valores, el resto no están en forma normal.
\begin{center}
\begin{tabular}{c|c}
\textbf{Programas} & \textbf{No Programas}\\
\hline
$(\lambdaAbs{x}{Bool}{x})~true$ & $\lambdaAbs{x}{ Nat}{pred(succ(y))}$ \\
\blue{$\lambdaAbs{x}{ Nat}{pred(succ(x))}$}    & $(\lambdaAbs{x}{ Bool}{pred(isZero(x))})~true$ \\
$(\lambdaAbs{f}{ Nat\to  Bool}{f~0})~(\lambdaAbs{x}{ Nat}{isZero(x)})$  & $(\lambdaAbs{f}{ Nat\to  Bool}{x})~(\lambdaAbs{x}{ Nat}{isZero(x)})$ \\
\red{$(\lambdaAbs{f}{ Nat \to  Bool}{f~pred(0)})~(\lambdaAbs{x}{ Nat}{isZero(x)})$} &  \\
$\lambdaFix{(\lambdaAbs{y}{ Nat}{succ(y)})}$ & \\
$\lambdaLetrec{f}{\lambdaAbs{x}{ Nat}{succ(f~x)}}{f~0}$ & \\
\end{tabular}
\end{center}


\vspace*{5mm}
\begin{multicols}{2}
\subsection{Ejercicio 13}
\textbf{1) }Cuando definimos $\to$, lo hacemos como una función deterministica que dada una expresión, siempre hay una única que regla que aplica para dar un paso en la reducción. Por lo que una expresión siempre reduce a la misma expresión, en un paso.

\textbf{2) } Al haber, en cada paso, un único camino por el que reducir, si reducimos $M\twoheadrightarrow N$ y $M\twoheadrightarrow N'$, ambas reducciones en $n$ pasos, entonces $N = N'$

\textbf{3)} No es cierto, con la mayoría de las expresiones seguimos un camino de reducción que nos llevará a una forma normal. En estos casos, $M'$ y $M''$ no pueden ser iguales, sin embargo, hay expresiones recursivas como $\lambdaFix{\lambdaAbs{x}{Bool->Bool}{x}}$ que cuando se las trata de reducir siempre evalúan a si mismas y, entonces vale siempre que $M' = M''$ .

\subsection{Ejercicio 14}
\textbf{1)} Cuando $M \twoheadrightarrow 0$, tenemos que $succ(pred(0)) = 1$ y que $pred(succ(0)) = 0$, por lo que no siempre es lo mismo evaluar estas dos expresiones. 

\textbf{2)} $isZero(succ(M))\twoheadrightarrow false$ solo cuando $M$ es de tipo Nat y su forma normal es $0$. 

\textbf{3)} Como dice el enunciado, hay infinitas expresiones que evalúan a cero, por lo tanto $isZero(pred(M))$ vale cuando $M\twoheadrightarrow 0$.

\subsection{Ejercicio 15}
\textbf{1)} Con la nueva regla, las abstracciones $\lambdaAbs{x}{\sigma}{M}$ pasan a ser expresiones reducibles mientras $M$ sea una expresión reducible, por lo que debemos considerarla un valor solo cuando su expresión interna es un valor.

\textbf{2)} Además, esta nueva regla nos da un nuevo camino para reducirla cuando es aplicada a un valor, es decir, en una expresión del tipo $\lambdaAbs{x}{\sigma}{M}~N$, podemos elegir entre usar esta regla o la regla E-Abs/$\beta$, por lo que hay que modificar esta regla para que solo sea apicable cuando $M$ está en su forma normal $F$, o sea no es posible seguir reduciendola. Entonces:
$$V~::=~ ...~|~\lambdaAbs{x}{\sigma}{V}$$
$$\frac{}{\lambdaAbs{x}{\sigma}{F}~V\to \replaceBy{F}{x}{V}}(\text{E-Abs/}\beta)$$

\textbf{3)} Ahora, si tratamos de reducir $(\lambdaAbs{x}{ Nat\to Nat}{x~\underline{23}})~(\lambdaAbs{x}{ Nat}{0})$ nos encontraremos con que la expresión dentro de la primer abstracción es reducible pues $x~\underline{23}$ no es un valor. Sin embargo,  $x$ puede ser cualquier función de tipo  $ Nat\to Nat$, o sea que tiene una variable libre, pero no sabemos como ni cuanta veces ocurre dentro de ella. Esto evita que podamos reducirla, trabando la computación.
\end{multicols}


\newpage

\subsection{Ejercicio 16}
\begin{multicols}{2}
Para poder utilizar la reducción \textit{call-by-name} debemos eliminar la regla E-App/$v$ que nos dice que debemos reducir la expresión usada como parámetro hasta obtener un valor y remplazar E-App2/$\beta$ por:
\vfil

$$\frac{}{(\lambdaAbs{x}{\sigma}{M})~N \to \replaceBy{M}{x}{N}}(\text{E-App/}\beta)$$

Esta nueva regla, nos permite remplazar $x$ por $N$, sin importar si $N$ está en forma normal o no. Esto evitará que si la expresión $M$ no usa a $x$, evaluemos $N$ inncesariamente. La próxima reducción se hizo utilizando los cambios mencionados:
\end{multicols}

\begin{equation*}
\begin{split}
&(\lambdaAbs{f}{ Nat\to Nat}{\lambdaAbs{g}{ Nat\to Nat}{\lambdaAbs{x}{ Nat}{f~(g~x)}}})~(\lambdaAbs{x}{ Nat}{succ(x)})~(\lambdaAbs{x}{ Nat}{succ(x)})~\underline{5} \\
    &\reduceTo{\text{E-App}/\beta} ((\lambdaAbs{g}{ Nat\to Nat}{\lambdaAbs{x}{ Nat}{(\lambdaAbs{x}{ Nat}{succ(x)})~(g~x)}})~(\lambdaAbs{x}{ Nat}{succ(x)})~\underline{5}
\reduceTo{\text{E-App}/\beta} (\lambdaAbs{x}{ Nat}{(\lambdaAbs{x}{ Nat}{succ(x)})~((\lambdaAbs{x}{ Nat}{succ(x)})~x)})~\underline{5}\\
&\reduceTo{\text{E-App}/\beta} (\lambdaAbs{x}{ Nat}{(\lambdaAbs{x}{ Nat}{succ(x)})~((\lambdaAbs{x}{ Nat}{succ(x)})~x)})~\underline{5} 
\reduceTo{\text{E-App}/\beta} (\lambdaAbs{x}{ Nat}{succ(x)})~((\lambdaAbs{x}{ Nat}{succ(x)})~\underline{5})
\reduceTo{\text{E-App}/\beta} succ((\lambdaAbs{x}{ Nat}{succ(x)})~\underline{5}) \\
&\reduceTo{\text{E-Succ; E-App}/\beta} succ(succ(\underline{5})) 
\end{split}
\end{equation*}

\newpage
\section*{\centering Extensiones}
\subsection{Ejercicio 17}\label{p2:e17}
\paragraph{1)}
\begin{multicols}{2}

 \begin{forest} tikzQtree,
[$\lambdaListCase{0 :: succ(0) :: \List{ Nat}}{false}{isZero(x)}$,
    [$0 :: succ(0) :: \List{ Nat}$,
        [$0$]
        [$succ(0) :: \List{ Nat}$,
            [$succ(0)$]
            [$\List{ Nat}$]
        ]
    ]
    [$false$]
    [$isZero(x)$,
        [$x$]
    ]
]
\end{forest}

 \begin{forest} tikzQtree,
[$\lambdaListFold{1 :: 2 :: 3 ::(\lambdaAbs{x}{[ Nat]}{x})~\List{ Nat}}{0}{h+r}$,
    [$1 :: 2 :: 3 :: (\lambdaAbs{x}{[ Nat]}{x})~\List{ Nat}$
        [$1$]
        [$2 :: 3 :: (\lambdaAbs{x}{[ Nat]}{x})~\List{ Nat}$,
            [$2$]
            [$3 :: (\lambdaAbs{x}{[ Nat]}{x})~\List{ Nat}$,
                [$3$]
                [$(\lambdaAbs{x}{[ Nat]}{x})~\List{ Nat}$,
                    [$\lambdaAbs{x}{[ Nat]}{x}$,
                        [$x$]
                    ]
                    [$\List{ Nat}$]
                ]
            ]
        ]
    ]
    [$0$]
    [$h+r$,
        [$h$]
        [$r$]
    ]
]
\end{forest}

\end{multicols}

\paragraph{2)}
\begin{multicols}{2}
$$\frac{}{\judgeType{\Gamma}{\List{\sigma}}{[\sigma]}}(\text{T-Vacio})$$

\vspace*{5mm}
$$\frac{\judgeType{\Gamma}{M}{\sigma}\hspace*{5mm}\judgeType{\Gamma}{N}{[\sigma]}}{\judgeType{\Gamma}{M::N}{[\sigma]}}(\text{T-}::)$$

\vspace*{5mm}
$$\frac{\judgeType{\Gamma}{M}{[\sigma]}\hspace*{5mm}\judgeType{\Gamma}{N}{\tau}~\hspace*{5mm}\judgeType{\Gamma,h:\sigma,t:[\sigma]}{O}{\tau}}{\judgeType{\Gamma}{\lambdaListCase{M}{N}{O}}{\tau}}(T-Case)$$

\vspace*{5mm}
$$\frac{\judgeType{\Gamma}{M}{[\sigma]}\hspace*{5mm}\judgeType{\Gamma}{N}{\tau}~\hspace*{5mm}\judgeType{\Gamma,h:\sigma,r:\tau}{O}{\tau}}{\judgeType{\Gamma}{\lambdaListFold{M}{N}{O}}{\tau}}(T-Fold)$$

\end{multicols}

\paragraph{3)} $\Gamma = \{x: Bool,~y:[ Bool]\}$

\vspace*{5mm}
    \begin{scprooftree}
   \def\extraVskip{5pt}

    \AxiomC{Demostración de abajo (1)}
    \UnaryInfC{$\judgeType{\Gamma}{x :: x :: y}{[ Bool]}$}
    
        \AxiomC{$y:[ Bool]\in\Gamma$}
    \RightLabel{T-Var}        
    \UnaryInfC{$\judgeType{\Gamma}{y}{[ Bool]}$}       

        \AxiomC{Demostración abajo (2)}
    \UnaryInfC{$\judgeType{\Gamma,h: Bool,r:[ Bool]}{\lambdaIf{h}{r}{\List{ Bool}}}{[ Bool]}$}
\RightLabel{T-Fold}
\TrinaryInfC{$\judgeType{\Gamma}{\lambdaListFold{x :: x :: y}{y}{\lambdaIf{h}{r}{\List{ Bool}}}}{[ Bool]}$}
    \end{scprooftree}

\vspace*{2cm}(1)
\begin{scprooftree}
            \AxiomC{$x: Bool\in\Gamma$}
        \RightLabel{T-Var}
        \UnaryInfC{$\judgeType{\Gamma}{x}{ Bool}$}

                \AxiomC{$x: Bool\in\Gamma$}
            \RightLabel{T-Var}
            \UnaryInfC{$\judgeType{\Gamma}{x}{ Bool}$}
                
                \AxiomC{$y:[ Bool]\in\Gamma$}
            \RightLabel{T-Var}        
            \UnaryInfC{$\judgeType{\Gamma}{y}{[ Bool]}$}  
                  
        \RightLabel{T-$::$}        
        \BinaryInfC{$\judgeType{\Gamma}{ x :: y}{[ Bool]}$}
        
    \RightLabel{T-$::$}
    \BinaryInfC{$\judgeType{\Gamma}{x :: x :: y}{[ Bool]}$}
\end{scprooftree}

\vspace*{2cm}(2)
\begin{scprooftree}
          \AxiomC{$h: Bool\in\Gamma,h: Bool,r:[ Bool]$}
        \RightLabel{T-Var}        
        \UnaryInfC{$\judgeType{\Gamma,h: Bool,r:[ Bool]}{h}{ Bool}$}       
 
           \AxiomC{$r:[ Bool]\in\Gamma,h: Bool,r:[ Bool]$}
         \RightLabel{T-Var}        
         \UnaryInfC{$\judgeType{\Gamma,h: Bool,r:[ Bool]}{r}{[ Bool]}$}       
        
          \AxiomC{}
        \RightLabel{T-Vacio}        
        \UnaryInfC{$\judgeType{\Gamma,h: Bool,r:[ Bool]}{\List{ Bool}}{[ Bool]}$}       
        
    \TrinaryInfC{$\judgeType{\Gamma,h: Bool,r:[ Bool]}{\lambdaIf{h}{r}{\List{ Bool}}}{[ Bool]}$}

\end{scprooftree}

\paragraph{4)} Los nuevos valores son $V~::= ...~|~\List{\sigma}~|~V::V$
\paragraph{5)}

\begin{multicols}{2}
$$\frac{M_1\to M_1'}{M_1 :: M_2 \to M'_1::M_2}(\text{E-}::\text{1})$$

\vspace*{5mm}
$$\frac{M_2\to M_2'}{V :: M_2 \to V::M'_2}(\text{E-}::\text{2})$$

\vspace*{5mm}
$$\frac{}{\lambdaListCase{\List{\sigma}}{N}{O}\to N}(\text{E-Case}[~])$$

\vspace*{5mm}
$$\frac{}{\lambdaListCase{V_1::V_2}{N}{O}\to\multiReplaceBy{O}{h\leftarrow V_1,~t\leftarrow V_2}}(\text{E-Case}::)$$
\end{multicols}
\vspace*{5mm}
$$\frac{M\to M'}{\lambdaListCase{M}{N}{O}\to\lambdaListCase{M'}{N}{O}}(\text{E-Case}\text{1})$$

\vspace*{5mm}
$$\frac{}{\lambdaListFold{\List{\sigma}}{N}{O}\to N}(\text{E-Fold}[~])$$

\vspace*{5mm}
$$\frac{}{\lambdaListFold{V_1::V_2}{N}{O}\to \multiReplaceBy{O}{h\leftarrow V_1,~r\leftarrow(\lambdaListFold{V_2}{N}{O})}}(\text{E-Fold}::)$$

\vspace*{5mm}
$$\frac{M\to M'}{\lambdaListFold{M}{N}{O}\to\lambdaListFold{M'}{N}{O}}(\text{E-Fold}\text{1})$$


\subsection{Ejercicio 18}
\begin{multicols}{2}
$M~:= ...~|~map(N,M)$

$$\frac{\judgeType{\Gamma}{N}{[\sigma]}\hspace*{5mm}\judgeType{\Gamma}{M}{\sigma\to\tau}}{\judgeType{\Gamma}{map(N,M)}{[\tau]}}(T-Map)$$

\vspace*{5mm}
Los valores del lenguaje no cambian.

\vspace*{5mm}
$$\frac{}{map{V}{\List{\sigma}}\to \List{\tau}}(\text{E-Map}[~])$$

\vspace*{5mm}
$$\frac{}{map(V_1,~V_2 :: V_3)\to (V_1~V_2)::map(V_1,V_3)}(\text{E-Map}::)$$

\vspace*{5mm}
$$\frac{M\to M'}{map(V, M)\to map(V,M')}(\text{E-Map1})$$

\vspace*{5mm}
$$\frac{M_1\to M'_1}{map(M_1, M_2)\to map(M'_1,M_2)}(\text{E-Map2})$$
\end{multicols}

\newpage
\subsection{Ejercicio 19}
\paragraph{1)}
\begin{multicols}{2}
$$\frac{\judgeType{\Gamma,x_1:\sigma_1,\dots,x_n:\sigma_n}{M}{\tau}}{\judgeType{\Gamma}{\mu x_1:\sigma_1,\dots,x_n:\sigma_n.M}{\tau}}(\text{T-}\mu)$$

\vspace*{5mm}
$$\frac{\judgeType{\Gamma}{M}{\{x_1:\sigma_1,\dots,x_n:\sigma_n\}\to\tau}\hspace*{5mm}\judgeType{\Gamma}{N}{\sigma_i}\hspace*{5mm}\text{para } n > 1}{\judgeType{\Gamma}{M~\#_i~N}{\{x_1:\sigma_1,\dots,x_{i-1}:\sigma_{i-1},x_{i+1}:\sigma_{i+1},\dots,x_n:\sigma_n\}\to\tau}}(\text{T-}\#_i)$$

\vspace*{5mm}
$$\frac{\judgeType{\Gamma}{M}{\{x_1:\sigma_1\}\to\tau}\hspace*{5mm}\judgeType{\Gamma}{N}{\sigma_1}}{\judgeType{\Gamma}{M~\#_1~N}{\tau}}(\text{T-}\#_1)$$
\end{multicols}

\vspace*{5mm}
\paragraph{2)} $V~::=~\dots~|~\mu x_1:\sigma_1,\dots,x_n:\sigma_n.M$
\begin{multicols}{2}
$$\frac{M\to M'}{M~\#_i~N \to M'~\#_i~N}(\text{E-}\#1)$$

\vspace*{5mm}
$$\frac{N\to N'}{V~\#_i~N \to V~\#_i~N'}(\text{E-}\#2)$$\

$$\frac{}{(\mu x_1:\sigma_1.M)~\#_1~V \to \replaceBy{M}{x_1}{V}}(\text{E-}\#3)$$
\vfill

\end{multicols}
\vspace*{5mm}
$$\frac{}{(\mu x_1:\sigma_1,\dots,x_n:\sigma_n.M)~\#_i~V \to \mu x_1:\sigma_1,\dots,x_{i-1}:\sigma_{i-1},x_{i+1}:\sigma_{i+1},\dots,x_n:\sigma_n.(\replaceBy{M}{x_i}{V})}(\text{E-}\#4) \text{ para todo } n > 1$$

\paragraph{3)} $\lambdaAbs{x}{\sigma}{M}\equalDef \mu x:\sigma.M$ y $M_1~M_2\equalDef M_1~\#_1~M_2$

\newpage
\subsection{Ejercicio 20}
Acá usamos la extensión con registros vista en la teórica.

$M~:=~\dots~|~unionReg(M,M)$

Los valores no cambian.

$$\frac{\judgeType{\Gamma}{M}{\{l_i:\sigma_i^{i\in 1..n}\}}\hspace*{5mm}\judgeType{\Gamma}{N}{\{l_i:\sigma_i^{i\in n+1..m}\}}\hspace*{5mm}l_i=l_j \Rightarrow i = j}{\judgeType{\Gamma}{unionReg(M,N)}{\{l_i:\sigma_i^{i\in 1..m}\}}}(T-UnionReg)$$

\begin{multicols}{2}

\vspace*{5mm}
$$\frac{M\to M'}{unionReg(M,N)\to unionReg(M',N)}(E-UnionReg1)$$

\vspace*{5mm}
$$\frac{N\to N'}{unionReg(V,N)\to unionReg(V,N')}(E-UnionReg2)$$

\vspace*{5mm}
$$\frac{}{unionReg(\{l_i:V_i^{i\in 1..n}\},\{l_i:V_i^{i\in n+1..m}\})\to \{l_i:V_i^{i\in 1..m}\}}(E-UnionReg3)$$
\end{multicols}

\subsection{Ejercicio 21}
\begin{multicols}{2}
$\textbf{Not} \equalDef \lambdaAbs{x}{Bool}{\lambdaIf{x}{false}{true}}$

\vspace*{5mm}
$\textbf{And} \equalDef \lambdaAbs{x}{Bool}{\lambdaAbs{y}{Bool}{\lambdaIf{x}{y}{false}}}$

\vfill

$\textbf{Or} \equalDef \lambdaAbs{x}{Bool}{\lambdaAbs{y}{Bool}{\lambdaIf{x}{true}{y}}}$

\vspace*{5mm}
$\textbf{Xor} \equalDef \lambdaAbs{x}{Bool}{\textbf{Or}~(\textbf{And}~x~(\textbf{Not}~y))~(\textbf{And}~(\textbf{Not}~x)~y)}$
\end{multicols}

\subsection*{Ejercicio 22}
\paragraph{1)}

\begin{multicols}{2}
$$\frac{\judgeType{\Gamma}{M_i}{\sigma_i\to\tau}\hspace*{5mm}i\in1..n\hspace*{5mm}\sigma_i=\sigma_j\Rightarrow i =j }{\judgeType{\Gamma}{[(M_1,\dots,M_n)]}{Union(\sigma_1,\dots,\sigma_n)_\tau}}(T-Union)$$

\vspace*{5mm}
$$\frac{\judgeType{\Gamma}{N}{\sigma_i}\hspace*{5mm}\judgeType{\Gamma}{M}{Union(\sigma_1,\dots,\sigma_n)_\tau}\hspace*{5mm}\text{para algún }  i\in1..n }{\judgeType{\Gamma}{M~N}{\tau}}(T-UnionApp)$$
\end{multicols}

\newpage
\paragraph{2)} $\Gamma = \{y: Nat\}$

\vspace*{5mm}
    \begin{scprooftree}
   \def\extraVskip{5pt}
        
        \AxiomC{$y:Nat\in\Gamma$}
    \RightLabel{T-Var}
    \UnaryInfC{$\judgeType{\Gamma}{y}{Nat}$}

        \AxiomC{Demostrado en (1)}
        \UnaryInfC{$\judgeType{\Gamma}{\lambdaAbs{x}{ Bool}{y}}{ Bool\to Nat}$}
          
        \AxiomC{Demostrado en (2)}
        \UnaryInfC{$\judgeType{\Gamma}{\lambdaAbs{x}{ Nat}{x}}{ Nat\to Nat}$}
        
        \AxiomC{$ Bool\neq Nat$}
    \RightLabel{T-Union}    \TrinaryInfC{$\judgeType{\Gamma}{[(\lambdaAbs{x}{ Bool}{y},\lambdaAbs{x}{ Nat}{x})]}{Union( Nat, Bool)_{ Nat}}$}

    \AxiomC{$ Nat\in\{ Bool, Nat\}$}
    
\RightLabel{T-UnionApp}
\TrinaryInfC{$\judgeType{\Gamma}{[(\lambdaAbs{x}{ Bool}{y},\lambdaAbs{x}{Nat}{x})]~y}{ Nat}$}
\end{scprooftree}

\vspace*{1cm}
\begin{multicols}{2}

(1)    \begin{scprooftree}
   \def\extraVskip{5pt}
                \AxiomC{$y: Nat\in\Gamma,x: Bool$}
            \RightLabel{T-Var}
            \UnaryInfC{$\judgeType{\Gamma,x: Bool}{y}{ Nat}$}
        \RightLabel{T-Abs}
        \UnaryInfC{$\judgeType{\Gamma}{\lambdaAbs{x}{ Bool}{y}}{ Bool\to Nat}$}
    \end{scprooftree}

(2)    \begin{scprooftree}
   \def\extraVskip{5pt}
                   \AxiomC{$x: Nat\in\Gamma,x: Nat$}
               \RightLabel{T-Var}
               \UnaryInfC{$\judgeType{\Gamma,x: Nat}{x}{ Nat}$}
           \RightLabel{T-Abs}
           \UnaryInfC{$\judgeType{\Gamma}{\lambdaAbs{x}{ Nat}{x}}{ Nat\to Nat}$}
    \end{scprooftree}
\end{multicols}

\paragraph{3)} $V~:=~\dots~|~[(V_1,\dots,V_n)]$. Cada valor, sera una expresión lamda que podremos reducir cuando la apliquemos a otra expresión.

\paragraph{4)}
\begin{multicols}{2}
$$\frac{M_j\to M_j'}{[(V_i^{i\in 1..j-1}, M_j, M_i^{i\in j+1..n})]\to[(V_i^{i\in 1..j-1}, M'_j, M_i^{i\in j+1..n})]}(\text{E-Un1})$$

$$\frac{\forall~\Gamma~\hspace*{5mm}\judgeType{\Gamma}{V_i}{\sigma_i\to\tau}\hspace*{5mm}\judgeType{\Gamma}{V}{\sigma_i}\hspace*{5mm}\text{para algún }i\in 1..n}{[(V_i^{i\in 1..n})]~V\to V_i~V}(\text{E-AppUnion})$$
\end{multicols}

\subsection{Ejercicio 23}
Usamos la extensión del ejercicio \ref{p2:e17}

\paragraph{1)}
\begin{multicols}{2}
$head_\sigma \equalDef \lambdaAbs{xs}{[\sigma]}{\lambdaListCase{xs}{\bot_\sigma}{h}}$ 

$tail_\sigma \equalDef \lambdaAbs{xs}{[\sigma]}{\lambdaListCase{xs}{\List{\sigma}}{t}}$
\end{multicols}

\paragraph{2)}
$iterate_\sigma\equalDef \lambdaFix{(\lambdaAbs{g}{(\sigma\to\sigma)\to\sigma\to[\sigma]}{\lambdaAbs{f}{\sigma\to\sigma}{\lambdaAbs{x}{\sigma}{x::(g~f~(f~x))}}})}$

\paragraph{3)} 

$isNull_\sigma \equalDef \lambdaAbs{xs}{[\sigma]}{\lambdaListCase{xs}{true}{false}}$

\vspace*{5mm}
$zip_{\sigma,\tau}\equalDef \lambdaFix{(\lambdaAbs{g}{[\sigma]\to[\tau]\to[\sigma\times\tau]}{\lambdaAbs{xs}{[\sigma]}{\lambdaAbs{ys}{[\tau]}{\lambdaIf{\textbf{Or}~(isNull_\tau~ys)~(isNull_\sigma~xs)}{\List
{\sigma\times\tau}}{(head_\tau~ys,head_\sigma~xs):(g~(tail_\sigma~xs)~(tail_\tau~ys)}}}})}$

\paragraph{4)}
$take_\sigma\equalDef \lambdaFix{(\lambdaAbs{f}{ Nat\to[\sigma]}{\lambdaAbs{n}{ Nat}{\lambdaAbs{xs}{[\sigma]}{\lambdaIf{\textbf{Or}~isZero(n)~(isNull~xs)}{\List{\sigma}}{(head~xs)::(f~pred(n)~(tail~xs))}}}})}$

\subsection{Ejercicio 24}
\paragraph{1)}
\begin{multicols}{2}
$$\frac{\judgeType{\Gamma}{M}{\sigma}}{\judgeType{\Gamma}{detener(M)}{\text{det}(\sigma)}}(\text{T-Det})$$

\vspace*{5mm}
$$\frac{\judgeType{\Gamma}{M}{\text{det}(\sigma)}}{\judgeType{\Gamma}{continuar(M)}{\sigma}}(\text{T-Cont})$$

\vspace*{5mm}
$$\frac{\judgeType{\Gamma}{M}{\text{det}(\sigma)\to\tau}~\hspace*{5mm}\judgeType{\Gamma}{N}{\sigma}}{\judgeType{\Gamma}{M~N}{\tau}}(\text{T-DetApp})$$
\end{multicols}

\paragraph{2)} $\Gamma = \{y:Bool\}$

\vspace*{5mm}
\begin{scprooftree}
\def\extraVskip{5pt}

           \AxiomC{Demostrado en (1)}
       \UnaryInfC{$\judgeType{\Gamma}{\lambdaAbs{x}{\text{det}(Bool)}{\lambdaIf{y}{continuar(x)}{false}}}{\text{det}( Bool)\to Bool}$}
            \AxiomC{}
        \RightLabel{T-Zero}
        \UnaryInfC{$\judgeType{\Gamma}{0}{ Nat}$}
        
    \RightLabel{T-isZeroZero}
    \UnaryInfC{$\judgeType{\Gamma}{isZero(0)}{ Bool}$}
\RightLabel{T-DetApp}
\BinaryInfC{$\judgeType{\Gamma}{(\lambdaAbs{x}{\text{det}(Bool)}{\lambdaIf{y}{continuar(x)}{false}})~isZero(0)}{ Bool}$}
\end{scprooftree}

\vspace*{1cm}
(1)\begin{scprooftree}
            \AxiomC{$y: Bool\in\Gamma,x:\text{det}( Bool)$}
        \RightLabel{T-Var}
        \UnaryInfC{$\judgeType{\Gamma,x:\text{det}( Bool)}{y}{ Bool}$}
        
                \AxiomC{$x:\text{det}( Bool)\in\Gamma,x:\text{det}( Bool)$}
            \RightLabel{T-Var}
            \UnaryInfC{$\judgeType{\Gamma,x:\text{det}( Bool)}{x}{\text{det}( Bool)}$}
        \RightLabel{T-Cont}
        \UnaryInfC{$\judgeType{\Gamma,x:\text{det}( Bool)}{continuar(x)}{ Bool}$}   

            \AxiomC{}
        \RightLabel{T-False}
        \UnaryInfC{$\judgeType{\Gamma,x:\text{det}( Bool)}{false}{ Bool}$}
        
    \RightLabel{T-If}
    \TrinaryInfC{$\judgeType{\Gamma,x:\text{det}( Bool)}{\lambdaIf{y}{continuar(x)}{false}}{ Bool}$}
\RightLabel{T-Abs}
\UnaryInfC{$\judgeType{\Gamma}{\lambdaAbs{x}{\text{det}(Bool)}{\lambdaIf{y}{continuar(x)}{false}}}{\text{det}( Bool)\to Bool}$}
\end{scprooftree}

\newpage
\paragraph{3)} $V~::=~...~|~detener(M)$

\begin{multicols}{2}

$$\frac{M\to M'}{continuar(M)\to continuar(M')}(\text{E-Cont})$$

\hspace*{5mm}
$$\frac{}{continuar(detener(M))\to M}(\text{E-Cont1})$$

$$\frac{\judgeType{\Gamma}{N}{\sigma}}{(\lambdaAbs{x}{\text{det}(\sigma)}{M})~N \to (\lambdaAbs{x}{\text{det}(\sigma)}{M})~detener(N)}(\text{E-AppDet})$$
\end{multicols}

Debemos modificar la regla E-App2 y E-AppAbs para mantener el determinismo:

\begin{multicols}{2}
$$\frac{\judgeType{\Gamma}{V}{\sigma\to\tau}\hspace*{5mm}\judgeType{\Gamma}{N}{\sigma}\hspace*{5mm}M_2\to M_2'}{V~M_2\to V~M'_2}(\text{T-App2})$$

$$\frac{\judgeType{\Gamma}{V}{\sigma}}{(\lambdaAbs{x}{\sigma}{M})~V\to \replaceBy{M}{x}{V}}(\text{T-Abs2})$$
\end{multicols}
\end{document}

