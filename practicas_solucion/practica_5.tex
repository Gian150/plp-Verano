\documentclass[10pt,a4paper]{article}

\usepackage[pdftex,
pdfauthor={Gianfranco Zamboni},
pdftitle={Resumen: Paradigmas de Lenguajes de Programación},
pdfsubject={},
pdfkeywords={Resumen , Computacion, FCEyN, UBA, Paradigmas de Lenguajes de Programación, Imperativo, Funcional, Cálculo Lambda, Programación Orientada a Objetos, Objetos, Programación Lógica},
pdfproducer={Latex with hyperref},
pdfcreator={pdflatex}]{hyperref}

\usepackage{amsmath}
\usepackage{ amssymb }
\usepackage{bussproofs}

\usepackage[spanish]{babel}


\usepackage[utf8]{inputenc} % para poder usar tildes en archivos UTF-8
\usepackage{graphicx}
\usepackage{xcolor}
\usepackage{pifont}

\usepackage{lscape}
\usepackage{minted}
\usepackage{a4wide} % márgenes un poco más anchos que lo usual
\usepackage[titletoc,toc,page]{appendix}
\usepackage{tikz}
\usepackage{forest}
\usepackage{multicol}

\setlength{\columnsep}{1cm}

\ifthenelse{\paperwidth < \paperheight}{\usepackage{fancyhdr}
\pagestyle{fancy}

%\renewcommand{\chaptermark}[1]{\markboth{#1}{}}
\renewcommand{\sectionmark}[1]{\markright{\thesection\ - #1}}

\fancyhf{}

\fancyhead[LO]{Sección \rightmark} % \thesection\ 
\fancyfoot[LO]{\small{Paradigmas de lenguajes de programación}}
\fancyfoot[RO]{\thepage}
\renewcommand{\headrulewidth}{0.5pt}
\renewcommand{\footrulewidth}{0.5pt}
\setlength{\hoffset}{-0.25in}
\setlength{\textwidth}{16cm}
%\setlength{\hoffset}{-1.1cm}
%\setlength{\textwidth}{16cm}
\setlength{\headsep}{0.5cm}
\setlength{\textheight}{25cm}
\setlength{\voffset}{-0.4in}
\setlength{\headwidth}{\textwidth}
\setlength{\headheight}{13.1pt}

\renewcommand{\baselinestretch}{1.1}  % line spacing}{\usepackage{fancyhdr}
\pagestyle{fancy}

%\renewcommand{\chaptermark}[1]{\markboth{#1}{}}
\renewcommand{\sectionmark}[1]{\markright{\thesection\ - #1}}

\fancyhf{}

\fancyhead[LO]{\rightmark} % \thesection\ 
\fancyfoot[LO]{\small{PLP - Prácticas}}
\fancyfoot[RO]{\thepage}
\renewcommand{\headrulewidth}{0.5pt}
\renewcommand{\footrulewidth}{0.5pt}
\setlength{\hoffset}{-0.25in}
\setlength{\textwidth}{25cm}
%\setlength{\hoffset}{-1.1cm}
%\setlength{\textwidth}{16cm}
\setlength{\headsep}{0.5cm}
\setlength{\textheight}{16cm}
\setlength{\voffset}{-0.4in}
\setlength{\headwidth}{\textwidth}
\setlength{\headheight}{13.1pt}

\renewcommand{\baselinestretch}{1.1}  % line spacing
}





\newenvironment{centrado}
    {
     \begin{center}
     \begin{minipage}{0.8\textwidth}
 }    
    {
     \end{minipage}
     \end{center}
    }

\newcommand{\rel}{\ensuremath{\mathcal{R}}}

\newcommand{\equalDef}{\overset{def}{=}}
\newcommand{\equalDot}{\overset{\cdot}{=}}

\newcommand{\lambdaAbs}[3]{\lambda #1: #2 . #3}
\newcommand{\lambdaAssign}[2]{#1~:=~#2}
\newcommand{\lambdaApp}[2]{#1~#2}
\newcommand{\lambdaIf}[3]{if~ #1~ then~ #2~ else~ #3}
\newcommand{\lambdaTrue}{true}
\newcommand{\lambdaFalse}{false}
\newcommand{\lambdaLet}[4]{let~#1:#2 = #3~in~#4}
\newcommand{\lambdaRef}[1]{ref~#1}
\newcommand{\lambdaVar}[1]{#1}
\newcommand{\lambdaValue}[1]{\color{red}#1\color{black}}
\newcommand{\lambdaFix}[1]{fix~#1}

\newcommand{\lambdaAbsI}[2]{\lambda #1. #2}



\newcommand{\blue}[1]{\color{blue}#1\color{black}}
\newcommand{\replaceBy}[3]{#1\{#2\leftarrow#3\}}

\newcommand{\judgeType}[3]{#1\triangleright #2 : #3}


\newenvironment{scprooftree}[1]%
{\gdef\scalefactor{#1}\begin{center}\proofSkipAmount \leavevmode}%
    {\scalebox{\scalefactor}{\DisplayProof}\proofSkipAmount \end{center} }


\tikzset{
    every leaf node/.style={text=red, align=center},
    every tree node/.style={text=blue, align=center},
}

\forestset{tikzQtree/.style={for tree={if n children=0{
                node options=every leaf node/.try}{node options=every tree node/.try}, text centered}}}
                
                
\DeclareMathOperator{\Erase}{Erase}
\DeclareMathOperator{\Nat}{Nat}
\DeclareMathOperator{\Bool}{Bool}
\DeclareMathOperator{\Union}{Union}

\newcommand{\WFunc}{\mathbb{W}}

\newcommand{\red}[1]{{\color{red}#1}}%\renewcommand{\appendixtocname}{Apéndices}
\newcommand{\green}[1]{{\color{green!40!black}#1}}%\renewcommand{\appendixpagename}{Apéndices}





\setcounter{section}{5}

%-----------------------------------------------
%-----------------------------------------------
\usepackage[utf8]{inputenc} % para poder usar tildes en archivos UTF-8
\usepackage[spanish]{babel} % para que comandos como \today den el resultado en castellano
\usepackage{amsmath}
\usepackage{listings}
\usepackage{color}
\usepackage{algorithm}
\usepackage{algpseudocode}
\usepackage{mathtools}
\DeclarePairedDelimiter\ceil{\lceil}{\rceil}
\DeclarePairedDelimiter\floor{\lfloor}{\rfloor}
\newcommand{\quotes}[1]{``#1''}


\usepackage{graphicx}
\usepackage{wrapfig}

\usepackage{tikz}
\usetikzlibrary{graphs}

\usepackage{listings}
\usepackage{color}
\definecolor{lightgray}{rgb}{.9,.9,.9}
\definecolor{white}{rgb}{1,1,1}
\definecolor{darkgray}{rgb}{.4,.4,.4}
\definecolor{purple}{rgb}{0.65, 0.12, 0.82}

\lstdefinelanguage{JavaScript}{
  keywords={typeof, new, true, false, catch, function, return, null, catch, switch, var, if, in, while, do, else, case, break},
  keywordstyle=\color{blue}\bfseries,
  ndkeywords={class, export, boolean, throw, implements, import, this},
  ndkeywordstyle=\color{darkgray}\bfseries,
  identifierstyle=\color{black},
  sensitive=false,
  comment=[l]{//},
  morecomment=[s]{/*}{*/},
  commentstyle=\color{purple}\ttfamily,
  stringstyle=\color{red}\ttfamily,
  morestring=[b]',
  morestring=[b]"
}

\lstset{
   language=JavaScript,
   backgroundcolor=\color{white},
   extendedchars=true,
   basicstyle=\footnotesize\ttfamily,
   showstringspaces=false,
   showspaces=false,
   numberstyle=\footnotesize,
   numbersep=9pt,
   tabsize=2,
   breaklines=true,
   showtabs=false,
   captionpos=b
}
%-----------------------------------------------
%-----------------------------------------------

\begin{document}
\title{PLP - Práctica 5: Programación Orientada a Objetos}

\date{\today}

\author{Zamboni, Gianfranco}

\maketitle
\setcounter{page}{1}

\section*{\centering Programación en JS}
\subsection{Ejercicio 1}
\subsubsection{Objeto:}
    \begin{lstlisting}
    function c1i(r,i){
    	this.r = r;
    	this.i = i;
    }
    \end{lstlisting}
\subsubsection{Modificando receptor:}
    \begin{lstlisting}
    c1i.sumar = function (complejo) {
    	this.r += complejo.r;
    	this.i += complejo.i;
    }
    \end{lstlisting}
\subsubsection{Sin modificar receptor:}
    \begin{lstlisting}
    c1i.sumar2 = function (complejo) {
    	return (new c1i(this.r + complejo.r, this.i + complejo.i));
    }
    \end{lstlisting}
\subsubsection{Sumar.Sumar:}
    El resultado era indefinido, entonces:
    \begin{lstlisting}
    c1i.prototype.sumar2 = function (complejo){
    	return (new c1i(this.r + complejo.r, this.i + complejo.i));
    }
    \end{lstlisting}
\subsubsection{Restar:}
    \begin{lstlisting}
    var c = c1i.sumar2(c1i);
    c.protptype.restar = function(complejo){
    	return new c1i(this.r - complejo.r,this.i - complejo.i);	
    }
    \end{lstlisting}
    c1i.restar(c): C1i no tiene definida la operación restar.
\subsubsection{Mostrar:}
    \begin{lstlisting}
    c1i.prototype.mostrar = function () {
    	if(this.i != 1){
    		console.log(this.r+" "+this.i+"i");
    	}else{
    		console.log(this.r+" "+"i");
    	}
    }
    \end{lstlisting}
    C tiene definida la operación mostrar por herencia.
\subsection{Ejercicio 2: a,b y c.}
    \begin{lstlisting}
    function t(a,b) {
    	this.ite = a;
    	this.mostrar = "Verdadero";
    	this.not = "Falso";
    	this.and = function(val){ 
    		if (this.ite == val.ite){
    			return "Verdadero";	
    		}else{
    			return "Falso";
    		}};
    }
    function f(a,b) {
    	this.ite = b;
    	this.mostrar = "Falso";
    	this.not = "Verdadero";
    	this.and = function(val) {
    		return "Falso";
    	}
    }
    \end{lstlisting}
    % console.log((new t(t,f)).mostrar);
    % console.log((new f(t,f)).mostrar);
    % var unTrue = new t(t,f);
    % var unFalse = new f(t,f);
    % console.log(unTrue.and(unTrue));
    % console.log(unTrue.and(unFalse));
    % console.log(unFalse.and(unTrue));
    % console.log(unFalse.and(unFalse));
\subsection{Ejercicio 3: a y b.}
    \begin{lstlisting}
    var cero = {};
    cero.esCero = true;
    cero.succ = function(){ 
    	return new sucesor(this);
    };
    cero.toNumber = 0;
    function sucesor(objeto) {
    	this.esCero = false;
    	this.succ = function(){ 
    		return new sucesor(this);
    	};
    	this.pred = objeto;
    	this.toNumber = objeto.toNumber+1;
    };
    \end{lstlisting}
    % var uno = new sucesor(cero);
    % var dos = new sucesor(uno);
    % console.log(cero);
    % console.log(uno.pred.esCero);
    % console.log(dos.pred.esCero);
    % console.log(dos.toNumber);
\subsubsection{///C}
    \begin{lstlisting}
    \end{lstlisting}
\subsection{Ejercicio 4}
\subsubsection{}
    \begin{lstlisting}
    \end{lstlisting}
\subsubsection{}
    \begin{lstlisting}
    \end{lstlisting}
\subsubsection{}
    \begin{lstlisting}
    \end{lstlisting}
\subsubsection{}
    \begin{lstlisting}
    \end{lstlisting}
\subsection{Ejercicio 5}
\subsection{Ejercicio 6}
\subsubsection{}
    \begin{lstlisting}
    \end{lstlisting}
\subsubsection{}
    \begin{lstlisting}
    \end{lstlisting}
\subsection{Ejercicio 7}
\subsubsection{}
    \begin{lstlisting}
    \end{lstlisting}
\subsubsection{}
    \begin{lstlisting}
    \end{lstlisting}
\subsubsection{}
    \begin{lstlisting}
    \end{lstlisting}
\subsubsection{}
    \begin{lstlisting}
    \end{lstlisting}

\section*{\centering Cálculo de Objetos}

\subsection{Ejercicio 8}
\subsubsection{}
    \begin{lstlisting}
    \end{lstlisting}
\subsubsection{}
    \begin{lstlisting}
    \end{lstlisting}
\subsection{Ejercicio 9}
\subsubsection{}
    \begin{lstlisting}
    \end{lstlisting}
\subsubsection{}
    \begin{lstlisting}
    \end{lstlisting}
\subsubsection{}
    \begin{lstlisting}
    \end{lstlisting}
\subsection{Ejercicio 10}
\subsection{Ejercicio 11}
\subsubsection{}
    \begin{lstlisting}
    \end{lstlisting}
\subsubsection{}
    \begin{lstlisting}
    \end{lstlisting}
\subsection{Ejercicio 12}
\subsubsection{}
    \begin{lstlisting}
    \end{lstlisting}
\subsubsection{}
    \begin{lstlisting}
    \end{lstlisting}
\subsubsection{}
    \begin{lstlisting}
    \end{lstlisting}
\subsubsection{}
    \begin{lstlisting}
    \end{lstlisting}
\subsection{Ejercicio 13}
\subsubsection{}
    \begin{lstlisting}
    \end{lstlisting}
\subsubsection{}
    \begin{lstlisting}
    \end{lstlisting}
\subsubsection{}
    \begin{lstlisting}
    \end{lstlisting}
\subsubsection{}
    \begin{lstlisting}
    \end{lstlisting}
\subsubsection{}
    \begin{lstlisting}
    \end{lstlisting}
\subsubsection{}
    \begin{lstlisting}
    \end{lstlisting}

\end{document}
