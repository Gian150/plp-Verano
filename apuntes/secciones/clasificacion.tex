\section{Clasificación}
Se usan \textbf{clases} que modelan \textbf{conceptos abstractos} del dominio del problema a resolver y definen el comportamiento y la forma de un conjunto de objetos (sus \textbf{instancias}). Todo \textbf{objeto} es una instancia de una clase.

\paragraph{Componentes de una clase}
Todas las clases tienen un \textbf{nombre} que usado para referenciarse a la misma. Dentro de ellas se definen las variables de instancias (colaboradores internos) de los objetos) y los métodos que saben responder esas instancias (sus nombres, sus parámetros y su cuerpo).

\subsection{Self/This}
Todas las clases tienen definida una pseudovariable que, durante la evaluación de un método, referencia al receptor del mensaje que activó dicha evaluación. No puede ser modificada por medio de una asignación y se liga automáticamente al receptor cuando comienza la evaluación del método.

\begin{minted}{smalltalk}
!classDefinition: #Node 
	instanceVariableNames: 'leftchild, rightchild'
	...
sum: 
	^ (self leftchild) sum + (self rightchild) ! !
	
!classDefinition: #Leaf 
	instanceVariableNames: 'value'
	...
sum: 
	^self value ! !

\end{minted}

Vemos que los métodos acceden a sus variables de instancia, enviándose a si mismos el mensaje asociado a cada una de ellas. En muchos lenguajes, para facilitar la escritura de un programa, la mención de \texttt{self} se hace implícitamente.

\subsection{Jerarquía de clases}
Cuando escribimos un programa en este paradigma, es común que creemos nuevas clases que extiendan a las ya existentes con nuevas variables de instancia o clase o que modifiquen el comportameiento de unos o varios métodos.

Para evitar tener que escribir toda una clase de cero, hacemos que la clase que estamos creando \textbf{herede} los atributos y los métodos de la clase pre-existente (la \textbf{super-clase}) que queremos extender. De esta forma, la nueva clase tendrá todo lo que tenía la super-clase y, además, las modificaciones que nosotros querramos agregarle.

La herencia define una relación transitiva. Si una clase $A$ tiene como super-tipo a otra clase $B$ y $C$ es super-tipo de $B$, entonces $C$ tambien es super-tipo de $A$. Llamaremos \textbf{ancestros} a todos los supertipos de $A$ y \textbf{descendientes} a todos los tipos que tienen a $A$ como ancestro.

\subsection{Tipos de herencia}
Hay dos tipos de herencia: \textbf{simple} y \textbf{múltiple}. La herencia simple permite que una clase tenga una única clase padre, mientras que la herencia múltiple deja que una clase tenga varios padres.

Además, todas las clases deben heredar de una clase \texttt{Object} que es la raíz del árbol o cadena de herencias.

La mayoría de los lenguajes orientados a objetos utilizan la herencia simple ya que la herencia múltiple complica el proceso de method dispath (que asocia los mensajes de un objeto con sus respectivos métodos). 

Supongamos que tenemos dos clases $A$ y $B$ incomparables y una clase $C$ que es subclase de $A$ y $B$ simultaneamente. Si $A$ y $B$ definen (o heredan) dos métodos diferentes para un mismo mensaje $m$, entonces cuando enviemos dicho mensaje a $C$ deberemos decidir cuál de los dos métodos asociados debemos evaluar. Esta selección se puede realizar de dos formas:
\begin{itemize}
\item Estableciendo un \textbf{orden de búsqueda} sobre las superclases de un clase asignando un nivel de prioridad a cada una de ellas.
\item U obligando al programador a \textbf{redefinir} el método en $C$ si $C$ hereda dos métodos distintos para el mismo mensaje.
\end{itemize}

\subsubsection{Method Dispatch}

Como dijimos, el \textbf{method Dispatch} es el método mediante el cual asociamos un mensaje a su método correspondiente en el objeto. Por lo general, este método se realiza de manera dinámica, es decir se realizan durante tiempo de ejecución. Sin embargo, dependiendo del contexto, hay situaciones en las que realizar este proceso de manera estática es necesario. 

Un ejemplo de esto, es cuando el lenguaje nos permite hacer uso de \texttt{super}, una pseudovariable que \textbf{referencia al objeto que recibe el mensaje} y \textbf{cambia} su proceso de activación al momento de recibirlo.
Cuando usamos una expresión de la forma \texttt{super msg} en el cuerpo de un método $m$, el \textbf{method lookup} (la búsqueda del método realizada por el method dispath), comienze a realizarse desde el padre de la \textbf{clase anfitriona} de m.

Algunos lenguajes, además, nos permiten pasarle como parámetro una clase a partir de la cual se debe empezar la búsqueda de la siguiente forma: \texttt{super[A] msg}, siempre y cuando $A$ sea un ancestro de la clase anfitriona del método.