\section{Programación funcional en Haskell}
\paragraph{Tipos elementales}
\begin{centrado}
	\begin{minted}{haskell}
1               -- Int          Enteros
'a'             -- Char         Caracteres
1.2             -- Float        Números de punto flotante
True            -- Bool         Booleanos
[1,2,3]         -- [Int]        Listas
(1, True)       -- (Int, Bool)  Tuplas, pares
length          -- [a] -> Int   Funciones
length [1,2,3]  -- Int          Expresiones
\x -> x         -- a -> a       Funciones anónimas
	\end{minted}
\end{centrado}

\paragraph{Guardas}
\begin{centrado}
	\begin{minted}{haskell}
signo n | n >= 0    = True
        | otherwise = False
	\end{minted}
\end{centrado}

\paragraph{Pattern Matching}
\begin{centrado}
	\begin{minted}{haskell}
longitud [] = 0
longitud (x:xs) = 1 + (longitud xs)
	\end{minted}
\end{centrado}

\paragraph{Polimorfismo paramétrico}
\begin{centrado}
	\begin{minted}{haskell}
todosIguales :: Eq a => [a] -> Bool
todosIguales [] = True
todosIguales [x] = True
todosIguales (x:y:xs) = x == y && todosIguales(y:xs)
	\end{minted}
\end{centrado}

\paragraph{Clases de tipo}
\begin{centrado}
	\begin{minted}{haskell}
Eq a    -- Tipos con comparación de igualdad
Num a   -- Tipos que se comportan como los números
Ord a   -- Tipos orden
Show a  -- Tipos que pueden ser representados como strings
	\end{minted}
\end{centrado}

\paragraph{Definición de listas}
\begin{centrado}
	\begin{minted}[breaklines]{haskell}
[1,2,3,4,5]                 -- Por extensión
[1 .. 4]                    -- Secuencias aritméticas
[ x | x <- [1..], esPar x ] -- Por compresión

-- Las listas pueden ser infinitas, solo hay que tener cuidado cuando las usamos. Ejemplo de lista infinita:
	
infinitosUnos :: [Int]
infinitosUnos = 1 : infinitosUnos

puntosDelCuadrante :: [(Int, Int)]
puntosDelCuadrante = [ (x, s-x) | s <- [0..], x <-[0..s] ]
	\end{minted}
\end{centrado}

\paragraph{Funciones de alto orden}
\begin{centrado}
	\begin{minted}[breaklines]{haskell}
mejorSegun :: (a -> a -> Bool) -> [a] -> a
mejorSegun _ [x] = x
mejorSegun f (x : xs) | f x (mejorSegun f xs) = x
                      | otherwise = mejorSegun f xs
\end{minted}
\end{centrado}

\subsection{Otros tipos útiles}
\paragraph{Formula}
\begin{centrado}
	\begin{minted}[breaklines]{haskell}
data Formula = Proposicion String | No Formula 
                                  | Y Formula Formula
                                  | O Formula Formula
                                  | Imp Formula Formula
                                  
foldFormula :: (String -> a) -> (Formula -> a) -> 
               (Formula -> Formula -> a) -> (Formula -> Formula -> a) 
               -> (Formula -> Formula -> a) -> Formula -> a
foldFormula fp fn fy fo fImp form = case form of :
		Proposicion s -> fp s
		No sf -> fn (rec sf)
		Y sf1 sf2 -> fy (rec sf1) (rec sf2)
		O sf1 sf2 -> fo (rec sf1) (rec sf2)
		Impl sf1 sf2 -> fImpl (rec sf1) (rec sf2)
	where rec = foldForm fp fn fy fo fImp
	\end{minted}
	\end{centrado}

\paragraph{Rosetree}
\begin{centrado}
	\begin{minted}[breaklines]{haskell}
data Rosetree = Rose a [Rosetree]
-- Hay varias formas de definir el fold para esta estructura
foldRose :: (a -> [b] -> b) -> Rosetree a -> b
foldRose f ( Rose x l ) = f x ( map ( foldRose f ) l )
	
foldRose2 :: ( a -> c -> b) -> ( b -> c -> c ) -> c 
            -> Rosetree a -> b
foldRose2 g f z (Rose x l) = 
          g x ( foldr f z ( map ( foldRose g f z ) l ) )
		
\end{minted}
\end{centrado}
