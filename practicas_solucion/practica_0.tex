\section{Pre-Práctica de Programación Funcional}

\subsection{Ejercicio 1}
\mintinline{haskell}{null :: Foldable t => t a -> Bool} indica si una estructura está vacía. El tipo a debe ser de la clase Foldable, esto es, son tipos a los que se les puede aplicar la función foldr. La notación "t a" indica que es un tipo parámetrico, es decir, un tipo t que usa a otro tipo a, por ejemplo, si le pasamos a la función una lista de enteros, entonces a = Int y t = [Int]

\vspace*{5mm}
\mintinline{haskell}{head :: [a] -> a} devuelve el primer elemento de una lista.

\vspace*{5mm}
\mintinline{haskell}{tail :: [a] -> [a]} devuelve los últimos elementos de una lista (todos los elementos, salvo el primero).

\vspace*{5mm}
\mintinline{haskell}{init :: [a] -> [a]} devuelve los primeros elementos de una lista (todos los elementos salvo el último).

\vspace*{5mm}
\mintinline{haskell}{last :: [a] -> a} devuelve el último elemento de una lista.


\vspace*{5mm}
\mintinline{haskell}{take :: Int -> [a] -> [a]} devuelve los primeros n elementos de una lista

\vspace*{5mm}
\mintinline{haskell}{drop :: Int -> [a] -> [a]} devuelve los últimos n elementos de una lista

\vspace*{5mm}
\mintinline{haskell}{(++) :: [a] -> [a] -> [a]} concatena dos listas

\vspace*{5mm}
\mintinline{haskell}{concat :: Foldable t => t [a] -> [a]} concatena todas las listas de un contenedor de listas que soporte la operación foldr.

\vspace*{5mm}
\mintinline{haskell}{(!!) :: [a] -> Int -> a} devuelve el elemento de una lista 
\mintinline{haskell}{l} que se encuentra en la \mintinline{haskell}{n}-ésima posición. La numeración comienza desde 0.

\vspace*{5mm}
\mintinline{haskell}{elem :: (Eq a, Foldable t) => a -> t a -> Bool}: Dada una estructura T que soporta la operación foldr y que almacene elementos del tipo a que puedan ser comparados por medio de la igualdad y dado un elemento A de ese tipo, indica si A aparecen en T.

\subsection{Ejercicio 2}
\begin{centrado}
\begin{minted}[breaklines]{haskell}
-- a) La función abs de Prelude ya hace esto
valorAbsoluto :: Float -> Float
valorAbsoluto x | x < 0     = -x
                | otherwise =  x
\end{minted}
\end{centrado}
\begin{centrado}
	\begin{minted}[breaklines]{haskell}
-- b) 
bisiesto :: Int -> Bool
bisiesto x = (x `mod` 4) == 0
	\end{minted}
\end{centrado}
\begin{centrado}
	\begin{minted}[breaklines]{haskell}
--c)
factorial :: Int -> Int
factorial 1 = 1
factorial x = x * factorial (x-1)
	\end{minted}
\end{centrado}
\begin{centrado}
	\begin{minted}[breaklines]{haskell}
cantDivisoresPrimos :: Int -> Int
cantDivisoresPrimos x = length (filter esPrimo (divisores x))
	\end{minted}
\end{centrado}

\begin{centrado}
	\begin{minted}[breaklines]{haskell}
-- Auxiliares 

esPrimo :: Int -> Bool
esPrimo x = length (divisores x) == 2

divisores :: Int -> [Int]
divisores x = [ y | y <- [1..x], x `mod` y == 0 ];
	\end{minted}
\end{centrado}

\subsection{Ejercicio 3}
\begin{centrado}
\begin{minted}[breaklines]{haskell}
--a)
inverso :: Float -> Maybe Float
inverso 0 = Nothing
inverso x = Just (1/x)
\end{minted}
\end{centrado}

\begin{centrado}
\begin{minted}[breaklines]{haskell}
-- b)
aEntero :: Either Int Bool -> Int
aEntero (Left x) = x
aEntero (Right x) | x == True = 1
                  | otherwise = 0
\end{minted}
\end{centrado}

\subsection{Ejercicio 4}
\begin{centrado}
\begin{minted}[breaklines]{haskell}
--a)
limpiar :: String -> String -> String
limpiar xs ys = [ y | y <- ys, not(elem y xs) ]
\end{minted}
\end{centrado}
\begin{centrado}
	\begin{minted}[breaklines]{haskell}
-- b)
difPromedio :: [Float] -> [Float]
difPromedio xs = map (\y -> y - promedio xs) xs 
    where promedio xs = (sum xs) / (genericLength xs)
\end{minted}
\end{centrado}
\begin{centrado}
	\begin{minted}[breaklines]{haskell}
-- c)
todosIguales :: [Int] -> Bool
todosIguales = 
	foldr (\y  rec -> ((length xs == 1) || (y == (head xs))) 
		&& rec) True
\end{minted}
\end{centrado}

\subsection{Ejercicio 5}
\begin{centrado}
\begin{minted}[breaklines, tabsize=4]{haskell}
data AB a = Nil | Bin (AB a) a (AB a)
\end{minted}
\end{centrado}
\begin{centrado}
	\begin{minted}[breaklines, tabsize=4]{haskell}
-- a)
vacioAB:: AB a -> Bool
vacioAB Nil = True
vacioAB (Bin _ _ _) = False
\end{minted}
\end{centrado}
\begin{centrado}
	\begin{minted}[breaklines, tabsize=4]{haskell}
-- b)
negacionAB :: AB Bool -> AB Bool
negacionAB Nil = Nil
negacionAB (Bin l x r) = 
	Bin (negacionAB l) (not x) (negacionAB r)
\end{minted}
\end{centrado}
\begin{centrado}
	\begin{minted}[breaklines, tabsize=4]{haskell}
-- c)
productoAB :: AB Int -> Int
productoAB Nil = 1
productoAB (Bin l x r) = x * (productoAB l) * (productoAB r)
\end{minted}
\end{centrado}