\documentclass[10pt,a4paper]{article}

\usepackage[pdftex,
pdfauthor={Gianfranco Zamboni},
pdftitle={Resumen: Paradigmas de Lenguajes de Programación},
pdfsubject={},
pdfkeywords={Resumen , Computacion, FCEyN, UBA, Paradigmas de Lenguajes de Programación, Imperativo, Funcional, Cálculo Lambda, Programación Orientada a Objetos, Objetos, Programación Lógica},
pdfproducer={Latex with hyperref},
pdfcreator={pdflatex}]{hyperref}

\usepackage{amsmath}
\usepackage{ amssymb }
\usepackage{bussproofs}

\usepackage[spanish]{babel}


\usepackage[utf8]{inputenc} % para poder usar tildes en archivos UTF-8
\usepackage{graphicx}
\usepackage{xcolor}
\usepackage{pifont}

\usepackage{lscape}
\usepackage{minted}
\usepackage{a4wide} % márgenes un poco más anchos que lo usual
\usepackage[titletoc,toc,page]{appendix}
\usepackage{tikz}
\usepackage{forest}
\usepackage{multicol}

\setlength{\columnsep}{1cm}

\ifthenelse{\paperwidth < \paperheight}{\usepackage{fancyhdr}
\pagestyle{fancy}

%\renewcommand{\chaptermark}[1]{\markboth{#1}{}}
\renewcommand{\sectionmark}[1]{\markright{\thesection\ - #1}}

\fancyhf{}

\fancyhead[LO]{Sección \rightmark} % \thesection\ 
\fancyfoot[LO]{\small{Paradigmas de lenguajes de programación}}
\fancyfoot[RO]{\thepage}
\renewcommand{\headrulewidth}{0.5pt}
\renewcommand{\footrulewidth}{0.5pt}
\setlength{\hoffset}{-0.25in}
\setlength{\textwidth}{16cm}
%\setlength{\hoffset}{-1.1cm}
%\setlength{\textwidth}{16cm}
\setlength{\headsep}{0.5cm}
\setlength{\textheight}{25cm}
\setlength{\voffset}{-0.4in}
\setlength{\headwidth}{\textwidth}
\setlength{\headheight}{13.1pt}

\renewcommand{\baselinestretch}{1.1}  % line spacing}{\usepackage{fancyhdr}
\pagestyle{fancy}

%\renewcommand{\chaptermark}[1]{\markboth{#1}{}}
\renewcommand{\sectionmark}[1]{\markright{\thesection\ - #1}}

\fancyhf{}

\fancyhead[LO]{\rightmark} % \thesection\ 
\fancyfoot[LO]{\small{PLP - Prácticas}}
\fancyfoot[RO]{\thepage}
\renewcommand{\headrulewidth}{0.5pt}
\renewcommand{\footrulewidth}{0.5pt}
\setlength{\hoffset}{-0.25in}
\setlength{\textwidth}{25cm}
%\setlength{\hoffset}{-1.1cm}
%\setlength{\textwidth}{16cm}
\setlength{\headsep}{0.5cm}
\setlength{\textheight}{16cm}
\setlength{\voffset}{-0.4in}
\setlength{\headwidth}{\textwidth}
\setlength{\headheight}{13.1pt}

\renewcommand{\baselinestretch}{1.1}  % line spacing
}





\newenvironment{centrado}
    {
     \begin{center}
     \begin{minipage}{0.8\textwidth}
 }    
    {
     \end{minipage}
     \end{center}
    }

\newcommand{\rel}{\ensuremath{\mathcal{R}}}

\newcommand{\equalDef}{\overset{def}{=}}
\newcommand{\equalDot}{\overset{\cdot}{=}}

\newcommand{\lambdaAbs}[3]{\lambda #1: #2 . #3}
\newcommand{\lambdaAssign}[2]{#1~:=~#2}
\newcommand{\lambdaApp}[2]{#1~#2}
\newcommand{\lambdaIf}[3]{if~ #1~ then~ #2~ else~ #3}
\newcommand{\lambdaTrue}{true}
\newcommand{\lambdaFalse}{false}
\newcommand{\lambdaLet}[4]{let~#1:#2 = #3~in~#4}
\newcommand{\lambdaRef}[1]{ref~#1}
\newcommand{\lambdaVar}[1]{#1}
\newcommand{\lambdaValue}[1]{\color{red}#1\color{black}}
\newcommand{\lambdaFix}[1]{fix~#1}

\newcommand{\lambdaAbsI}[2]{\lambda #1. #2}



\newcommand{\blue}[1]{\color{blue}#1\color{black}}
\newcommand{\replaceBy}[3]{#1\{#2\leftarrow#3\}}

\newcommand{\judgeType}[3]{#1\triangleright #2 : #3}


\newenvironment{scprooftree}[1]%
{\gdef\scalefactor{#1}\begin{center}\proofSkipAmount \leavevmode}%
    {\scalebox{\scalefactor}{\DisplayProof}\proofSkipAmount \end{center} }


\tikzset{
    every leaf node/.style={text=red, align=center},
    every tree node/.style={text=blue, align=center},
}

\forestset{tikzQtree/.style={for tree={if n children=0{
                node options=every leaf node/.try}{node options=every tree node/.try}, text centered}}}
                
                
\DeclareMathOperator{\Erase}{Erase}
\DeclareMathOperator{\Nat}{Nat}
\DeclareMathOperator{\Bool}{Bool}
\DeclareMathOperator{\Union}{Union}

\newcommand{\WFunc}{\mathbb{W}}

\newcommand{\red}[1]{{\color{red}#1}}%\renewcommand{\appendixtocname}{Apéndices}
\newcommand{\green}[1]{{\color{green!40!black}#1}}%\renewcommand{\appendixpagename}{Apéndices}





\setcounter{section}{6}


\begin{document}
  \title{PLP - Práctica 6: Resolución en Lógica}

  \date{\today}

  \author{Zamboni, Gianfranco}

  \maketitle
  \setcounter{page}{1}

\section*{Resolución en Lógica Proposicional}
\subsection{Ejercicio 1}
\begin{tabular}{c|c|c}
\textbf{Formula} & \textbf{FNC} & \textbf{FC} \\[1pt]
$p \supset p$ & $\lnot p \lor p$ & $\{\lnot p, p\}$ \\[1pt] 
$(p \land q) \supset p$ & $\lnot p \lor \lnot q \lor p$ & $\{\lnot p, \lnot q, p\}$ \\[1pt]
$(p \lor q) \supset p$ & $(\lnot p \lor p) \land (\lnot q \lor p)$ &  $\{\{\lnot p, p\},\{\lnot q, p\}\}$\\[1pt]
$\lnot(p \iff \lnot p)$ & $\lnot p \lor p$ & $\{\{\lnot p, p\}\}$ \\[1pt]
$\lnot(p\land q) \supset (\lnot p \lor \lnot q) $&  $(p \lor \lnot p \lor \lnot q)\land(q \lor \lnot p \lor \lnot q)$ & $\{\{p, \lnot p, \lnot q\},\{q, \lnot p, \lnot q\} \}$\\[1pt]
$(p \land q) \lor (p \land r)$ & $p \land (p \lor r) \land (q \lor p) \land (q \lor r)$ & $\{\{p\}, \{p, r\}, \{q, p\}, \{q, r\}\}$ \\[1pt]
$(p \land q) \supset r $ & $\lnot p \lor \lnot q \lor r$ & $\{\lnot p,\lnot q, r\}$ \\[1pt]
$ p \supset (q \supset r)$ & $\lnot p \lor \lnot q \lor r$ & $\{\lnot p, \lnot q, r\}$ \\[1pt]
\end{tabular}

\subsection{Ejercicio 2}
\paragraph{I.} Para probar que una fórmula es una tatulogía hay que negarla, escribirla en forma clausal y ver que es insatisfactible agregando resolventes hasta llegar a la resolvente $\Box$. Las tautologías son:
\begin{itemize}
\item $(p \supset p)$. 

Su negación $(p \land \lnot p)$ escrita en forma clausal es $S = \{\{p\},\{\lnot p\}\}$. La resolvente entre $\{p\}$ y $\{\lnot p\}\}$ es $\Box$. Entonces $S$ es insatisfacible.

\item $(p \land q) \supset p$.

Su negación $(p \land q \land \lnot p)$ escrita en forma clausal es $S = \{\{p\},\{q\},\{\lnot p\}\} $. Resolviendo $\{p\}$ y $\{\lnot p\}\}$ obtenemos $\Box$. Entonces $S$ es insatisfactible.

\item $\lnot(p \iff \lnot p)$.

Su forma normal clausal es $(p \lor \lnot p)$ y su negación $(p \land \lnot p)$ que es el primer item, entonces es una tautología.

\item $\lnot(p\land q) \supset (\lnot p \lor \lnot q) $.

Su negación $((\lnot p \lor \lnot q)\land p \land q$ escrita en forma clausal es $S = \{\{\lnot p,\lnot q\},\{p\},\{q\}\}$.

De $\{p\}$ y $\{\lnot p,\lnot q\}$, obtenemos la resolvente $\{\lnot q,\}$ y de $\{\lnot q,\}$ y $\{\ q,\}$ obtenemos $\Box$. Entonces es insatisfactible.
    \end{itemize}

\paragraph{II.} Queremos ver que
$ ((\lnot p \supset q) \land (p \supset q) \land (\lnot p \supset \lnot q)) \supset  (p \land q)$. Osea que debemos negarla y usar resolución para ver que la fórmula negada es insatisfactible.

\subsubsection*{Paso a forma clausal de la fórmula:}

\begin{align*}
\lnot[((\lnot p \supset q) \land (p \supset q) \land (\lnot p \supset \lnot q)) \supset  (p \land q)] &\hspace{5mm}\text{Negación} \\[1pt]
\lnot[\lnot((p \lor q) \land (\lnot p \lor q) \land (p \lor \lnot q)) \lor  (p \land q)] &\hspace{5mm}\text{Eliminación de implicas}\\[1pt]
\lnot\lnot((p \lor q) \land (\lnot p \lor q) \land (p \lor \lnot q)) \land  \lnot(p \land q) & \\[1pt]
(p \lor q) \land (\lnot p \lor q) \land (p \lor \lnot q) \land  (\lnot p \land \lnot q) & \hspace{5mm}\text{FNN y FNC}\\[1pt]
\{\blue{\underbrace{\{p , q\}}_{C_1}} , \blue{\underbrace{\{\lnot p , q\}}_{C_2}} , \blue{\underbrace{\{p , \lnot q\}}_{C_3}} , \blue{\underbrace{\{\lnot p , \lnot q\}}_{C_4}}\} & \hspace{5mm}\text{Forma Clausal}
\end{align*}

\subsubsection*{Resolución:}
\begin{center}
\begin{forest}resolucion,
 [ $\red{\Box}$,
    [$\{q\}$,
        [$C_1$]
        [$C_2$]
    ]
    [$\{\lnot q\}$,
        [$C_3$]
        [$C_4$]
    ]
 ]
\end{forest}    
\end{center}

\subsection{Ejercicio 3}
Valen las siguientes proposiciones:
\begin{multicols}{2}
\begin{itemize}
    \item $p \supset a \leadsto \lnot p \lor a \leadsto \blue{C_1 = \{\lnot p, a\}}$
    \item $\lnot p \supset c \leadsto p\lor c \leadsto \blue{C_2 = \{p,c\}}$
    \item $\lnot(a\land c) \leadsto \lnot a \lor \lnot c \leadsto \blue{C_3 = \{\lnot a, \lnot c\}}$
    \item $a \leadsto \blue{C_4 = \{a\}}$
    \item $\lnot l\leadsto \blue{C_5 = \{\lnot l\}}$
    \vfill\null
\end{itemize}
\end{multicols}

Queremos probar  que vale $(p\land\lnot l)\lor (\lnot p\land l)$ usando resolución. Primero la negamos y la pasamos a forma clausal:

\begin{align*}
    \lnot[(p \land \lnot l)\lor (\lnot p\land l)] &\hspace{5mm}\text{Negación}\\[1pt]
    \lnot(p \land \lnot l)\land \lnot(\lnot p\land l) &\hspace{5mm}\\[1pt]
    (\lnot p \lor l)\land (p \lor \lnot l) &\hspace{5mm}\text{FNN y FNC}\\[1pt]
    \{\blue{\underbrace{\{\lnot p , l\}}_{C_6}},\blue{\underbrace{\{p , \lnot l\}}_{C_7}}\} &\hspace{5mm}\text{Forma Clausal}\\[1pt]
\end{align*}

Y resolvemos:
\begin{center}
\begin{forest}resolucion,
 [ $\red{\Box}$,
    [$\{\lnot a\}$,
        [$\{c\}$,
            [$\{\lnot p\}$,
                [$C_6$]
                [$C_5$]
            ]
            [$C_2$]
        ]
        [$C_3$]
    ]
    [$C_4$]
 ]
\end{forest}    
\end{center}

\newpage
\section*{Unificación en Lógica de Primer Orden}
\subsection{Ejercicio 4}
\begin{multicols}{2}
\begin{enumerate}
    \item $P(f(x)) $ unifica con:
        \begin{enumerate}
        \item $P(f(a))$ si $\sigma = \{a/x\}$
        \end{enumerate}
    \item $P(a) $ unifica con:
        \begin{enumerate}
            \item $P(x)$ si $\sigma =\{a/x\}$
        \end{enumerate}
    \item $P(y) $ unifica con:
        \begin{enumerate}
            \item $P(x) $ si $\sigma = \{y/x\}$
            \item $P(f(a)) $ si $\sigma = \{f(a)/y\}$
        \end{enumerate}
    \vfill\null
    \columnbreak
    \item $Q(x,f(y)) $ unifica con:
    \begin{enumerate}
        \item $Q(f(y),x)$ si $\sigma = \{f(y)/x\}$
    \end{enumerate}
    \item $Q(x,f(z)) $ unifica con:
    \begin{enumerate}
        \item $Q(f(y),x)$ si $\sigma = \{f(y)/x,~y/z\}$
        \item $Q(f(y),f(x))$ si $\sigma = \{f(y)/x,~f(y)/z\}$
        \item $Q(f(y),y)$ si $\sigma = \{f(z)/y,~f(f(z))/x\}$
    \end{enumerate}
    \item $Q(x,f(a)) $ unifica con:
        \begin{enumerate}
            \item $Q(f(y),x) $ si $\sigma = \{a/y,~f(a)/x\}$
            \item $Q(f(y),y) $ si $\sigma = \{ f(a)/y,~f(f(a))/x\}$
        \end{enumerate}
    \end{enumerate}
\end{multicols}

\subsection{Ejercicio 5}
\begin{enumerate}
    \item $f(x,x,y) \doteq f(a,b,z) \goesTo{}{x/a}~f(a,a,y) \doteq f(a,b,z)\goesTo{}{} \falla$ ($a$ no unifica con $b$)
    \item $f(x) \doteq y\goesTo{}{f(x)/y} f (x)\doteq f(x)$ y el MGU es $\{f(x)/y\}$.
    \item $f(g(c,y),x) \doteq f(z,g(z,a)) \goesTo{}{g(c,y)/z} f(g(c,y),x) \doteq f(g(c,y),g(g(c,y),a)) \\[10pt] \goesTo{}{g(g(c,y),a)/x} f(g(c,y),g(g(c,y),a)) \doteq f(g(c,y),g(g(c,y),a))$
    
    $$MGU = \{g(g(c,y),a)/x,~g(c,y)/z\}$$
    \item $f(a) \doteq g(y) \goesTo{}{} \falla$ ($f$ y $g$ son funciones distintas).
    
    \item $f(x) \doteq x \goesTo{}{} \falla$ ($x \in FV(f(x))$)
    \item $g(x,y) \doteq g(f(y),f(x)) \goesTo{}{f(y)/x} g(f(y),y) \doteq g(f(y),f(f(y))) \goesTo{}{} \falla$ ($y \in FV(f(f(y)))$)
\end{enumerate}

\newpage
\section*{Resolución en Lógica de Primer Orden}
\subsection{Ejercicio 6}
\begin{multicols}{2}
\paragraph{I.}
\begin{align*}
\forall x.  \forall y. ( \lnot Q(x,y) \supset \lnot P(x,y))& \\[1pt]
\forall x.  \forall y. ( \lnot \lnot Q(x,y) \lor \lnot P(x,y))& \\[1pt]
\forall x.  \forall y. ( Q(x,y) \lor \lnot P(x,y))&
\end{align*}


\paragraph{II.}
\begin{align*}
\forall x.  \forall y. ((P(x,y) \land Q(x,y)) \supset R(x,y))& \\[1pt]
\forall x.  \forall y. (\lnot (P(x,y) \land Q(x,y)) \lor R(x,y))& \\[1pt]
\forall x.  \forall y. (\lnot P(x,y) \lor \lnot Q(x,y) \lor R(x,y))&
\end{align*}
\vfill\null
\columnbreak
\paragraph{III.}
\begin{align*}
\forall x.  \exists y. (P(x,y) \supset Q(x,y))& \\[1pt]
\forall x.  \exists y. (\lnot P(x,y) \lor Q(x,y))&
\end{align*}

\end{multicols}

\subsection{Ejercicio 7}
\begin{multicols}{2}
\paragraph{I.} 
\begin{align*}
\exists x. \exists y. (x<y) \\[1pt]
\exists y. (c<y) \\[1pt]
c<d &\hspace{5mm}\text{FSK}\\[1pt]
\{\{c<d\}\}
\end{align*}
\vfill\null
\paragraph{II.}
\begin{align*}
\forall x. \exists y. (x<y) & \\[1pt]
\forall x. (x<f(x)) &\hspace{5mm}\text{FSK} \\[1pt]
\{\{x<f(x)\}\} &\hspace{5mm}\text{FC}
\end{align*}
\vfill\null
\end{multicols}
\begin{multicols}{2}
\paragraph{III.} 
\begin{align*}
\forall x. \lnot(P(x) \land \forall y.(\lnot P(y) \lor Q(y))) & \\[1pt]
\forall x. \lnot P(x) \lor \lnot\forall y.(\lnot P(y) \lor Q(y)) & \\[1pt]
\forall x. \lnot P(x) \lor \exists y.\lnot(\lnot P(y) \lor Q(y)) & \\[1pt]
\forall x. \lnot P(x) \lor \exists y.(\lnot\lnot P(y) \land \lnot Q(y))& \\[1pt]
\forall x. \lnot P(x) \lor \exists y. (P(y) \land \lnot Q(y))\\[1pt]
\forall x.\lnot P(x) \lor  (P(c) \land \lnot Q(c)) \\[1pt]
\forall x. (\lnot P(x) \lor  P(c)) \land (\lnot P(x)\lor \lnot Q(c))  \\[1pt]
\{\{\lnot P(x), P(c)\},\{\lnot P(z), \lnot Q(c)\}\} \\[1pt]
\end{align*}
\vfill\null
\columnbreak
\paragraph{IV.}
\begin{align*}
 \exists x. \forall y. (P(x,y) \land Q(x) \land \lnot R(y))&\hspace{5mm}\text{FNN} \\[1pt]
\forall y. (P(c,y) \land Q(c) \land \lnot R(y)) \\[1pt]
\{\{P(c,y)\}, \{Q(c)\}, \{\lnot R(z)\}\}\\[1pt]
\end{align*}
\vfill\null
\end{multicols}

\paragraph{V.}
\begin{align*}
\forall x. (P(x) \land \exists y. (Q(y) \lor \forall z. \exists w. (P(z) \land \lnot Q(w))) \\[1pt]
\forall x. (P(x) \land (Q(f(x)) \lor \forall z. \exists w. (P(z) \land \lnot Q(w))) \\[1pt]
\forall x. (P(x) \land (Q(f(x)) \lor \forall z. (P(z) \land \lnot Q(f(x,z))) \\[1pt]
\forall x.\forall z.  (P(x) \land (Q(f(x)) \lor (P(z) \land \lnot Q(f(x,z))) \\[1pt]
\forall x.\forall z.  (P(x) \land (Q(f(x)) \lor P(z)) \land (Q(f(x) \lor \lnot Q(f(x,z)) \\[1pt]
\{\{ P(x)\} , \{Q(f(x_1)), P(z)\} , \{Q(f(x_2)), \lnot Q(f(x_2,z))\} \}
\end{align*}
       
\subsection{Ejercicio 8}
\paragraph{I.} La cláusula $\{p,\lnot p\}$, si se resuelve consigo misma da la resolvente $\Box$.
\paragraph{II.} Las cláusulas $\{p,q\}$ y $\{\lnot p, \lnot q\}$ arrojan las siguientes resolventes:
\begin{itemize}
    \item Si resolvemos los dos literales, entonces arroja $\Box$.
    \item Si resolvemos solo con $p$, entonces arroja $\{q,\lnot q\}$.
    \item Si resolvemos solo con $q$, entonces arroja $\{p,\lnot p\}$.
\end{itemize}
\paragraph{III.} Las clausulas de la forma$\{p,\lnot p, q\}$ y $\{\lnot p, p,\lnot q\}$ deben unificar los tres literales al mismo tiempo para conseguir la clausula $\Box$.

\subsection{Ejercicio 9}
\paragraph{I.} $\exists x. \forall y. R(x,y) \supset \forall y. \exists x. R(x,y) \goesTo{}{} \exists x.\forall y. [ R(x,y) \supset [\forall w.\exists z. R(z,w) ]]$
\begin{multicols}{2}
\textbf{Negación:}
\begin{align*}
\lnot\exists x. \forall y. [ R(x,y) \supset [\forall w. \exists z. R(z,w) ]] & \\[1pt]
\lnot\exists x. \forall y. [ \lnot R(x,y) \lor [\forall w. \exists z. R(z,w) ]] \\[1pt]
\forall x.\lnot \forall y. [ \lnot R(x,y) \lor [\forall w. \exists z. R(z,w) ]]  \\[1pt]
\forall x.\exists y. \lnot [ \lnot R(x,y) \lor [\forall w. \exists z. R(z,w) ]] \\[1pt]
\forall x.\exists y. [ R(x,y) \land \lnot[\forall w. \exists z. R(z,w) ]] \\[1pt]
\forall x.\exists y. [ R(x,y) \land [\exists w. \lnot\exists z. R(z,w) ]]  \\[1pt]
\forall x.\exists y. [ R(x,y) \land [\exists w. \forall z. \lnot R(z,w) ]] \\[1pt]
\forall x.\exists y. \exists w. \forall z. [ R(x,y) \land \lnot R(z,w) ] \\[1pt]
\forall x.\exists w. \forall z. [ R(x,f(x)) \land \lnot R(z,w) ]  \\[1pt]
\forall x.\forall z. [ R(x,f(x)) \land \lnot R(z,g(x)) ]  \\[1pt]
\{ \{R(x_1,f(x_1))\}, \{\lnot R(z_2,g(x_2))\} \}  \\[1pt]
\end{align*}

\noindent\textbf{Resolución:} 

No podemos aplicar ningún paso de resolución a S porque las cláusulas no unifican:

\noindent$R(x_1,f(x_1)) \doteq R(z,g(x_2)) \goesTo{}{x_1/x_2} \falla$ porque $f(x_2)$ y $g(x_2)$ no unifican.
\vfill\null
\end{multicols}
\paragraph{II.} $\forall x. \exists y. R(x,y) \supset \exists y. \forall x. R(x,y)  \goesTo{}{}  \forall x. \exists y. [R(x,y) \supset [\exists w. \forall z. R(z,w)]]$

\begin{multicols}{2}
\textbf{Negación:}
\begin{align*}
\lnot\forall x. \exists y. [ R(x,y) \supset [\exists w. \forall z. R(z,w) ]]  \\[1pt]
\lnot\exists x. \exists y. [ \lnot R(x,y) \lor [\exists w. \forall z. R(z,w) ]] \\[1pt]
\forall x.\lnot \exists y. [ \lnot R(x,y) \lor [\exists w. \forall z. R(z,w) ]]  \\[1pt]
\exists x.\forall y. \lnot [ \lnot R(x,y) \lor [\exists w. \forall z. R(z,w) ]]  \\[1pt]
\exists x.\forall y. [ R(x,y) \land \lnot[\exists w. \forall z. R(z,w) ]]  \\[1pt]
\exists x.\forall y. [ R(x,y) \land [\forall w. \lnot\forall z. R(z,w) ]]  \\[1pt]
\exists x.\forall y. [ R(x,y) \land [\forall w. \exists z. \lnot R(z,w) ]] \\[1pt]
\exists x.\forall y. \forall w. \exists z. [ R(x,y) \land \lnot R(z,w) ] \\[1pt]
\forall y. \forall w. \exists z. [ R(c,y) \land \lnot R(z,w) ] \\[1pt]
\forall y. \forall w.[ R(c,y) \land \lnot R(f(y,w),w) ] \\[1pt]
\{ \{R(c,y_1)\},\{\lnot R(f(y_2,w_2),w_2)\}\} \\[1pt]
\end{align*}

\noindent\textbf{Resolución:} 

No podemos aplicar ningún paso de resolución a S  porque $c$ y $f(y_2,w_2)$ no unifican.

\end{multicols}

\newpage
\paragraph{III.} $\exists x. [P(x) \supset \forall x. P(x)] \goesTo{}{} \exists x. [P(x) \supset \forall y. P(y)]$
\begin{multicols}{2}
\textbf{Negación:}
\begin{align*}
\lnot\exists x. [P(x) \supset \forall y. P(y)] & \\[1pt]
\lnot\exists x. [\lnot P(x) \lor \forall y. P(y)] &\\[1pt]
\forall x. \lnot[\lnot P(x) \lor \forall y. P(y)] & \\[1pt]
\forall x. [P(x) \land \lnot\forall y. P(y)]  \\[1pt]
\forall x. [P(x) \land \lnot\forall y. P(y)]  \\[1pt]
\forall x. [P(x) \land \exists y. \lnot P(y)]  \\[1pt]
\forall x. [P(x) \land \exists y. \lnot P(y)]  \\[1pt]
\forall x.\exists y. [P(x) \land  \lnot P(y)]  \\[1pt]
\forall x.[P(x) \land  \lnot P(f(x))]\\[1pt]
\{\{P(x_1)\}, \{\lnot P(f(x_2))\}\}  \\[1pt]
\end{align*}

\textbf{Resolución:}
\begin{center}
	\begin{forest} resolucion,
[$\red{\Box}$ 
	[$\{P(x_1)\}$,edge label={node[midway,right] {$x_1\leftarrow f(x_2)$}}]
    [$\{\lnot P(f(x_2))\}$]
]
	\end{forest}
\end{center}
La negación de la formula es insatisfactible cuando $\sigma = \{ f(x_2)/x\}$ por lo que la fórmula es valida.
\end{multicols}


\paragraph{IV.} $\exists x.[P(x) \lor Q(x)] \supset [\exists x.P(x) \lor \exists x.Q(x)] \goesTo{}{} \exists x.[P(x) \lor Q(x)] \supset [\exists y.P(y) \lor \exists z.Q(z)]$

\begin{multicols}{2}
\textbf{Negación:}
\begin{align*}
\lnot\exists x.[P(x) \lor Q(x)] \supset [\exists y.P(y) \lor \exists z.Q(z)] & \\[1pt]
\lnot\exists x.\lnot [P(x) \lor Q(x)] \lor [\exists y.P(y) \lor \exists z.Q(z)] & \\[1pt]
\forall x.\lnot\lnot [P(x) \lor Q(x)] \land \lnot[\exists y.P(y) \lor \exists z.Q(z)] & \\[1pt]
\forall x.[P(x) \lor Q(x)] \land [\forall y.\lnot(P(y) \lor \exists z.Q(z))] & \\[1pt]
\forall x.[P(x) \lor Q(x)] \land [\forall y.\lnot P(y) \land \lnot\exists z.Q(z)] & \\[1pt]
\forall x.[P(x) \lor Q(x)] \land [\forall y.\lnot P(y) \land \forall z.\lnot Q(z)] & \\[1pt]
\forall x.[P(x) \lor Q(x)] \land [\forall y.\lnot P(y) \land \lnot\exists z.Q(z)] & \\[1pt]
\forall x.\forall y.\forall z.[P(x) \lor Q(x)] \land \lnot P(y) \land \lnot Q(z) & \\[1pt]
\{\{P(x),Q(x)\}, \{\lnot P(y)\}, \{\lnot Q(z)\}\}  \\[1pt]
\end{align*}

\textbf{Resolución:}
\begin{center}
	\begin{forest} resolucion,
[$\red{\Box}$ 
	[$\{Q(y)\}$,edge label={node[midway,right] {$y\leftarrow z$}}
	    [$\{P(x) \comma Q(x)\}$
	    ,edge label={node[midway,right]{$x\leftarrow y$}}]
	    [$\{\lnot P(y)\}$]
	]
	[$\{\lnot Q(z)\}$]
]
	\end{forest}
\end{center}
La negación de la formula es insatisfactible con $\sigma = \{ z/y,~z/x\}$ por lo que la fórmula es valida.
\end{multicols}

\paragraph{V.}$ \forall x. [P(x) \lor Q(x)] \supset [\forall x. P(x) \lor \forall x. Q(x)] \leadsto \forall x. [[P(x) \lor Q(x)] \supset [\forall y. [P(y) \lor \forall z. Q(z)]]]$

\begin{multicols}{2}
\textbf{Negación:}
\begin{align*}
\lnot \forall x. [[P(x) \lor Q(x)] \supset [\forall y. [P(y) \lor \forall z. Q(z)]]] \\[1pt]
\lnot\forall x. [\lnot [P(x) \lor Q(x)] \lor [\forall y. [P(y) \lor \forall z. Q(z)]]] \\[1pt]
\exists x. [\lnot\lnot [P(x) \lor Q(x)] \land \lnot[\forall y. [P(y) \lor \forall z. Q(z)]]] \\[1pt]
\exists x. [P(x) \lor Q(x)] \land [\exists y.\lnot[P(y) \lor \forall z. Q(z)]] \\[1pt]
\exists x. [P(x) \lor Q(x)] \land [\exists y.\lnot P(y) \land \lnot\forall z. Q(z)]\\[1pt]
\exists x. [P(x) \lor Q(x)] \land [\exists y.\lnot P(y) \land\exists z. \lnot Q(z)]\\[1pt]
\exists x.\exists y. \exists z. [P(x) \lor Q(x)] \land [\lnot P(y) \land \lnot Q(z)]\\[1pt]
\exists y. \exists z. [P(c) \lor Q(c)] \land [\lnot P(y) \land \lnot Q(z)]\\[1pt]
\exists z. [P(c) \lor Q(c)] \land [\lnot P(d) \land \lnot Q(z)]\\[1pt]
[P(c) \lor Q(c)] \land \lnot P(d) \land \lnot Q(e)\\[1pt]
\{\{P(c), Q(c)\}, \{ \lnot P(d)\} , \{\lnot Q(e)\}\}\\[1pt]
\end{align*}

\textbf{Resolución:}
Las clausulas no unifican entre si porque usan distintas constantes.
\end{multicols}

\paragraph{VI.}$ [\exists x. P(x) \land \forall x. Q(x)] \supset \exists x. [P(x) \land Q(x)] \leadsto [\exists x. P(x) \land \forall y. Q(y)] \supset \exists z. [P(z) \land Q(z)]$

\begin{multicols}{2}
\textbf{Negación:}
\begin{align*}
\lnot([\exists x. P(x) \land \forall y. Q(y)] \supset \exists z. [P(z) \land Q(z)]) \\[1pt]
\lnot(\lnot[\exists x. P(x) \land \forall y. Q(y)] \lor \exists z. [P(z) \land Q(z)]) \\[1pt]
\lnot\lnot[\exists x. P(x) \land \forall y. Q(y)] \land \lnot\exists z. [P(z) \land Q(z)]) \\[1pt]
[\exists x. P(x) \land \forall y. Q(y)] \land \forall z. \lnot[P(z) \land Q(z)]) \\[1pt]
\exists x. P(x) \land \forall y. Q(y) \land \forall z. [\lnot P(z) \lor \lnot Q(z)]) \\[1pt]
\exists x.\forall y. \forall z. P(x) \land  Q(y) \land [\lnot P(z) \lor \lnot Q(z)]) \\[1pt]
\forall y. \forall z. P(c) \land  Q(y) \land [\lnot P(z) \lor \lnot Q(z)]) \\[1pt]
\{\{P(c)\}, \{Q(y)\}, \{\lnot P(z) , \lnot Q(z)\} \}\\[1pt]
\end{align*}

\textbf{Resolución:}
\begin{center}
	\begin{forest} resolucion,
[$\red{\Box}$ 
	[$\{\lnot Q(c)\}$,edge label={node[midway,right] {$y\leftarrow c$}}
	    [$\{\lnot P(z) \comma \lnot Q(z)\}$
	    ,edge label={node[midway,right]{$z\leftarrow c$}}]
	    [$\{P(c)\}$]
	]
	[$\{Q(y)\}$]
]
	\end{forest}
\end{center}
La negación de la formula es insatisfactible con $\sigma = \{ c/y,~c/z\}$ por lo que la fórmula es valida.
\end{multicols}

\paragraph{VII.}$\forall x. \exists y. \forall z. \exists w. [P(x,y) \lor \lnot P(w,z)]$

\begin{multicols}{2}
\textbf{Negación:}
\begin{align*}
\lnot \forall x. \exists y. \forall z. \exists w. [P(x,y) \lor \lnot P(w,z)]  \\[1pt]
\exists x. \lnot \exists y. \forall z. \exists w. [P(x,y) \lor \lnot P(w,z)] \\[1pt]
\exists x. \forall y. \lnot \forall z. \exists w. [P(x,y) \lor \lnot P(w,z)]  \\[1pt]
\exists x. \forall y. \exists z. \lnot \exists w. [P(x,y) \lor \lnot P(w,z)] \\[1pt]
\exists x. \forall y. \exists z. \forall w. \lnot [P(x,y) \lor \lnot P(w,z)]  \\[1pt]
\exists x. \forall y. \exists z. \forall w. [\lnot P(x,y) \land \lnot \lnot P(w,z)] \\[1pt]
\exists x. \forall y. \exists z. \forall w. [\lnot P(x,y) \land P(w,z)] \\[1pt]
\forall y. \exists z. \forall w. [\lnot P(c,y) \land P(w,z)] \\[1pt]
\forall y.\forall w. [\lnot P(c,y) \land P(w,f(y))] \\[1pt]
\forall y. \forall w. [\lnot P(c,y) \land P(w,f(y))] \\[1pt]
\{\{ \lnot P(c,y_1) \}, \{P(w,f(y_2)) \} \}
\end{align*}

\textbf{Resolución:}
\begin{center}
	\begin{forest} resolucion,
[$\red{\Box}$ 
	[$\{ \lnot P(c\comma y_1) \}$,edge label={node[midway,right] {$w\leftarrow c,~y\leftarrow f(y_2)$}}]
	[$\{P(w \comma f(y_2))$]
]
	\end{forest}
\end{center}
La negación de la formula es insatisfactible con $\sigma = \{ c/w,~f(y_2)/y\}$ por lo que la fórmula es valida.
\end{multicols}


\paragraph{VIII.} Este es muy largo, pero todos los literales terminan con constantes distintas y ninguno tiene variables.


\subsection{Ejercicio 10}
\paragraph{I.} Modus Ponens: $((P \supset Q) \land P) \supset Q$
\begin{multicols}{2}
\textbf{Negación:}
\begin{align*}
\lnot[((P \supset Q) \land P) \supset Q] \\[1pt]
\lnot[\lnot((\lnot P \lor Q) \land P) \lor Q] \\[1pt]
\lnot\lnot((\lnot P \lor Q) \land P) \land \lnot Q \\[1pt]
(\lnot P \lor Q) \land P \land \lnot Q \\[1pt]
\{\{\lnot P, Q\}, \{P\}, \{Q\}\}\} \\[1pt]
\end{align*}
\textbf{Resolución:}
\begin{center}
	\begin{forest} resolucion,
[$\red{\Box}$ 
	[$\{ Q \}$,
    	[$\{\lnot P\comma Q\}$]
    	[$\{P\}$]
	]
	[$\{ \lnot Q \}$]
]
	\end{forest}
\end{center}
\end{multicols}

\newpage
\paragraph{II.} Modus Tollens: $((P \lor Q) \land \lnot P) \supset Q$
\begin{multicols}{2}
\textbf{Negación:}
\begin{align*}
\lnot[((P \lor Q) \land \lnot P) \supset Q] \\[1pt]
\lnot[\lnot((P \lor Q) \land \lnot P) \lor Q] \\[1pt] \lnot\lnot((P \lor Q) \land \lnot P) \land \lnot Q \\[1pt]
(P \lor Q) \land \lnot P \land \lnot Q \\[1pt]
\{\{P, Q\}, \{\lnot P\}, \{\lnot Q\}\}\} \\[1pt]
\end{align*}
\textbf{Resolución:}
\begin{center}
	\begin{forest} resolucion,
[$\red{\Box}$ 
	[$\{ Q \}$,
    	[$\{ P\comma Q\}$]
    	[$\{\lnot P\}$]
	]
	[$\{ \lnot Q \}$]
]
	\end{forest}
\end{center}
\end{multicols}

\subsection{Ejercicio 11}
\paragraph{I.} $\{P(x), \lnot P(x), Q(a)\}$ no es una cláusula de Horn. 

Su fórmula de primer orden es $\forall x.P(x)\lor \lnot P(x)\lor Q(a)$.

\paragraph{II.} $\{P(x), \lnot Q(y), \lnot R(x,y) \}$ es una cláusula de de definición.

Su fórmula de primer orden es $\forall x\forall y.P(x) \lor \lnot Q(y) \lor R(x,y)$

\paragraph{III.} $\{\lnot P(x,x,z), \lnot Q(x,y), \lnot Q(y,z) \}$ es una cláusula de Horn pero no de definición.

Su fórmula de primer orden: $\forall x.\forall z.\forall y. \lnot P(x,x,z) \lor \lnot Q(x,y) \lor \lnot Q(y,z)$
\paragraph{IV.} $\{M(1,2,x) \}$ es una cláusula de definición.

Su fórmula de primer orden es $\forall x. M(1,2,x)$.

\subsection{Ejercicio 12}
    Condiciones son necesarias para que una demostración por resolución sea SLD:
    \begin{enumerate}
        \item Realizarse de manera lineal (utilizando en cada paso el resolvente obtenido en el paso anterior).
        \item Utilizar únicamente cláusulas de Horn.
        \item Empezar por una cláusula objetivo (sin literales positivos).
        \item Empezar por una cláusula que provenga de la negación de lo que se quiere demostrar.
        \item Utilizar la regla de resolución binaria en lugar de la general.
    \end{enumerate}

\subsection{Ejercicio 13}
Enunciado expresado en cláusulas:

\begin{itemize}
    \item Alana es un robot japonés.
    \begin{align*}
    R(\texttt{alan})\land J(\texttt{alan}) \\
    \{\blue{\underbrace{\{R(\texttt{alan})\}}_{C_1}},\blue{\underbrace{\{J(\texttt{alan})\}}_{C_2}}\}
    \end{align*}
    \item Cualquier robot que puede resolver un problema lógico es \texttt{inteligente}.
    \begin{align*}
\forall x.[ R(x) \land \exists y. PL(y) \land Res(x,y)] \supset I(x) \\[1pt]
\forall x.\lnot [ R(x) \land \exists y. PL(y) \land Res(x,y)] \lor I(x) \\[1pt]
\forall x. \lnot R(x) \lor \lnot \exists y. PL(y) \lor \lnot Res(x,y) \lor I(x) \\[1pt]
\forall x. \lnot R(x) \lor \lnot \exists y. PL(y) \lor \lnot Res(x,y) \lor I(x) \\[1pt]
\forall x. \lnot R(x) \lor \forall y. \lnot PL(y) \lor \lnot Res(x,y) \lor I(x) \\[1pt]
\forall x.\forall y. \lnot R(x) \lor  \lnot PL(y) \lor \lnot Res(x,y) \lor I(x) \\[1pt]
\blue{C_3 = \{ \lnot R(x_3),\lnot PL(y_3),\lnot Res(x_3,y_3) ,I(x_3) \}  }\\[1pt]
   \end{align*}
    
    \item Todos los robots japoneses pueden resolver todos los problemas de esta práctica.
        \begin{align*}
\forall x.[ R(x) \land J(x) ] \supset [\forall y. Pr(y) \land Res(x,y)] \\[1pt]
\forall x.\lnot[ R(x) \land J(x) ] \lor [\forall y. Pr(y) \land Res(x,y)] \\[1pt]
\forall x.\forall y.[\lnot R(x) \lor \lnot J(x) ] \lor [ Pr(y) \land Res(x,y)] \\[1pt]
\forall x.\forall y.[\lnot R(x) \lor \lnot J(x) \lor Pr(y) ] \land [ \lnot R(x) \lor \lnot J(x) \lor Res(x,y)] \\[1pt]
\{\blue{\underbrace{\{\lnot R(x_4), \lnot J(x_4), Pr(y_4)\}}_{C_4}},\blue{\underbrace{\{ \lnot R(x_5), \lnot J(x_5), Res(x_5,y_5)\}}_{C_5}}\}
       \end{align*}
\item Todos los problemas de esta práctica son lógicos.
\begin{align*}
\forall x. Pr(x) \supset PL(x) \\[1pt]
\forall x. \lnot Pr(x) \lor PL(x) \\[1pt]
\blue{C_6 =\{\lnot Pr(x_6),PL(x_6)\}} \\[1pt]
\end{align*}
\item Existe al menos un problema en esta práctica.
\begin{align*}
\exists x. Pr(x) \\[1pt]
Pr(c) \\[1pt]
\blue{C_7 = \{Pr(c)\}} \\[1pt]
\end{align*}
\end{itemize}
Queremos ver para que $x$ vale $I(x)$, entonces lo negamos y usamos resolución para conseguir alguna sustitución que lo haga.

\textbf{Resolución:}
\begin{center}
	\begin{forest} resolucion,
[$\red{\Box}$ 
	[$\{\lnot J(\texttt{alan})\}$
    	[$\{ \lnot R(x) \comma \lnot J(x)\}$, edge label={node[midway,right] {$x\leftarrow \texttt{alan}$}},
        	[$\{ \lnot R(x) \comma \lnot J(x) \comma \lnot Pr(y_3)\}$, edge label={node[midway,right] {$y_3\leftarrow c$}}
            	[$\{ \lnot R(x) \comma \lnot J(x) \comma \lnot PL(y_3)\}$
            	, edge label={node[midway,right] {$x_6\leftarrow y_3$}}
                	[$\{ \lnot R(x) \comma \lnot PL(y_3) \comma \lnot Res(x\comma y_3)\}$, edge label={node[midway,right] {$x_5\leftarrow x,~y_5\leftarrow y_3$}}
                    	[$\lnot I(x)$,edge label={node[midway,right] {$x_3\leftarrow x$}}]
                    	[$C_3$]
                	]
                	[$C_5$]
            	]
            	[$C_6$]
        	]
        	[$C_7$]
    	]
    	[$C_1$]
	]
	[$C_2$]
]
	\end{forest}
\end{center}

La sustitución resultante es $\sigma = \{ alan/x,~c/y3,~c/x_6,~alan/x_5,~c/y_5,~alan/x_3\}$. Entonces $alan$ es un robot \texttt{inteligente}.

\subsection{Ejercicio 14}
\textbf{Cláusulas:}
\begin{itemize}
\item $C_1 = \{\lnot suma(x_1,y_1,z_1), suma(x_1,suc(y_1),suc(z_1))\}$
\item $C_2 =\{suma(x_2,cero,x_2)\}$
\item $C_3 =\{\lnot suma(x_3,y_3,z_3), par(y_3)\}$
\end{itemize}

\textbf{Resolución:}
\begin{center}
	\begin{forest} resolucion,
[$\red{\Box}$ 
        	[$\{\lnot suma(x_1'\comma cero\comma z_1')\}$,
            edge label={node[midway,right] {$x_2\leftarrow x_1',~z_1' \leftarrow x_1'$}}
            	[$\{\lnot suma(x_1\comma suc(cero)\comma z_1)\}$
            	, edge label={node[midway,right] {$x_1\leftarrow x_1',~y_1'\leftarrow cero, z_1 \leftarrow suc(z_1')$}}
                	[$\{\lnot suma(x_3 \comma suc(suc(cero)) \comma z_3)\}$, edge label={node[midway,right] {$x_3\leftarrow x_1,~y_1\leftarrow suc(cero), z_3\leftarrow suc(z_1)$}}
                    	[$\{\lnot par(suc(suc(cero)))\}$,edge label={node[midway,right] {$y_3\leftarrow suc(suc(cero))$}}]
                    	[$C_3$]
                	]
                	[$C_1$]
            	]
            	[$C_1$]
        	]
        	[$C_2$]
]
	\end{forest}
\end{center}

Y la sustitución resultante es 
\begin{align*}
\sigma = \{ &x_1'/x_2,~x_1'/z_1',~x_1'/x_1, cero/y_1', suc(x_1')/z_1, x_1'/x_3, \\ &suc(cero)/y_1, suc(suc(x_1'))/z_3, suc(suc(cero))/y_3\}
\end{align*}

La resolución usada es resolución SLD porque en cada paso resolvimos únicamente un literal (es lineal), comenzamos con un goal compuesto solo de negaciones y todas las cláusulas son fórmulas de Horn.

\subsection{Ejercicio 15}
\paragraph{I.} Renombro las variables. $c\leftarrow x$ y $e\leftarrow y$ para que no sea confuso.
\begin{enumerate}[a)]
\item $\forall x. ( V(x) \lor \exists y.(P(y, x)))$
\begin{align*}
\forall x. ( V(x) \lor \exists y.(P(y, x))) \\[1pt]
\forall x.\exists y. ( V(x) \lor P(y, x)) \\[1pt]
\blue{C_1 = \{V(x_1), P(f(x_1), x_1)\}} \\[1pt]
\end{align*}
\item $\lnot \exists x.(V (x) \land \exists y.(P(y, x)))$
\begin{align*}
\lnot \exists x.(V (x) \land \exists y.(P(y, x)))\\[1pt]
\forall x.(\lnot V(x) \lor \lnot\exists y.(P(y, x)))\\[1pt]
\forall x.(\lnot V(x) \lor \forall y.\lnot P(y, x))\\[1pt]
\forall x.\forall y.(\lnot V(x) \lor \lnot P(y, x))\\[1pt]
\blue{C_2 = \{\lnot V(x_2) , \lnot P(y_2, x_2)\}}\\[1pt]
\end{align*}
\item $\forall y.\forall x.(P(y, I(x)) \iff P(y, x))$
\begin{align*}
\forall y.\forall x.(P(y, I(x)) \iff P(y, x)) \\[1pt]
\forall y.\forall x.[P(y, I(x)) \supset P(y, x)]\land [P(y, x)\supset P(y, I(x)) ] \\[1pt]
\forall y.\forall x.[\lnot P(y, I(x)) \lor P(y, x)]\land [\lnot P(y, x)\lor P(y, I(x)) ] \\[1pt]
\{\blue{\underbrace{\{\lnot P(y_3, I(x_3)), P(y_3, x_3)\}}_{C_3}},\blue{\underbrace{\{\lnot P(y_4, x_4), P(y_4, I(x_4))\}}_{C_4}}\\[1pt]
\end{align*}
\end{enumerate}

\paragraph{II.} Queremos ver que vale  $\forall x.(V (I(x)) \supset V (x))$, entonces lo negamos, lo pasamos a forma clausal usamos resolución para inferir la insatisfactibilidad de la negación.

\textbf{Negación:}
\begin{align*}
\lnot\forall x.(V (I(x)) \supset V (x)) \\[1pt]
\lnot\forall x.(\lnot V (I(x)) \lor V (x)) \\[1pt]
\exists x.(\lnot\lnot V (I(x)) \land \lnot V (x)) \\[1pt]
\exists x.(V (I(x)) \land \lnot V (x)) \\[1pt]
V (I(c)) \land \lnot V (c) \\[1pt]
\{\blue{\underbrace{\{V (I(c))\}}_{G_1}},\blue{\underbrace{\{\lnot V (c)\}}_{G_2}}\} \\[1pt]
\end{align*}

 No vamos a poder usar resolución SLD, por que la cláusula $C_1$ no es una fórmula de Horn (tiene más de un literal positivo) y hay cláusulas de nuestro Goal que no están negadas.
\textbf{Resolución:}
\begin{center}
	\begin{forest} resolucion,
[$\red{\Box}$ 
        	[$\{V(c)\}$,
            	[$\{\lnot P(y_2\comma c)\}$
            	, edge label={node[midway,right] {$x_1\leftarrow c,~y_2\leftarrow f(c)$}}
                	[$\{\lnot P(y_2\comma I(c))\}$, edge label={node[midway,right] {$y_4\leftarrow y_2,~x_4\leftarrow c$}}
                    	[$G_1$,edge label={node[midway,right] {$x_2\leftarrow I(c)$}}]
                    	[$C_2$]
                	]
                	[$C_4$]
            	]
            	[$C_1$]
        	]
        	[$G_2$]
]
	\end{forest}
\end{center}
Luego vale que $\forall x.(V (I(x)) \supset V (x))$.
\subsection{Ejercicio 16} No lo pude demostrar.
%\paragraph{a)} Los errores son:
%\begin{itemize}
%\item Negó mal los existenciales.
%\item No negó la fórmula antes de usar resolución.
%\item En el primer paso de la resolución sustituye una constante por una variable.
%\item En el segundo paso de la resolución consigue una resolvente a partir de dos literales negados.
%\end{itemize}
%
%\newpage
%\paragraph{b)} $(\exists x. \texttt{enBar}(x)) \supset \exists y. ( \texttt{enBar}(y) \land ( \texttt{\texttt{bebe}}(y) \supset \forall z. ( \texttt{enBar}(z) \supset \texttt{\texttt{bebe}}(z) ) ) )$
%
%\textbf{Negación:}
%\begin{align*}
%\lnot[(\exists x. \texttt{enBar}(x)) \supset \exists y. ( \texttt{enBar}(y) \land ( \texttt{bebe}(y) \supset \forall z. ( \texttt{enBar}(z) \supset \texttt{bebe}(z) ) ) )] \\[1pt]
%\lnot [\lnot (\exists x. \texttt{enBar}(x)) \lor \exists y. ( \texttt{enBar}(y) \land (\lnot \texttt{bebe}(y) \lor \forall z. ( \lnot \texttt{enBar}(z) \lor \texttt{bebe}(z) ) ) )] \\[1pt]
%(\exists x. \texttt{enBar}(x)) \land \lnot\exists y. ( \texttt{enBar}(y) \land (\lnot \texttt{bebe}(y) \lor \forall z. ( \lnot \texttt{enBar}(z) \lor \texttt{bebe}(z) ) ) ) \\[1pt]
%(\exists x. \texttt{enBar}(x)) \land \forall y. (\lnot\texttt{enBar}(y) \lor \lnot(\lnot \texttt{bebe}(y) \lor \forall z. ( \lnot \texttt{enBar}(z) \lor \texttt{bebe}(z) ) ) ) \\[1pt]
%(\exists x. \texttt{enBar}(x)) \land \forall y. (\lnot\texttt{enBar}(y) \lor (\texttt{bebe}(y) \land \lnot\forall z. ( \lnot \texttt{enBar}(z) \lor \texttt{bebe}(z) ) ) ) \\[1pt]
%(\exists x. \texttt{enBar}(x)) \land \forall y. (\lnot\texttt{enBar}(y) \lor (\texttt{bebe}(y) \land \exists z. ( \texttt{enBar}(z) \land \lnot\texttt{bebe}(z) ) ) ) \\[1pt]
%\exists x.\forall y.\exists z.  \texttt{enBar}(x) \land (\lnot\texttt{enBar}(y) \lor (\texttt{bebe}(y) \land   \texttt{enBar}(z) \land \lnot\texttt{bebe}(z) )) \\[1pt]
%\forall y.\exists z.  \texttt{enBar}(c) \land (\lnot\texttt{enBar}(y) \lor (\texttt{bebe}(y) \land   \texttt{enBar}(z) \land \lnot\texttt{bebe}(z) )) \\[1pt]
%\forall y.  \texttt{enBar}(c) \land (\lnot\texttt{enBar}(y) \lor (\texttt{bebe}(y) \land   \texttt{enBar}(f(y)) \land \lnot\texttt{bebe}(f(y)) )) \\[1pt]
%\forall y.  \texttt{enBar}(c) \land (\lnot\texttt{enBar}(y) \lor \texttt{bebe}(y)) \land  (\lnot\texttt{enBar}(y) \lor  \texttt{enBar}(f(y))) \land (\lnot\texttt{enBar}(y) \lor \lnot\texttt{bebe}(f(y)) ) \\[1pt]
%\{\blue{\underbrace{\{\texttt{enBar}(c)\}}_{C_1}},\blue{\underbrace{\{\lnot\texttt{enBar}(y_2), \texttt{bebe}(y_2)\}}_{C_2}},\blue{\underbrace{\{\lnot\texttt{enBar}(y_3),  \texttt{enBar}(f(y_3))\}}_{C_3}}  , \blue{\underbrace{\{\lnot\texttt{enBar}(y_4), \lnot\texttt{bebe}(f(y_4)) \}}_{C_4}}\} \\[1pt]
%\end{align*}
%\textbf{Resolución:}
%\begin{center}
%	\begin{forest} resolucion,
%[$\red{\Box}$ 
%        	[$\{V(c)\}$,
%            	[$\{\lnot P(y_2\comma c)\}$
%            	, edge label={node[midway,right] {$x_1\leftarrow c,~y_2\leftarrow f(c)$}}
%                	[$\{\lnot P(y_2\comma I(c))\}$, edge label={node[midway,right] {$y_4\leftarrow y_2,~x_4\leftarrow c$}}
%                    	[$C_3$,edge label={node[midway,right] {$y_4\leftarrow y_3$}}]
%                    	[$C_4$]
%                	]
%                	[$C_4$]
%            	]
%            	[$C_1$]
%        	]
%        	[$G_2$]
%]
%	\end{forest}
%\end{center}
\subsection{Ejercicio 17}
\begin{itemize}
        \item $\blue{C_1 = \{\texttt{esContacto}(x_1,f(x_1))\}}$
        \item $\blue{C_2 = \{ \lnot \texttt{esContacto}(x_2,y_2), \texttt{esContacto}(y_2,x_2)\}}$
\end{itemize}

\paragraph{I.} La demostración no es correcta, en el último paso, realiza la sustitución $f(x)\leftarrow c$ pero $f(x)$ no unificia con $c$.

\paragraph{II.} En el paso 4, la sustitución no es correcta, la resultante obtenida en ese paso debería ser $\{\lnot\texttt{esContacto}(d,x)\}$.

\paragraph{III.} Si, hay que seguir los mismos pasos que en el inciso anterior pero corrigiendo ese error:

\begin{center}
	\begin{forest} resolucion,
[$\red{\Box}$ 
                	[$\{\lnot\texttt{esContacto}(d \comma x)\}$, edge label={node[midway,right] {$x_1\leftarrow d,~x\leftarrow f(d)$}}
                    	[$\{\lnot \texttt{esContacto}(x\comma d)\}$,edge label={node[midway,right] {$y_2\leftarrow x,~x_2\leftarrow d$}}]
                    	[$C_2$]
                	]
                	[$C_1$]
            	]
        	]
]
	\end{forest}
\end{center}

\subsection{Ejercicio 18}
    \begin{itemize}
        \item $C_1 = \{ \lnot \texttt{Progenitor}(x_1,y_1),\texttt{Descenciente}(y_1,x_1)\}$ 
         \item $C_2 =\{ \lnot \texttt{Abuelo}(x_2,y_2), \texttt{Progenitor}(x_2, \texttt{medio}(x_2,y_2))\}$ 
        \item $C_3 = \{ \lnot \texttt{Descenciente}(x_3,y_3), \lnot \texttt{Descenciente}(y_3,z_3), \texttt{Descenciente}(x_3,z_3)\}$ 
        \item $C_4 =\{ \lnot \texttt{Abuelo}(x_4,y_4), \texttt{Progenitor}(\texttt{medio}(x_4,y_4),y_4)\}$ 
     \end{itemize}

Queremos ver que: $\forall x. \forall y. (\texttt{Abuelo}(x,y) \supset \texttt{Descenciente}(y,x))$ 
\textbf{Negación:}
\begin{align*}
\lnot\forall x. \forall y. (\texttt{Abuelo}(x,y) \supset \texttt{Descenciente}(y,x)) \\[1pt]
\lnot\forall x. \forall y. (\lnot \texttt{Abuelo}(x,y) \lor \texttt{Descenciente}(y,x) \\[1pt]
\exists x. \lnot \forall y. (\lnot \texttt{Abuelo}(x,y) \lor \texttt{Descenciente}(y,x)) \\[1pt]
\exists x. \exists y. \lnot (\lnot \texttt{Abuelo}(x,y) \lor \texttt{Descenciente}(y,x))\\[1pt]
\exists x. \exists y. (\lnot \lnot \texttt{Abuelo}(x,y) \land \lnot \texttt{Descenciente}(y,x))\\[1pt]
\exists x. \exists y. (\texttt{Abuelo}(x,y) \land \lnot \texttt{Descenciente}(y,x))\\[1pt]
\exists y. (\texttt{Abuelo}(c,y) \land \lnot \texttt{Descenciente}(y,c))\\[1pt]
\texttt{Abuelo}(c,d) \land \lnot \texttt{Descenciente}(d,c)\\[1pt]
\{\blue{\underbrace{\{\texttt{Abuelo}(c,d)\}}_{G_1}},\blue{\underbrace{\{\lnot \texttt{Descenciente}(d,c)\}}_{G_2}}\}\\[1pt]
\end{align*}

\textbf{Resolución:}
\begin{center}
	\begin{forest} resolucion,
[$\Box$,
    [$\{\texttt{Descenciente}(d\comma c)\}$
        [$\{ \lnot \texttt{Abuelo}(c\comma d) \comma \texttt{Descenciente}(d\comma c)\}\}$,
            [ $\{ \lnot \texttt{Progenitor}(\texttt{medio}(c\comma d)\comma x_3)\comma \texttt{Descenciente}(x_3\comma c)\}\}$,
            , edge label={node[midway,right] {$x_4\leftarrow c,~y_4 \leftarrow d,~x_3\leftarrow d$}}
                [$\{ \lnot \texttt{Descenciente}(x_3\comma \texttt{medio}(c\comma d))\comma \texttt{Descenciente}(x_3\comma c)\}\}$,
                 edge label={node[midway,right] {$y_1'\leftarrow x_3,~x_1' \leftarrow \texttt{medio}(c,d)$}}
                    [$\{\texttt{Descenciente}(\texttt{medio}(c\comma d) \comma c)\}$, edge label={node[midway,right] {$y_3\leftarrow \texttt{medio}(c,d),~z_3 \leftarrow c$}}
                        [$\{ \texttt{Progenitor}(c\comma \texttt{medio}(c\comma d))\}$, edge label={node[midway,right] {$x_1\leftarrow c,~y_1\leftarrow \texttt{medio}(c,d)$}}
                            [$G_1$,edge label={node[midway,right] {$x_2\leftarrow c,~y_2\leftarrow d$}}]
                            [$C_2$]
                        ]
                        [$C_1$]
                    ]
                    [$C_3$]
                ]
                [$C_1$]
            ]
            [$C_4$]
        ]
        [$G_1$]
    ]
    [$G_2$]
]
	\end{forest}
\end{center}

 \subsection{Ejercicio 19}
 \begin{itemize}
 \item $R$ es \textbf{irreflexiva}
 \begin{align*}
 \forall x. \lnot R(x,x) \\[1pt]
 \blue{C_1 = \{ \lnot R(x_1,x_1)\}}
 \end{align*}
 \item $R$ es \textbf{símetrica}
 \begin{align*}
 \forall x. \forall y. (R(x,y) \supset R(y,x)) \\[1pt]
 \forall x. \forall y. (\lnot R(x,y) \lor R(y,x))\\[1pt]
 \blue{C_2 = \{ \lnot R(x_2,y_2), R(y_2,x_2)\} }
 \end{align*}
 \item $R$ es \textbf{transitiva}
 \begin{align*}
\forall x. \forall y. \forall z. ((R(x,y) \land R(y,z)) \supset R(x,z))\\[1pt]\forall x. \forall y. \forall z. ( \lnot (R(x,y) \land R(y,z)) \lor R(x,z))\\[1pt]
\forall x. \forall y. \forall z. ( (\lnot R(x,y) \lor \lnot R(y,z)) \lor R(x,z))\\[1pt]
\blue{C_3 = \{ \lnot R(x_3,y_3), \lnot R(y_3,z_3), R(x_3,z_3)\}} \\[1pt]
 \end{align*}
\end{itemize}

$R$ es vacía es $\forall x. \lnot \exists y. R(x,y)$, y queremos probrar que no existe una relación no vacia que pueda cumplir todas las propiedades a la vez, osea que  $\lnot(\forall x. \lnot \exists y. R(x,y))$ es insatisfactible si valen las primeras tres propiedades. Entonces lo pasamos a forma clausal y usamos resolución para probarlo.
\begin{align*}
\lnot(\forall x. \lnot \exists y. R(x,y)) \\[1pt]
\exists x. \lnot \forall y. \lnot R(x,y) \\[1pt]
\exists x. \exists y. \lnot \lnot R(x,y)\\[1pt]
\exists x. \exists y. R(x,y)\\[1pt]
\exists y. R(c,y)\\[1pt]
R(c,d)\\[1pt]
\blue{G = \{R(c,d)\}}\\[1pt]
\end{align*}

\textbf{Resolución:}
\begin{center}
	\begin{forest} resolucion,
[$\Box$,
    [$\{ \lnot R(c\comma d)\}$
        [$\{\lnot R(z_3\comma d)\comma R(c\comma z_3)\}$, edge label={node[midway,right] {$z_3\leftarrow c$}}
            [$\{ R(d\comma z_3)\comma R(c\comma z_3)\}$, edge label={node[midway,right] {$x_1\leftarrow c,~y_1\leftarrow \texttt{medio}(c,d)$}}
                [$G$,edge label={node[midway,right] {$x_3\leftarrow c,~y_3\leftarrow d$}}]
                [$C_3$]
            ]
            [$C_2$]
        ]
        [$C_1$]
    ]
    [$G$]
]
	\end{forest}
\end{center}

\newpage
\subsection{Ejercicio 20}
\paragraph{I.} El programa reducirá siempre usando la primer regla de \texttt{mayorOIgual}, sin embargo cuando llegue a la expresión \texttt{mayorOIgual(N,suc(cero))}, $N$ no estará correctamente instanciada y podrá unificar con $succ(X)$ para cualquier $X$. Luego, intentará reducir la expresión \texttt{mayorOIgual(X,suc(cero))} y volverá a pasar lo mismo. Entonces el programa se cuelga.

\paragraph{II.}
\begin{itemize}
    \item $C_1 = \{\texttt{natural}(\texttt{cero}) \}$
    \item $C_2 = \{\texttt{natural}(\texttt{suc}(x_2)), \lnot \texttt{natural}(x_2) \}$
    \item $C_3 = \{\texttt{mayorOIgual}(\texttt{suc}(x_3),y_3), \lnot \texttt{mayorOIgual}(x_3,y_3) \}$
    \item $C_4 = \{\texttt{mayorOIgual}(x_4,x_4), \lnot \texttt{natural}(x_4) \}$
\end{itemize}

El goal es $ G = \{ \lnot \texttt{mayorOIgual}(\texttt{suc}(\texttt{suc}(N)),\texttt{suc}(\texttt{cero}))\}$

\begin{center}
	\begin{forest} resolucion,
[$\Box$,
    [$\{ \lnot \texttt{natural}(\texttt{cero})\}$
        [$\{\lnot \texttt{natural}(\texttt{suc}(\texttt{cero}))\}$, edge label={node[midway,right] {$x_2\leftarrow \texttt{cero}$}}
            [$\{ \lnot \texttt{mayorOIgual}(\texttt{suc}(N)\comma \texttt{succ}(\texttt{cero})) )\}$, edge label={node[midway,right] {$x_4\leftarrow \texttt{suc}(\texttt{cero}),~N\leftarrow \texttt{cero}$}}
                [$G$,edge label={node[midway,right] {$x_3\leftarrow \texttt{suc}(N),~y_3\leftarrow \texttt{suc}(\texttt{cero})$}}]
                [$C_3$]
            ]
            [$C_4$]
        ]
        [$C_2$]
    ]
    [$C_1$]
]
	\end{forest}
\end{center}

 La sustitución resultado es: $\sigma = \{\texttt{cero}/x_2,~\texttt{suc(cero)}/x_4,~\texttt{cero}/N,~\texttt{suc(cero)}/x_3,~ \texttt{suc(cero)}/y_3\}$ Por lo que $N = \texttt{cero}$ es una solución.
\paragraph{III.} Es resolución SLD porque usamos cláusulas de horn, resolución binaria y lineal. Sin embargo, no usamos la misma técnica de selección que usa Prolog.

\subsection{Ejercicio 21}
\begin{itemize}
\item $C_1 = \{\texttt{analfabeto}(x_1), \lnot \texttt{vivo}(x_1), \lnot \texttt{noSabeLeer}(x_1) \}$
\item $C_2 = \{\texttt{vivo}(x_2), \lnot \texttt{delfin}(x_2) \}$
\item $C_3 = \{\texttt{inteligente}(\texttt{flipper})\}$
\item $C_4 = \{\texttt{inteligente}(\texttt{alan})\}$
\item $C_5 = \{\texttt{noSabeLeer}(x_5), \lnot \texttt{mesa}(x_5)\}$
\item $C_6 = \{\texttt{noSabeLeer}(x_6), \lnot \texttt{delfin}(x_6)\}$
\item $C_7 = \{\texttt{delfin}(\texttt{flipper})\}$
\end{itemize}
Queremos probar que existe alguien inteligente pero analfabeto, es decir que $\exists x.\texttt{inteligente}(x)\land\texttt{analfabeto}(x)$.

\textbf{Negación:}
\begin{align*}
\lnot\exists x.\texttt{inteligente}(x)\land\texttt{analfabeto}(x)\\[1pt]
\forall x.\lnot\texttt{inteligente}(x)\lor\lnot\texttt{analfabeto}(x)\\[1pt]
\blue{G = \{\lnot\texttt{inteligente}(x)\lor\lnot\texttt{analfabeto}(x)\}}\\[1pt]
\end{align*}

\textbf{Resolución:}
\begin{center}
	\begin{forest} resolucion,
[$\Box$,
    [$\{ \lnot \texttt{delfin}(\texttt{flipper})\}$
        [$\{\lnot \texttt{noSabeLeer}(\texttt{flipper})\comma \lnot \texttt{delfin}(\texttt{flipper})\}$, edge label={node[midway,right] {$x_6\leftarrow \texttt{flipper}$}}
            [$\{\lnot \texttt{vivo}(\texttt{flipper})\comma \lnot \texttt{noSabeLeer}(\texttt{flipper})\}$, edge label={node[midway,right] {$x_2\leftarrow \texttt{flipper}$}}
                [$\{\lnot \texttt{analfabeto}(\texttt{flipper})\}$,edge label={node[midway,right] {$x_1\leftarrow \texttt{flipper}$}}
                    [$G$,,edge label={node[midway,right] {$x\leftarrow \texttt{flipper}$}}]
                    [$C_3$]
                ]
                [$C_1$]
            ]
            [$C_2$]
        ]
        [$C_6$]
    ]
    [$C_7$]
]
	\end{forest}
\end{center}

%  \subsection{Ejercicio 22}

\end{document}
