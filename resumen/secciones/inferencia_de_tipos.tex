\section{Inferencia de tipos}

Queremos modificar el lenguaje de cálculo lambda para que las expresiones no necesiten las notaciones de tipos explicitas. Para esto debemos definir términos sin información de tipos en los que la información faltante pueda ser \textbf{inferida} de manera sencilla. Esto es, debemos convertir dichos términos en términos bien tipados del cálculo lambda sin ningún problema.

Este nuevo lenguaje, nos evitará la sobrecarga de tener que declarar y manipular todos los tipos al momento de escribir un programa. Sin embargo, debemos tener en cuenta que, durante la compilación de los mismos, hay que hacer la inferencia de tipos, es decir, el compilador se deberá encargar de pasar el lenguaje que definamos a uno lambda tipado antes de poder compilar el programa.

\subsubsection*{Términos}
El lenguaje sin tipos tendrá todos los términos del lenguaje $\lambda$ con el que estuvimos trabajando hasta ahora, con la diferencia de que si en ellos había una notación de tipo, entonces la obviamos:

\begin{equation*}
	\begin{split}
		M~::=~&x \\
		|~~~&true~|~false~|~\lambdaIf{M}{P}{Q} \\
		|~~~&0~|~succ(0)~|~isZero(M) \\
		|~~~& \lambdaAbsI{x}{M}~|~M~N~|~\lambdaFix{M}
	\end{split}
\end{equation*}

Ahora, si bien la mayoría de los términos son iguales a los términos originales, necesitariamos alguna forma de convertir los términos del lambda cálculo a términos no tipados y viceversa. Para el primer caso, definimos la función $\Erase$ que dado un término del lambda cálculo, \textbf{elimina} las anotaciones de tipos de las abstracciones que contenga. Por ejemplos: 

$$\Erase(\lambdaAbs{x}{Nat}{\lambdaAbs{f}{Nat\to Nat}{f~x}}) = \lambdaAbsI{x}{\lambdaAbsI{f}{f~x}}$$

\paragraph{Chequeos de tipo}
Realizar el chequeo de tipo es determinar, para un término estándar (del lenguaje $\lambda$ tipado) $M$, si existe $\Gamma$ y $\sigma$ tales que $\judgeType{\Gamma}{M}{\sigma}$ es derivable. Osea que nos indica si $M$ es un término tipable o no.

Este chequeo  es facil de realizar, ya que solo hay que seguir la estructura sintáctica de $M$ para reconstruir una derivación de juicio. 

\subsubsection*{Definición de inferencia}
En cambio, con la inferencia de tipos, dado un término $U$ sin notaciones de tipo, se trata hallar un término estándar (con anotaciones de tipos) $M$ tal que:
\begin{enumerate}
	\item $\judgeType{\Gamma}{M}{\sigma}$ para algún $\Gamma$ y $\sigma$, y
	\item $\Erase(M) = U$
\end{enumerate}

Lo que estamos diciendo es que queremos encontrar una expresión bien tipada $M$ del lenguaje lambda que sea equivalente a $U$. Si encontramos este $M$, $U$ será de tipo $\sigma$, sino $U$ será una expresión no tipable en nuestro lenguaje.


\subsubsection{Variables de tipo}
Supongamos que tenemos la expresión $U = \lambdaAbsI{x}{x}$. En este caso, si queremos tipar $U$, nos damos cuenta que puede ser la función identidad de cualquier tipo. Como cualquiera de estas expresiones es igual de válida necesitamos escribir esto en nuestra solución, para eso usamos las \textbf{variables de tipo}.

Una \textbf{variables de tipo} s es una variable que representa una expresión de tipo arbitraria e indica que no importa por que expresión de tipo la remplacemos, tendremos una solución válida. Esto nos permitirá escribir que la expresión $M$ resultante de inferir los tipos de $U$ será $M=\lambdaAbs{x}{\textbf{s}}{x}$ donde \textbf{s} puede ser cualquier tipo de nuestro lenguaje.

Debemos agregar esta nueva expresión a las expresiones  de tipo del cálculo lambda:

$$\sigma~::=~\text{s}~|~Nat~|~Bool~|~\sigma\to\tau$$


\subsection{Sustitución de tipos}
Una función $S$ de sustitución es una función que mapea variables de tipo en expresiones de tipo y puede ser aplicada a expresiones de tipos ($S\sigma$), términos ($SM$) y contextos de tipado ($S\Gamma$).

Describimos $S$ usando la notación $\{\sigma_1/t_1,\dots,\sigma_n/t_n\}$ indicando que la variable $t_i$ debe ser remplazada por $\sigma_1$. Además, definimos el \textbf{conjunto soporte} de $S$ al conjunto $\{t_1,\dots,t_n\}$ como el conjunto que representa las variables que afecta $S$.

Por ejemplo, si $S = \{  Bool/t \}$, entonces $S(\lambdaAbs{x}{\text{t}}{x}) = \lambdaAbs{x}{ Bool}{x}$ y el tipo soporte de $S$ es $\{t\}$.

La sustitución cuyo soporte es $\emptyset$, es la \textbf{sustitución identidad}.


\hspace*{5mm}
Si tenemos dos juicios de tipado $\judgeType{\Gamma}{M}{\sigma}$ y $\judgeType{\Gamma'}{M'}{\sigma'}$ tales que $\judgeType{\Gamma'}{M'}{\sigma'}$ es el resultado de aplicar alguna función de sustitución $S$ a $\judgeType{\Gamma}{M}{\sigma}$, entonces decimos que $\judgeType{\Gamma'}{M'}{\sigma'}$ es instancia de $\judgeType{\Gamma}{M}{\sigma}$

\paragraph{Composición de sustituciones} La composición de sustituciones de $S$ y $T$, denotada $S\circ T$, es la sustitución que se comporta como sigue:

$$(S\circ T)(\sigma) = S(T\sigma)$$

\paragraph{Preorden de sustituciones} Una sustitución $S$ es \textbf{más general} que $T$ si existe una sustitución $U$ tal que $T = U\circ S$, es decir, si $T$ es una instancia de $S$.


\subsubsection{Unificación}
El algoritmo de inferencia que vamos a proponer analiza un término (sin notaciones de tipos) a partir de sus subtérminos. Una vez obtenida la información para cada uno de los subtérminos debe determinar si la información de cada uno de ellos es consistente (\textbf{consistencia})y, si lo es, sintetizar la información del término original a partir de esta (\textbf{Síntesis}).

Para realizar la síntesis debemos \textbf{compatibilizar} la información de tipos de cada subtérmino, por cada variable $x$ del término tenemos que tomar los tipos que le asigno cada súbtermino y unificarlos. Es decir, debemos encontrar una sustitución $S$ que nos permita remplazar los tipos que dió cada subexpresión por un tipo único. Veamos un ejemplo:

Sea $M = x~y + x~(y+1)$, del primer súbtermino $x~y$ ténemos que $x :: \text{s}\to \text{t}$ e $y :: \text{s}$, del subtérmino $x~(y+1)$ tenemos que $x::Nat\to \text{u}$ e $y :: Nat$. Ahora, $x$ e $y$ pueden tener un solo tipo en $M$, por lo que necesitamos una sustitución que nos permita unificar $\text{s}\to \text{t}$ con $Nat\to \text{u}$ y $s$ con $Nat$. En este caso podemos definir $S = \{Nat/\text{s},~\text{u}/\text{t}\}$, concluyendo que $x :: Nat\to\text{t}$ e $y:Nat$.

\paragraph{Ecuación de unificación} Es una expresión de la forma $\sigma_1 \equalDot \sigma_2$ cuya solución es una sustitución tal que $S\sigma_1 = S\sigma_2$. Por lo general tendremos un conjunto de ecuaciones de unificación y la solución a dicho conjunto será la sustitución que unifica todas las expresiones.

En el ejemplo anterior, las ecuaciones de unificación hubiesen sido $\{\text{s}\to\text{t}\equalDot Nat\to\text{u},~\text{s}\equalDot Nat\}$

Diremos que una sustitución $S$ es un \textbf{unificador más general (MGU)} de $\{\sigma_1 \equalDot \sigma_1',\dots,\sigma_n \equalDot \sigma_n'\}$, si es solución de ese conjunto y es más general que cualquier otra de sus soluciones.



\paragraph{Teorema} Si $\{\sigma_1 \equalDot \sigma_1',\dots,\sigma_n \equalDot \sigma_n'\}$ tiene solución, entonces existe un MGU y además es único salvo renombre de variables.

\subsubsection{Algoritmo de unificación de Martelli-Montanari}

Dado un conjunto de ecuaciones de unifiación $\{\sigma_1 \equalDot \sigma_1',\dots,\sigma_n \equalDot \sigma_n'\}$, vamos a presentar un algoritmo no-deterministico que consiste en \textbf{reglas de simplificación} que reescriben conjuntos de pares de tipos a unificar (\textit{goals}).

Las secuencias que terminan en un \textit{goal} vacío son \textbf{exitosas}, el resto, son \textbf{fallidas}. Si una secuencia es exitosa, entonces los pasos en los que realizamos sustituciones serán soluciones parciales al problema y la composición de todas ellas será el MGU.

\subsubsection{Reglas de reducción}

\begin{enumerate}
	\item \textbf{Descomposición}
	
	$\{\sigma_1\to\sigma_2 \equalDot\tau_1\to\tau_2\}\cup G\mapsto \{\sigma_1\equalDot\tau_1,~\sigma_2 \equalDot\tau_2\}\cup G$
	\item \textbf{Eliminación de par trivial}
	
	$\{Nat \equalDot Nat\}\cup G\mapsto G$
	
	$\{ Bool \equalDot Bool\}\cup G\mapsto G$
	
	$\{\text{s} \equalDot\text{s}\}\cup G\mapsto G$
	\item \textbf{Swap} Si $\sigma$ no es una variable,
	
	$\{\sigma \equalDot\text{s}\}\cup G\mapsto \{\text{s}\equalDot\sigma\}\cup G$
	
	\item \textbf{Eliminación de variable} Si $s\notin FV(\sigma)$
	
	$\{\text{s}\equalDot\sigma\}\cup G\mapsto_{\sigma/s} G[\sigma/s]$
	
	\item \textbf{Falla}
	
	$\{\sigma\equalDot\tau\}\cup G\mapsto \red{\texttt{falla}}$, con $(\sigma,\tau)\in T\cup T^{-1}$ y $T =\{( Bool,Nat), (Nat, \sigma_1\to\sigma_2), ( Bool, \sigma_1\to\sigma_2\}$. Acá, la notación $T^{-1}$ se refiere al conjunto con cada tupla de $T$ invertida.
	
	\item \textbf{Occur Check} Si $s\neq\sigma$ y $s\in FV(\sigma)$
	
	$\{\text{s}\equalDot\sigma\}\cup G\mapsto \red{\texttt{falla}}$
\end{enumerate}

\subsubsection{Propiedades del algoritmo}
El algoritmo de Martinelli-Montanari siempre termina. Sea $G$ un conjunto de pares, entonces:
\begin{itemize}
	\item Si $G$ tiene un unificador, el algoritmo termina exitosamente y retorna un MGU.
	\item Si $G$ no tiene un unificador, el algoritmo termina con \red{\texttt{falla}}.
\end{itemize}

\subsubsection{Ejemplo de aplicación}
\begin{equation*}
	\begin{split}
		&\{(Nat\to r)\to(r\to u) \equalDot t\to(s\to s)\to t\}
		\mapsto^1 \{Nat\to r\equalDot t,~r\to u\equalDot (s\to s)\to t\} \\
		&\mapsto^3 \{t\equalDot Nat\to r,~r\to u\equalDot (s\to s)\to t\} 
		\mapsto^4_{Nat\to r/t} \{r\to u\equalDot (s\to s)\to (Nat\to r)\} \\
		&\mapsto^1 \{r\equalDot(s\to s),~u\equalDot Nat\to r\} 
		\mapsto^4_{s\to s/r} \{u\equalDot Nat\to (s\to s)\}
		\mapsto^4_{Nat\to(s\to s)/u} \emptyset \\
	\end{split}
\end{equation*}

Entonces, el MGU es 

$\{Nat\to(s\to s)/u\}\circ\{s\to s/r\}\circ\{Nat\to r/t\} = \{Nat\to(s\to s)/u,~s\to s/r,~Nat\to (s\to s)/t\}$

\subsection{Función de inferencia \texorpdfstring{$\mathbb{W}$}{W}}
Vamos a definir una función $\mathbb{W}$ que dada una expresión $U$ sin notación de tipos, nos devolverá un juicio de tipado con una expresión tipada $M$ que corresponde a $U$. Esta función, la ejecutaremos de manera recursiva sobre las sub-expresiones de $U$ y sustituirá, si es posible, los tipos de cada una de ellas para que tengan ``sentido'' en $U$.




\subsubsection{Propiedades deseables de \texorpdfstring{$\WFunc$}{W}}
Dado un término $U$, $\WFunc(U)$ nos devolverá, si tiene éxito, una terna de tres elementos que serán un contexto de tipado $\Gamma$ una expresión $M$ y un $\sigma$ (notamos $\WFunc(U) = \judgeType{\Gamma}{M}{\sigma}$).

Queremos que $\WFunc$ sea \textbf{correcto} y \textbf{completo}.

\paragraph{Correctitud} $\WFunc(U) = \judgeType{\Gamma}{M}{\sigma}$ implica que $\Erase(M) = U$ y $\judgeType{\Gamma}{M}{\sigma}$ es derivable. Osea que $M$ es una expresión de tipo $\sigma$ en un contexto $\Gamma$ tal que si le borramos las notaciones de tipo, se convierte en $U$.

\paragraph{Completitud} Si $\judgeType{\Gamma}{M}{\sigma}$ es derivable y $\Erase(M) = U$, entonces:
$\WFunc(U)$ tiene éxito y produce un juicio $\judgeType{\Gamma'}{M'}{\sigma'}$ que es instancia del mismo. En otras palabras, si $U$ se puede obtener a partir de una expresión $M$, entonces $\WFunc$ deberá devolver el juicio de tipado que corresponde a $M$ o uno más general (esto es con variables de tipos, si resulta que $U$ podría ser de otros tipos).

\subsubsection{Algoritmo de inferencia}
El objetivo es definir $\WFunc$ por recursión sobre la estructura de $U$, por lo que definirla, primero, para las construcciones más simples y luego para las expresiones compuetas. Además, el algoritmo se valdrá del algoritmo de unificación para combinar los resultados de los pasos recursivos y, así, obtener un tipado consistente.

\subsubsection{Constantes y variables}
\begin{equation*}
	\begin{split}
		\WFunc(\red{true}) &\equalDef \judgeType{\emptyset}{true}{Bool} \\
		\WFunc(\red{false}) &\equalDef \judgeType{\emptyset}{false}{Bool} \\
		\WFunc(\red{x}) &\equalDef \judgeType{\{x:s\}}{x}{s}, \text{ \textit{s} variable fresca } \\
		\WFunc(\red{0}) &\equalDef \judgeType{\emptyset}{0}{Nat} \\
	\end{split}
\end{equation*}

\subsubsection{Caso \textit{succ}}
$\WFunc(\red{succ(U)}) \equalDef \judgeType{S\Gamma}{S~succ(M)}{Nat}$
\begin{centrado}
	\begin{itemize}
		\item $\WFunc(U) = \judgeType{\Gamma}{M}{\tau}$
		\item $S = MGU\{\tau\equalDot Nat\}$
	\end{itemize}
\end{centrado}

\subsubsection{Caso \textit{pred}}
$\WFunc(\red{pred(U)}) \equalDef \judgeType{S\Gamma}{S~pred(M)}{Nat}$
\begin{centrado}
	\begin{itemize}
		\item $\WFunc(U) = \judgeType{\Gamma}{M}{\tau}$
		\item $S = MGU\{\tau\equalDot Nat\}$
	\end{itemize}
\end{centrado}

\subsubsection{Caso \textit{isZero}}
$\WFunc(\red{isZero(U)}) \equalDef \judgeType{S\Gamma}{S~isZero(M)}{Bool}$
\begin{centrado}
	\begin{itemize}
		\item $\WFunc(U) = \judgeType{\Gamma}{M}{\tau}$
		\item $S = MGU\{\tau\equalDot Nat\}$
	\end{itemize}
\end{centrado}

\subsubsection{Caso \textit{ifThenElse}}
$\WFunc(\red{\lambdaIf{U}{V}{W}}) \equalDef \judgeType{S\Gamma_1\cup S\Gamma_2\cup S\Gamma_3}{S~(\lambdaIf{M}{P}{Q})}{S\sigma}$
\begin{centrado}
	\begin{itemize}
		\item $\WFunc(U) = \judgeType{\Gamma_1}{M}{\rho}$
		\item $\WFunc(V) = \judgeType{\Gamma_2}{P}{\sigma}$
		\item $\WFunc(W) = \judgeType{\Gamma_3}{Q}{\tau}$
		\item $S = MGU\{\sigma_1\equalDot \sigma_2~|~x:\sigma_1\in\Gamma_i~\land~x:\sigma_2\in\Gamma_j,~i\neq j\}\cup\{\sigma\equalDot\tau\,~\rho\equalDot Bool\}$
	\end{itemize}
\end{centrado}

\subsubsection{Caso aplicación}
$\WFunc(\red{U~V}) \equalDef \judgeType{S\Gamma_1\cup S\Gamma_2}{S~(M~N)}{St}$
\begin{centrado}
	\begin{itemize}
		\item $\WFunc(U) = \judgeType{\Gamma_1}{M}{\tau}$
		\item $\WFunc(V) = \judgeType{\Gamma_2}{N}{\rho}$
		\item $S = MGU\{\sigma_1\equalDot \sigma_2~|~x:\sigma_1\in\Gamma_i~\land~x:\sigma_2\in\Gamma_j,~i\neq j\}\cup\{\tau\equalDot\rho\to t\}$ con $t$ variable fresca
	\end{itemize}
\end{centrado}

\subsubsection{Caso abstracción}
$$\WFunc(\red{\lambdaAbsI{x}{U}}) \equalDef \judgeType{\Gamma \backslash\{x:\tau\}}{\lambdaAbs{x}{\tau}{M}}{\tau\to\rho}$$


Sea $\WFunc(U) = \judgeType{\Gamma}{M}{\rho}$, si $\Gamma$ tiene información de tipos para $x$, es decir $x:\tau\in\Gamma$ para algún $\tau$, entonces:

$$\WFunc(\red{\lambdaAbsI{x}{U}}) \equalDef \judgeType{\Gamma \backslash\{x:\tau\}}{\lambdaAbs{x}{\tau}{M}}{\tau\to\rho}$$

Si $\Gamma$ no tiene información de tipos para $x$ ($x\notin \text{Dom}(\Gamma)$), entonces elegimos una variable fresca $s$ y

$$\WFunc(\red{\lambdaAbsI{x}{U}}) \equalDef \judgeType{\Gamma}{\lambdaAbs{x}{s}{M}}{s\to\rho}$$

\subsubsection{Caso \textit{fix}}
$\WFunc(\red{\lambdaFix{(U)}}) \equalDef \judgeType{S\Gamma}{S~\lambdaFix{(M)}}{St}$
\begin{centrado}
	\begin{itemize}
		\item $\WFunc(U) = \judgeType{\Gamma_1}{M}{\tau}$
		\item $S = MGU\{\tau\equalDot t\to t\}$ con $t$ variable fresca
	\end{itemize}
\end{centrado}

\subsubsection{Complejidad del algoritmo}
Tanto la unificación como la inferencia para cálculo lambda se puede realizar en tiempo lineal. Sin embargo, el tipo principal asociado a un término sin anotaciones puede ser \red{exponencial} en el tamaño del término.

\subsubsection{Extensión del algoritmo a nuevos tipos}
Para extender el algoritmo a otros tipos debemos agregar los casos correspondientes al nuevo tipo teniendo en cuenta que los llamados recursivos devuelven un contexto, un término y un tipo sobre los que no podemos asumir nada.
Si la nueva regla tiene tipos iguales o contextos repetidos, debemos unificarlos. Y si la regla liga alguna variable, entonces vamos a poder dividir en dos casos: 
Si alguno de los contextos recursivos tiene información sobre esa variable, entonces sacamos su tipo del contexto que la contenga, sino le asignamos una variable fresca de tipo. Si la regla tiene restricciones adicionales, se incorporan como posibles fallas. Veamos un ejemplo para la extensión de listas (definida en el ejercicio 17 de la práctica 2).

\subsubsection{Extensión del algoritmo para listas}
$\WFunc([]) = \judgeType{\emptyset}{\List{t}}{t}$ con $t$ variable fresca.

$\WFunc(U_1::U_2) = \judgeType{S\Gamma_1\cup S\Gamma_2}{S(M_1::M_2)}{S[\sigma_1]}$

\begin{centrado}
	\begin{itemize}
		\item $\WFunc(U_i) = \judgeType{\Gamma_i}{M_i}{\sigma_i}$
		\item $S = MGU\{\sigma_1\equalDot \sigma_2~|~x:\sigma_1\in\Gamma_i~\land~x:\sigma_2\in\Gamma_j,~i\neq j\}\cup\{[\sigma_1]\equalDot\sigma_2\} $
	\end{itemize}
\end{centrado}


$\WFunc(\lambdaListCase{U}{U_2}{U_3}) =$

\quad$\judgeType{S\Gamma_1\cup S\Gamma_2\cup S\Gamma_3\backslash\{h,t\}}{S(\lambdaListCase{M_1}{M_2}{M_3})}{S\sigma_2}$

\begin{centrado}
	\begin{itemize}
		\item $\WFunc(U_i) = \judgeType{\Gamma_i}{M_i}{\sigma_i}$ con $i = 1,2,3$
		\item $S = MGU\{\sigma_1\equalDot \sigma_2~|~x:\sigma_1\in\Gamma_i~\land~x:\sigma_2\in\Gamma_j,~i\neq j\}\cup\{\sigma_1\equalDot [t_1],~t_1\equalDot \tau_h,~\tau_t\equalDot \sigma_1,~\sigma_2\equalDot\sigma_3\} $ con 
		\[ \tau_h = \begin{cases} 
		\sigma_h & \text{si } h:\sigma_h\in\Gamma_3 \\
		t_2 & \text{sino} \\
		\end{cases} \hspace*{5mm}\text{ y }\hspace*{5mm}
		\tau_t = \begin{cases} 
		\sigma_t & \text{si } t:\sigma_t\in\Gamma_3 \\
		t_2 & \text{sino} \\
		\end{cases}
		\]
	\end{itemize}
\end{centrado}