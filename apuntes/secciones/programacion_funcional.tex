
\section{Programación Funcional}

\paragraph{Valor} Entidad matemática abstracta con ciertas propiedades.

\paragraph{Expresión} 
Secuencia de símbolos utilizada para denotar un valor. Hay dos tipos de expresiones:
\begin{itemize}
	\item \textbf{Atómicas ó formas formales}: Son expresiones que representan a un valor. Por ejemplo: $2$, \textit{false}, $(3, True)$.
	\item \textbf{Compuestas}: Se construyen combinando otras expresiones. Los valores que denotan estas expresiones pueden ser inferidos a partir de esta combinación.
\end{itemize}

Hay expresiones a las que, a pesar de contener símbolos válidos, no se les puede dar un significado debido a errores sintácticos (mal escritas) o a errores de tipo (que denotan operaciones sobre tipos incorrectos). A éstas las llamamos \textbf{expresiones mal formadas}.

La programación funcional, se basa en la transformación de la información. Es decir, dada una expresión bien formada, se la va transformando (\textbf{reduciendo}) de manera iterativa por expresiones equivalentes más simples hasta obtener una que denote un valor.

\paragraph{Transparencia referencial} Asi se llama a la propiedad que nos permite remplazar una expresión de un programa por otra con igual valor, sin modificar el comportamiento del mismo.

Ésta solo se puede asegurar en ambientes en los que una función siempre da el mismo valor sin importar el contexto en la que es usada (ejemplo son las funciones de haskell, siempre dan el mismo resultado para las mismas entradas). 

En otros paradigmas, como imperativo, en donde los programas tienen efectos colaterales (se crea, modifican y destruyen variables constantemente) esto no es cierto. Por ejemplo, en el siguiente programa:

\begin{centrado}
	\begin{minted}{C++}
		int x = 5;
		
		int suma(int y) {
			x++;
			return  x + y;
		}
	\end{minted}
\end{centrado}

Si llamamos dos veces a la función suma de la siguiente manera: \mintinline{C++}{suma(1)}, la primera vez, el resultado será siete, la segunda vez será ocho. El resultado cambia porque depende del valor de $x$ que es modificado por la función cada vez que es llamada.

\subsection{Tipos}
Son una forma de particionar el universo de valores de acuerdo a ciertas propiedades. Se clasifican en dos categorias:
\begin{itemize}
	\item \textbf{Básicos} ó primitivos: Son los tipos que ya vienen definidos en el lenguaje por literales y representan valores. Ej: Int, Bool, Float, etc.
	\item \textbf{Tipos compuestos} (pares, listas, etc.) Son aquellos que se definen a partir de otros tipos.
\end{itemize}

Cada tipo de dato tiene asociado distintas operaciones que nos permiten manipularlos. Las expresiones bien formadas o son valores de algún tipo, o son una composición de estas operaciones aplicadas a un valor del tipo adecuado (no todos los tipos soportan las mismas operaciones). Entonces, a toda expresión bien formada se le puede asignar un tipo que sólo depende de los componentes de la expresión (strong-typing) y además puede ser inferido a partir de su constitución.

\paragraph{Notación:}   En haskell \haskell{e :: A } se lee “la expresión \haskell{e} tiene tipo \haskell{A}” y significa que el valor \haskell{e} pertenece al conjunto de valores denotado por \haskell{A}.

\subsubsection{Propiedades deseables de un lenguaje funcional}
Se busca que un lenguaje le asigne un tipo de manera automática al mayor número posible de expresiones ``con sentido'' y que no le asigne ningún tipo al mayor número posible de expresiones mal formadas. Además, se busca que el tipo de la expresión se mantenga si es reducida.

Otra cosa a tener en cuenta es que los tipos ofrecidos por el lenguaje deben ser descriptivos y razonablemente sencillos de leer.

\paragraph{Inferencia de tipos} Dada una expresión \haskell{e} determinar si tiene tipo o no y, si lo tiene, cuál es.

\paragraph{Chequeo de tipos} Dada una expresión tipable \haskell{e} y un tipo \haskell{A}, determinar si \haskell{e :: A} o no.

\subsection{Tipo Función}
Un programa, en el paradigma funcional, es una función descripta por un conjunto de ecuaciones (expresiones) que definen uno o más valores. Estas ecuaciones son evaluadas (reducidas) hasta llegar a una expresión atómica que nos indique el valor de las mismas.

\paragraph{Funciones} Son valores especiales que representan transformación de datos. Se aplican a elementos de un conjunto definido por el tipo de entrada y devuelve un elemento del tipo de salida. En haskell, este tipo se escribe:  \haskell{ A -> B}.

Al ser valores, las funciones pueden ser pasadas como argumentos o ser resultado de otras.

\paragraph{Funciones de alto orden} Son funciones que manipulan otras funciones. Es decir, no se límitan a funciones que tomen como parámetro a otras, sino que tambien incluye aquellas funciones que generan nuevas funciones.

\paragraph{Lenguaje Funcional Puro} Lenguaje de expresiones con transparencia referencial y funciones como valores. Su modelo de cómputo es la reducción realizada mediante el reemplazo de iguales por iguales.

\paragraph{Polimorfismo paramétrico} Se da cuando una función tiene un parámetro que puede ser instanciado de diferentes maneras. Es decir, cuando uno de sus parámetros puede ser de cualquier tipo. En este caso, el sistema puede asignarle un tipo que sea más general que todos los tipos que toma de tal manera que cada vez que la función sea llamada se la transforma para ese uso particular. Por ejemplo, una función que imprima un objeto en pantalla podría funcionar con cualquier tipo de dato.

Hay funciones que, a pesar de poseer polimorfismo paramétrico, no aceptan cualquier clase de tipo. Sino que requieren que los tipos con las que son llamadas tengan ciertas propiedades. Por ejemplo, que posean relaciones de igualdad (\haskell{Eq}) o relación de orden (\haskell{Ord}).

\begin{centrado}
	\begin{minted}{haskell}
		esMenor :: (Ord a) => a -> a -> Bool 
		esMenor x y = x < y
	\end{minted}
\end{centrado} 

La función es menor toma cualquier tipo $a$ que tenga definido una relación de orden.

\subsubsection{Evaluación}
Cuando se computa el valor de una expresión, decimos que la estamos \textit{evaluando}. Dependiendo del orden en el que el lenguaje lo haga, el tipo de evaluación se  clasifica en:

\paragraph{Evaluación Estricta} Si una parte de una expresión se indefine, entonces la expresión se indefine. La evaluación eager, en la que un en lenguaje computa una expresión apenas es definida, es de este tipo. 

\paragraph{Evaluación no Estricta} Puede pasar que una expresión esté definida a pesar de que alguna de sus partes se haya indefinido. La evaluación lazy, en la que un lenguaje solo computa una expresión cuando es necesaria para saber el valor de otra, es de este tipo.

Haskell usa evaluación lazy de izquierda a derecha: Resuleve primero las partes más externas de la expresión y luego, si es necesario, sus partes.

\paragraph{Currificación} Hay una correspondencia entre cada función de múltiples parámetros y una de alto orden que retorna una función intermedia que completa el trabajo. En otras palabras, por cada función \textit{f} definida como:
\begin{centrado}
	\begin{minted}{haskell}
f :: (a,b) -> c
f (x,y) = e
	\end{minted}
\end{centrado} 
existe un función \textit{f'} tal que:
\begin{centrado}
	\begin{minted}{haskell}
f' :: a -> b -> c 
f' x y = e
	\end{minted}
\end{centrado} 

La currificación nos da la posibilidad de realizar funciones parciales y nos permite tratar el código de manera más modular al momento de inferir tipos y transformar programas.

\paragraph{Evaluación parcial} En haskell, una expresión no es evaluada completamente salvo que sea necesario.

Este tipo de evaluación, junto con la currificación, nos permiten llamar a las funciones con un subconjunto de sus parámetros de entrada para generar nuevas funciones que tomen solo los parámetros que no fueron especificados en esa llamada.

Un \textbf{ejemplo}: Supongamos que tenemos la función \haskell{suma}:

\begin{centrado}
	\begin{minted}{haskell}
		suma :: Int -> Int -> Int
		suma x y = x + y
	\end{minted}
\end{centrado} 

Podemos usarla crear una función \haskell{sumarUno :: Int -> Int} instanciando su primer parámetro en uno:

\begin{centrado}
	\begin{minted}{haskell}
		sumarUno :: Int -> Int
		sumarUno y = suma 1
	\end{minted}
\end{centrado} 


\subsection{Inducción/Recursion}

La inducción es un mecanismo que nos permite definir conjuntos infinitos, probar propiedades sobre sus elementos y definir funciones recursivas sobre ellos con garantía de terminación.

\paragraph{Principio de extensionalidad:} Dadas dos expresiones A y B, si ambas denotan el mismo valor entonces A puede ser remplazada por B y B por A sin que esto afecte al resultado de una ecuación.

\subsubsection{Inducción estructural}
Una definición inductiva de un conjunto $\rel$ consiste en dar condiciones de dos tipos:
\begin{itemize}
	\item reglas base ($z\in\rel$) que afirman que algún elemento simple $x$ pertenece a $\rel$
	\item reglas inductivas ($x_1\in\rel,\dots,x_n\in\rel\Rightarrow x\in\rel$) que afirman que un elemento compuesto $x$ pertenece a
	$\rel$ siempre que sus partes $x_1,\dots,x_n$ pertenezcan a $\rel$
	(y $x$ no satisface ninguna de las otras regla dadas)
\end{itemize}

$\rel$ debe ser el menor conjunto que satisfaga todas las reglas dadas.

\subsubsection{Funciones recursivas}
Sea \haskell{S} un conjunto inductivo, y \haskell{T} uno cualquiera. Una definición recursiva estructural de una función \haskell{f :: S -> T} es una definición de la siguiente forma:
\begin{itemize}
	\item Por cada elemento base \haskell{z}, el valor de \haskell{(f z)} se da directamente usando valores previamente definidos
	\item Por cada elemento inductivo \haskell{y} con partes inductivas \haskell{y1}, ..., \haskell{yn}, el valor de \haskell{(f y)} se da usando valores previamente definidos y los valores \haskell{(f y1)}, ..., \haskell{(f yn)}.
\end{itemize}

\subsubsection{Principio de inducción}
Sea $S$ un conjunto inductivo, y sea $P$ una propiedad sobre los elementos de S. Si se cumple que:
\begin{itemize}
	\item para cada elemento $z\in S$ tal que $z$ cumple con una regla base, $P(z)$ es verdadero, y
	\item para cada elemento $y\in S$ construido en una regla inductiva utilizando los elementos \\ $y_1, ..., y_n$, si $P(y_1 ), ..., P(y_n)$ son verdaderos entonces $P(y)$ lo es
	
\end{itemize}

entonces $P(x)$ se cumple para todo $x\in S$.

\subsection{Parametrización}
Dado un conjunto de funciones que se comportan de la misma manera buscamos encontrar alguna forma de crear una función que las genere automáticamente. 

\paragraph{Esquema de funciones} Dado un conjunto de funciones ``parecidas'', el esquema de estas funciones son los que no permiten parametrizar correctamente alguno de los parámetros.

La parametrización nos permitirá crear definiciones más concisas y modulares, reutilizar código y demostrar propiedades generales de manera más fácil. Un ejemplo de esto son los esquemas recursivos (Sección \ref{sec:funcional.sub:esquemas_recursion}) que generalizan las funciones que hacen recursión sobre un tipo de datos.

\subsection{Tipos algebraicos}

\subsubsection{Definición de tipos}
Hay dos formas de definir un tipo de dato:
\begin{itemize}
	\item \textbf{De manera algebraica:} Establecemos qué \textit{forma} tendrá cada \textit{elemento} y damos un mecanismo único para inspeccionar cada uno de ellos.
	\item \textbf{De manera abstracta:} Determinamos cuales serán las \textit{operaciones} que manipularán los elementos, \textbf{SIN} decir cuál será la forma exacta del tipo ni de las operaciones que definimos.
\end{itemize}

\subsubsection{Tipos algebraicos en Haskell}
Los definimos mediante \textbf{constantes} llamadas \textit{constructores} cuyos nombres comienzan con mayúscula. Éstas no tienen asociada una regla de reducción y pueden tener argumentos.

Para implementar un tipo algebraico nuevo en Haskell, usamos la clausula \haskell{data} que introduce su nombre, los nombres de su constructores y sus argumentos.

\textbf{Ejemplos:}
\begin{centrado}
	\begin{minted}{haskell}
data Sensacion = Frio | Calor
	\end{minted}
\end{centrado}

El tipo \haskell{Sensacion} tiene como constructores a las constantes \haskell{Frio} y \haskell{Caliente}

\begin{centrado}
	\begin{minted}{haskell}
		data Shape = Circle Float | Rect Float Float
	\end{minted}
\end{centrado}

Los constructores del tipo \haskell{Forma} toman como parámetros elementos de tipo \haskell{Float}.

\begin{centrado}
	\begin{minted}{haskell}
data Maybe = Nothing | Just a
	\end{minted}
\end{centrado}
\haskell{Maybe} tiene todos los elementos del tipo a con \haskell{Just} y la constante \haskell{Nothing}. En este caso \haskell{a} es un tipo génerico, es decir, puede ser cualquier tipo.

\vspace*{5mm}

Son considerados tipos algebraicos porque:
\begin{itemize}
	\item toda combinación válida de constructores y valores es elemento del tipo algebraico (y solo ellas lo son)
	\item y porque dos elementos de un tipo algebraico son iguales si y solo si están construidos utilizando los mismos constructores aplicados a los mismos valores.
\end{itemize}
Al principio de esta sección, dijimos que además de establecer la forma que tiene el tipo, debemos dar un mecanismo único de inspección. En Haskell, se usa \textbf{Pattern Matching} para esto.


\subsubsection{Pattern Matching}
El pattern matching es la búsqueda de patrones especiales (en nuestro caso, los constructores del tipo) dentro de una expresión. Cuando la búsqueda tiene éxito, nos permita inspeccionar el valor de la expresión analizada.

Por ejemplo:

\begin{centrado}
	\begin{minted}{haskell}
		value :: Maybe -> Int 
		value Nothing = 0
		value (Just a) = 1
	\end{minted}
\end{centrado}

En esta función, la expresión \haskell{value (Just 2)} tiene valor $1$. Haskell, analiza las reglas de pattern matching provistas una a una y cuando encuentra que la expresión coincide con la segunda, evalúa al valor indicado.



\subsubsection{Tipos especiales}
\paragraph{Tupla} Este es un tipo algebraico con sintaxis especial. Una tupla es un estructura que posee varios elementos de distintos tipos. Por ejemplo: \haskell{(Float,Int)} es una tupla cuyo primer elemento es un \haskell{Float} y su segundo elemento es un \haskell{Int}.

\paragraph{Maybe} El tipo \haskell{Maybe}, definido en el último ejemplo, nos permite expresar la posibilidad de que el resultado sea erróneo sin necesidad de usar casos especiales. De esta forma, logramos evitar el uso de $\bot$ (indefinido) hasta que el programador lo decida lo que permite controlar errores.

\paragraph{Either} El tipo \haskell{Either} representa la unión disjunta de dos conjuntos (los elementos de uno se identifican con \haskell{Left} y los del otro con \haskell{Right}). Sirve para mantener el tipado fuerte y poder devolver elementos de distintos tipos o para representar el origen de un valor.
\begin{centrado}
	\begin{minted}{haskell}
data Either = Left a | Right b
	\end{minted}
\end{centrado}

\subsubsection{Expresividad}
Los tipos algebraicos no pueden representar cualquier cosa. Los números racionales son pares de enteros (numerador, denominador) cuya igualdad puede no depender de los valores con los que fueron construidos o incluso pueden llegar a no ser validos. Esto es así porque no todo par de enteros es un número racional, por ejemplo el (1,0). 

Además recordemos que la igualdad de dos elementos de un tipo algebraico solo se da si estos fueron construidos exactamente de la misma forma. Si seguimos con el ejemplo de los racionales, sabemos que hay valores iguales con distinto numerador y denominador como el (4,2) y el (2,1).

\subsubsection{Clases de tipos algebraicos}

\paragraph{Enumerativos} Solo constructores sin argumentos.

\paragraph{Productos} Un único constructor con varios argumentos.

\paragraph{Sumas} Varios constructores con argumentos.

\paragraph{Recursivos} Utilizan el tipo definido como argumento.

\subsection{Tipos algebraicos recursivos}
Un tipo algebraico recursivo tiene al menos uno de los constructores con el tipo que se define como argumento y es la concreción, en Haskell, de un conjunto definido inductivamente.

Cada constructor define un caso de una definición inductiva de un conjunto. Si tiene al tipo definido como argumento, entonces es un caso inductivo, sino es un caso base.

El pattern matching proporciona una forma de analizar estos casos y de acceder a los elementos inductivos que forman a un elemento dado. Por esta razón, se pueden definir funciones recursivas.

A estos tipos, les damos un significado a través de funciones definidas recursivamente. Éstas manipulan simbólicamente al tipo. Sin embargo, estas manipulaciones, por si solas no tienen un significado sino que se lo dan las propiedades que dichas manipulaciones deben cumplir.

\paragraph{Naturales} Notación unaria para expresar tipos naturales.
\begin{centrado}
	\begin{minted}[breaklines]{haskell}
data N = Z | S N
	\end{minted}
\end{centrado}
En este caso, \haskell{Z} y \haskell{S N} son el cero y el sucesor de un numero. (Notar que ese significado, es el que le damos nosotros y el que vamos a utilizar cuando escribamos funciones que nos permitan operar sobre este tipo. Para el lenguaje son solamente valores que pueden ser transformados por las funciones que nosotros definamos pero no tienen una semántica definida).

\paragraph{Listas} Definición equivalente a las listas de Haskell
\begin{centrado}
	\begin{minted}[breaklines]{haskell}
data List a = Nil | Cons a (List a)
	\end{minted}
\end{centrado}

\paragraph{Árboles}
Un árbol es un tipo algebraico tal que al menos un elemento compuesto tiene dos componentes inductivas.

\begin{centrado}
	\begin{minted}[breaklines]{haskell}
data Arbol a = Hoja a | Nodo a (Arbol a) (Arbol a)
	\end{minted}
\end{centrado}

\subsection{Esquemas de recursión} \label{sec:funcional.sub:esquemas_recursion}
Cuando tenemos un conjunto de funciones que manipulan ciertas estructuras de manera similar, podemos abstraer este comportamiento en funciones de alto orden que nos facilitarán su escritura.

A continuación, veremos unos ejemplos de esquemas sobre listas: 
\subsubsection{Map}
Aplica una función a cada elemento de una lista:
\begin{centrado}
	\begin{minted}{haskell}
map :: (a -> b) -> [a] -> [b]
map _ [] = []
map f (x:xs) = (f x) : (map f xs)
	\end{minted}
\end{centrado} 

\textbf{Ejemplo:}
\begin{centrado}
	\begin{minted}{haskell}
doble x = x + x
dobleL = map doble  
	\end{minted}
\end{centrado} 

\haskell{dobleL} calcula el doble de cada elemento de una lista.

\subsubsection{Filter}
Dada una lista \haskell{l} y un predicado \haskell{p}, selecciona todos los elementos de \haskell{l} que cumplen \haskell{p}.

\begin{centrado}
	\begin{minted}{haskell}
filter :: (a -> Bool) -> [a] -> [a]
filter _ [] = []
filter p (x:xs) | (p x)     = x : (filter p xs)
                | otherwise = filter p xs  
	\end{minted}
\end{centrado}

\textbf{Ejemplo}
\begin{centrado}
	\begin{minted}{haskell}
masQueCero = filter (>0)
	\end{minted}
\end{centrado}

\haskell{masQueCero} se queda con todos los elementos mayores de una lista

\subsubsection{Fold}
\textbf{fold} expresa el patrón de recursión estructural sobre listas como función de alto orden. Es decir, realiza recursión sobre una lista.

Dada una lista \haskell{l} y una función \haskell{f} que denota un valor que depende de todos los elementos de la lista \haskell{l} y un valor inicial \haskell{z}, aplica y combina las soluciones parciales obtenidas por \haskell{f} de manera  ``iterativa''. 
Hay dos tipos de fold: \haskell{foldr} (acumula desde la derecha) y \haskell{foldl} (acumula desde la izquierda).

\begin{centrado}
	\begin{minted}{haskell}
foldr :: (a -> b -> b) -> b -> [a] -> b
foldr _ z [] = z
foldr f z (x:xs) = f x (foldr f z xs)
		
		
foldl :: (b -> a -> b) -> b -> [a] -> b
foldl f z [] = z
foldl f z (x : xs) = foldl f (f z x) xs
	\end{minted}
\end{centrado}

\textbf{Ejemplos}
\begin{centrado}
	\begin{minted}{haskell}
sumatoria :: [Int] -> Int
sumatoria = foldr (\x rec -> x + rec ) 0
	\end{minted}
\end{centrado}
Para la sumatoria, la primer función que pasamos como parámetro toma el primer elemento de la lista y lo suma al resultado de la recursión. $0$ es el valor de la sumatoria cuando la lista pasada como parámetro es la lista vacía.

Map y Filter pueden ser implementados usando recursión estructural:
\begin{centrado}
	\begin{minted}{haskell}
map f = foldr (\x rec -> (f x): rec) []
filter p = foldr (\x rec -> if (p x) then x:rec else rec) []
	\end{minted}
\end{centrado}


\subsubsection{Recursión primitiva}
Recordemos de Logica y Computabilidad: una función h es recursiva primitiva si \textit{h} es de la forma:

\begin{align*}
	h(x_1,\dots,x_n,0) &= f(x_1,\dots,x_n) \\
	h(x_1,\dots,x_n,t+1) &= g(h(x_1,\dots,x_n, t),x_1,\dots, x_n, t) \\
\end{align*}

Es decir, el caso recursivo de \textit{h} no solo depende de la descomposición de sus parámetros, sino que también depende de ellos.

En Haskell, podemos definir una función que dada una lista \haskell{l}, un caso base \haskell{z} y un caso recursivo primitivo \haskell{f}, aplique la definición de \haskell{z} y \haskell{f} a la lista:

\begin{centrado}
	\begin{minted}{haskell}
recr :: b -> (a -> [a] -> b -> b) -> [a] -> b
recr z _ []= z
recr z f (x:xs) = f x xs (recr z f xs)
	\end{minted}
\end{centrado}
\vspace*{2mm}

Como ejemplo, escibimos la función \haskell{insertar} de una lista con recursión primitiva:

\begin{centrado}
	\begin{minted}{haskell}
-- Insert ordenado con pattern matching
insert :: a -> [a] -> [a]
insert x [] = [x]
insert x (y:ys) = if x<y then (x:y:ys) else (y:insert x ys)
		
-- Insert ordenado con recursión primitiva
insert x = recr [x] (\y ys rec -> if x<y then (x:y:ys) else (y:rec))
	\end{minted}
\end{centrado}

\subsubsection{Divide \& Conquer}
La técnica de Divide \& Conquer consiste en dividir un problema en problemas más fáciles de resolver y, luego, combinando los resultados parciales, obtener un resultado general. En este caso, \haskell{DivideConquer} es un tipo de función, osea que define una familia de funciones que toman como parámetro 4 funciones y un elemento de tipo \texttt{a} y devuelve un valor de tipo \texttt{b}:
\begin{centrado}
	\begin{minted}[breaklines]{haskell}
type DivideConquer a b  = (a -> Bool) -> (a -> b) -> (a -> [a]) 
    -> ([b] -> b) -> a -> b                         
	\end{minted}
\end{centrado}
Las funciones que toma como parámetro son:
\begin{itemize}
	\item \haskell{esTrivial :: a -> Bool} que devuelve verdadero si elemento de tipo \haskell{a} es el caso base del problema.
	\item \haskell{resolver :: a -> b} que resuelve el problema cuando el elemento de tipo \haskell{a} es el caso trivial
	\item \haskell{repartir :: a -> [a]} que divide al elemento de tipo \haskell{a} en la cantidad de subproblemas necesarios para resolver el problema.
	\item \haskell{combinar :: [b] -> b} que resuelve todos los subproblemas obtenidos por \haskell{repartir} y combina sus soluciones para obtener el resultado final.
\end{itemize}

\textbf{Ejemplo}

Vamos a definir el Divide \& Conquer para listas:
\begin{centrado}
	\begin{minted}[breaklines]{haskell}
divideConquerListas :: DivideConquer [a] b
-- Esto significa que DivideConquerLista es de tipo 
-- ([a] -> Bool) -> ([a] -> b) -> ([a] -> [[a]]) -> ([b] -> b)
-- -> [a] -> b
		
divideConquerListas esTrivial resolver repartir combinar l =
if (esTrivial l) then resolver l
else combinar (map dc (repartir l))
where dc = divideConquerListas esTrivial resolver repartir combinar	
	\end{minted}
\end{centrado}

Supongamos que queremos ordenar listas usando \texttt{mergesort}, un algoritmo que usa esta técnica, entonces una posible implementación sería:

\begin{centrado}
	\begin{minted}[breaklines,tabsize=4]{haskell}
	mergesort :: Ord a => [a] -> [a]
	mergesort = divideConquerListas ((<=1).length) 
									id  
									partirALaMitad
									(\[xs,ys] -> merge xs ys)	
	-- id es la función identidad de haskell

	partirALaMitad :: [a] -> [[a]]
	partirALaMitad xs = [ take i xs, drop i xs ] 
	where i = (div (length xs) 2)
	
	merge :: Ord a => [a] -> [a] -> [a]
	merge = foldr 
	(\y rec -> (filter (<= y) rec) ++ [y] ++ (filter (>y) rec))	 
	\end{minted}
\end{centrado}


\paragraph{Otros esquemas de recursión} Los esquemas de recursión que nombramos no son los únicos que existen y, además, pueden ser definidos para otros tipos recursivos, no solo para listas.

\subsubsection{La función fold y como definirla}
Todo tipo algebraico tiene asociado un patrón de inducción estructural. En particular, dado un tipo algebraico recursivo \haskell{T} podemos definir la función \haskell{foldrT:: * -> a} (donde * son los parámetros de la función) que representa a dicha induccion. A continuación damos algunas propiedades que debe cumplir \haskell{fold} para asegurarnos que la definimos correctamente:
\begin{itemize}
	\item Por cada constructor base de \haskell{T} debe tomar un parámetro de tipo \haskell{a} que será el elemento devuelto por la función si cae en alguno de dichos casos.
	\item Por cada constructor recursivo debe tomar una función que tome como parámetros a cada elemento del constructor que no sea del tipo \haskell{T} y un parámetro de tipo \haskell{a} por cada elemento del tipo \haskell{T}  del constructor. Esta función devuelve un elemento del tipo \haskell{a} y es la que resolverá recursivamente el caso planteado usando la segunda clase de parámetros.
	\item Por último, si la función está bien implementada, si remplazamos cada parámetro por el contructor correspondiente entonces la función resultante debería ser la función identidad del tipo \haskell{T}.
\end{itemize}


Al momento de definir \haskell{fold} ayuda mucho plantear el esquema de recursión del tipo. A continuación un ejemplo:

\paragraph{Ejemplo} Queremos definir el fold sobre el tipo \haskell{Arbol}.

\begin{centrado}
\begin{minted}[breaklines,tabsize=4]{haskell}
data Arbol a = Hoja a | Nodo a (Arbol a) (Arbol a)

-- Primero debemos definir el tipo de fold. Sabemos que es una fución que va de algo a algun tipo b.

foldrArbol :: * -> b

-- ¿Que es *?.  Los casos bases, en nuestro caso, será cuando el arbol sea de la forma Hoja x, con x :: a entonces necesitaremos una función que devuelva un valor de tipo b a partir del valor x.

foldrArbol :: * -> (a -> b) -> b

-- Ahora, debemos pensar el caso inductivo. El único constructor recursivo que tenemos es Nodo. Nodo toma un elemento de tipo a y dos árboles por lo que deberemos hacer doble recursión (una por cada subárbol). Osea que la solución de nuestra función depende del valor del primer parámetro (que es de tipo a) y de la soluciones dos recursiones (que son de tipo b):

foldrArbol :: (a -> b -> b -> b) -> (a -> b) -> b

-- Ya podemos implementarlo usando pattern matching, el caso base es aplicar la segunda función pasada como parámetro al valor de la hoja:

foldrArbol fcr fcb (Hoja x) = fcb x 

-- El caso inductivo es aplicar la primer función al elemento y al resultado de las recursiones:

foldrArbol fcr fcb (Node x ai ad) = 
	fcr x (foldrArbol fcr fcb ai) (foldrArbol fcr fcb ad)

-- fcr = funcion caso recursivo
-- fcb = funcion caso base
-- ai = arbol izquierdo
-- ad = arbol derecho
\end{minted}
\end{centrado}