
\section{Prototipado}
Lo lenguajes basados en prototipado de objetos se caracterizan por la ausencia de clases. Proveen constructores para la creación de objetos particulares y la herramientas necesarias para crear procedimientos que generen objetos.

En este tipo de paradigma, creamos objetos concretos (llamados \textbf{prototipos}) que se interpretan como representantes canónicos de cierto conjunto de objetos y, a partir de ellos, generamos otras intancias (\textbf{clones}) que pueden ser modificadas sin afectar al prototipo.

Cuando clonamos realizamos lo que se llama una \textbf{shallow copy} del objeto clonado, es decir copiamos cada atributo, con su método, del objeto original en el nuevo. Esto significa que cada atributo tiene exactamente la misma definición en ambos objetos.

Una consecuencia de esto es que si un atributo es una referencia a un objeto $A$, entonces en el nuevo objeto será una referencia a $A$ y cualquier modificación que le hagamos se verá reflejada en ambos objetos.

\subsection{Cálculo de objetos no tipado \texorpdfstring{($\varsigma$ cálculo)}{}}
Usaremos un lenguaje cuya única estructura computacional son los \textbf{Objetos}. Estos objetos son una colección de atributos nombrados (\textbf{registros}) que están asociados a métodos con una única variable ligada (que representa a \texttt{self}/\texttt{this}) y un cuerpo que produce un resultado.

Todos los objetos proveen dos operaciones:

\paragraph{Envío de mensajes:} Que nos permite invocar un método para que el objeto ejecute.

\paragraph{Redefinición de un método:} Que nos permite reemplazar el cuerpo de un atributo por otro.

\subsubsection{Sintaxis}
\begin{tabular}{lllll}
	$a,b$ &$::=$& &$x$ & Variables \\
 	      &     & $|$ &$[\OOAtributo{l_i}{x_i}{b_i}^{i\in1..n}]$ &  Objetos\\
 	      &     & $|$ &$a.l$ &  Selección/ Envío de mensajes \\
 	      &     & $|$ &$\OORedefinicion{a.l}{x}{b}$ &  Redefinición de un método.	       	       	      
\end{tabular}

\vspace*{5mm}
El objeto $[~]$ es el objeto vacío y no proporciona ningún método.

En este lenguajes, como todos los átributos son métodos, simulamos los colaboradores internos de un objeto con métodos que no utilizan el parámetro \texttt{self}. Por ejemplo:

$$o \equalDef [\OOAtributo{l_1}{x_1}{[~]},~
				\OOAtributo{l_2}{x_2}{x_2.l_1}]$$

$o.l_1$ retorna un objeto vacío. Y $o.l_2$ envia el mensaje $l_1$ a \texttt{self} (representado por el parámetro $x_2$). 

\paragraph{Notación} Cuando un objeto tenga un atributo de la forma $\OOAtributo{l}{x}{b}$ y $x$ no se usa en $b$ podemos escribir $l = b$ y a la reasignación $\OORedefinicion{o.l}{x}{b}$ como $o.l := b$.

\paragraph{Variables libres}
$\varsigma$ es un ligador de variables, cuando lo usamos en una expresión de la forma $\varsigma(x)b$ siempre liga la variable $x$ que se le pasa como párametro a \texttt{self}.

De manera análoga a $FV$ del cálculo $\lambda$ definimos fv para objetos y diremos que un término $a$ es \textbf{cerrado} si fv($a$) = $\emptyset$:

\begin{center}
\begin{tabular}{ll}
	$\text{fv}(\varsigma(x)b)$ &$= \text{fv}(b)\backslash \{x\} $\\
	$\text{fv}(x)$ &$= \{x\} $\\
	$\text{fv}([\OOAtributo{l_i}{x_i}{b_i}^{i\in 1..n}])$ &$=  \bigcup^{1\in 1..n} \text{fv}(\varsigma(x)b)$\\
	$\text{fv}(a.l)$ &$= \text{fv}(a) $\\
	$\text{fv}(\OORedefinicion{a.l}{x}{b})$ &$= \text{fv}(a.l)\cup \text{fv}(\varsigma(x)b) $\\
\end{tabular}
\end{center}
\paragraph{Sustitución} La función de sustitución de variables libres para objetos está definida de la siguiente forma:

\begin{center}
\begin{tabular}{lll}
	$x\{x \leftarrow c\}$ &$= c$ & \\
	$y\{x \leftarrow c\}$ &$= y$ & si $x\neq y$\\
	$([\OOAtributo{l_i}{x_i}{b_i}^{i\in 1..n}])\{x \leftarrow c\}$ &$=  [l_i = (\varsigma(x_i)b_i)\{x \leftarrow c\}^{i\in 1..n}]$ & \\
	$(a.l)\{x \leftarrow c\}$ &$= (a\{x \leftarrow c\}).l $ & \\
	$(\OORedefinicion{a.l}{x}{b})\{x \leftarrow c\}$ &$= (a\{x \leftarrow c\}).l \leftleftharpoons (\varsigma(x)b)\{x \leftarrow c\} $ & \\
	$(\varsigma(y)b)\{x \leftarrow c\}$ &$= (\varsigma(y')(b\{y \leftarrow y'\}\{x \leftarrow c\})) $ & si $y'\notin$fv$(\varsigma(y)b)\cup$fv$(c)\cup\{x\}$ \\
\end{tabular}
\end{center}

Notemos que en el último caso, remplazamos $y$ por $y'$ por si $y = x$ asegurandonos, de esta manera, que no cambiamos el significado de la expresión.

\paragraph{$\alpha$-conversión} En objetos decimos que dos objetos ($o_1 \equiv o_2$) son equivalentes si saben responder a los mismo mensajes y si los métodos asociados a cada uno de ellos son equivalentes en ambos objetos. Y dos métodos $\varsigma(x)b$ y $\varsigma(y)b_y$ son equivalente si $b_y\{y\leftarrow x\}\longrightarrow b$, es decir si la única diferencia entre ambos métodos es el nombre de las variables.


\begin{align*}
	o_1 \equalDef [l_1 = [~],~\OOAtributo{l_2}{x_2}{x_2.l_1}] \\
	o_2 \equalDef [\OOAtributo{l_2}{x_3}{x_3.l_1},~l_1 = [~]]
\end{align*}

son equivalentes porque ambos objetos tiene los atributos $l_1$ y $l_2$ y $\varsigma(x_2) x_2.l_1 =_\alpha \varsigma(x_3) x_3.l_1$.


\subsubsection{Semántica operacional}
Todos los objetos son considerados valores.
$$V~::=~[\OOAtributo{l_i}{x_i}{b_i}^{1\in 1..n}]$$

A diferencía del cálculo $\lambda$, usaremos el método de reducción \textbf{big-step} para evaluar expresiones. Este método nos permite saber el valor que representa la expresión en un solo paso.

$$\frac{}{v\longrightarrow v}[\text{Obj}]$$
\vspace*{5mm}
$$\frac{a\longrightarrow v'\hspace*{5mm} v'\equiv [\OOAtributo{l_i}{x_i}{b_i}^{i\in 1..n}]\hspace*{5mm} b_j\{x_j\leftarrow v'\}\longrightarrow v\hspace*{5mm} j\in1..n}{a.l_j\longrightarrow v}[\text{Sel}]$$

\vspace*{5mm}
$$\frac{a\longrightarrow [\OOAtributo{l_i}{x_i}{b_i}^{i\in 1..n}]\hspace*{5mm} j\in1..n}{\OORedefinicion{a.l_j}{x}{b}\longrightarrow [\OOAtributo{l_j}{x}{b},~\OOAtributo{l_i}{x_i}{b_i}^{i\in 1..n-\{j\}}]}[\text{Upd}]$$


La regla Obj nos dice que los objetos no reducen (o reducen a si mismos).

Sel nos indica que el resultado de enviar un mensaje es el valor que obtenemos al remplazar el párametro del método asociado por el mismo objeto (esto es la ligación a \texttt{self}).

Upd es el comportamiento de la redifinición, que devuelve un objeto con los mismos atributos que $a$ pero remplazando el $j-$ésimo atributo por la nueva definición.


\paragraph{Ejemplo de reducción}

\begin{center}
	\begin{scprooftree}
		\def\extraVskip{5pt}
		\AxiomC{$\OOReduccion{o}{o}$}
			
			\AxiomC{$\OOReduccion{o}{o}$}
			
				\AxiomC{}
			\RightLabel{[Obj]}
			\UnaryInfC{$\OOReduccion{[~]\{x\leftarrow o\}}{[~]}$}
			\RightLabel{[Sel]}
		\BinaryInfC{$\OOReduccion{(x.a)\{x\leftarrow o\}}{[~]}$}
		
		\RightLabel{[Sel]}
		\BinaryInfC{$\OOReduccion{[a=[~],~\OOAtributo{b}{x}{x.a}].b}{[~]}$}
	\end{scprooftree}
\end{center}

\paragraph{Indefinición} Similar al cálculo $\lambda$, podemos definir expresiones que se indefinen pero, en este caso, no es necesario que introduzcamos ninguna estructura nueva. Por ejemplos, si intentamos evaluar la expresión $[\OOAtributo{a}{x}{x.a}].a$ nos daremos cuenta de que su reducción es infinita.

\paragraph{Codificación de funciones (Cálculo $\lambda$)}
Modelamos las expresiones del cálculo $\lambda$ como objetos con un atributo $val$ que nos indica su valor, las funciones, además tiene el atributo $arg$ que representa al argumento de la función. El argumento de una función permanecerá indefinido hasta que aparezca en una aplicación.
\begin{align*}
\OORep{x} &\equalDef x\\
\OORep{M~N} &\equalDef \OORep{M}.arg :=~\OORep{N}\\
\OORep{\lambdaAbsI{x}{M}} &\equalDef 
[\OOAtributo{val}{y}{\OORep{M}\{x\leftarrow y.arg\}},~\OOAtributo{arg}{y}{y.arg}]\\
\end{align*}

Cuando querramos representar un método que espera parámetros, usaremos la definición de función para escribirlo: $\varsigma(x)\OORep {\lambdaAbsI{y}{M}}$.Podemos hacer abuso de notacion y escribir $\lambda(y)M$ en vez de $\varsigma(x)\OORep {\lambdaAbsI{y}{M}}$ y $M(N)$, en vez de $\OORep{M~N}$.

\subsubsection{Traits}
Un trait es una colección de métodos que parametrizan cierto comportamientos. Estos objetos no especifican variables de estado ni acceden a su estado.

El trait y sus métodos por si solo no son utilizables, ya que el trait no provee los estados necesarios para evaluarlos correctamente. Solo lo usaremos para definir métodos que pueden ser evaluados por varios objetos con el objetivo de no tener que repetir siempre las mismas definiciones.

Los vamos a representar como una colección de \textbf{pre-metodos} (que no usan el parámetro \text{self}). Por lo que un trait tendrá la siguiente forma:
$$\texttt{t} = [l_i = \lambdaAbsI{y_i}{b_i}^{i\in 1..n}]$$

Vamos a decir que un trait es completo cuando provea todos los atributos necesarios para que un objeto pueda ser utilizado. Y, cuando es completo, podremos definir una función $new$ como un constructor que crea un objeto con las mismas etiquetas que el trait y que asocia a cada una de ellas un método que invoca al método del trait correspondiente con su primer parámetro ligado a \texttt{self}:

$$new \equalDef \lambdaAbsI{z}{[\OOAtributo{l_i}{s}{z.l_i(s)^{i\in 1..n}}]}$$

El trait \texttt{CompT} no es completo y, por lo tanto, si queremos hacer un $new$ de un objeto, nos devolverá un objeto inutilizable:

\begin{align*}
	\texttt{CompT}\equalDef [~& \\ &\OOAtributo{eq}{t}{\lambda(x)\lambda(y)\lambdaIf{(x.comp(y)) == 0}{\texttt{true}}{\texttt{false}}},~\\
	&\OOAtributo{leq}{t}{\lambda(x)\lambda(y)\lambdaIf{(x.comp(y)) < 0}{\texttt{true}}{\texttt{false}}} \\
	]~&
\end{align*}

\begin{align*}
new~\texttt{CompT}\longrightarrow [~& \\ &\OOAtributo{eq}{x}{\lambda(y)\texttt{CompT}.eq(x,y)},~\\
&\OOAtributo{leq}{x}{\lambda(y)\texttt{CompT}.leq(x,y)} \\
]~&
\end{align*}

Sin embargo, este objeto sigue sin poder ser utilizado.

El siguiente ejemplo es un trait completo, si bien no puede ser usado, se puede ver que cuando creamos un objeto, le agrega el atributo $get$, $inc$ que usan un atributo $v$ y además le agrega el atributo $v$, por lo que estos dos métodos podrán ser ejecutados desde el nuevo objeto.

\vspace*{5mm}
\begin{align*}
\texttt{Contador}\equalDef [~& v = 0, \\
 & inc = \lambda(s) s.v := s.v + 1, \\
 & get = \lambda(s) s.v \\
 ]~&  \\
\end{align*}

\begin{align*}
new~\texttt{Contador}\equalDef [~& v = Contador.v, \\
 & inc = \varsigma(s) Contador.inc(s), \\
 & get = \varsigma(s) Contador.get(s) \\
 ]~&  \\
\end{align*}


Cuando un trait completo tenga dentro suyo a la función $new$, diremos que ese trait es una \textbf{clase}:
\begin{align*}
\texttt{C}\equalDef [~& \\ &\OOAtributo{new}{z}{[\OOAtributo{l_i}{s}{z.l_i(s)^{i\in 1..n}}]}~\\
&l_i = \lambda(x_{1})\dots\lambda(x_{n_i})B_i \\
]~&
\end{align*}
\paragraph{Herencia} Cuando queremos que una clase ``herede'', lo que hacemos es crear un nuevo trait que contenga todas las etiquetas del trait original y asocie cada una de esas etiquetas al método correspondiente del trait original y le agregamos los atributos que deseamos para extenderlo. Además, modificamos el constructor $new$ para que tome en cuenta los nuevos atributos.

\vspace*{5mm}
\begin{align*}
\texttt{Contador}\equalDef [~&
new = \varsigma(z)[\OOAtributo{inc}{s}{z.inc(s)},~v = z.v,~ \OOAtributo{get}{s}{z.get(s)} ], \\
 & v = \lambda(s)0, \\
 & inc = \lambda(s) s.v := s.v + 1, \\
 & get = \lambda(s) s.v \\
 ]~&  \\
\end{align*}

Y su sublcase \texttt{ContadorR}:

\vspace*{5mm}
\begin{tabular}{ll}
$\texttt{ContadorR}\equalDef [$ &
$new = \varsigma(z)[$
\\ & $\quad\OOAtributo{inc}{s}{z.inc(s)},$ \\ 
 & $\quad v = z.v,$ \\
 & $\quad \OOAtributo{get}{s}{z.get(s)},$ \\
 & $\quad \OOAtributo{reset}{\lambda(s)}{z.reset(s)} $ \\
 & $ ],$ \\
& $v = Contador.v$ \\
& $inc = \lambda(y)\texttt{Contador}.inc(y)$ \\
& $get = \lambda(y)\texttt{Contador}.get(y)$ \\
& $reset = \lambda(s) s.v := 0,$ \\
& $]$ \\
\end{tabular}


\paragraph{Otras consideraciones del lenguaje}
El lenguaje que definimos se parece mucho al funcional. En la versión imperativa, donde se mantiene un store con referencias a objetos, se ofrece la función \texttt{Clone}$(a)$ que crea un nuevo objeto con las mismas etiquetas de $a$ y cada componente comparte los métodos con las componentes de $a$.

Además, el lenguaje no nos deja agregar o eliminar dinámicamente métodos en un objeto y no nos deja extraer sus métodos, es decir, no es posible tener un método fuera de un objeto.

Hay otras versiones de este cálculo que incluyen sistemas de tipado.
