
\documentclass[10pt,a4paper]{article}

\usepackage[pdftex,
pdfauthor={Gianfranco Zamboni},
pdftitle={Resumen: Paradigmas de Lenguajes de Programación},
pdfsubject={},
pdfkeywords={Resumen , Computacion, FCEyN, UBA, Paradigmas de Lenguajes de Programación, Imperativo, Funcional, Cálculo Lambda, Programación Orientada a Objetos, Objetos, Programación Lógica},
pdfproducer={Latex with hyperref},
pdfcreator={pdflatex}]{hyperref}

\usepackage{amsmath}
\usepackage{ amssymb }
\usepackage{bussproofs}

\usepackage[spanish]{babel}


\usepackage[utf8]{inputenc} % para poder usar tildes en archivos UTF-8
\usepackage{graphicx}
\usepackage{xcolor}
\usepackage{pifont}

\usepackage{lscape}
\usepackage{minted}
\usepackage{a4wide} % márgenes un poco más anchos que lo usual
\usepackage[titletoc,toc,page]{appendix}
\usepackage{tikz}
\usepackage{forest}
\usepackage{multicol}

\setlength{\columnsep}{1cm}

\ifthenelse{\paperwidth < \paperheight}{\usepackage{fancyhdr}
\pagestyle{fancy}

%\renewcommand{\chaptermark}[1]{\markboth{#1}{}}
\renewcommand{\sectionmark}[1]{\markright{\thesection\ - #1}}

\fancyhf{}

\fancyhead[LO]{Sección \rightmark} % \thesection\ 
\fancyfoot[LO]{\small{Paradigmas de lenguajes de programación}}
\fancyfoot[RO]{\thepage}
\renewcommand{\headrulewidth}{0.5pt}
\renewcommand{\footrulewidth}{0.5pt}
\setlength{\hoffset}{-0.25in}
\setlength{\textwidth}{16cm}
%\setlength{\hoffset}{-1.1cm}
%\setlength{\textwidth}{16cm}
\setlength{\headsep}{0.5cm}
\setlength{\textheight}{25cm}
\setlength{\voffset}{-0.4in}
\setlength{\headwidth}{\textwidth}
\setlength{\headheight}{13.1pt}

\renewcommand{\baselinestretch}{1.1}  % line spacing}{\usepackage{fancyhdr}
\pagestyle{fancy}

%\renewcommand{\chaptermark}[1]{\markboth{#1}{}}
\renewcommand{\sectionmark}[1]{\markright{\thesection\ - #1}}

\fancyhf{}

\fancyhead[LO]{\rightmark} % \thesection\ 
\fancyfoot[LO]{\small{PLP - Prácticas}}
\fancyfoot[RO]{\thepage}
\renewcommand{\headrulewidth}{0.5pt}
\renewcommand{\footrulewidth}{0.5pt}
\setlength{\hoffset}{-0.25in}
\setlength{\textwidth}{25cm}
%\setlength{\hoffset}{-1.1cm}
%\setlength{\textwidth}{16cm}
\setlength{\headsep}{0.5cm}
\setlength{\textheight}{16cm}
\setlength{\voffset}{-0.4in}
\setlength{\headwidth}{\textwidth}
\setlength{\headheight}{13.1pt}

\renewcommand{\baselinestretch}{1.1}  % line spacing
}





\newenvironment{centrado}
    {
     \begin{center}
     \begin{minipage}{0.8\textwidth}
 }    
    {
     \end{minipage}
     \end{center}
    }

\newcommand{\rel}{\ensuremath{\mathcal{R}}}

\newcommand{\equalDef}{\overset{def}{=}}
\newcommand{\equalDot}{\overset{\cdot}{=}}

\newcommand{\lambdaAbs}[3]{\lambda #1: #2 . #3}
\newcommand{\lambdaAssign}[2]{#1~:=~#2}
\newcommand{\lambdaApp}[2]{#1~#2}
\newcommand{\lambdaIf}[3]{if~ #1~ then~ #2~ else~ #3}
\newcommand{\lambdaTrue}{true}
\newcommand{\lambdaFalse}{false}
\newcommand{\lambdaLet}[4]{let~#1:#2 = #3~in~#4}
\newcommand{\lambdaRef}[1]{ref~#1}
\newcommand{\lambdaVar}[1]{#1}
\newcommand{\lambdaValue}[1]{\color{red}#1\color{black}}
\newcommand{\lambdaFix}[1]{fix~#1}

\newcommand{\lambdaAbsI}[2]{\lambda #1. #2}



\newcommand{\blue}[1]{\color{blue}#1\color{black}}
\newcommand{\replaceBy}[3]{#1\{#2\leftarrow#3\}}

\newcommand{\judgeType}[3]{#1\triangleright #2 : #3}


\newenvironment{scprooftree}[1]%
{\gdef\scalefactor{#1}\begin{center}\proofSkipAmount \leavevmode}%
    {\scalebox{\scalefactor}{\DisplayProof}\proofSkipAmount \end{center} }


\tikzset{
    every leaf node/.style={text=red, align=center},
    every tree node/.style={text=blue, align=center},
}

\forestset{tikzQtree/.style={for tree={if n children=0{
                node options=every leaf node/.try}{node options=every tree node/.try}, text centered}}}
                
                
\DeclareMathOperator{\Erase}{Erase}
\DeclareMathOperator{\Nat}{Nat}
\DeclareMathOperator{\Bool}{Bool}
\DeclareMathOperator{\Union}{Union}

\newcommand{\WFunc}{\mathbb{W}}

\newcommand{\red}[1]{{\color{red}#1}}%\renewcommand{\appendixtocname}{Apéndices}
\newcommand{\green}[1]{{\color{green!40!black}#1}}%\renewcommand{\appendixpagename}{Apéndices}





\setcounter{section}{6}


\begin{document}
  \title{PLP - Práctica 7: Programación Lógica}

  \date{\today}

  \author{Zamboni, Gianfranco}

  \maketitle
  \setcounter{page}{1}


\section*{\ El motor de búsqueda de prolog}
\subsection{Ejercicio 1}
Base de conocimiento:
\begin{itemize}
    \item padre(juan, carlos).
    \item padre(juan, luis).
    \item padre(carlos, daniel).
    \item padre(carlos, diego).
    \item padre(luis, pablo).
    \item padre(luis, manuel).
    \item padre(luis, ramiro).
    \item abuelo(X,Y) :- padre(X,Z), padre(Z,Y).
\end{itemize}
Respuestas:
\begin{enumerate}
\item abuelo(X, manuel): padre(X,Z), padre(Z,manuel) \\
Juan
\item Definición:
hijo(X,Y) :- padre(Y,X). \\
hermano(X,Y) :- padre(Z,X), padre(Z,Y). \\
descendiente(X,Y) :- padre(Y,X). \\
descendiente(X,Y) :- padre(X,Z), descendiente(Z,Y).
\item \blue{arbol}
\item abuelo(Juan,X).
\item hermano(Pablo,X).
\item Regla: \\
ancestro(X, X). \\
ancestro(X, Y) :- ancestro(Z, Y), padre(X, Z).
\item el problema es que se genera una consulta que no termina, porque todo el tiempo el arbol se abre por ancestro().
\item Solucion: \\
ancestro(X, X). \\
ancestro(X, Y) :- padre(X, Z), ancestro(Z, Y).
\end{enumerate}
\subsection{Ejercicio 2}
\subsection{Ejercicio 3}
\section*{\ Estructuras, instanciación y reversibilidad}
\subsection{Ejercicio 4}
\subsection{Ejercicio 5}
\subsection{Ejercicio 6}
\subsection{Ejercicio 7}
\subsection{Ejercicio 8}
\subsection{Ejercicio 9}
\subsection{Ejercicio 10}
\subsection{Ejercicio 11}
\subsection{Ejercicio 12}
\section*{\ Generate and test}
\subsection{Ejercicio 13}
\subsection{Ejercicio 14}
\subsection{Ejercicio 15}
\section*{\ Negación por falla}
\subsection{Ejercicio 16}
\subsection{Ejercicio 17}
\subsection{Ejercicio 18}
\subsection{Ejercicio 19}

\subsection{Ejercicio 20}
\subsection{Ejercicio 21}
\subsection{Ejercicio 22}

\end{document}
