\documentclass[10pt,a4paper]{article}

\usepackage[pdftex,
pdfauthor={Gianfranco Zamboni},
pdftitle={Resumen: Paradigmas de Lenguajes de Programación},
pdfsubject={},
pdfkeywords={Resumen , Computacion, FCEyN, UBA, Paradigmas de Lenguajes de Programación, Imperativo, Funcional, Cálculo Lambda, Programación Orientada a Objetos, Objetos, Programación Lógica},
pdfproducer={Latex with hyperref},
pdfcreator={pdflatex}]{hyperref}

\usepackage{amsmath}
\usepackage{ amssymb }
\usepackage{bussproofs}

\usepackage[spanish]{babel}


\usepackage[utf8]{inputenc} % para poder usar tildes en archivos UTF-8
\usepackage{graphicx}
\usepackage{xcolor}
\usepackage{pifont}

\usepackage{lscape}
\usepackage{minted}
\usepackage{a4wide} % márgenes un poco más anchos que lo usual
\usepackage[titletoc,toc,page]{appendix}
\usepackage{tikz}
\usepackage{forest}
\usepackage{multicol}

\setlength{\columnsep}{1cm}

\ifthenelse{\paperwidth < \paperheight}{\usepackage{fancyhdr}
\pagestyle{fancy}

%\renewcommand{\chaptermark}[1]{\markboth{#1}{}}
\renewcommand{\sectionmark}[1]{\markright{\thesection\ - #1}}

\fancyhf{}

\fancyhead[LO]{Sección \rightmark} % \thesection\ 
\fancyfoot[LO]{\small{Paradigmas de lenguajes de programación}}
\fancyfoot[RO]{\thepage}
\renewcommand{\headrulewidth}{0.5pt}
\renewcommand{\footrulewidth}{0.5pt}
\setlength{\hoffset}{-0.25in}
\setlength{\textwidth}{16cm}
%\setlength{\hoffset}{-1.1cm}
%\setlength{\textwidth}{16cm}
\setlength{\headsep}{0.5cm}
\setlength{\textheight}{25cm}
\setlength{\voffset}{-0.4in}
\setlength{\headwidth}{\textwidth}
\setlength{\headheight}{13.1pt}

\renewcommand{\baselinestretch}{1.1}  % line spacing}{\usepackage{fancyhdr}
\pagestyle{fancy}

%\renewcommand{\chaptermark}[1]{\markboth{#1}{}}
\renewcommand{\sectionmark}[1]{\markright{\thesection\ - #1}}

\fancyhf{}

\fancyhead[LO]{\rightmark} % \thesection\ 
\fancyfoot[LO]{\small{PLP - Prácticas}}
\fancyfoot[RO]{\thepage}
\renewcommand{\headrulewidth}{0.5pt}
\renewcommand{\footrulewidth}{0.5pt}
\setlength{\hoffset}{-0.25in}
\setlength{\textwidth}{25cm}
%\setlength{\hoffset}{-1.1cm}
%\setlength{\textwidth}{16cm}
\setlength{\headsep}{0.5cm}
\setlength{\textheight}{16cm}
\setlength{\voffset}{-0.4in}
\setlength{\headwidth}{\textwidth}
\setlength{\headheight}{13.1pt}

\renewcommand{\baselinestretch}{1.1}  % line spacing
}





\newenvironment{centrado}
    {
     \begin{center}
     \begin{minipage}{0.8\textwidth}
 }    
    {
     \end{minipage}
     \end{center}
    }

\newcommand{\rel}{\ensuremath{\mathcal{R}}}

\newcommand{\equalDef}{\overset{def}{=}}
\newcommand{\equalDot}{\overset{\cdot}{=}}

\newcommand{\lambdaAbs}[3]{\lambda #1: #2 . #3}
\newcommand{\lambdaAssign}[2]{#1~:=~#2}
\newcommand{\lambdaApp}[2]{#1~#2}
\newcommand{\lambdaIf}[3]{if~ #1~ then~ #2~ else~ #3}
\newcommand{\lambdaTrue}{true}
\newcommand{\lambdaFalse}{false}
\newcommand{\lambdaLet}[4]{let~#1:#2 = #3~in~#4}
\newcommand{\lambdaRef}[1]{ref~#1}
\newcommand{\lambdaVar}[1]{#1}
\newcommand{\lambdaValue}[1]{\color{red}#1\color{black}}
\newcommand{\lambdaFix}[1]{fix~#1}

\newcommand{\lambdaAbsI}[2]{\lambda #1. #2}



\newcommand{\blue}[1]{\color{blue}#1\color{black}}
\newcommand{\replaceBy}[3]{#1\{#2\leftarrow#3\}}

\newcommand{\judgeType}[3]{#1\triangleright #2 : #3}


\newenvironment{scprooftree}[1]%
{\gdef\scalefactor{#1}\begin{center}\proofSkipAmount \leavevmode}%
    {\scalebox{\scalefactor}{\DisplayProof}\proofSkipAmount \end{center} }


\tikzset{
    every leaf node/.style={text=red, align=center},
    every tree node/.style={text=blue, align=center},
}

\forestset{tikzQtree/.style={for tree={if n children=0{
                node options=every leaf node/.try}{node options=every tree node/.try}, text centered}}}
                
                
\DeclareMathOperator{\Erase}{Erase}
\DeclareMathOperator{\Nat}{Nat}
\DeclareMathOperator{\Bool}{Bool}
\DeclareMathOperator{\Union}{Union}

\newcommand{\WFunc}{\mathbb{W}}

\newcommand{\red}[1]{{\color{red}#1}}%\renewcommand{\appendixtocname}{Apéndices}
\newcommand{\green}[1]{{\color{green!40!black}#1}}%\renewcommand{\appendixpagename}{Apéndices}





\setcounter{section}{6}


\begin{document}
  \title{PLP - Práctica 6: Resolución en Lógica}

  \date{\today}

  \author{Zamboni, Gianfranco}

  \maketitle
  \setcounter{page}{1}

  \section*{\ Resolución en Lógica Proposicional}
  \subsection{Ejercicio 1}
    \subsubsection{FNC:}
    \begin{enumerate}
    \item $-p \vee p$
    \item $-p \vee -q \vee p$
    \item $(-p \vee p) \wedge (-q \vee p)$
    \item $(-p \vee p) \wedge (p \vee -p)$
    \item $(q \vee -p \vee -q) \wedge (p \vee -p \vee -q)$
    \item $(p \vee p) \wedge (p \vee r) \wedge (q \vee p) \wedge (q \vee r)$
    \item $(r \vee -p) \wedge (-q \vee r)$
    \item $-p \vee -q \vee r$
    \end{enumerate}
    \subsubsection{\blue{FNClausal:}}
  \subsection{Ejercicio 2}
  \subsubsection{}
    \begin{enumerate}
        \item $(-p \vee p)$ es una tautología $  \Leftrightarrow \{\{p\},\{-p\}\} $ es insatisfacible \\
        $S=\{\{p\},\{-p\}\}$ \\
        $\{p\}-(p) \cup \{-p\}-(-p)=\{\}$ \\
        $S=\{\{p\},\{-p\},\{\}\} \Rightarrow $ es insatisfacible
        \item $((-p \vee -q) \vee p) $ es una tautología$ \Leftrightarrow \{\{p\},\{q\},\{-p\}\} $ es insatisfacible \\
        $S=\{\{p\},\{q\},\{-p\}\}$ \\
        $\{p\}-(p) \cup \{-p\}-(-p)=\{\}$ \\
        $S=\{\{p\},\{q\},\{-p\},\{\}\} \Rightarrow $ es insatisfacible
        \item $((-p \vee p)\wedge(-q \vee p))$ es una tautología $ \Leftrightarrow \{\{p, q\},\{p, -p\},\{-p, q\},\{-p, -p\}\} $ es insatisfacible \\
        $S=\{\{p, q\},\{p, -p\},\{-p, q\},\{-p, -p\}\}$ \\
        $\{p, q\}-(p) \cup \{-p, q\}-(-p)=\{q\}$ \\
        $S=\{\{p, q\},\{p, -p\},\{-p, q\},\{-p, -p\},\{q\}\}$ \\
        No podemos aplicar ningún paso de resolución a S es decir generar una cláusula nueva, por lo tanto, no puede llegarse a una refutación a partir S. Entonces, S debe ser satisfacible. Sea v(p)=F,v(q)=V.
        \item $((-p \vee p)\wedge(p \vee -p))$ es una tautología $ \Leftrightarrow \{\{p, -p\},\{p, p\},\{-p, -p\},\{-p, p\}\} $ es insatisfacible \\
        $S=\{\{p, -p\},\{p, p\},\{-p, -p\},\{-p, p\}\}$ \\
        $\{p, -p\}-(p,-p) \cup \{-p, p\}-(p,-p)=\{\}$ \\
        $S=\{\{p, -p\},\{p, p\},\{-p, -p\},\{-p, p\},\{\}\} \Rightarrow $ es insatisfacible
        \item $((q \vee -p \vee -q)\wedge(p \vee -p \vee -q))$ es una tautología $ \Leftrightarrow \{\{-q, -p\},\{-q, p\},\{-q, q\},\{p, -p\},\{p, p\},\{p, q\},\{q, -p\},\{q, p\},\{q, q\}\} es$ insatisfacible
        $S=\{\{-q, -p\},\{-q, p\},\{-q, q\},\{p, -p\},\{p, p\},\{p, q\},\{q, -p\},\{q, p\},\{q, q\}\}$ \\
        $\{-q, -p\}-(-q, -p) \cup \{p, q\}-(p, q)=\{\}$ \\
        $S=\{\{-q, -p\},\{-q, p\},\{-q, q\},\{p, -p\},\{p, p\},\{p, q\},\{q, -p\},\{q, p\},\{q, q\},\{\}\} \Rightarrow $ es insatisfacible
        \item $((p \vee p)\wedge(p \vee r)\wedge(q \vee p)\wedge(q \vee r))$ es una tautología $ \Leftrightarrow \{\{-p, -p, -q, -q\},\{-p, -p, -q, -r\},\{-p, -p, -p, -q\},\{-p, -p, -p, -r\},\{-p, -p, -q, -q\},\{-p, -p, -q, -r\},\{-p, -p, -p, -q\},\{-p, -p, -p, -r\},\{-p, -r, -q, -q\},\{-p, -r, -q, -r\},\{-p, -r, -p, -q\},\{-p, -r, -p, -r\},\{-p, -r, -q, -q\},\{-p, -r, -q, -r\},\{-p, -r, -p, -q\},\{-p, -r, -p, -r\}\} $ es insatisfacible \\
        $S=\{\{-p, -p, -q, -q\},\{-p, -p, -q, -r\},\{-p, -p, -p, -q\},\{-p, -p, -p, -r\},\{-p, -p, -q, -q\},\{-p, -p, -q, -r\},\{-p, -p, -p, -q\},\{-p, -p, -p, -r\},\{-p, -r, -q, -q\},\{-p, -r, -q, -r\},\{-p, -r, -p, -q\},\{-p, -r, -p, -r\},\{-p, -r, -q, -q\},\{-p, -r, -q, -r\},\{-p, -r, -p, -q\},\{-p, -r, -p, -r\}\}$ \\
        No podemos aplicar ningún paso de resolución a S, por lo tanto, no puede llegarse a una refutación a partir S. Entonces, S debe ser satisfacible. Sea v(p)=v(q)=v(r)=F.
        \item $((r \vee -p)\wedge(-q \vee r))$ es una tautología $ \Leftrightarrow \{\{-r, q\},\{p, q\},\{-r, -r\},\{p, -r\}\} $ es insatisfacible \\
        $S=\{\{-r, q\},\{p, q\},\{-r, -r\},\{p, -r\}\}$ \\
        No se puede generar el conjunto vacío
        \item $(-p \vee -q \vee r)$ es una tautología $ \Leftrightarrow \{\{p\},\{q\},\{-r\}\} $ es insatisfacible \\
        $S=\{\{p\},\{q\},\{-r\}\}$ \\
        No podemos aplicar ningún paso de resolución a S, por lo tanto, no puede llegarse a una refutación a partir S. Entonces, S debe ser satisfacible. Sea v(p)=v(q)=T,v(r)=F.
    \end{enumerate}
\subsubsection{}
$((-p \Rightarro w q) \wedge (p \Rightarro w q) \wedge (-p \Rightarrow -q)) \Rightarrow (p \wedge q)$ es una tautología $\Leftrightarrow$ \\
$ \{\{-p, p, -p, p\},\{-p, p, q, p\},\{-p, p, -p, q\},\{-p, p, q, q\},\{-q, p, -p, p\},\{-q, p, q, p\},\{-q, p, -p, q\},$ \\
$\{-q, p, q, q\} \{-p, -q, -p, p\},\{-p, -q, q, p\},\{-p, -q, -p, q\},\{-p, -q, q, q\} \{-q, -q, -p, p\},\{-q, -q, q, p\},$ \\
$\{-q, -q, -p, q\},\{-q, -q, q, q\}\}$ es insatisfacible \\

$S=\{\{-p, p, -p, p\},\{-p, p, q, p\},\{-p, p, -p, q\},\{-p, p, q, q\},\{-q, p, -p, p\}, \\ \{-q, p, q, p\},\{-q, p, -p, q\},\{-q, p, q, q\} \{-p, -q, -p, p\},\{-p, -q, q, p\}, \\ \{-p, -q, -p, q\},\{-p, -q, q, q\} \{-q, -q, -p, p\},\{-q, -q, q, p\},\{-q, -q, -p, q\},\{-q, -q, q, q\}\}$ \\ \\
$S1=\{-q, -q, q, p\}\{-q, p, q, p\}=\{-q, p, p, p\}$ \\
$S2=\{-p, -q, -p, q\}\{-q, p, p, p\}=\{-q, p\}$ \\
$S3=\{-q, p, -p, q\}\{-p, p, -p, q\}=\{q, -p\}$ \\
$S4=\{-q, p\}\{q, -p\}=\{\} $ \\ \\
$\Rightarrow$ \\
$S\cup S1 \cup S2 \cup S3 \cup S4 = \{\{-p, p, -p, p\},\{-p, p, q, p\},\{-p, p, -p, q\},\{-p, p, q, q\},\{-q, p, -p, p\}, \\ \{-q, p, q, p\},\{-q, p, -p, q\},\{-q, p, q, q\} \{-p, -q, -p, p\},\{-p, -q, q, p\}, \\ \{-p, -q, -p, q\},\{-p, -q, q, q\},\{-q, -q, -p, p\},\{-q, -q, q, p\},\{-q, -q, -p, q\}, \\ \{-q, -q, q, q\}\},\{-q, p, p, p\},\{-q, p\},\{q, -p\},\{\}$ es insatisfacible

  \subsection{Ejercicio 3}
  \section*{\ Unificación en Lógica de Primer Orden}
  \subsection{Ejercicio 4}
    \begin{enumerate}
    \item $P(f(x)) $ unifica con: \\
    $P(f(a)) $ si $ x \doteq a$
    \item $P(a) $ unifica con: \\
    $P(x) $ si $ x \doteq a$ \\
    $P(g(z)) $ si $ g(z) \doteq a$
    \item $P(y) $ unifica con: \\
    $P(x) $ si $ y \doteq x$ \\
    $P(f(a)) $ si $ y \doteq f(a)$ \\
    $P(g(z)) $ si $ y \doteq g(z)$
    \item $Q(x,f(y)) $ unifica con: \\
    $Q(f(y),x) $ si $ x \doteq f(y)$
    \item $Q(x,f(z)) $ unifica con: \\
    $Q(f(y),x) $ si $ x \doteq f(y) $ y $ z \doteq y$ \\
    $Q(f(y),f(x)) $ si $ x \doteq f(y) $ y $ z \doteq f(y)$ \\
    $Q(f(y),y) $ si $ y \doteq f(z) y x \doteq f(f(z))$
    \item $Q(x,f(a)) $ unifica con: \\
    $Q(f(y),x) $ si $ x \doteq f(a) $ y $ y \doteq a$ \\
    $Q(f(y),f(x)) $ si $ x \doteq a $ y $ a \doteq f(y)$ \\
    $Q(f(y),y) $ si $ y \doteq f(a) $ y $ x \doteq f(f(a))$
    \end{enumerate}
  \subsection{Ejercicio 5}
  \section*{\ Resolución en Lógica de Primer Orden}
  \subsection{Ejercicio 5}
  \subsection{Ejercicio 6}
  \subsection{Ejercicio 7}
  \subsection{Ejercicio 8}
  \subsection{Ejercicio 9}
  \subsection{Ejercicio 10}
  \subsection{Ejercicio 11}
  \subsection{Ejercicio 12}
  \subsection{Ejercicio 13}
  \subsection{Ejercicio 14}
  \subsection{Ejercicio 15}
  \subsection{Ejercicio 16}
  \subsection{Ejercicio 17}
  \subsection{Ejercicio 18}
  \subsection{Ejercicio 19}
  \subsection{Ejercicio 20}
  \subsection{Ejercicio 21}
  \subsection{Ejercicio 22}

\end{document}
