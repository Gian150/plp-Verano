
\documentclass[10pt,a4paper]{article}

\usepackage[pdftex,
pdfauthor={Gianfranco Zamboni},
pdftitle={Resumen: Paradigmas de Lenguajes de Programación},
pdfsubject={},
pdfkeywords={Resumen , Computacion, FCEyN, UBA, Paradigmas de Lenguajes de Programación, Imperativo, Funcional, Cálculo Lambda, Programación Orientada a Objetos, Objetos, Programación Lógica},
pdfproducer={Latex with hyperref},
pdfcreator={pdflatex}]{hyperref}

\usepackage{amsmath}
\usepackage{ amssymb }
\usepackage{bussproofs}

\usepackage[spanish]{babel}


\usepackage[utf8]{inputenc} % para poder usar tildes en archivos UTF-8
\usepackage{graphicx}
\usepackage{xcolor}
\usepackage{pifont}

\usepackage{lscape}
\usepackage{minted}
\usepackage{a4wide} % márgenes un poco más anchos que lo usual
\usepackage[titletoc,toc,page]{appendix}
\usepackage{tikz}
\usepackage{forest}
\usepackage{multicol}

\setlength{\columnsep}{1cm}

\ifthenelse{\paperwidth < \paperheight}{\usepackage{fancyhdr}
\pagestyle{fancy}

%\renewcommand{\chaptermark}[1]{\markboth{#1}{}}
\renewcommand{\sectionmark}[1]{\markright{\thesection\ - #1}}

\fancyhf{}

\fancyhead[LO]{Sección \rightmark} % \thesection\ 
\fancyfoot[LO]{\small{Paradigmas de lenguajes de programación}}
\fancyfoot[RO]{\thepage}
\renewcommand{\headrulewidth}{0.5pt}
\renewcommand{\footrulewidth}{0.5pt}
\setlength{\hoffset}{-0.25in}
\setlength{\textwidth}{16cm}
%\setlength{\hoffset}{-1.1cm}
%\setlength{\textwidth}{16cm}
\setlength{\headsep}{0.5cm}
\setlength{\textheight}{25cm}
\setlength{\voffset}{-0.4in}
\setlength{\headwidth}{\textwidth}
\setlength{\headheight}{13.1pt}

\renewcommand{\baselinestretch}{1.1}  % line spacing}{\usepackage{fancyhdr}
\pagestyle{fancy}

%\renewcommand{\chaptermark}[1]{\markboth{#1}{}}
\renewcommand{\sectionmark}[1]{\markright{\thesection\ - #1}}

\fancyhf{}

\fancyhead[LO]{\rightmark} % \thesection\ 
\fancyfoot[LO]{\small{PLP - Prácticas}}
\fancyfoot[RO]{\thepage}
\renewcommand{\headrulewidth}{0.5pt}
\renewcommand{\footrulewidth}{0.5pt}
\setlength{\hoffset}{-0.25in}
\setlength{\textwidth}{25cm}
%\setlength{\hoffset}{-1.1cm}
%\setlength{\textwidth}{16cm}
\setlength{\headsep}{0.5cm}
\setlength{\textheight}{16cm}
\setlength{\voffset}{-0.4in}
\setlength{\headwidth}{\textwidth}
\setlength{\headheight}{13.1pt}

\renewcommand{\baselinestretch}{1.1}  % line spacing
}





\newenvironment{centrado}
    {
     \begin{center}
     \begin{minipage}{0.8\textwidth}
 }    
    {
     \end{minipage}
     \end{center}
    }

\newcommand{\rel}{\ensuremath{\mathcal{R}}}

\newcommand{\equalDef}{\overset{def}{=}}
\newcommand{\equalDot}{\overset{\cdot}{=}}

\newcommand{\lambdaAbs}[3]{\lambda #1: #2 . #3}
\newcommand{\lambdaAssign}[2]{#1~:=~#2}
\newcommand{\lambdaApp}[2]{#1~#2}
\newcommand{\lambdaIf}[3]{if~ #1~ then~ #2~ else~ #3}
\newcommand{\lambdaTrue}{true}
\newcommand{\lambdaFalse}{false}
\newcommand{\lambdaLet}[4]{let~#1:#2 = #3~in~#4}
\newcommand{\lambdaRef}[1]{ref~#1}
\newcommand{\lambdaVar}[1]{#1}
\newcommand{\lambdaValue}[1]{\color{red}#1\color{black}}
\newcommand{\lambdaFix}[1]{fix~#1}

\newcommand{\lambdaAbsI}[2]{\lambda #1. #2}



\newcommand{\blue}[1]{\color{blue}#1\color{black}}
\newcommand{\replaceBy}[3]{#1\{#2\leftarrow#3\}}

\newcommand{\judgeType}[3]{#1\triangleright #2 : #3}


\newenvironment{scprooftree}[1]%
{\gdef\scalefactor{#1}\begin{center}\proofSkipAmount \leavevmode}%
    {\scalebox{\scalefactor}{\DisplayProof}\proofSkipAmount \end{center} }


\tikzset{
    every leaf node/.style={text=red, align=center},
    every tree node/.style={text=blue, align=center},
}

\forestset{tikzQtree/.style={for tree={if n children=0{
                node options=every leaf node/.try}{node options=every tree node/.try}, text centered}}}
                
                
\DeclareMathOperator{\Erase}{Erase}
\DeclareMathOperator{\Nat}{Nat}
\DeclareMathOperator{\Bool}{Bool}
\DeclareMathOperator{\Union}{Union}

\newcommand{\WFunc}{\mathbb{W}}

\newcommand{\red}[1]{{\color{red}#1}}%\renewcommand{\appendixtocname}{Apéndices}
\newcommand{\green}[1]{{\color{green!40!black}#1}}%\renewcommand{\appendixpagename}{Apéndices}





\setcounter{section}{7}

\begin{document}
  \title{PLP - Práctica 7: Programación Lógica}

  \date{\today}

  \author{Gianfranco Zamboni}

  \maketitle
  \setcounter{page}{1}


\section*{El motor de búsqueda de prolog}

\subsection{Ejercicio 1}
\subsubsection*{Base de conocimiento:}

\begin{multicols}{2}
\begin{centrado2}
\begin{minted}[tabsize=4]{prolog}
padre(juan, carlos).   %(1)
padre(juan, luis).     %(2)
padre(carlos, daniel). %(3) 
padre(carlos, diego).  %(4)
\end{minted}
\end{centrado2}
\begin{centrado2}
\begin{minted}[tabsize=4]{prolog}
padre(luis, pablo).   %(5)
padre(luis, manuel).  %(6)
padre(luis, ramiro).  %(7)
abuelo(X,Y) :-        %(8)
    padre(X,Z), 
    padre(Z,Y).
\end{minted}
\end{centrado2}
\end{multicols}

\paragraph{I.} La consulta \prolog{abuelo(X, manuel)}, devuele \prolog{X = juan}.

\subsubsection*{II.}
\begin{multicols}{2}
\begin{centrado2}
\begin{minted}[tabsize=4]{prolog}
hijo(X, Y) :-       %(9)
	padre(Y, X).

hermano(X, Y) :-    %(10)
	padre(Z, X), 
	padre(Z, Y),
	X \= Y.
\end{minted}
\end{centrado2}
\vfill\null
\columnbreak
\begin{centrado2}
\begin{minted}[tabsize=4]{prolog}
descendiente(X, Y) :-    %(11)
	hijo(X,Y).
descendiente(X,Y) :-     %(12)
	hijo(X, Z),
	descendiente(Z, Y).
\end{minted}
\end{centrado2}
\end{multicols}

\paragraph{IV.} \prolog{abuelo(juan,X)}
\paragraph{V.} \prolog{hermano(pablo,X)}
\paragraph{VII.} \prolog{ancestro(juan,X)} devuelve \prolog{X = juan} por la primer regla y luego, si se pide otro resultado, se cuelga porque \prolog{Z} no está bien instanciado.
\paragraph{VIII.} Con dar vuelta las condiciones de la segunda regla se arregla este problema
\begin{centrado}
\begin{minted}[tabsize=4]{prolog}
ancestro(X, X). \\
ancestro(X, Y) :- padre(X, Z), ancestro(Z, Y).
\end{minted}
\end{centrado}

\newpage
\subsection{Ejercicio 2}
\paragraph{I.}
\begin{center}
	\begin{forest}
[$\lnot\prolog{vecino(5,Y,[5,6,5,3])}$,
    [$\Box \{\prolog{Y}\leftarrow \prolog{6}\}$,edge label={node[midway,left=2mm,font=\footnotesize] {$(1)~\prolog{X}\leftarrow \prolog{5},~\prolog{Y}\leftarrow\prolog{6},\prolog{LS}\leftarrow\prolog{[5,3]}$}},
    ]
    [$\lnot\prolog{vecino(5,Y,[6,5,3])}$,edge label={node[midway,right=2mm,font=\footnotesize] {$(2)~ \prolog{X}\leftarrow \prolog{5},~\prolog{Y}\leftarrow\prolog{Y},\prolog{W}\leftarrow\prolog{5},\prolog{LS}\leftarrow\prolog{[6,5,3]}$}}
        [$\lnot\prolog{vecino(5,Y,[5,3])}$,edge label={node[midway,right=2mm,font=\footnotesize] {$(2)~ \prolog{X}\leftarrow \prolog{5},~\prolog{Y}\leftarrow\prolog{Y},\prolog{W}\leftarrow\prolog{6},\prolog{LS}\leftarrow\prolog{[5,3]}$}},
            [$\Box \{\prolog{Y}\leftarrow \prolog{3}\}$,edge label={node[midway,left=2mm,font=\footnotesize] {$(1)~\prolog{X}\leftarrow \prolog{5},~\prolog{Y}\leftarrow\prolog{3},\prolog{LS}\leftarrow\prolog{[]}$}},
                ]
            [$\lnot\prolog{vecino(5,Y,[3])}$,edge label={node[midway,right=2mm,font=\footnotesize] {$(2)~ \prolog{X}\leftarrow \prolog{5},~\prolog{Y}\leftarrow\prolog{Y},\prolog{W}\leftarrow\prolog{5},\prolog{LS}\leftarrow\prolog{[3]}$}},
                [$\lnot\prolog{vecino(5,Y,[])}$,edge label={node[midway,right=2mm,font=\footnotesize] {$(2)~ \prolog{X}\leftarrow \prolog{5},~\prolog{Y}\leftarrow\prolog{Y},\prolog{W}\leftarrow\prolog{3},\prolog{LS}\leftarrow\prolog{[]}$}},
                    [$\falla$]
                ]
            ]
        ]
    ]
]
	\end{forest}
\end{center}

\paragraph{II.} En este caso, si invertimos las reglas, obtendremos exactamente los mismos resultado y el árbol generado por prolog, es el espejo del inciso anterior.

\subsection{Ejercicio 3}
\subsubsection*{Bases de conocimiento}
\begin{minted}[tabsize=4]{prolog}
natural(0).
natural(suc(X)) :- natural(X).

menorOIgual(X, suc(Y)) :- menorOIgual(X, Y).
menorOIgual(X,X) :- natural(X).
\end{minted}

\paragraph{I.} Prolog  unifica \prolog{menorOIgual(0,X)} con la regla \prolog{menorOIgual(X, suc(Y))} remplazando \prolog{X} (de la regla) por \prolog{0} e instanciando el \prolog{X} de la consulta con \prolog{suc(Y)}, obteniendo la resolvente  \prolog{menorOIgual(0, Y)} que se resuelve de la misma manera. Entonces, entra en un ciclo infinito en el que siempre unifica con la primer regla.
\paragraph{II.} Esto se debe a que Prolog no tiene suficiente información sobre \prolog{X} como para descartar la primer regla (\prolog{X} no está correctamente instanciada) por lo que siempre es posible hacerla coincidir con cualquier regla.

\paragraph{III.} Otra vez, con cambiar el orden de evaluación de las reglas, el problema se arregla:
\begin{minted}[tabsize=4]{prolog}
menorOIgual(X,X) :- natural(X).
menorOIgual(X, suc(Y)) :- menorOIgual(X, Y).
\end{minted}

\section*{\ Estructuras, instanciación y reversibilidad}

\subsection{Ejercicio 4}
\begin{centrado}
\begin{minted}[tabsize=4]{prolog}
%concatenar(?Lista1, ?Lista2, ?Lista3)
concatenar([], Lista2, Lista2).
concatenar([X | T1], Lista2, [X| T3]) :-
	concatenar(T1, Lista2, T3).
\end{minted}
\end{centrado}

\subsection{Ejercicio 5}
\begin{multicols}{2}
\paragraph{I.}
\begin{centrado2}
\begin{minted}[tabsize=4]{prolog}
%last(?L, ?U)
last([ X ], X ).
last([ _ | T ], Y) :- last(T,Y).
\end{minted}
\end{centrado2}
\paragraph{II.}
\begin{centrado2}
\begin{minted}[tabsize=4]{prolog}
%tienenLaMismaLongitud(+L, +L1)
tienenLaMismaLongitud(L1, L2) :-
	length(L1, N ),
	length(L2, N ).

%reverse(+L, -L1)
reverse([],[]).
reverse( [ X | T ], R) :-
	tienenLaMismaLongitud([X | T], R),
	reverse(T, RT ),
	append(RT, [ X ], R).
\end{minted}
\end{centrado2}
\paragraph{III.}
\begin{centrado2}
\begin{minted}[tabsize=4]{prolog}
%maxlista(+L, -M).
maxlista([E],E).
maxlista([E|T],E):-
    maxlista(T,M),E>=M.
maxlista([E|T],M):-
    maxlista(T,M),E=<M.

%minlista(+L, -M).
minlista([E],E).
minlista([E|T],M):-
    minlista(T,M),E>=M.
minlista([E|T],E):-
    minlista(T,M),E=<M.
\end{minted}
\end{centrado2}

\paragraph{IV.}
\begin{centrado2}
\begin{minted}[tabsize=4]{prolog}
%prefijo(?P, +L).
prefijo([], _).
prefijo([E|T2], [E|T]):-
    prefijo(T2,T).
\end{minted}
\end{centrado2}

\paragraph{V.}
\begin{centrado2}
\begin{minted}[tabsize=4]{prolog}
%sufijo(?S, +L).
sufijo(S, L):-
    prefijo(P,L),
    append(P, S, L).
\end{minted}
\end{centrado2}

\paragraph{VI.}
\begin{centrado2}
\begin{minted}[tabsize=4]{prolog}
%sublista(?S, +L).
sublista([], _).
sublista(XS, L) :-
    prefijo(P, L),
    sufijo(S, L),
    append(P, XS, P1),
    append(P1, S, L),
    length(XS, N),
    N >= 1.
\end{minted}
\end{centrado2}
\paragraph{VII.}
\begin{centrado2}
\begin{minted}[tabsize=4]{prolog}
%pertenece(?X, +L).
pertenece(X, [X|_]).
pertenece(X, [Y|T]):-
    Y \= X,
    pertenece(X,T).
\end{minted}
\end{centrado2}
\end{multicols}

\subsection{Ejercicio 6}
\begin{centrado}
\begin{minted}[tabsize=4]{prolog}
aplanar([],[]).
aplanar([ [] | T ], Res) :-
	aplanar(T, Res).
aplanar([ [X | T1 ] | T ], Res) :-
	aplanar([ X | T1 ], Y),
	aplanar(T, RecT),
	append(Y, RecT, Res).
aplanar([ X | T ], [X | Res]) :-
	not(is_list(X)),
	aplanar(T, Res).
\end{minted}
\end{centrado}

\newpage
\subsection{Ejercicio 7}
\paragraph{I.}
\begin{centrado}
\begin{minted}[tabsize=4]{prolog}
%palindromo(+L, -L1)
palindromo(L, L1) :-
	reverse(L, A),
	append(L,A,L1).
\end{minted}
\end{centrado}

\paragraph{II.}
\begin{centrado}
\begin{minted}[tabsize=4]{prolog}
%doble(+L, -L1)
doble([], []).
doble([ X | T], [X, X | Rec ]) :-
	doble(T, Rec).
\end{minted}
\end{centrado}

\paragraph{III.}
\begin{centrado}
\begin{minted}[tabsize=4]{prolog}
%iesimo(?I, +L, -X)
iesimoAux(1, [X | _], X).
iesimoAux(I, [_ | T], Y) :-
	I1 is I - 1,
	iesimo(I1, T, Y).

iesimo(I, L, X) :- 
	length(L, N),
	between(1,N, I),
	iesimoAux(I, L, X).
\end{minted}
\end{centrado}

\subsection{Ejercicio 8}
\paragraph{I.} Se debe instanciar \prolog{X} en un valor especifico. Entonces devuelve todos los números a partir de ese valor. Si se instancia \prolog{Y} en un valor menor que \prolog{X}, entonces se cuelga porque siempre se puede unificar con la segunda regla. Si se instancia \prolog{Y} en un valor mayor, entonces devolverá como primer resultado el valor de \prolog{Y} y luego se colgará por la misma razón que antes.

Y si no se instancia \prolog{X} entonces tira error porque no tiene suficiente información sobre la variable como para realizar unificación.

\paragraph{II.}
\begin{centrado}
\begin{minted}[tabsize=4]{prolog}
%desde2(+X,?Y)
desde2(X, Y) :- 
	nonvar(Y),
	Y >= X.
desde2(X,Y) :-
	var(Y),
	desde(X,Y).
\end{minted}
\end{centrado}

\newpage
\subsection{Ejercicio 9}
\paragraph{I.}
\begin{centrado}
\begin{minted}[tabsize=4]{prolog}
%interseccionAux(+L1, +L2, +L3, -L4)
interseccionAux([], _, _, []).
interseccionAux([ X | T ], L2, Usados, Resultado) :-
	not(member(X, L2)),
	interseccionAux(T, L2, Usados, Resultado).
interseccionAux([ X | T ], L2, Usados, [ X | L4 ]) :-
	member(X, L2),
	not(member(X, Usados)),
	interseccionAux(T, L2, [ X | Usados], L4).
interseccionAux([ X | T ], L2, Usados, L4 ) :-
	member(X, Usados),
	interseccionAux(T, L2, Usados, L4).

%intersección(+L1, +L2, -L3)
interseccion(L1, L2, L3) :-
	interseccionAux(L1, L2, [], L3).

\end{minted}
\end{centrado}

\paragraph{II.}
\begin{centrado}
\begin{minted}[tabsize=4]{prolog}
sufijoDeLongitud(L, N, S) :-
	sufijo(S, L),
	length(S, N).

prefijoDeLongitud(L, N, S) :-
	prefijo(S, L),
	length(S, N).

%split(+N,+L, -L1, -L2) -- N y L deben estar definidos
split(N, L, L1, L2) :-
	length(L, M),
	N1 is M - N,
	prefijoDeLongitud(L, N, L1),
	sufijoDeLongitud(L, N1, L2).

\end{minted}
\end{centrado}

\paragraph{III.}
\begin{centrado}
\begin{minted}[tabsize=4]{prolog}
%borrar(+ListaOriginal, +X, -ListaSinXs)
borrar([], _, []).
borrar([X | T], X, ListaSinXs) :-
	borrar(T, X, ListaSinXs).
borrar([ Y | T], X, [ Y | Rec ]) :-
	X \= Y,
	borrar(T, X, Rec).
\end{minted}
\end{centrado}
\paragraph{IV.}
\begin{centrado}
\begin{minted}[tabsize=4]{prolog}
%sacarDuplicados(+L1, -L2)
sacarDuplicados([], []).
sacarDuplicados([ X | T], [ X | Rec ]) :-
	borrar(T, X, T1),
	sacarDuplicados(T1, Rec).
\end{minted}
\end{centrado}
\newpage
\paragraph{V.}
\begin{centrado}
\begin{minted}[tabsize=4]{prolog}
%concatenarTodas( +LL, -L)
concatenarTodas([], []).
concatenarTodas([ X | T], Res) :-
	concatenarTodas(T, Rec),
	concatenar(X, Rec, Res).

%todosSusMiembrosSonSublitas(+LListas, +L)
todosSusMiembrosSonSublitas([], _).
todosSusMiembrosSonSublitas([ X | XS], L) :-
	sublista(X, L),
	todosSusMiembrosSonSublitas(XS, L).
%reparto(+L, +N, -LListas)
reparto(L, N, LListas) :-
	length(LListas, N),
	todosSusMiembrosSonSublitas(LListas, L),
	concatenarTodas(LListas, L).
\end{minted}
\end{centrado}

\paragraph{VI.}
\begin{centrado}
\begin{minted}[tabsize=4]{prolog}
%repartoSinVacias(+L, -LListas)
repartoSinVacias(L, LListas) :-
	length(L, N),
	between(1, N, X),
	reparto(L, X, LListas),
	not(member([], LListas)).
\end{minted}
\end{centrado}

\subsection{Ejercicio 10}
\begin{centrado}
\begin{minted}[tabsize=4]{prolog}
%intercalar(?L1, ?L2, ?L3) -- Funciona para todas las combinaciones posibles.
intercalar([], [], []).
intercalar([], L, L) :-
	length(L, N),
	N >= 1.
intercalar(L, [], L) :-
	length(L, N),
	N >= 1.
intercalar([ X | T1], [ Y | T2 ], [ X, Y | T3 ] ) :-
	intercalar(T1, T2, T3).
\end{minted}
\end{centrado}

\subsection{Ejercicio 11}
\begin{centrado}
\begin{minted}[tabsize=4]{prolog}
%arbolEjemplo
arbolEjemplo(bin(bin(nil,1,nil),2,bin(bin(nil,10,nil),20,bin(nil,30,nil)))).

%vacio(?A)
vacio(nil).

%raiz(?A, ?R)
raiz(bin(_,X,_), X).

%altura(+A, -H)
altura(nil, 0).
altura(bin(I, _, D), H) :-
	altura(I, HI),
	altura(D, HD),
	H is max(HI, HD) + 1.

%cantidadDeNodos(+A, -N)
cantidadDeNodos(nil, 0).
cantidadDeNodos(bin(I, _, D), N) :-
	cantidadDeNodos(I, NI),
	cantidadDeNodos(D, ND),
	N is NI + ND + 1.
\end{minted}
\end{centrado}

\subsection{Ejercicio 12}
\begin{centrado}
\begin{minted}[tabsize=4]{prolog}
%inorder(+AB, -Lista)
inorder(nil, []).
inorder(bin(I, X, D), Inorder) :-
	inorder(I, LI),
	inorder(D, LD),
	append(LI, [X | LD], Inorder).

%arbolConInorder(-L, +AB)
arbolConInorder([], nil).
arbolConInorder(XS, bin(AI, X, AD)) :-
	reparto(XS, 2, [LI, [X | LD]]),
	arbolConInorder(LI, AI),
	arbolConInorder(LD, AD).

%abb(+T)
aBB(nil).
aBB(bin(I, _, D)) :-
	aBB(I),
	aBB(D).

%aBBInsertar(+X, +T1, -T2)
aBBInsertar(X, nil, bin(nil, X, nil)).
aBBInsertar(X, bin(AI, Y, AD), bin(AIM, Y, AD)) :-
	X =< Y,
	aBBInsertar(X, AI, AIM).
aBBInsertar(X, bin(AI, Y, AD), bin(AI, Y, ADM)) :-
	X >= Y,
	aBBInsertar(X, AD, ADM).
\end{minted}
\end{centrado}

\section*{Generate and test}

\subsection{Ejercicio 13}
\begin{centrado}
\begin{minted}[tabsize=4]{prolog}
%coprimos(-X,-Y)
coprimos(X,Y) :-
	desde(2, X),
	between(2, X, Y),
	1 is gcd(X,Y).
\end{minted}
\end{centrado}
\subsection{Ejercicio 14}
\begin{centrado}
\begin{minted}[tabsize=4]{prolog}
listasQueSuman([],0,0).
listasQueSuman([X|XS],S,N):-
    between(0,S,X),
    N2 is N-1,
    length(XS,N2), 
    S2 is S-X, 
    listasQueSuman(XS,S2,N2).

cuadradoSemiLatinoAux(_,0,[],_).
cuadradoSemiLatinoAux(M,N,[X|XS],S):- 
	N2 is N-1, 
	listasQueSuman(X,S,M),
	cuadradoSemiLatinoAux(M,N2,XS,S).

cuadradoSemiLatino(N,XS):-
	length(XS,N),
	desde(0,S),
	cuadradoSemiLatinoAux(N,N,XS,S).
\end{minted}
\end{centrado}

\subsection{Ejercicio 15}
(Solución dada en clases)
\begin{centrado}
\begin{minted}[tabsize=4]{prolog}
%ladoValido(+A, +B, +C)
ladoValido(A,B,C) :-
	S is B+C,
	A < S,
	D is abs(B-C),
	A > D.

%esTriangulo(+T)
esTriangulo(tri(A,B,C)) :-
	ladoValido(A,B,C),
	ladoValido(B,C,A),
	ladoValido(C,A,B).

%perim  etro(?T,?P)
perimetro(tri(A,B,C), P) :-
	desde2(3,P),
	M is P-2,
	between(1,M,A),
	N is P - A -1,
	between(1, N, B),
	C is P - A - B,
	esTriangulo(tri(A,B,C)).
\end{minted}
\end{centrado}

\section*{\ Negación por falla}

\subsection{Ejercicio 16}
\begin{centrado}
\begin{minted}[tabsize=4]{prolog}
%diferenciaSimétrica(Lista1, +Lista2, -Lista3)
diferenciaSimetrica([], L2, L2).
diferenciaSimetrica([ X | L1], L2, [ X | XS]) :-
	not(member(X, L2)),
	diferenciaSimetrica(L1, L2, XS).
diferenciaSimetrica([ X | L1], L2, XS) :-
	member(X, L2),
	borrar(L2, X, L2SinX),
	diferenciaSimetrica(L1, L2SinX, XS).
\end{minted}
\end{centrado}

\subsection{Ejercicio 17}
\paragraph{I.} La consulta busca todos los valores para los que valga $P(Y)\land \lnot Q(Y)$.
\paragraph{II.} Si se invierten el orden de los literales y queda \prolog{not(q(Y)),p(Y)}, entonces prolog se cuelga porque no puede instanciar \prolog{Y} correctamente.

\subsection{Ejercicio 18}
\begin{centrado}
\begin{minted}[tabsize=4]{prolog}
%sumList
sumList([], 0).
sumList([X | XS], N) :-
	sumList(XS, N1),
	N is N1 + X.

%diff
diff(L1, L2, DL) :-
	sumList(L1, N1),
	sumList(L2, N2),
	DL is abs(N1-N2).

%corteMásParejo(+L,-L1,-L2)
corteMasParejo(L, L1, L2) :-
	append(L1, L2, L),
	diff(L1,L2,DL),
	not((append(M1,M2,L), 
		diff(M1,M2,DM),
		DM < DL
		)).
\end{minted}
\end{centrado}
%
%\subsection{Ejercicio 19}
%\begin{itemize}
%\item
%\end{itemize}
%
%
%\subsection{Ejercicio 20}
%\begin{itemize}
%\item
%\end{itemize}
%
%\subsection{Ejercicio 21}
%\begin{itemize}
%\item
%\end{itemize}
%
%\subsection{Ejercicio 22}
%\begin{itemize}
%\item
%\end{itemize}
%
\end{document}
