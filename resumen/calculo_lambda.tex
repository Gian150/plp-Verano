\section{Cálculo Lambda Tipado}

El cálculo lambda es un modelo de computación turing completo basado en \textbf{funciones} introducido por \textbf{Alonzo Church}. Este modelo consiste en un conjunto de expresiones o terminos que representan abstracciones o aplicaciones de funciones y cuyos valores pueden ser determinados aplicando ciertas reglas sintacticas hasta obtener lo 	que se dice su forma normal, una expresión que, a falta de reglas no puede ser reducida de ninguna manera. En nuestro caso, estamos estudiando cálculo lambda tipado, es decir que habrá expresiones que, a pesar de estar bien formadas, no tendrán sentido.

\subsection{Expresiones de Tipos de \texorpdfstring{$\lambda^b$}{lambda b}}
El primer lenguaje lambda que usamos en la materia tiene dos \textbf{tipos} $Bool$ y $\sigma\rightarrow\theta$ que son los tipos de los valores booelanos y las funciones que van de un tipo $\sigma$ a un tipo $\theta$, respectivamente. Y lo notamos:
\begin{equation*}
	\sigma,\theta ~::=~ Bool ~|~ \sigma\rightarrow\theta
\end{equation*}

Una vez que definamos por completo el lenguaje lambda para estos dos tipos, esto es definir reglas de sintaxis, de tipado y de reducción de expresiones, vamos a extender el lenguaje con los naturales y, luego, con otros tipos de interés, como abstracciones de memoria y comandos.
\subsubsection{Términos de \texorpdfstring{$\lambda^b$}{lambda b}}
Ahora debemos definir los \textbf{términos} que nos permitirán escribir las expresiones válidas del tipado. Sea $\mathcal{X}$ un conjunto infinito enumerable de variables y $x\in\mathcal{X}$. Los \textbf{términos} de $\lambda^b$ están dados por:

\begin{equation*}
\begin{split}
    M, P, Q ~ ::&= ~ true \\
    & |~ false \\
    & |~ \lambdaIf{M}{P}{Q} \\
    & |~ \lambdaApp{M}{N} \\
    & |~ \lambdaAbs{x}{\sigma}{M} \\
    & |~ x
\end{split}
\end{equation*}

Esto significa que dados tres términos $M$, $P$ y $Q$, los términos válidos del lenguaje son:
\begin{itemize}
    \item $true$ y $false$ que representan las \textbf{constantes de verdad}.
    \item $ \lambdaIf{M}{P}{Q}$ que expresa el \textbf{condicional}.
    \item $\lambdaApp{M}{N}$ que indica la \textbf{aplicación} de la función denotada por el termino $M$ al argumento $N$.
    \item $\lambdaAbs{x}{\sigma}{M}$ que es una \textbf{función} (abstracción) cuyo parámetro formal es $x$ y cuyo cuerpo es $M$
    \item $x$, una \textbf{variable de términos}.
\end{itemize}

\subsubsection{Variables ligadas y libres}
Por como definimos el lenguaje, una variable $x$ puede ocurrir de dos formas: \textbf{libre} o \textbf{ligada}. Decimos que $x$ ocurre \textbf{libre} si no se encuentra bajo el alcance de una ocurrencia de $\lambda x$. Caso contrario ocurre ligada.

Por ejemplo:
$$\lambdaAbs{x}{Bool}{\lambdaIf{true}{\underbrace{x}_{ligada}}{\underbrace{y}_{libre}}} $$

Para conseguir las variables ligadas de una expresión, vamos a definir la función $FV$ que toma como parámetro una expresión y devuelve el conjunto de variables libres de la misma.

\begin{equation*}
\begin{split}
FV(x) &\equalDef {x} \\
FV(true) = FV(false) &\equalDef \phi \\
FV(\lambdaIf{M}{P}{Q}) &\equalDef FV(M)\cup FV(P)\cup FV(Q) \\
FV(\lambdaApp{M}{N}) &\equalDef FV(M)\cup FV(N) \\
FV(\lambdaAbs{x}{\sigma}{M}) &\equalDef FV(M) \backslash \{x\}
\end{split}
\end{equation*}

\subsubsection{Reglas de sustitución}

Una de las operaciones que podemos realizar sobre las expresiones del lenguaje es la \textbf{sustitución} que, dado un término $M$, sustituye todas las ocurrencias \textbf{libres} de una variable $x$ en dicho término por un término $N$. La notamos:

$$\replaceBy{M}{x}{N}$$

Esta operación nos sirve para darle semántica a la aplicación de funciones y es sencilla de definir, sin embargo debemos tener en cuenta algunos casos especiales.

\paragraph{$\alpha$-equivalencia} Dos terminos $M$ y $N$ que difieren solamente en el nombre de sus variables ligadas se dicen $\alpha$-equivalentes. Esta relación es una relación de equivalencia. Técnicamente, la sustitución está definida sobre clases de $\alpha$-equivalencia de términos


\paragraph{Captura de variables}\label{calculo_lambda:captura_variables} El primer problema se da cuando la sustitución que deseamos realizar sustituye una variable por otra con el mismo nombre que alguna de las variables ligadas de la expresión. Por ejemplo:
$$\replaceBy{(\lambdaAbs{z}{\sigma}{x})}{x}{z} = \lambdaAbs{z}{\sigma}{z}$$

En estos casos, si realizamos la sustitución cambiariamos el significado de la expresión (en el caso mostrado, estariamos convirtiendo la función constante que devuelve $x$ en la función identidad). Por esta razón debemos asegurarnos que cuando realizemos la operación $\replaceBy{\lambdaAbs{y}{\sigma}{M}}{x}{N}$, la variable ligada $y$ sea renombrada de tal manera que \textbf{no} ocurra libre en $N$.

\vspace*{5mm}
Entonces, teniendo en cuenta lo mencionado, definimos el comportamiento de la operación:

\begin{equation*}
\begin{split}
\replaceBy{x}{x}{N} &\equalDef N \\
\replaceBy{a}{x}{N} &\equalDef a \text{ si } a \in \{true,false\}\cup\mathcal{X}\backslash\{x\} \\
\replaceBy{(\lambdaIf{M}{P}{Q})}{x}{N} &\equalDef \lambdaIf{\replaceBy{M}{x}{N}}{\replaceBy{P}{x}{N}}{\replaceBy{Q}{x}{N}}\\
\replaceBy{(\lambdaApp{M_1}{M_2})}{x}{N} &\equalDef \lambdaApp{\replaceBy{M_1}{x}{N}}{\replaceBy{M_2}{x}{N}}\\
\replaceBy{(\lambdaAbs{y}{\sigma}{M})}{x}{N} &\equalDef \lambdaAbs{y}{\sigma}{\replaceBy{M}{x}{N}}~x\neq y,~y\notin~FV(N)
\end{split}
\end{equation*}

La condición $x\neq y,~y\notin~FV(N)$ está para que efectivamente no se produzca la situación mencionada en el parrafo anterior. Y \textbf{siempre} puede cumplirse, solo hay que renombrar las variables de manera apropiada.

\subsubsection{Árbol sintáctico}
Dada una expresión $M$, su árbol sintáctico es un árbol que tiene como raíz a $M$ y como hijos de la raíz a todos los subtérminos válidos de la expresión.
\paragraph{Ejemplos}
El árbol sintáctico de $true$ es:
\begin{center}
        \begin{forest} tikzQtree,
[$true$]
        \end{forest}
\end{center}

El árbol sintáctico de $\lambdaIf{x}{y}{\lambdaAbs{z}{Bool}{z}}$ es:

\begin{center}
    \begin{forest} tikzQtree,
[$\lambdaIf{x}{y}{\lambdaAbs{z}{Bool}{z}}$, [$x$][$y$][$\lambdaAbs{z}{Bool}{z}$[$z$]    ]
]
\end{forest}
\end{center}

\subsection{Sistema de tipado}
El sistema de tipado es un sistema formal de deducción (o derivación) que utiliza axiomas y reglas de tipado para caracterizar un subconjunto de los términos. A estos términos los llamamos \textbf{términos tipados}.

Como dijimos, vamos a estudiar lenguajes de cálculo lambda tipado, por lo que para que una expresión sea considerada una expresión válida del lenguaje no solo debe ser sintácticamente correcta sino que debemos poder inferir su tipo a través del sistema de tipado que definamos. Y si no es posible inferir el tipo de una expresión con el sistema dado, entonces no la consideraremos una expresión válida del lenguaje.

\paragraph{Contexto de tipado}: Es un conjunto de pares $x_i:\sigma_i$, anotado $\Gamma = \{x_1:\sigma_1, \dots, x_n:\sigma_n\}$ que nos indica los tipos de cada variable de un programa.

Dado un contexto de tipado $\Gamma$, un \textbf{juicio de tipado} es una expresion $\judgeType{\Gamma}{M}{\sigma}$ que se lee ``el término M tiene tipo $\sigma$ asumiendo el contexto de tipado $\Gamma$''. 

\subsubsection{Axiomas de tipado de \texorpdfstring{$\lambda^b$}{lambda b}}

\begin{equation*}
\begin{gathered}
    \frac{}{\judgeType{\Gamma}{\lambdaValue{true}}{Bool}}(\text{T-True}) \hspace*{2cm} \frac{}{\judgeType{\Gamma}{\lambdaValue{false}}{Bool}}(\text{T-False})\hspace*{2cm} \\
    \vspace*{5mm} \\
    \frac{x:\sigma\in\Gamma}{\judgeType{\Gamma}{\lambdaValue{x}}{\sigma}}(\text{T-Var})\hspace*{2cm}
\end{gathered}
\end{equation*}

\vspace*{5mm}
Los axiomas \textbf{T-True} y \textbf{T-False} nos dicen, que no importa el contexto en el que se encuentren los valores \textit{true} y \textit{false}, respectivamente, ambos valores serán de tipo \textit{Bool}. El axioma \textbf{T-Var}, nos dice que una variable libre $x$ es de $\sigma$ en un contexto $\Gamma$ entonces el par $x:\sigma$ se encuentra en $\Gamma$

\subsubsection{Reglas de tipado de \texorpdfstring{$\lambda^b$}{lambda b}}
\begin{equation*}
\begin{gathered}
\frac{\judgeType{\Gamma}{M}{Bool}\hspace*{5mm}\judgeType{\Gamma}{P}{\sigma}\hspace*{5mm}\judgeType{\Gamma}{Q}{\sigma}}{\judgeType{\Gamma}{\lambdaIf{M}{P}{Q}}{\sigma}}(\text{T-If}) \\
\vspace*{5mm} \\
\frac {\judgeType{\Gamma,x:\sigma}{M}{\tau}}
      {\judgeType{\Gamma}{\lambdaValue{\lambdaAbs{x}{\sigma}{M}}}{\sigma\to\tau}}(\text{T-Abs})\hspace*{2cm}
\frac{\judgeType{\Gamma}{M}{\sigma\to\tau}\hspace*{5mm}\judgeType{\Gamma}{N}{\sigma}}{\judgeType{\Gamma}{\lambdaApp{M}{N}}{\tau}}(\text{T-App})
\end{gathered}
\end{equation*}

\vspace*{5mm}
\textbf{T-If} nos dice que si $\lambdaIf{M}{P}{Q}$ es de tipo $\sigma$ en $\Gamma$, entonces $M$ es de tipo $Bool$ y $P$ y $Q$ son de tipo $\sigma$ en $\Gamma$.

\textbf{T-Abs} indica que si $\lambdaAbs{x}{\sigma}{M}$ es de tipo $\sigma\to\tau$ en $\Gamma$, entonces $M$ es de tipo $\tau$ y $x$ en de tipo $\sigma$ en $\Gamma$.

\textbf{T-App} significa que si $\lambdaApp{M}{N}$ es de tipo $\sigma\to\tau$ en $\Gamma$, entonces $M$ es de tipo $\tau$ en el contexto $\Gamma, x:\sigma$. Este es el contexto formado por la unión disjunta entre $\Gamma$ y $x:sigma$, o en castellano, el contexto que remplaza el tipo de $x$ en $Gamma$ por $\sigma$.

\subsubsection{Resultados básicos}

Si $\judgeType{\Gamma}{M}{\sigma}$ puede derivarse usando los axiomas y reglas de tipados decimos que el juicio es \textbf{derivable}. Además, si el juicio se puede derivar para algún $\Gamma$ y $\sigma$, entonces decimos que $M$ es \textbf{tipable}.

\paragraph{Unicidad de tipos} Si $\judgeType{\Gamma}{M}{\sigma}$ y $\judgeType{\Gamma}{M}{\tau}$ son derivables, entonces $\sigma = \tau$

\paragraph{Weakening + Strengthening} Si $\judgeType{\Gamma}{M}{\sigma}$ es derivable y $\Gamma\cap\Gamma'$ contiene a todas las variables libres de $M$, entonces $\judgeType{\Gamma'}{M}{\sigma}$

\paragraph{Sustitución} Si $\judgeType{\Gamma,x:\sigma}{M}{\tau}$ y $\judgeType{\Gamma}{N}{\sigma}$ son derivables, entonces $\judgeType{\Gamma}{\replaceBy{M}{x}{N}}{\tau}$ es derivable.

\subsubsection{Demostración de juicios de tipado}
Dado un sistema tipado, queremos ver si un juicio de tipado es correcto. Para hacer esto, iremos aplicando, al juicio, las reglas del sistema hasta llegar a sus axiomas o hasta llegar a una contradicción o incertidumbre. Si pasa lo primero, entonces el juicio es correcto, si pasa lo segundo, el juicio está mal.

\subsection{Semántica operacional}Ya definimos cuales serán los términos y expresiones válidas de nuestro lenguaje. El siguiente paso, es definir algún mecanismo que nos permita inferir el significado o \textbf{valor} de un término. 

Para lograr este objetivo definimos lo que se llama \textbf{semántica operacional}, un mecanismo que interpreta a los \textbf{términos como estados} de una máquina abstracta y define una \textbf{función de transición} que indica, dado un estado, cual es el siguiente.

De esta forma, el significado de un término $M$ es el estado final que alcanza la máquina al comenzar con $M$ como estado inicial.

Tenemos dos formas de definir la semántica:
\begin{itemize}
    \item \textbf{Small-step}: La función de transición describe un paso de computación, descomponiendo los términos compuestos en términos más simples y especificando el orden el que deben ser reducidos.
    \item \textbf{Big-step} (o \textbf{Natural Semantics}): La función de transición, en un paso, evalúa el termino a su resultado.
\end{itemize}

Nosotros vamos a usar la primer opción. Y la formulamos a través de \textbf{juicios de evaluación} 
$$M\to N$$ que se leen ``\textit{el término M reduce, en un paso, al término N}''.

Para establecer el significado de estos juicios, vamos a definir \textbf{axiomas de evaluación} y \textbf{reglas de evaluación}. Los axiomas nos indicarán cuales juicios de evaluación son siempre derivables y las reglas nos dirán que juicios son derivables dado un contexto. Las reglas de la semántica asumen que las expresiones están bien tipadas.

\subsubsection{Expresiones Booleanas}\label{calculo_lambda:semantica:booleanas}
Los valores de las  expresiones booleanas son:
$$ V~::=~true~|~false$$
y son usados para reducir el término $\lambdaIf{M_1}{M_2}{M_3}$ mediante los siguientes axiomas:

\begin{equation*}
\frac{}{\lambdaIf{\lambdaValue{true}}{M_1}{M_2} \to M_1}(\text{E-IfTrue})
\end{equation*}
\vspace*{5mm}
\begin{equation*}
\frac{}{\lambdaIf{\lambdaValue{false}}{M_1}{M_2} \to M_2}(\text{E-IfFalse})
\end{equation*}
\vspace*{5mm}
\begin{equation*}
\frac{M_1\to M'_1}{\lambdaIf{M_1}{M_2}{M_3}\to\lambdaIf{M'_1}{M_2}{M_3}}(\text{E-If})
\end{equation*}

\vspace*{5mm}
Estas reglas nos indican que dado un término del tipo $\lambdaIf{M_1}{M_2}{M_3}$, si $M_1 = true$, entonces podemos remplazar la expresión por $M_2$, si $M_1=false$ entonces podemos remplazar la expresión por $M_3$ y si $M_1$ es una expresión reducible a $M'_1$, entonces podemos remplazar la expresión por $\lambdaIf{M'_1}{M_2}{M_3}$.

Con estas reglas definimos la estrategia de evaluación del condicional que se corresponde el orden habitual en lenguajes de programación:
\begin{enumerate}
    \item Primero evaluamos la guarda del condicional
    \item y una vez que la guarda sea un valor, evaluamos la expresión del \textit{then} o del \textit{else} según corresponda.
\end{enumerate}

\subsubsection{Propiedades}

\paragraph{Determinismo} Si $M\to M'$ y $M\to M''$ entonces $M' = M''$, esto quiere decir que el valor que representa $M$ no cambia con las reducciones que le apliquemos.

\paragraph{Valores en forma normal} Una \textbf{forma normal} es un término que no puede evaluarse más. Consideraremos que terminamos de evaluar un término cuando conseguimos su forma normal.

Todos los valores tiene una forma normal, sin embargo hay que tener en cuenta que como estamos definiendo un lenguaje tipado, habrá formas normales que no representen ningún valor.

\subsubsection{Evaluación en muchos pasos}
El juicio de \textbf{evualuación de muchos pasos} $\twoheadrightarrow$ es la clausura reflexiva, transitiva de $\to$. Es decir, la menor relación tal que:
\begin{enumerate}
    \item Si $M\to M'$, entonces $M\twoheadrightarrow M'$
    \item $M\twoheadrightarrow M$ para todo $M$
    \item Si $M\twoheadrightarrow M'$ y $M' \twoheadrightarrow M''$, entonces $M\twoheadrightarrow M''$
\end{enumerate}

\paragraph{Unicidad de formas normales} Si $M\twoheadrightarrow U$ y $M\twoheadrightarrow V$ con $U$ y $V$ formas normales, entonces $U = V$

\paragraph{Terminación}
Para todo $M$ existe una forma normal $N$ tal que $M\twoheadrightarrow N$


\subsection{Semántica operacional de \texorpdfstring{$\lambda^b$}{lambda b}}
En la sección \ref{calculo_lambda:semantica:booleanas} definimos el comportamiento de las expresiones booleanas, sin embargo, nos falta definir como reducir términos del tipo $\lambdaAbs{x}{\sigma}{M}$ y $\lambdaApp{M}{N}$.

Lo primero a tener en cuenta, es que vamos a considerar a los términos de $\lambdaAbs{x}{\sigma}{M}$ como valores, sin si $M$ es reducible o nó. Entonce, nuestro conjunto de valores del lenguaje sería:

$$ V  ~::=~ true~|~false~|~\lambdaAbs{x}{\sigma}{M}$$

Por lo que todo término bien tipado y cerrado (sin variables libres) evalúa a alguna de estos términos. Si es de tipo $Bool$ evalúa a $true$ o $false$, si es de tipo $\sigma\to\tau$ evalúa a $\lambdaAbs{x}{\sigma{M}}$
A las reglas y axiomas definidos para los tipos booleanos agregamos los siguientes:

\begin{equation*}
\frac{M_1\to M'_1}{\lambdaApp{M_1}{M_2} \to 
\lambdaApp{M'_1}{M_2}}(\text{E-App1}~ /~ \mu)
\end{equation*}
\vspace*{5mm}
\begin{equation*}
\frac{M_2 \to M'_2}{\lambdaApp{\lambdaValue{V_1}}{M_2} \to 
	\lambdaApp{\lambdaValue{V_1}}{M'_2}}(\text{E-App2}~/~v)
\end{equation*}	
\vspace*{5mm}
\begin{equation*}
\frac{}{\lambdaApp{(\lambdaAbs{x}{\sigma}{M})}{\lambdaValue{V}} \to 
	\replaceBy{M}{x}{\lambdaValue{V}}}(\text{E-App2}~/~\beta)
\end{equation*}

\paragraph{Estado de error} Es un estado que \textbf{no es} un valor pero en el que la computación está trabada. Representa el estado en el cual el sistema de runtime de una implementación real generaría una excepción.

El sistema de tipado, nos garantiza que si un término cerrado está bien tipado entonces evalúa a un valor.


 \paragraph{Corrección}
La corrección de un término nos asegura dos cosas:	\textbf{Progreso} y \textbf{Preservación}.

El \textbf{progreso} asegura que si $M$ es un término cerrado y bien tipado, entonces $M$ es un valor o existe $M'$ tal que $M\to M'$. En otras palabras, nos asegura que la evaluación no puede trabarse para términos cerrados y bien tipados que no son valores. Y si un programa termina, entonces nos devuelve un valor.

La \textbf{preservación} asegura que la evaluación de un término $M$ cerrado y bien tipado preserva tipos. Es decir, no importa cuanta veces se reduzca $M$, el término resultante siempre es del tipo original.

$$\text{Si } \judgeType{\Gamma}{M}{\sigma} \text{ y } M\to N \text{ entonces } \judgeType{\Gamma}{N}{\sigma}$$	

\paragraph{Extendiendo el lenguaje}
Cuando queramos extender el lenguaje, debemos realizar los mismos pasos que realizamos para definir el lenguaje $\lambda^b$, esto es decir, agregar el nuevo tipo al conjunto de tipos, definir los términos de ese tipo, sus reglas de tipado y sus reglas semánticas, asegurándonos de que las nuevas reglas no interfieran con las ya definidas. Esto es, no debemos definir reglas que las contradigan o que den nuevas formas de inferir algo que ya se podía inferir con otras reglas.

En el apéndice de extensiones, muestro algunas extensiones que servirán como ejemplo.


\paragraph{Macros} Hay expresiones del lenguaje que usaremos con demasiada frecuencia, para estas expresiones podremos definir macros que simplificaran su escritura. Algunos ejemplos son:

$$Id_{Bool} \equalDef \lambdaAbs{x}{Bool}{x}$$
$$and \equalDef \lambdaAbs{x}{Bool}{\lambdaAbs{y}{Bool}{\lambdaIf{x}{y}{false}}}$$

\subsection{Extensión Naturales (\texorpdfstring{$\lambda^{bn}$}{lambda bn})}

\paragraph{Tipos}
$$\sigma, \tau ~::=~ Bool~|Nat~|~\sigma\to\tau$$

\paragraph{Términos}
$$ M~::=~ \dots~|~0~|~succ(M)~|~pred(M)~|~isZero(M) $$

Los términos significan:
\begin{itemize}
    \item $succ(M)$: evaluar $M$ hasta arrojar un número e incrementarlo.
    \item $pred(M)$: evaluar $M$ hasta arrojar un número y decrementar.
    \item $iszero(M)$: evaluar $M$ hasta arrojar un número, luego retornar $true/false$ según sea cero o no.
\end{itemize}

\paragraph{Axiomas y reglas de tipado}
\begin{equation*}
\frac{}{\judgeType{\Gamma}{0}{Nat}}(\text{T-Zero})
\end{equation*}
\vspace*{5mm}
\begin{equation*}
\frac{\judgeType{\Gamma}{M}{Nat}}
{\judgeType{\Gamma}{succ(M)}{Nat}}(\text{T-Succ})\hspace*{1cm}
\frac{\judgeType{\Gamma}{M}{Nat}}{\judgeType{\Gamma}{pred(M)}{Nat}}(\text{T-Pred})
\end{equation*}
\vspace*{5mm}
\begin{equation*}
\frac{\judgeType{\Gamma}{M}{Nat}}{\judgeType{\Gamma}{isZero(M)}{Bool}}(\text{T-IsZero})
\end{equation*}

\paragraph{Valores}
$$V~::=~\dots~|~\underline{n}\text{ donde } \underline{n} \text{ abrevia } succ^n(0)$$

\paragraph{Axiomas y reglas de evaluación}

\begin{equation*}
\frac{M_1\to M_1'}{succ(M_1)\to succ(M'_1)}(\text{E-Succ})
\end{equation*}
\vspace*{5mm}
\begin{equation*}
\frac{}{pred(0)\to 0}(\text{E-PredZero})\hspace*{1cm}\frac{}{pred(succ(\underline{n}))\to \underline{n}}(\text{E-PredSucc})
\end{equation*}
\vspace*{5mm}
\begin{equation*}
\frac{M_1\to M_1'}{pred(M_1)\to pred(M'_1)}(\text{E-Pred})
\end{equation*}
\vspace*{5mm}
\begin{equation*}
\frac{}{isZero(0)\to true}(\text{E-IsZeroZero})\hspace*{1cm}\frac{}{isZero(succ(\underline{n}))\to false}(\text{E-isZeroSucc})
\end{equation*}
\vspace*{5mm}
\begin{equation*}
\hspace*{1cm}\frac{M_1\to M'_1}{isZero(M_1)\to isZero(M_1')}(\text{E-isZero})
\end{equation*}



\subsection{Simulación de lenugajes imperativos}
Los lenguajes imperativos se caracterizan por su capacidad de asignar y modificar variables dentro de un programa. Esto lo hace a través de comandos, expresiones del lenguaje cuyo objetivo es crear un efecto sobre el estado de la computadora. 

Queremos extender el lenguaje $\lambda$ para que poder simular comandos y efectos sobre la memoria.

En un lenguaje imperativo \textbf{todas} las variables son \textbf{mutables}, es decir, que hay operaciones que pueden modificar su valor. Para lograr esto, hace uso de tres operaciones básicas:

\begin{itemize}
    \item \textbf{Asignación:} $x := M$ almacena en la referencia $x$ el valor de $M$
    \item \textbf{Alocación (Reserva de memoria)} $ref~M$ genera una referencia fresca cuyo contenido es el valor de $M$
    \item \textbf{Derreferenciación (Lectura):} $!x$ sigue la referencia $x$ y retorna su contenido.
\end{itemize}


Notemos que una vez que agreguemos estas expresiones al lenguaje lambda, este dejará de ser un lenguaje funcional \textbf{puro} (un lenguaje en el todas sus expresiones carecen de efecto).

Nos gustaría agregar las expresiones mencionadas a nuestro lenguaje, para esto primero debemos asignarles un tipo. 

\paragraph{Asignacion} Lo primero que debemos tener en cuenta, es que la igualdad ($x := M$) es una expresión de la cual no nos interesa saber su valor sino el efecto que tiene la misma sobre el contexto. Entonces, debemos definir un nuevo tipo que nos permita identificar cuando una expresión evaluada solo fue evaluada para generar un efecto. Nombraremos este tipo $Unit$ y su conjunto de valores será solo el valor $unit$. Podemos decir que este tipo cumple el rol de $void$ en C.

\paragraph{Macro punto y coma ( ; )} En lenguajes con efectos laterales, como el que estamos definiendo, esta macro nos servirá para definir el orden de evaluación de varias expresiones en \textbf{secuencia}.

$$M_1;M_2 \equalDef \lambdaApp{(\lambdaAbs{x}{Unit}{M_2})}{M_1} \hspace*{5mm} x\notin FV(M_2)$$

Por como definimos las reglas semánticas del lenguaje, esto significa que primero se evalúa $M_1$ y luego $M_2$. 

\subsubsection{Extensión con Referencias (\texorpdfstring{$\lambda^{bnu}$}{lambda bnu})}



\paragraph{Referencias}
Una referencia es una abstracción de una porción de memoria que se encuentra en uso. Usaremos el tipo $Ref~\sigma$ para diferenciar las expresiones que representan referencias. 

\paragraph{Representación}
Representaremos las posiciones con \textbf{direcciones simbólicas} o \textit{locations} usando etiquetas $l,l_1$ y definiremos a la \textbf{memoria} o \textit{store} como una función parcial $\mu$ que dada una dirección nos devuelve el valor almacenado en ella. Y notaremos:
\begin{itemize}
    \item $\mu[l\to V]$ es el store resultante de \textbf{pisar} $\mu(l)$ con $V$.
    \item $\mu\oplus(l\to V)$ es el \textbf{store extendido} resultante de ampliar $\mu$ con una nueva asociación $l \to V$ asumiendo que $l \notin Dom(\mu)$.    
\end{itemize}

\paragraph{Uso en semántica} Ahora necesitamos una forma de usar estas nuevas definiciones en nuestras evaluaciones, por lo que agregaremos las etiquetas al conjunto de valores y, a partir de ahora, los juicios de evaluación, tendrán la siguiente forma: 
$$M|\mu \to M'|\mu'$$
Esto significa que una expresión $M$ reduce a $M'$ y que afecta a $\mu$ de tal forma que pasa a ser $\mu'$, así reflejaremos los cambios de estado de la memoria.

Notemos que, a pesar de que agregamos las etiquetas $l$ como términos y valores, estás son solo producto de la formalización y \textbf{no} se pretende que sean usadas por el programador.

\paragraph{Uso en tipado} Con la posibilidad de modificar la memoria durante la ejecución de un programa, se hace necesaria la definición de un contexto que nos permita inferir el tipo del valor almacenado en las posiciones usadas. Introducimos el \textbf{contexto de tipado} $\Sigma$ para direcciones como una función parcial de direcciones a tipos. Y los juicios de tipado serán de la siguiente forma:

$$\judgeType{\Gamma|\Sigma}{M}{\sigma}$$

Indicando esto, que $M$ es de tipo $\sigma$ en el contexto $\Gamma$ cuando el estado de la memoria se corresponde con el contexto de tipado $\Sigma$.

\subsubsection{La extension de la que tanto hablamos}

\paragraph{Tipos}
$$\sigma, \tau ~::=~ Bool~|Nat~|~\blue{Unit}~|~\blue{Ref~\sigma}~|~\sigma\to\tau$$

\paragraph{Términos}

$$ M~::=~ \dots~|~unit~|~\lambdaRef{M}~|~!M~|~\lambdaAssign{M}{N} |~    l$$

\paragraph{Axiomas y reglas de tipado}
\begin{equation*}
\frac{}{\judgeType{\Gamma|\Sigma}{unit}{Unit}}(\text{T-Unit})\hspace*{1cm}\frac{\judgeType{\Gamma|\Sigma}{M_1}{\sigma}}{\judgeType{\Gamma|\Sigma}{\lambdaRef{M_1}}{Ref~\sigma}}(\text{T-Ref})
\end{equation*}
\vspace*{5mm}
\begin{equation*}
\frac{\judgeType{\Gamma|\Sigma}{M_1}{Ref~\sigma}}{\judgeType{\Gamma}{!M_1}{\sigma}}(\text{T-DeRef})
\end{equation*}
\vspace*{5mm}
\begin{equation*}
\frac{\judgeType{\Gamma|\Sigma}{M_1}{Ref~\sigma}\hspace*{5mm}\judgeType{\Gamma|\Sigma}{M_2}{\sigma}}{\judgeType{\Gamma}{\lambdaAssign{M_1}{M_2}}{Unit}}(\text{T-Assing})
\end{equation*}
\vspace*{5mm}
\begin{equation*}
\frac{\Sigma(l) = \sigma}{\judgeType{\Gamma|Signa}{l}{Ref~\sigma}}(\text{T-Loc})
\end{equation*}

\paragraph{Valores}
$$V~::=~\dots~|~unit~|~l$$

\paragraph{Axiomas y reglas semánticas}
\begin{equation*}
\frac{M_1|\mu\to M'_1|\mu'}{\lambdaApp{M_1}{M_2}|\mu\to \lambdaApp{M'_1}{M_2}|\mu'}(\text{E-App1})\hspace*{1cm}\frac{M_2|\mu\to M'_2|\mu'}{\lambdaApp{\lambdaValue{V_1}}{M_2}|\mu\to \lambdaApp{\lambdaValue{V_1}}{M'_2}|\mu'}(\text{E-App2})
\end{equation*}
\vspace*{5mm}
\begin{equation*}
\frac{}{(\lambdaApp{\lambdaAbs{x}{\sigma}{M})}{\lambdaValue{V}}|\mu\to \replaceBy{M}{x}{\lambdaValue{V}}|\mu'}(\text{E-AppAbs})
\end{equation*}
\vspace*{5mm}
\begin{equation*}
\frac{M_1|\mu\to M'_1|\mu'}{!M_1|\mu\to !M_1'|\mu'}(\text{E-DeRef})\hspace*{1cm}
\frac{\mu(l) = \lambdaValue{V}}{!l|\mu\to V|\mu}(\text{E-DerefLoc})
\end{equation*}
\vspace*{5mm}
\begin{equation*}
\frac{M_1|\mu\to M'_1|\mu'}{\lambdaAssign{M_1}{M_2}|\mu\to \lambdaAssign{M'_1}{M_2}|\mu'}(\text{E-Assign1})
\end{equation*}
\vspace*{5mm}
\begin{equation*}
\frac{M_2|\mu\to M'_2|\mu'}{\lambdaAssign{\lambdaValue{V}}{M_2}|\mu\to \lambdaAssign{\lambdaValue{V}}{M'_2}|\mu'}(\text{E-Assign2})
\end{equation*}
\vspace*{5mm}
\begin{equation*}
\frac{}{\lambdaAssign{l}{\lambdaValue{V}}|\mu\to unit|\mu[l\to \lambdaValue{V}]}(\text{E-Assign})
\end{equation*}
\vspace*{5mm}
\begin{equation*}
\frac{M_1|\mu\to M'_1|\mu'}{\lambdaRef{M_1}|\mu\to \lambdaRef{M'_1}|\mu'}(\text{E-Ref})\hspace*{1cm}
\frac{l\notin Dom(\mu)}{\lambdaRef{\lambdaValue{V}}|\mu\to l|\mu\oplus(l\to \lambdaValue{V})}(\text{E-RefV})
\end{equation*}

\subsubsection{Correción de tipos en un lenguaje con referencias}
Al agregar referencia, hay consecuencias. Una de ellas es que no todo término cerrado y bien tipado termina. Por lo que debemos reformular las definiciones de corrección del lenguaje, es decir, debemos indicar que significa el \textbf{progreso} y la \textbf{preservación} cuando hay referencias.

\paragraph{Preservación} La preservación nos aseguraba que no importa cuantas veces reduzcamos una expresión, esta debería mantener su tipo. Sin embargo, con las asignaciones podemos cambiar el tipo de ciertos valores durante la ejecución de un programa, lo que implica que la expresión podria cambiar su tipo. Para definir la preservación precisamos una noción de compatibilidad entre el store y el contexto de tipado que nos permita asegurar que si los valores no cambian su tipo, entonces la expresión mantiene su tipo.

Decimos que $\Gamma|\Sigma\triangleright\mu$ si y solo si $Dom(\Sigma) = Dom(\mu)$ y $\judgeType{\Gamma|Sigma}{\mu(l)}{\Sigma(l)}$ para todo $l\in Dom(\mu)$. Es decir, $\mu$ es compatible con $\Sigma$ si ambas funciones tiene el mismo dominio, y es cierto que los tipos de cada etiqueta de $\mu$ coinciden con los tipos que se les asignó en $\Sigma$.

Entonces definimos la preservación de la siguiente manera:
\begin{centrado}
    \textbf{Si}  $\judgeType{\Gamma|\Sigma}{M}{\sigma}$ y $M|\mu\to N|\mu'$ y $\Gamma|\Sigma\triangleright\mu$ \textbf{entonces existe un} $\Sigma'$ \textbf{que contiene a} $\Sigma$\textbf{ tal que} $\judgeType{\Gamma|\Sigma'}{N}{\sigma}$ y $\Gamma|\Sigma'\triangleright\mu'$
\end{centrado}

La nueva definición nos dice que dada una expresión $M$ de tipo $\sigma$ y un contexto $\Gamma\Sigma$ compatible con $\mu$, si pasa que cuando reducimos $M|\mu\to N|\mu'$, $mu'$ es compatible con $\Sigma'$, entonces la reducción tendrá el mismo tipo que $M$.

Para que $\mu'$ sea compatible con $\Sigma'$ puede haber dos posibilidad: O qué $\mu' = \mu$ y $\Sigma = \Sigma'$ o qué $\mu'$ sea una extensión de $\mu$, es decir que se haya creado una referencia nueva, en cuyo caso ninguno de los tipos fue modificado y $\Sigma'$ es $\Sigma$ extendido con el tipo de la nueva referencia.

\paragraph{Progreso} El progreso nos aseguraba que dada una expresión, entonces su ejecución termina en un valor o no termina. Para el nuevo lenguaje, hay que tener en cuenta el contexto de tipado:

\begin{centrado}
    Si $M$ es cerrado y bien tipado en un contexto de tipado de memoria $\Sigma$, entonces
    \begin{itemize}
        \item $M$ es un valor
        \item o bien para cualquier memoria $\mu$ que sea compatible con $\Sigma$, existe $M'$ y $\mu'$ tal que $M|\mu\to M'|\mu'$
    \end{itemize}
\end{centrado}

Esto quiere decir que solo se puede asegurar progreso cuando $\mu$ es compatible con $\Sigma$.

\subsection{Extensión con recursión}\label{lambda_calculo:recursion}
Queremos dar al lenguaje $\lambda$, la capacidad de interpretar expresiones recursivas. Definimos, entonces, la función $fix M$ que dado para función $M$ devuelve el punto fijo de dicha función, es decir, un valor $x$ tal que $M x$ evalúa a $x$.

Veamos un ejemplo, supongamos que tenemos la función $f(n) = \textbf{If } n=0 \textbf{ then } 1 \textbf{ else } n*f(n-1)$. Cuando evaluamos $f$ en $0$, obtenemos su definición para el valor $O$, cuando la evaluamos en $1$, la definimos para $0$ y $1$, con cada valor que tengamos, la iremos definiendo para ese valor y para todos los anteriores. 

Podríamos pensar que cuando evaluamos $n\to\inf$, obtenemos una función que se define para todos los valores naturales, es decir obtenemos la función factorial, propiamente dicha.

\paragraph{Términos}
$$M~:=~\dots~|~\lambdaFix{M}$$

\paragraph{Regla de tipado}
\begin{equation*}
    \frac{\judgeType{\Gamma}{M}{\sigma\to\sigma}}{\judgeType{\Gamma}{\lambdaFix{M}}{\sigma}}(\text{T-Fix})
\end{equation*}

\paragraph{Reglas de evaluación}
\begin{equation*}
    \frac{M_1\to M'_1}{\lambdaFix{M_1}\to\lambdaFix{M'_1}}(\text{E-Fix})
\end{equation*}
\vspace*{5mm}
\begin{equation*}
\frac{}{\lambdaFix{\lambdaAbs{x}{\sigma}{M}}\to\replaceBy{M}{x}{\lambdaFix{\lambdaAbs{x}{\sigma}{M}}}}(\text{E-FixBeta})
\end{equation*}