\section{Inferencia de tipos}
Queremos modificar el lenguaje de cálculo lambda para que las expresiones no necesiten las notaciones de tipos explicitas. Para esto, debemos definir términos a partir de los cuales se pueda \textbf{inferir} la información faltante. Esto es, debemos poder convertirlos en términos bien tipados del cálculo lambda sin ningún problema.

Este nuevo lenguaje nos evitará la sobrecarga de tener que declarar y manipular todos los tipos al momento de escribir un programa. Un compilador deberá realizar la inferencia de tipos antes de compilar el program.

\subsubsection*{Términos}
El lenguaje sin tipos tendrá todos los términos del lenguaje $\lambda$ con el que estuvimos trabajando hasta ahora, con la diferencia de que si en ellos había una notación de tipo, entonces la obviamos:

\begin{equation*}
	\begin{split}
		M~::=~&x \\
		|~~~&true~|~false~|~\lambdaIf{M}{P}{Q} \\
		|~~~&0~|~succ(0)~|~isZero(M) \\
		|~~~& \lambdaAbsI{x}{M}~|~M~N~|~\lambdaFix{M}
	\end{split}
\end{equation*}

Si bien la mayoría de los términos son iguales a los originales, necesitaríamos alguna forma de convertir los términos del lambda cálculo a no tipados y viceversa. 

\paragraph{Función Erase:} Dado un término del lambda cálculo, \textbf{elimina} las anotaciones de tipos de las abstracciones que contenga. Por ejemplos: 

$$\Erase(\lambdaAbs{x}{Nat}{\lambdaAbs{f}{Nat\to Nat}{f~x}}) = \lambdaAbsI{x}{\lambdaAbsI{f}{f~x}}$$

\paragraph{Chequeos de tipo:}
Realizar el chequeo de tipo es determinar, para un término estándar (del lenguaje $\lambda$ tipado) $M$, si existe $\Gamma$ y $\sigma$ tales que $\judgeType{\Gamma}{M}{\sigma}$ es derivable. Osea que nos indica si $M$ es un término tipable o no. Solo hay que seguir la estructura sintáctica de $M$ para reconstruir una derivación de juicio. 

\subsubsection*{Inferencia de tipos:} Dado un término $U$ sin notaciones de tipo, se trata hallar un término estándar (con anotaciones de tipos) $M$ bien tipado que sea equivalente. Osea tal que:
\begin{enumerate}
	\item $\judgeType{\Gamma}{M}{\sigma}$ para algún contexto $\Gamma$ y algún tipo $\sigma$, y
	\item $\Erase(M) = U$
\end{enumerate}

Si encontramos este $M$, $U$ será de tipo $\sigma$- Sino, $U$ será una expresión no tipable en nuestro lenguaje.


\subsubsection{Variables de tipo}
Una \textbf{variables de tipo} s es una variable que representa una expresión de tipo arbitraria e indica que tendremos una solución válida sin importar por que expresión de tipo la remplacemos.

Debemos agregar esta nueva expresión a las expresiones  de tipo del cálculo lambda:

$$\sigma,\tau~::=~\text{s}~|~Nat~|~Bool~|~\sigma\to\tau$$

Supongamos que tenemos la expresión $U = \lambdaAbsI{x}{x}$. En este caso, $U$ puede ser la función identidad de cualquier tipo.
Entonces, la expresión $M=\lambdaAbs{x}{\textbf{s}}{x}$ es el resultado de la inferencia de $U$ (\textbf{s} pudiendo ser cualquier tipo de nuestro lenguaje).



\subsection{Sustitución de tipos}
Una función $S$ de sustitución es una función que mapea variables de tipo en expresiones de tipo y puede ser aplicada a expresiones de tipos, términos y contextos de tipado.

Describimos $S$ usando la notación $\{\sigma_1/t_1,\dots,\sigma_n/t_n\}$. Esto significa que la variable $t_i$ debe ser remplazada por $\sigma_i$.

\paragraph{Conjunto soporte de $S$:} Conjunto $\{t_1,\dots,t_n\}$ que contiene a todas las variables que afecta $S$.

\paragraph{Ejemplos:}
\begin{itemize}
	\item Si $S = \{  Bool/t \}$, entonces $S(\lambdaAbs{x}{\text{t}}{x}) = \lambdaAbs{x}{ Bool}{x}$ y el conjunto soporte de $S$ es $\{t\}$.
	\item La sustitución cuyo soporte es $\emptyset$, es la \textbf{sustitución identidad}
\end{itemize}

\paragraph{Instancia de juicio de tipado:}
Si tenemos dos juicios de tipado $\judgeType{\Gamma}{M}{\sigma}$ y $\judgeType{\Gamma'}{M'}{\sigma'}$ tales que $$S(\judgeType{\Gamma}{M}{\sigma}) = \judgeType{S\Gamma}{SM}{S\sigma} =\judgeType{\Gamma'}{M'}{\sigma'}$$, entonces decimos que $\judgeType{\Gamma'}{M'}{\sigma'}$ es instancia de $\judgeType{\Gamma}{M}{\sigma}$. )

\paragraph{Composición de sustituciones} La composición de sustituciones de $S$ y $T$, denotada $S\circ T$, es la sustitución que se comporta como sigue:

$$(S\circ T)(\sigma) = S(T\sigma)$$

\paragraph{Preorden de sustituciones} Una sustitución $S$ es \textbf{más general} que $T$ si existe una sustitución $U$ tal que $T = U\circ S$, es decir, si $T$ es una instancia de $S$.


\subsubsection{Unificación}
El algoritmo de inferencia que vamos a proponer analiza un término (sin notaciones de tipos) a partir de sus subtérminos. Una vez obtenida la información de cada uno de ellos, debe determinar si es \textbf{consistente} y, si lo es, \textbf{sintetizar} la información del término original a partir de ésta.

Para realizar la síntesis, debemos \textbf{compatibilizar} la información de tipos de cada subtérmino. Por cada una de las variables del término, tenemos que tomar los tipos que le asigno cada uno de ellos y \textbf{unificarlos}. Es decir, debemos encontrar una sustitución $S$ que nos permita remplazar los tipos que dió cada subexpresión por un tipo único. 


\paragraph{Ecuación de unificación} Es una expresión de la forma $\sigma_1 \equalDot \sigma_2$ cuya solución es una sustitución tal que $S\sigma_1 = S\sigma_2$. Por lo general tendremos un conjunto de ecuaciones de unificación y la solución a dicho conjunto será la sustitución que unifica todas las expresiones.

\paragraph{Unificador más general (MGU):} Una sustitución $S$ que resuelva $\{\sigma_1 \equalDot \sigma_1',\dots,\sigma_n \equalDot \sigma_n'\}$ y que es más general que cualquier otra sustitución que la resuelva.

\paragraph{Teorema} Si $\{\sigma_1 \equalDot \sigma_1',\dots,\sigma_n \equalDot \sigma_n'\}$ tiene solución, entonces existe un MGU y además es único salvo renombre de variables.

\subsubsection{Algoritmo de unificación de Martelli-Montanari}
Es un algoritmo no deterministico que, a través de \textbf{reglas de simplificación}, nos permite reescribir un conjunto de ecuaciones de unificación (\textit{goal}) $\{\sigma_1 \equalDot \sigma_1',\dots,\sigma_n \equalDot \sigma_n'\}$. 

Si, cuando el algortimo termina, conseguimos  una secuencia de reescrituras que terminan en un \textit{goal} vacío, entonces la corrida fue \textbf{exitosa}, los pasos en los que realizamos sustituciones serán soluciones parciales al \textit{goal} original y la composición de todas ellas será su MGU. Si no sucede esto, entonces la secuencia será \textbf{fallida}.

\subsubsection{Reglas de reducción}

\begin{enumerate}
	\item \textbf{Descomposición}
	
	$\{\sigma_1\to\sigma_2 \equalDot\tau_1\to\tau_2\}\cup G\mapsto \{\sigma_1\equalDot\tau_1,~\sigma_2 \equalDot\tau_2\}\cup G$
	\item \textbf{Eliminación de par trivial}
	
	$\{Nat \equalDot Nat\}\cup G\mapsto G$
	
	$\{ Bool \equalDot Bool\}\cup G\mapsto G$
	
	$\{\text{s} \equalDot\text{s}\}\cup G\mapsto G$
	\item \textbf{Swap} Si $\sigma$ no es una variable de tipo,
	
	$\{\sigma \equalDot\text{s}\}\cup G\mapsto \{\text{s}\equalDot\sigma\}\cup G$
	
	\item \textbf{Eliminación de variable} Si $s\notin FV(\sigma)$
	
	$\{\text{s}\equalDot\sigma\}\cup G\mapsto_{\sigma/s} G[\sigma/s]$
	
	\item \textbf{Falla} Sea $T =\{( Bool,Nat), (Nat, \sigma_1\to\sigma_2), ( Bool, \sigma_1\to\sigma_2)\}$ y $T^{-1}$ el conjunto con cada tupla de $T$ invertida. Si $(\sigma,\tau)\in T\cup T^{-1}$:
	
	$\{\sigma\equalDot\tau\}\cup G\mapsto \red{\texttt{falla}}$.
	
	\item \textbf{Occur Check} Si $s\neq\sigma$ y $s\in FV(\sigma)$
	
	$\{\text{s}\equalDot\sigma\}\cup G\mapsto \red{\texttt{falla}}$
\end{enumerate}

\subsubsection{Propiedades del algoritmo}
El algoritmo de Martinelli-Montanari siempre termina. Sea $G$ un conjunto de pares, entonces:
\begin{itemize}
	\item Si $G$ tiene un unificador, el algoritmo termina exitosamente y retorna un MGU.
	\item Si $G$ no tiene un unificador, el algoritmo termina con \red{\texttt{falla}}.
\end{itemize}

\subsubsection{Ejemplo de aplicación}
\begin{equation*}
	\begin{split}
		&\{(Nat\to r)\to(r\to u) \equalDot t\to(s\to s)\to t\}
		\mapsto^1 \{Nat\to r\equalDot t,~r\to u\equalDot (s\to s)\to t\} \\
		&\mapsto^3 \{t\equalDot Nat\to r,~r\to u\equalDot (s\to s)\to t\} 
		\mapsto^4_{Nat\to r/t} \{r\to u\equalDot (s\to s)\to (Nat\to r)\} \\
		&\mapsto^1 \{r\equalDot(s\to s),~u\equalDot Nat\to r\} 
		\mapsto^4_{s\to s/r} \{u\equalDot Nat\to (s\to s)\}
		\mapsto^4_{Nat\to(s\to s)/u} \emptyset \\
	\end{split}
\end{equation*}

Entonces, el MGU es 

$\{Nat\to(s\to s)/u\}\circ\{s\to s/r\}\circ\{Nat\to r/t\} = \{Nat\to(s\to s)/u,~s\to s/r,~Nat\to (s\to s)/t\}$

\subsection{Función de inferencia \texorpdfstring{$\mathbb{W}$}{W}}
Dada una expresión $U$ sin notación de tipos, la función $\mathbb{W}$, devolverá un juicio de tipado con una expresión tipada $M$ que corresponde a $U$. Esta función será ejecutada de manera recursiva sobre las sub-expresiones de $U$ y sustituirá, si es posible, los tipos de cada una de ellas para que tengan ``sentido'' en $U$.




\subsubsection{Propiedades deseables de \texorpdfstring{$\WFunc$}{W}}
Dado un término $U$, $\WFunc(U)$ nos devolverá, si tiene éxito, una terna de tres elementos que contendrá un contexto de tipado $\Gamma$ una expresión $M$ y un $\sigma$ (notamos $\WFunc(U) = \judgeType{\Gamma}{M}{\sigma}$).

Queremos que $\WFunc$ sea \textbf{correcto} y \textbf{completo}.

\paragraph{Correctitud} Si $\WFunc(U) = \judgeType{\Gamma}{M}{\sigma}$ entonces vale que:
\begin{itemize}
	\item $\Erase(M) = U$ y
	\item $\judgeType{\Gamma}{M}{\sigma}$ es derivable.
\end{itemize}

\paragraph{Completitud} Si $\judgeType{\Gamma}{M}{\sigma}$ es derivable y $\Erase(M) = U$ entonces $\WFunc(U)$ tiene éxito y produce un juicio $\judgeType{\Gamma'}{M'}{\sigma'}$ que es instancia de $\judgeType{\Gamma}{M}{\sigma}$. 

\subsubsection{Algoritmo de inferencia}
$\WFunc$ va a hacer recusión sobre la estructura de $U$. Primero la vamos a definir para las construcciones más simples y, luego, para las compuestas. Además, usaremos el algoritmo de unificación para combinar los resultados de los pasos recursivos y, así, obtener un tipado consistente.

\subsubsection{Constantes y variables}
\begin{equation*}
	\begin{split}
		\WFunc(\red{true}) &\equalDef \judgeType{\emptyset}{true}{Bool} \\
		\WFunc(\red{false}) &\equalDef \judgeType{\emptyset}{false}{Bool} \\
		\WFunc(\red{x}) &\equalDef \judgeType{\{x:s\}}{x}{s}, \text{ \textit{s} variable fresca } \\
		\WFunc(\red{0}) &\equalDef \judgeType{\emptyset}{0}{Nat} \\
	\end{split}
\end{equation*}

\subsubsection{Caso \textit{succ}}
$\WFunc(\red{succ(U)}) \equalDef \judgeType{S\Gamma}{S~succ(M)}{Nat}$
\begin{centrado}
	\begin{itemize}
		\item $\WFunc(U) = \judgeType{\Gamma}{M}{\tau}$
		\item $S = MGU\{\tau\equalDot Nat\}$
	\end{itemize}
\end{centrado}

\subsubsection{Caso \textit{pred}}
$\WFunc(\red{pred(U)}) \equalDef \judgeType{S\Gamma}{S~pred(M)}{Nat}$
\begin{centrado}
	\begin{itemize}
		\item $\WFunc(U) = \judgeType{\Gamma}{M}{\tau}$
		\item $S = MGU\{\tau\equalDot Nat\}$
	\end{itemize}
\end{centrado}

\subsubsection{Caso \textit{isZero}}
$\WFunc(\red{isZero(U)}) \equalDef \judgeType{S\Gamma}{S~isZero(M)}{Bool}$
\begin{centrado}
	\begin{itemize}
		\item $\WFunc(U) = \judgeType{\Gamma}{M}{\tau}$
		\item $S = MGU\{\tau\equalDot Nat\}$
	\end{itemize}
\end{centrado}

\subsubsection{Caso \textit{ifThenElse}}
$\WFunc(\red{\lambdaIf{U}{V}{W}}) \equalDef \judgeType{S\Gamma_1\cup S\Gamma_2\cup S\Gamma_3}{S~(\lambdaIf{M}{P}{Q})}{S\sigma}$
\begin{centrado}
	\begin{itemize}
		\item $\WFunc(U) = \judgeType{\Gamma_1}{M}{\rho}$
		\item $\WFunc(V) = \judgeType{\Gamma_2}{P}{\sigma}$
		\item $\WFunc(W) = \judgeType{\Gamma_3}{Q}{\tau}$
		\item $S = MGU\{\sigma_1\equalDot \sigma_2~|~x:\sigma_1\in\Gamma_i~\land~x:\sigma_2\in\Gamma_j,~i\neq j\}\cup\{\sigma\equalDot\tau\,~\rho\equalDot Bool\}$
	\end{itemize}
\end{centrado}

\subsubsection{Caso aplicación}
$\WFunc(\red{U~V}) \equalDef \judgeType{S\Gamma_1\cup S\Gamma_2}{S~(M~N)}{St}$
\begin{centrado}
	\begin{itemize}
		\item $\WFunc(U) = \judgeType{\Gamma_1}{M}{\tau}$
		\item $\WFunc(V) = \judgeType{\Gamma_2}{N}{\rho}$
		\item $S = MGU\{\sigma_1\equalDot \sigma_2~|~x:\sigma_1\in\Gamma_i~\land~x:\sigma_2\in\Gamma_j,~i\neq j\}\cup\{\tau\equalDot\rho\to t\}$ con $t$ variable fresca
	\end{itemize}
\end{centrado}

\subsubsection{Caso abstracción}
$$\WFunc(\red{\lambdaAbsI{x}{U}}) \equalDef \judgeType{\Gamma \backslash\{x:\tau\}}{\lambdaAbs{x}{\tau}{M}}{\tau\to\rho}$$


Sea $\WFunc(U) = \judgeType{\Gamma}{M}{\rho}$, si $\Gamma$ tiene información de tipos para $x$, es decir $x:\tau\in\Gamma$ para algún $\tau$, entonces:

$$\WFunc(\red{\lambdaAbsI{x}{U}}) \equalDef \judgeType{\Gamma \backslash\{x:\tau\}}{\lambdaAbs{x}{\tau}{M}}{\tau\to\rho}$$

Si $\Gamma$ no tiene información de tipos para $x$ ($x\notin \text{Dom}(\Gamma)$), entonces elegimos una variable fresca $s$ y

$$\WFunc(\red{\lambdaAbsI{x}{U}}) \equalDef \judgeType{\Gamma}{\lambdaAbs{x}{s}{M}}{s\to\rho}$$

\subsubsection{Caso \textit{fix}}
$\WFunc(\red{\lambdaFix{(U)}}) \equalDef \judgeType{S\Gamma}{S~\lambdaFix{(M)}}{St}$
\begin{centrado}
	\begin{itemize}
		\item $\WFunc(U) = \judgeType{\Gamma_1}{M}{\tau}$
		\item $S = MGU\{\tau\equalDot t\to t\}$ con $t$ variable fresca
	\end{itemize}
\end{centrado}

\subsubsection{Complejidad del algoritmo}
Tanto la unificación como la inferencia para cálculo lambda se puede realizar en tiempo lineal. Sin embargo, el tipo principal asociado a un término sin anotaciones puede ser \red{exponencial} en el tamaño del término.

\subsubsection{Extensión del algoritmo a nuevos tipos}
Para extender el algoritmo a otros tipos, debemos agregar los casos correspondientes teniendo en cuenta que los llamados recursivos devuelven un contexto, un término y un tipo sobre los que no podemos asumir nada.
Si la nueva regla tiene tipos iguales o contextos repetidos, debemos unificarlos. Y si la regla liga alguna variable entonces vamos a poder dividir en dos casos: 
\begin{itemize}
	\item Si alguno de los contextos recursivos tiene información sobre esa variable, entonces sacamos su tipo del contexto que la contenga.
	\item Sino, le asignamos una variable fresca de tipo.
\end{itemize}

Si la regla tiene restricciones adicionales, se incorporan como posibles fallas.

\subsubsection{Extensión del algoritmo para listas}
$\WFunc([]) = \judgeType{\emptyset}{\List{t}}{t}$ con $t$ variable fresca.

$\WFunc(U_1::U_2) = \judgeType{S\Gamma_1\cup S\Gamma_2}{S(M_1::M_2)}{S[\sigma_1]}$

\begin{centrado}
	\begin{itemize}
		\item $\WFunc(U_i) = \judgeType{\Gamma_i}{M_i}{\sigma_i}$
		\item $S = MGU\{\sigma_1\equalDot \sigma_2~|~x:\sigma_1\in\Gamma_i~\land~x:\sigma_2\in\Gamma_j,~i\neq j\}\cup\{[\sigma_1]\equalDot\sigma_2\} $
	\end{itemize}
\end{centrado}


$\WFunc(\lambdaListCase{U}{U_2}{U_3}) =$

\quad$\judgeType{S\Gamma_1\cup S\Gamma_2\cup S\Gamma_3\backslash\{h,t\}}{S(\lambdaListCase{M_1}{M_2}{M_3})}{S\sigma_2}$

\begin{centrado}
	\begin{itemize}
		\item $\WFunc(U_i) = \judgeType{\Gamma_i}{M_i}{\sigma_i}$ con $i = 1,2,3$
		\item $S = MGU\{\sigma_1\equalDot \sigma_2~|~x:\sigma_1\in\Gamma_i~\land~x:\sigma_2\in\Gamma_j,~i\neq j\}\cup\{\sigma_1\equalDot [t_1],~t_1\equalDot \tau_h,~\tau_t\equalDot \sigma_1,~\sigma_2\equalDot\sigma_3\} $ con 
		\[ \tau_h = \begin{cases} 
		\sigma_h & \text{si } h:\sigma_h\in\Gamma_3 \\
		t_2 & \text{sino} \\
		\end{cases} \hspace*{5mm}\text{ y }\hspace*{5mm}
		\tau_t = \begin{cases} 
		\sigma_t & \text{si } t:\sigma_t\in\Gamma_3 \\
		t_2 & \text{sino} \\
		\end{cases}
		\]
	\end{itemize}
\end{centrado}